%Formattazione
%Scelgo report al posto di article per poter dividere l'elaborato in 3 capitoli
\documentclass[a4paper,10pt]{article}
\usepackage[utf8]{inputenc}

%Pacchetti che avevo già in altri file
%\usepackage{ucs}
\usepackage[british]{babel}
%Per i circuiti quantistici
%\usepackage[braket, qm]{qcircuit}
\usepackage{amsmath}
\usepackage{amssymb}
\usepackage{amsthm}
\usepackage{physics}
\usepackage{centernot}

%Pacchetti aggiunti dall'internet:
%Questo Pacchetto serve per modificare le impostazioni del line spacing inline
\usepackage{setspace}
\usepackage{graphicx}
%Per centrare il circuito di QFT (Chapter_2)
\usepackage{changepage}
%Per le liste nelle descrizioni degli algoritmi
\usepackage{enumitem}
%Per la sfera di Bloch
\usepackage{tikz}
%Per i diagrammini commutativi
\usetikzlibrary{cd}
\usetikzlibrary{arrows.meta}
\usetikzlibrary{decorations.pathmorphing} %For squiggly lines
% To use some colours
\usepackage{xcolor}

%Per le appendici
\usepackage[title]{appendix}

%Per il Frontespizio
\usepackage{tabularx}
\usepackage{geometry}

%Dal Darione
\usepackage{layaureo}

\usepackage{mathtools}
\setlist[description]{leftmargin=3.2em,labelindent=3.2em}

%Comandi miei
%Sillabazione
%\hyphenation{distinto}
\newcommand{\R}{\mathbb{R}}
\newcommand{\C}[1]{\mathbb{C}^{#1}}
\newcommand{\Zn}[1]{\mathbb{Z}/{#1}\mathbb{Z}}
\newcommand{\Z}{\mathbb{Z}}
\newcommand{\N}{\mathbb{N}}
\newcommand{\K}{\mathbb{K}}
\newcommand{\T}{\mathbb{T}}
\newcommand{\bket}[2]{\braket{#1\,}{\, #2}}
\newcommand{\bbket}[3]{\mel{#1\,}{\, #2 \,}{\, #3}}
\renewcommand{\op}[2]{\ket{#1}\! \bra{#2}}
%New useful operators:
\DeclareMathOperator{\sign}{sign}
\DeclareMathOperator{\ima}{im}
\DeclareMathOperator{\Ima}{Im}
\DeclareMathOperator{\coim}{Coim}
\DeclareMathOperator{\coker}{coker}
\DeclareMathOperator{\fct}{Fct}

\newtheorem{post}{Postulate}
\newtheorem{thm}{Theorem}[section]
\newtheorem{cor}[thm]{Corollary}
\newtheorem{lem}[thm]{Lemma}
\newtheorem{prop}[thm]{Proposition}

\newtheoremstyle{algoritmo}
{1.7em}%〈Space above〉
{1em}%〈Space below〉
{}%〈Body font〉
{}%〈Indent amount〉
{\bfseries}%〈Theorem head font〉
{}%〈Punctuation after theorem head〉
{.5em}%〈Space after theorem head〉
{\thmname{#1}\thmnumber{ #2}\thmnote{: #3.}}%〈Theorem head spec(can be left empty, meaning ‘normal’)〉

\newtheoremstyle{definizione}
{1.0em}
{0.7em}
{}
{}
{\bfseries}
{}
{1.0em}
{\thmname{#1}\thmnumber{ #2}\thmnote{: #3. \\*}}


\theoremstyle{algoritmo}
\newtheorem*{algo}{Algoritmo}
\newtheorem*{ex}{Example}

\theoremstyle{definizione}
\newtheorem{defn}[thm]{Definition}%[section]
%\theoremstyle{definizione}
\newtheorem{rem}{Remark}

\numberwithin{equation}{section}

%Simbolo QED
\renewcommand{\qedsymbol}{\ensuremath{\blacksquare}}
\usepackage{hyperref}

\begin{document}

\tableofcontents
\section{Preliminaries}
Here a couple of preliminariy definitions, before we move on to categories.

\begin{defn}[$R$-module]
	A {\em left} $R$-module $\left(M, +, \cdot\right)$ is an abelian group $\left(M, +\right)$ on which is defined a map
	\begin{align}
		\cdot\colon R \cross M &\to M \\
		\left(r, m\right) &\mapsto rm
	.\end{align} 
	It is called scalar multiplication and satisfies
	\begin{enumerate}
		\item for any $r \in R$ the induced map
			 \begin{align}
				\dot{r}\colon M &\to M \\
				m &\mapsto rm
			\end{align} 
			is a homomorphism of abelian groups.
		\item The map that sends each $r \in R$ to its associated endomorphism (as in the above)
			\begin{align}
				\phi\colon R &\to \mathrm{End}_{\Z}\left( M \right) \\
				r &\mapsto \dot{r}
			\end{align} 
			is a morphism of rings.
	\end{enumerate}
	If $\phi$, instead of being a homomorphism is an antihomomorphism (i.e. it is a homomorphism
	\begin{align}
		\phi\colon R^{op} &\to \mathrm{End}_{ \Z}\left( M \right)
	,\end{align} 
	from the opposite ring, in which the operations are computed in the opposite direction), then $M$ is a {\em right} $R$-module.
	We denote {\em left} $R$-modules as $\prescript{}{R}{M}$, wheareas {\em right} $R$-modules as $M_R$.
\end{defn}

\begin{defn}[Bimodule]
	Let $R,S$ be rings.
	An abelian group $\left(M, +\right)$ is an $R,S$-bimodule ${}_{R}{M}_{S}$ iff 
	$\prescript{}{R}{M}$ is a left $R$-module, $M_S$ is a right $R$-module and
	\begin{equation}
		r(xs) = (rx)s
	\end{equation} 
	for any $r \in R$, $s \in S$, $x \in M$.
\end{defn}

\begin{defn}[Tensor product of modules]
	Let $S$ be a ring, $M_S \in \mathsf{Mod}\text{-}S$ and ${}_SN \in S\text{-}\mathsf{Mod}$.
	A map $\beta\colon M \cross N \to G$, to $G$ an abelian group, is called balanced iff
	it satisfies the following
	\begin{align}
		\beta(m + m', n) &= \beta(m,n) + \beta(m',n) \qquad & \,\forall\, m, m' \in M \text{ and } \,\forall\, n \in N\\
		\beta(m, n + n') &= \beta(m,n) + \beta(m,n') \qquad & \,\forall\, m \in M \text{ and } \,\forall\, n, n' \in N\\
		\beta(ms, n) &= \beta(m,sn) \qquad & \,\forall\, m \in M,\ \,\forall\, n \in N \text{ and } \,\forall\, s \in S
	.\end{align} 
	The tensor product of $M$ and $N$ is the pair $\left(M \otimes_{S} N, \tau \right)$,
	with $M \otimes_{S} N$ an abelian group and $\tau\colon M \cross N \to M \otimes_{S} N$ a map s.t.
	$\,\forall\, \beta\colon M \cross N \to G$ a balanced map, $\exists\, !\, \alpha\colon M \otimes_{S} N \to G$ an abelian group morphism
	s.t. the following diagram commutes
	\begin{equation}
	\begin{tikzcd}
		M \cross N \arrow[r, "\tau", rightarrow] \arrow[d, "\beta"', rightarrow] &
		M \otimes_{S} N \arrow[dl, "\alpha", rightarrow] \\
		G &
	\end{tikzcd}
	,\end{equation} 
	i.e. s.t. $\alpha \circ \tau = \beta$.
	In such a case we say that every balanced map $\beta$ factors through $\tau$ via an abelian group morphism.
\end{defn}

\begin{rem}[Construction of the tensor product]
	Consider $M$ and $N$ as before.
	Consider $M \cross N$ as a set.
	Let $\Z^{M \cross N}$ be the free abelian group with basis $\left(m, n\right) \in M \cross N$.
	Consider $H \triangleleft \Z^{M \cross N}$ generated by the elements of the form
	\begin{align}
		\big\{& \left(m + m', n\right) - \left(m, n\right) - \left(m',n \right),
	\left(m, n + n'\right) - \left(m, n\right) - \left(m, n'\right),\\
		      &\left(ms, n\right) - \left(m, sn\right)
		     \ \big|\ m,m' \in M,\ n,n' \in N,\ s \in S \big\}
	.\end{align} 
	Let $\tau\colon M \cross N \to \Z^{M \cross N}/H$, defined by $\left(m, n\right) \xmapsto{\tau} \left(m, n\right) + H$,
	then $\left(\Z^{M \cross N}, \tau\right)$ is a tensor product of $M$ and $N$.
\end{rem}

\begin{rem}[Tensor product as a module]
	Given $N_S \in \mathsf{Mod}\text{-}S$ and ${}_S M_R$ and $S$-$R$ bimodule,
	we want to construct a tensor product of the two, which is also a right $R$-module.
	(We can choose $S = \mathrm{End}_{R}\left( M \right)$, or $S = \Z$ for example).
	We define, for any $L_R \in \mathsf{Mod}\text{-}R$, the set
	\begin{equation}
		\beta \in \mathrm{Bal}(N \cross M, L_R)
	,\end{equation} 
	consisting of $\beta$ balanced maps (as defined above) s.t. $\beta(x, yr) = \beta(x,y) r$
	for all $x \in N$, $y \in M$ and $r \in R$.
Then, in this situation, we define the tensor product $N \otimes_{S} M$ as the right $R$-module, with a map $\tau$, s.t.
the following diagram commutes
\begin{equation}
\begin{tikzcd}
	N \cross M \arrow[r, "\tau", rightarrow] \arrow[d, "\beta"', rightarrow] &
	N \otimes_S M \arrow[dl, "\exists\, \alpha", dashrightarrow] \\
	L_R &
\end{tikzcd}
\end{equation} 
for all $L_R$, and $\beta \in \mathrm{Bal}(N \cross M, L_R)$.
In particular this gives a bijection
\begin{equation}
\begin{tikzcd}
	\mathrm{Bal}(N \cross M, L_R) \arrow[r, "\varphi", leftrightarrow] &
	\mathrm{Hom}_{R} \left( N \otimes_S M, L_R \right)
\end{tikzcd}
.\end{equation} 
\end{rem}


\section{Category theory}

\subsection{Categories and morphisms}

\begin{defn}[Category]
	A category $\mathsf{C}$ si determined by the following elements:
	\begin{itemize}
		\item $\mathrm{Ob}(\mathsf{C})$ a {\em class} of objects,
		\item $\,\forall\, X,Y \in \mathrm{Ob}(\mathsf{C})$ the data of a set of {\em arrows}  with {\em source} $X$ and {\em target} $Y$, denoted with $\mathrm{Hom}_{\mathsf{C}} \left( X, Y \right)$, whose elements are called {\em morphisms},
		\item an operation of composition, that acts as follows
			\begin{align}
				\circ\colon \mathrm{Hom}_{\mathsf{C}} \left( X, Y \right) \cross \mathrm{Hom}_{\mathsf{C}} \left( Y, Z \right) &\to \mathrm{Hom}_{\mathsf{C}} \left( X, Z \right)\\
				\left(f, g\right) &\mapsto g \circ f
			,\end{align} 
			for any $X,Y,Z \in \mathrm{Ob}(\mathsf{C})$ and is associative, i.e.
			\begin{equation}
				h \circ (g \circ f) = (h \circ g) \circ f
			,\end{equation} 
			whenever defined, i.e. $\,\forall\, X \xrightarrow{f} Y \xrightarrow{g} Z \xrightarrow{h} W$.
	\end{itemize}
	Also the set $\mathrm{End}_{\mathsf{C}}\left(X\right) \coloneqq \mathrm{Hom}_{\mathsf{C}} \left( X, X \right)$ always contains the element $id_X$, that is defined to act as: given any $f \in \mathrm{Hom}_{C} \left( X, Y \right)$, $g \in \mathrm{Hom}_{\mathsf{C}} \left( Z, X \right)$ 
	\begin{equation}
	f \circ id_X = f, \quad id_X \circ g = g
	.\end{equation} 
\end{defn}

\begin{ex}\leavevmode\vspace{-.2\baselineskip}
	\begin{itemize}
		\item $\mathsf{Sets}$: $\mathrm{Ob} \left(\mathsf{Sets}\right)$ are sets, and morphisms are set theoretic maps,
		\item $\mathsf{Top}$: $\mathrm{Ob} \left(\mathsf{Top}\right)$ are topological spaces, morphisms are continuous maps,
		\item $\mathsf{Semigroups}$: $\mathrm{Ob} \left(\mathsf{Semigrousp}\right)$ are sets with an associative operation, morphisms are homomorphisms of semigroups,
		\item $\mathsf{Monoids}$: $\mathrm{Ob} \left(\mathsf{Monoids}\right)$ are semigroups with a unit, morphisms are monoid morphisms,
		\item Clearly one can construct a lot more examples, we'll stop here.
	\end{itemize}
\end{ex} 

\begin{defn}[Opposite category]
	Given a category $\mathsf{C}$, one can define the opposite category $\mathsf{C}^{op}$, characterized by
	\begin{itemize}
		\item $\mathrm{Ob}(\mathsf{C}^{op}) \coloneqq \mathrm{Ob}(\mathsf{C})$,
		\item $\mathrm{Hom}_{\mathsf{C}^{op}} \left( X, Y \right) \coloneqq \mathrm{Hom}_{\mathsf{C}} \left( Y, X \right)$, with composition given by
			\begin{equation}
				g^{op} \circ_{\mathsf{C}^{op}} f^{op} \coloneqq \left( f \circ_{\mathsf{C}} g \right)^{op}
			.\end{equation} 
	\end{itemize} 
\end{defn}

\begin{defn}[iso-mono-epi morphisms]
	Let $X \xrightarrow{f} Y$ be a morphism in a category $\mathsf{C}$, then it is a(n)
	\begin{description}
		\item[monomorphism:] iff $\,\forall\, Z \begin{matrix} g_1 \\ \rightrightarrows \\ g_2 \end{matrix} X$ s.t. $f \circ g_1 = f \circ g_2 \implies g_1 = g_2$.
			We denote it with $f\colon Y \rightarrowtail Z$.
			It is said that $f$ is {\em left} erasable
		\item[epimorphism:] iff $\,\forall\, X \begin{matrix} h_1 \\ \rightrightarrows \\ h_2 \end{matrix} Y$ s.t. $h_1 \circ f = h_2 \circ f \implies h_1 = h_2$.
			We denote it with $f\colon Y \twoheadrightarrow Z$.
			It is said that $f$ is {\em right} erasable
		\item[isomorphism:] iff $\exists\,  Y \xrightarrow{g} X$ s.t. $g \circ f = id_X$ and $f \circ g = id_Y$.
	\end{description} 	
\end{defn}

\begin{rem}
	Note that if $X \xrightarrow{f} Y$ is an {\em iso}, then it is also {\em mono} and {\em epi}, but the converse is not always true.

	If, moreover, $X \xrightarrow{f} Y$ is an iso, we say that $X$ and $Y$ are {\em isomorphic} and we denote it with $X \simeq_{\mathsf{C}} Y$ (especially if we do not want to explicitly cite the isomorphism).
\end{rem}

\begin{defn}[Subcategory]
	A category $\mathsf{C}'$ is a subcategory of $\mathsf{C}$, denoted with $\mathsf{C}' \subset \mathsf{C}$ iff
	\begin{itemize}
		\item $\mathrm{Ob}(\mathsf{C}') \subset \mathrm{Ob}(\mathsf{C})$
		\item $\,\forall\, X,Y \in \mathrm{Ob}(\mathsf{C}')$, we have $\mathrm{Hom}_{\mathsf{C}'} \left( X, Y \right) \subset \mathrm{Hom}_{\mathsf{C}} \left( X, Y \right)$,
	\end{itemize} 
	and the two categories have the same composition and identities.
\end{defn}

\begin{defn}[Full subcategory]
	$\mathsf{C}' \subset \mathsf{C}$ is said to be a {\em full} subcategory iff $\,\forall\, X, Y \in \mathrm{Ob}(\mathsf{C}')$ 
	\begin{equation}
	\mathrm{Hom}_{\mathsf{C}'} \left( X, Y \right) = \mathrm{Hom}_{\mathsf{C}} \left( X, Y \right)
	.\end{equation} 
\end{defn}

\begin{defn}[Discrete/finite/grupoid]
	A category $\mathsf{C}$ is said to be
	\begin{description}
		\item[Discrete:] iff the only morphisms are the identities.
			Note that a set can be naturally identified as a {\em discrete} category.
		\item[Finite:] iff the family $\mathrm{Mor}(\mathsf{C})$ of all the morphisms in $\mathsf{C}$ (and, as a consequence $\mathrm{Ob}(\mathsf{C})$) is a finite set.
		\item[Grupoid:] iff all the morphisms are isomoprhisms.
			Note that a group $G$ can be identified with a {\em grupoid} category $\mathsf{C}$ with only one element $X \in \mathrm{Ob}\mathsf{C}$ and
			\begin{equation}
				\mathrm{Hom}_{\mathsf{C}} \left( X, X \right) \coloneqq G
			.\end{equation} 
	\end{description} 
\end{defn}

\begin{defn}[Product category]
	Let $\mathsf{C}$ and $\mathsf{D}$ be two categories, one can define their product $\mathsf{C}\cross \mathsf{D}$ as the category characterized by
	\begin{itemize}
		\item $\mathrm{Ob} \left(\mathsf{C}\cross \mathsf{D}\right) \coloneqq \mathrm{Ob} \left(\mathsf{C}\right) \cross \mathrm{Ob} \left(\mathsf{D}\right)$,
		\item $\mathrm{Hom}_{\mathsf{C}\cross \mathsf{D}} \left( (X,Y), (X',Y') \right) \coloneqq \mathrm{Hom}_{\mathsf{C}} \left( X, Y \right) \cross \mathrm{Hom}_{\mathsf{D}} \left( X', Y' \right)$,
		\item $\left(f, g\right)\circ_{\mathsf{C}\cross \mathsf{D}}\left(f', g'\right) \coloneqq \left(f \circ_{\mathsf{C}}g, f' \circ_{\mathsf{D}} g' \right)$.
	\end{itemize} 
\end{defn}

\begin{defn}[Initial/terminal/zero object]
	An object $X \in \mathrm{Ob} \left(\mathsf{C}\right)$ is said to be
	\begin{description}
		\item[Initial:] iff $\,\forall\, Y \in \mathrm{Ob} \left(\mathsf{C}\right)$ we have $\mathrm{Hom}_{\mathsf{C}} \left( X, Y \right) = \left\{ \mathrm{pt} \right\}$,
		\item[Terminal:] iff $\,\forall\, Y \in \mathrm{Ob} \left(\mathsf{C}\right)$ we have $\mathrm{Hom}_{\mathsf{C}} \left( Y, X \right) = \left\{ \mathrm{pt} \right\}$,
		\item[Zero:] iff it is both an {\em initial} and {\em terminal} object.
	\end{description} 
	In the above list we have denoted with $\left\{ \mathrm{pt} \right\}$ the singleton, i.e. any set with only one element.
\end{defn}

\begin{defn}[Zero morphism]
	Let $\mathsf{C}$ be category with a zero object $0_{\mathsf{C}}$.
	Given $X, Y \in \mathrm{Ob} \left(\mathsf{C}\right)$ we can define the
	$0$-morphism from $X$ into $Y$ as the unique map
	\begin{equation}
		X \xrightarrow{\alpha} 0_{\mathsf{C}} \xrightarrow{\beta} Y
	.\end{equation} 
\end{defn}

\subsection{Functors}

\begin{defn}[Functor]
	Given two categories $\mathsf{C}$ and $\mathsf{D}$, a functor $F$ between them is defined by:
	\begin{itemize}
		\item a map $F\colon \mathrm{Ob} \left(\mathsf{C}\right) \to \mathrm{Ob} \left(\mathsf{D}\right)$,
		\item a collection of maps, also denoted by $F$, given $\,\forall\, X,Y \in \mathrm{Ob} \left(\mathsf{C}\right)$
			\begin{equation}
			F\colon \mathrm{Hom}_{\mathsf{C}} \left( X, Y \right) \to \mathrm{Hom}_{\mathsf{D}} \left( FX, FY \right) 
			,\end{equation} 
			s.t. $F(id_X) = id_Y$ and $\,\forall\, f,g$ these maps preserve composition, i.e. 
			\begin{equation}
			 F \left( g \circ_{\mathsf{C}} f \right) = F(g) \circ_{\mathsf{D}} F(f)
			.\end{equation}
	\end{itemize}
\end{defn}

\begin{defn}[Full/faithful/essentially surjective/conservative functors]
	Let $\mathsf{C} \xrightarrow{F} \mathsf{D}$ be a functor, then it is said to be
	\begin{description}
		\item[Full] iff $\,\forall\, X,Y \in \mathrm{Ob} \left(\mathsf{C}\right)$ the map $\mathrm{Hom}_{\mathsf{C}} \left( X, Y \right) \xrightarrow{F} \mathrm{Hom}_{\mathsf{D}} \left( FX, FY \right)$ is surjective,
		\item[Faithful] iff $\,\forall\, X,Y \in \mathrm{Ob} \left(\mathsf{C}\right)$ the map $\mathrm{Hom}_{\mathsf{C}} \left( X, Y \right) \xrightarrow{F} \mathrm{Hom}_{\mathsf{D}} \left( FX, FY \right)$ is injective,
		\item[Fully faithful] iff $\,\forall\, X,Y \in \mathrm{Ob} \left(\mathsf{C}\right)$ the map $\mathrm{Hom}_{\mathsf{C}} \left( X, Y \right) \xrightarrow{F} \mathrm{Hom}_{\mathsf{D}} \left( FX, FY \right)$ is bijective,
		\item[Essentially surjective] iff $\,\forall\, Y \in \mathrm{Ob} \left(\mathsf{D}\right)\ \exists\, X \in \mathrm{Ob} \left(\mathsf{C}\right) \text{ s.t. } FX \simeq_{\mathsf{D}} Y$,
		\item[Conservative] iff $X \xrightarrow{f} Y$ is an isomorphism in $\mathsf{C}$ as soon as $F(f)$ is an isomorphism in $\mathsf{D}$.
	\end{description} 
\end{defn}

\begin{rem}
	A fully faithful functor $F\colon \mathsf{C} \to \mathsf{D}$ is conservative.
\end{rem}


\begin{defn}[Concrete category]
	A category $\mathsf{C}$ is called {\em concrete} iff it is equipped with a faithful functor to $\mathsf{Sets}$.
\end{defn}


\begin{defn}[Contravariant functor]
	We define a {\em contravariant} functor from $\mathsf{C}$ to $\mathsf{C}'$ to be a functor from $\mathsf{C}^{op}$ to $\mathsf{C}'$, i.e. it satisfies
	\begin{equation}
		F(g \circ f) = F(f) \circ F(g)
	.\end{equation}
	We denote with $\mathrm{op}\colon \mathsf{C} \to \mathsf{C}^{op}$ to be the contravariant functor associated with $id_{\mathsf{C}^{op}}$.
	Sometimes functors are called {\em covariant} in order to emphasize the fact that they are not {\em contravariant}.
\end{defn}

\begin{rem}
	Notice that, given $F\colon \mathsf{C} \to \mathsf{D}$ and $G\colon \mathsf{D} \to \mathsf{E}$ functors, then
	\begin{itemize}
		\item if both $F$ and $G$ are either covariant or contravariant, then $F \circ G$ is covariant,
		\item if one of them is covariant and the other is contravariant, then $F \circ G$ is contravariant.
	\end{itemize}
\end{rem}

\begin{defn}[Bifunctor]
	A {\em bifunctor} $F$ from $\left(\mathsf{C}, \mathsf{D}\right)$ to $\mathsf{E}$ is a functor from the product category, i.e.
	\begin{equation}
	F\colon \mathsf{C}\cross \mathsf{D} \to \mathsf{E} 
	.\end{equation}
	In particular, fixed $X \in \mathsf{C}$ and $Y \in \mathsf{D}$,
	then $F \left(X, - \right)\colon \mathsf{D} \to \mathsf{E}$ and
	$F \left( - , Y \right)\colon \mathsf{C} \to \mathsf{E}$ are functors.
	Moreover, for any morphism $f\colon X \to X'$ in $\mathsf{C}$ and
	$g\colon Y \to Y'$ in $\mathsf{D}$, then the following diagram commutes:
	\begin{equation}
	\begin{tikzcd}
		F(X,Y) \arrow[r, "{F(X,g)}", rightarrow]\arrow[d, "{F(f,Y)}"', rightarrow] & F(X,Y')\arrow[d, "{F(f,Y')}", rightarrow] \\
		F(X',Y) \arrow[r, "{F(X',g)}"', rightarrow] & F(X',Y')
	\end{tikzcd}
	.\end{equation} 
\end{defn}

\begin{ex}
	Given a category $\mathsf{C}$, there is a natural bifunctor
	\begin{equation}
	F = \mathrm{Hom}_{\mathsf{C}} \left( -, - \right)\colon \mathsf{C}^{op} \cross \mathsf{C} \to \mathsf{Sets}
	.\end{equation} 
	It is defined as follows.
	On objects it acts as
	\begin{align}
		F\colon \mathsf{C}^{op} \cross \mathsf{C} &\to \mathsf{Sets} \\
		\left(C, D\right) &\mapsto \mathrm{Hom}_{\mathsf{C}} \left( C, D \right)
	.\end{align} 
	On pairs of morphisms $C' \xrightarrow{f} C$ and $D \xrightarrow{g} D'$, it acts as
	\begin{equation}
	\begin{tikzcd}
		\left(C, D\right) \arrow[r, "", rightarrow] \arrow[d, "{(f,g)}", rightarrow] &
		\mathrm{Hom}_{\mathsf{C}} \left( C, D \right) \arrow[d, "{F(f,g)}", rightarrow] &
		\alpha \arrow[d, "", mapsto]\\
		\left(C', D'\right) \arrow[r, "", rightarrow] &
		\mathrm{Hom}_{\mathsf{C}} \left( C', D' \right) &
		g \circ \alpha \circ f
	\end{tikzcd}
	.\end{equation} 
	Clearly $F$ is covariant in both variables.
\end{ex} 	

\begin{defn}[Morphism of functors]
	Given two functors $F,G\colon \mathsf{C} \to \mathsf{D}$, a {\em morphism of functors} (sometimes called {\em natural transformation}) $\theta\colon F \to G$ (sometimes denoted with $F \xRightarrow{\theta} G$) 
	is the data, for any $X \in \mathsf{C}$, of a map $\theta(X)\colon FX \to GX$ s.t. $\,\forall\, f\colon X \to X'$ in $\mathsf{C}$ the following diagram commutes
	\begin{equation}
	\begin{tikzcd}
		FX \arrow[r, "{\theta(X)}", rightarrow] \arrow[d, "{F(f)}"', rightarrow] & GX \arrow[d, "{G(f)}", rightarrow] \\
		FX' \arrow[r, "{\theta(X')}"', rightarrow] & GX'
	\end{tikzcd}
	,\end{equation} 
	i.e. $G(f) \circ \theta(X) = \theta(X') \circ F(f)$.

	Some authors denote one such transformation with the following diagram
	\begin{equation}
	\begin{tikzcd}[row sep=tiny]
%		& \ \arrow[dd, "\theta"', Rightarrow] & \\
%		\mathsf{C} \arrow[rr, "", rightarrow, bend right, bend right] \arrow[rr, "", rightarrow, bend left, bend left] & & \mathsf{D}\\
%															       & \ & 
		C \arrow[r, ""{name=U, below}, rightarrow, bend left=35] 
		\arrow[r, ""{name=D}, rightarrow, bend right=35] &
		B
		\arrow[Rightarrow, from=U, to=D] 
	\end{tikzcd}
	\end{equation} 
\end{defn}

\begin{defn}[Natural isomorphic functors]
	Let $\mathsf{C}$ and $\mathsf{D}$ be two categories, and $G,F\colon \mathsf{C} \to \mathsf{D}$ be two functors.
	We say that $F$ is {\em naturally isomorphic} to $G$ iff one of the following (equivalent) conditions is satisfied:
	\begin{itemize}
		\item there exist two natural transformations $\eta\colon F \to G$ and $\theta\colon G \to F$ s.t.
			\begin{equation}
			id_G = \eta \circ \theta \quad \text{ and } \quad \theta \circ \eta = id_F
			,\end{equation} 
		\item there exists a natural transformation $\eta\colon F \to G$ s.t. $\eta_X\colon FX \to GX$ is an isomorphism in $\mathsf{D}$ for every $C \in \mathrm{Ob} \left(\mathsf{C}\right)$.
	\end{itemize}
\end{defn}

\begin{defn}[Category of functors]
	We denote by $\mathsf{D}^{\mathsf{C}} \coloneqq \mathsf{Fct}\left(\mathsf{C}, \mathsf{D} \right)$ the {\em category of functors} from $\mathsf{C}$ to $\mathsf{D}$,
	whose elements are functors $F\colon \mathsf{C} \to \mathsf{D}$ and whose morphisms are the above mentioned morphisms of functors.
\end{defn}

\begin{rem}
	In general the category of functors is a {\em large category}, in the sense that its objects might not be sets.
	Though, if we start from a {\em small} category, i.e. if $\mathrm{Ob} \left(\mathsf{C}\right)$ is a set,
	then $\mathsf{Fct}\left(\mathsf{C}, \mathsf{D} \right) $ is a small category.

	In such case, fixed $F, G$ functors from $\mathsf{C}$ to $\mathsf{D}$, then a natural transformation is
	\begin{equation}
	\eta = \left\{ \eta_X \right\}_{X \in \mathrm{Ob} \left(\mathsf{C}\right)} \in \prod_{X \in \mathrm{Ob} \left(\mathsf{C}\right)} \mathrm{Hom}_{\mathsf{D}} \left( FX, GX \right)
	.\end{equation} 
	It is important to notice that the infinite product of sets is still a set, hence
	\begin{equation}
	\mathrm{Nat}\, \left(F, G\right) \subset \prod_{X \in \mathrm{Ob} \left(\mathsf{C}\right)} \mathrm{Hom}_{\mathsf{C}} \left( FX, GX \right)
	.\end{equation} 
\end{rem}

\begin{ex}
	Fix $\mathsf{I} \coloneqq \left(I, \le\right)$ a poset (a small category) and a category $\mathsf{C}$.
	An element $F \in \mathsf{Fct}\left(\mathsf{I}, \mathsf{C} \right) = \mathsf{C}^{\mathsf{I}}$ is a functor
	\begin{equation}
	F\colon \mathsf{I} \to \mathsf{C}
	\end{equation} 
	that associates to each element $i \in I$ an object $F(i) \in \mathrm{Ob} \left(\mathsf{C}\right)$.
	Moreover, with regards to morphisms it acts as follows: given $i \leq j \le k$ we have $i \xrightarrow{\alpha} j \xrightarrow{\beta} k$ and $\beta \circ \alpha = \gamma \colon i \to k$ and the following commutative diagram
	\begin{equation}
	\begin{tikzcd}[column sep=tiny]
		F(i) \arrow[rr, "F(\gamma)", rightarrow] \arrow[rd, "F(\alpha)"', rightarrow] & & F(k)\\
			& F(j) \arrow[ru, "F(\beta)"', rightarrow] &
	\end{tikzcd}
	.\end{equation} 
	In particular, given $\mathsf{C} = \mathsf{Mod}\left( R \right)$, then $F \in \mathsf{Fct}\left(I, \mathsf{Mod}\left( R \right) \right)$ is a functor s.t., called $f_{ji} \coloneqq F( i \to j)$, then
	\begin{equation}
		f_{ki} = f_{kj} \circ f_{ji}
	.\end{equation} 
	This is called a {\em direct system of modules}.
\end{ex} 

\begin{defn}[Preadditive category]
	A category $\mathsf{C}$ is called {\em preadditive} iff it is a $\Z$ category, i.e. iff
	given any pair $X,Y \in \mathrm{Ob} \left(\mathsf{C}\right)$ the set $\mathrm{Hom}_{\mathsf{C}} \left( X, Y \right)$ is a $\Z$-module (an abelian group) and the composition of morphisms is a bilinear map.
\end{defn}

\begin{ex}
	$R\text{-}\mathsf{Mod}$, the category of left $R$-modules, and $\mathsf{Mod}\text{-}R$, the category of right $R$-modules, are all preadditive categories (even for $R$ division rings or fields).\newline
	$\mathsf{Rings}$ and $\mathsf{Groups}$ are not preadditive: the Hom sets do not have the structure of abelian group.
\end{ex} 

\begin{defn}[Additive functors]
	Given two preadditive categores $\mathsf{C}$ and $\mathsf{D}$, a functor $F\colon \mathsf{C} \to \mathsf{D}$ is called {\em additive} iff, for any $X,Y \in \mathrm{Ob} \left(\mathsf{C}\right)$, for any $f,g\colon X \to Y$, then
	\begin{equation}
		F(f+g) = F(f) + F(g)
	.\end{equation} 
\end{defn}

\begin{rem}
	For a samll preadditive category $\mathsf{C}$ and a preadditive category $\mathsf{D}$, then we denote with
	\begin{equation}
		\underline{\mathrm{Hom}_{} \left( \mathsf{C}, \mathsf{D} \right)}
	\end{equation} 
	the category of all additive functors from $\mathsf{C}$ to $\mathsf{D}$.
\end{rem}

\begin{ex}
	Given a ring $R$, we define the category $\underline{\mathsf{R}}$ with one object, $*$, characterized by
	\begin{equation}
		\mathrm{Hom}_{\underline{\mathsf{R}}} \left(* , * \right) \coloneqq R
	,\end{equation} 
	with the composition acting as the product in $R$.
	Clearly it is a preadditive category.
	Let's consider the category
	\begin{equation}
		\underline{\mathrm{Hom}_{} \left( \underline{\mathsf{R}}, \mathsf{Ab} \right)}
	.\end{equation} 
\end{ex}

\begin{defn}[Category of $\mathsf{C}$-modules]
	Given a small preadditive category $\mathsf{C}$, then the category
	\begin{equation}
		\underline{\mathrm{Hom}_{\mathsf{}} \left( \mathsf{C}, \mathsf{Ab} \right)}
	\end{equation} 
	of additive covariant (contravariant) functors, is called the category of {\em left (right) } $\mathsf{C}$-modules.
\end{defn}

\begin{defn}[Category isomorphism/equivalence]
	Given two categories $\mathsf{C}$ and $\mathsf{D}$ we say they are
	\begin{description}
		\item[isomorphic], notation $\mathsf{C} \cong \mathsf{D}$, iff there exist $F\colon \mathsf{C} \to \mathsf{D}$ and $G\colon \mathsf{D} \to \mathsf{C}$ s.t. $F \circ G = id_{\mathsf{D}}$ and $G \circ F = id_\mathsf{C}$,
		\item[equivalent], notation $\mathsf{C} \simeq \mathsf{D}$, iff there exist $F\colon \mathsf{C} \to \mathsf{D}$ and $G\colon \mathsf{D} \to \mathsf{C}$ s.t. $F \circ G \simeq id_{\mathsf{D}}$ and $G \circ F \simeq id_\mathsf{C}$.
			In this case we just asked for isomorphism of functors, which makes $F$ and $G$ {\em quasi-inverses}.
	\end{description} 

	Moreover an equivalence $F\colon \mathsf{C} \to \mathsf{D}^{op}$ is called a  {\em duality}.
\end{defn}

\begin{rem}
	Fixed a ring $R$, then 
	\begin{equation}
		\underline{\mathrm{Hom}_{\mathsf{}} \left( \underline{\mathsf{R}}, \mathsf{Ab} \right)} \cong R \text{-}\mathsf{Mod} 
		\quad \text{ and } \quad
		\underline{\mathrm{Hom}_{\mathsf{}} \left( \underline{\mathsf{R}}^{op}, \mathsf{Ab} \right)} \cong \mathsf{Mod}\text{-}R
	.\end{equation} 
\end{rem}

\begin{ex}[duality]
	Let $K$ be a division ring and $K$-$\mathsf{Vect}$ the category of finite dimensionale left $K$-Vector Spaces, then
	\begin{align}
		D\colon K \text{-}\mathsf{Vect} &\to \mathsf{Vect} \text{-}K \\
		V &\mapsto V^*
	\end{align} 
	is a duality.
\end{ex} 

\begin{prop}
	A functor $F\colon \mathsf{C} \to \mathsf{D}$ is an equivalence of categories iff it is {\em fully faithful} and {\em essentially surjective}.
\end{prop} 

\subsection{Yoneda lemma}

\begin{defn}[]
	Let $\mathsf{C}$ be a category, one defines the following:
	\begin{equation}
	\mathsf{C}^\wedge \coloneqq \mathsf{Fct}\left(\mathsf{C}^{op}, \mathsf{Sets} \right), \quad \mathsf{C}^\vee \coloneqq \mathsf{Fct}\left(\mathsf{C}^{op}, \mathsf{Sets}^{op} \right)
	,\end{equation} 
	and the functors
	\begin{align}
		h_\mathsf{C}\colon \mathsf{C} \to \mathsf{C}^\wedge &\text{ s.t. }
		X \mapsto \mathrm{Hom}_{\mathsf{\mathsf{C}}} \left( -, X \right)\\
		k_\mathsf{C}\colon \mathsf{C} \to \mathsf{C}^\vee &\text{ s.t. }
		X \mapsto \mathrm{Hom}_{\mathsf{\mathsf{C}}} \left( X, - \right)
	.\end{align}
\end{defn}

\begin{lem}[Yoneda]
	The functor $h_\mathsf{C}$ is fully faithful.
\end{lem} 

\begin{defn}[Representable functor]\leavevmode\vspace{-\baselineskip}
	\begin{enumerate}
		\item A functor $F\colon \mathsf{C}^{op} \to \mathsf{Sets}$ is {\em representable} iff there exists $X \in \mathsf{C}$ s.t. $F(Y) \simeq \mathrm{Hom}_{\mathsf{C}} \left( Y, X \right)$ functorially in $Y \in \mathsf{C}$.
			In other words we have $F \simeq h_\mathsf{C}(X)$ in $\mathsf{C}^{\wedge}$. 
			Such object $X$ is called a representative of $F$.
		\item A functor $G\colon \mathsf{C} \to \mathsf{Sets}$ is {\em corepresentable} iff there exists a representative $X \in \mathsf{C}$ s.t. $G(Y) \simeq \mathrm{Hom}_{\mathsf{C}} \left( X, Y \right)$ functorially in $Y \in \mathsf{C}$.
	\end{enumerate} 
\end{defn}

\begin{prop}
	Let $F\colon \mathsf{C}^{op} \to \mathsf{Sets}$ be a {\em representable} functor, i.e. $\exists\, X \in \mathrm{Ob} \left(\mathsf{C}\right)$ s.t.
	\begin{equation}
	F \simeq \mathrm{Hom}_{\mathsf{C}} \left( -, X \right)
	.\end{equation} 
	Then $X$ is unique up to isomorphism.
\end{prop} 

\begin{defn}[Adjoint functors]
	Let $F\colon \mathsf{C} \to \mathsf{D}$ and $G\colon \mathsf{D} \to \mathsf{C}$ be two functors.
	One says that $\left(F, G\right)$ is an {\em adjoint} pair, or equivalently that $F$ is a {\em left adjoint} to $G$ or that $G$ is a {\em right adjoint} to $F$,
	iff there exists an isomorphism of bifunctors:
	\begin{equation}
		\mathrm{Hom}_{\mathsf{D}} \left( F(-), - \right) \simeq
		\mathrm{Hom}_{\mathsf{C}} \left( -, G(-) \right)
	.\end{equation} 
\end{defn}

\begin{rem}
	Note that, given two categories $\mathsf{C}$ and $\mathsf{D}$ and a pair $\left(F, G\right)$ of {\em adjoint} functors, one has the following morphism of functors:
	\begin{equation}
	F \circ G \to id_\mathsf{D}, \quad G \circ F \to id_\mathsf{C}
	.\end{equation} 
\end{rem}

\section{Limits}
\subsection{Kernel and Cokernel}

\begin{defn}[(Co)kernel]
	Let $\mathsf{C}$ be a preadditive category, with a zero object.
	Let $A \xrightarrow{f} B$ a morphism in $\mathsf{C}$.
	\begin{itemize}
		\item A \textbf{kernel} of $f$ is a pair $\left(K, \epsilon\right)$, with $K \xrightarrow{\epsilon} A$ satisfying
	\begin{description}
		\item[K1] $f \circ \epsilon = 0$,
		\item[K2] for any $\epsilon': K' \to A$ s.t. $f \circ \epsilon' = 0$, then
			$\exists\, ! K' \xrightarrow{\alpha} K$ s.t. $\epsilon \circ \alpha = \epsilon'$, i.e. s.t. the following diagram commutes
			\begin{equation}
			\begin{tikzcd}
				K \arrow[r, "\epsilon", rightarrow] & A \arrow[r, "f", rightarrow] & B\\
				    & K' \arrow[lu, "\exists\, ! \alpha", dashrightarrow] \arrow[u, "\epsilon'"', rightarrow] \arrow[ru, "0"', rightarrow] & 
			\end{tikzcd}
			.\end{equation} 
	\end{description} 
	\item A \textbf{cokernel} of $f$ is a kernel of $B \xrightarrow{f} A$ in $\mathsf{C}^{op}$.
		In other words it is a pair $\left(C, p\right)$, with $B \xrightarrow{p} C$ s.t.
	\begin{description}
		\item[CK1] $p \circ f = 0$,
		\item[CK2] for any $p': B \to C'$ s.t. $p' \circ f = 0$, then
			$\exists\, ! C \xrightarrow{\gamma} C'$ s.t. $\gamma \circ p = p'$, i.e. s.t. the following diagram commutes
			\begin{equation}
			\begin{tikzcd}
				A \arrow[r, "f", rightarrow] \arrow[rd, "0"', rightarrow]  & B \arrow[r, "p", rightarrow] \arrow[d, "p'", rightarrow] & C \arrow[ld, "\exists\, ! \gamma", dashrightarrow] \\
				    & C'& 
			\end{tikzcd}
			.\end{equation} 
	\end{description} 
	\end{itemize}
	We denote with the uppercase Ker the object $K$, and with the lowercase ker the morphism $\epsilon: K \to A$.\newline
	Analogously for the cokernel, we denote with the uppercase Coker the object $C$, and with the lower case coker the morphism $p: B \to C$.
\end{defn}

\begin{rem}
	Property \textbf{K2} grants that Ker satisfies a universal property (U.P.).
	Objects that satisfy universal properties are unique up to a unique isomorphism. 
\end{rem}

\begin{defn}[(Co)equalizer]
	Let $f,g$ be two parallel morphisms $A \rightrightarrows B$ in a category $\mathsf{C}$.
	\begin{itemize}
		\item An \textbf{equalizer} of $f$ and $g$ is a pair $\left(C, e\right)$, with $C \xrightarrow{e} A$, satisfying
	\begin{description}
		\item[eq1] $f \circ e = g \circ e$,
		\item[eq2] for $\left(C', e'\right)$ with $C' \xrightarrow{e'} A$ s.t. $f \circ e' = g \circ e'$, then
			$\exists\, ! \alpha: C' \to C$ s.t. $e \circ \alpha = e'$, i.e. the following diagram commutes
			\begin{equation}
			\begin{tikzcd}
				C \arrow[r, "e", rightarrow] & A \arrow[r, "f", rightarrow, shift left=.5ex] \arrow[r, "g"', rightarrow, shift right=.5ex] & B\\
				    & C' \arrow[lu, "\exists\, ! \alpha", dashrightarrow] \arrow[u, "e'"', rightarrow] & 
			\end{tikzcd}
			.\end{equation} 
	\end{description}
	\item A \textbf{coequalizer} of $f$ and $g$ is an equalizer of $f$ and $g$ in $\mathsf{C}^{op}$.
		In other words it is a pair $\left(C, p\right)$, with $B \xrightarrow{p} C$ s.t.
		\begin{description}
			\item[coeq1] $p \circ f = p \circ g$,
			\item[coeq2] for $\left(C', p'\right)$ with $B \xrightarrow{p'} C'$ s.t. $p' \circ f = p' \circ g$, then $\exists\, ! \gamma: C \to C'$, with $\gamma \circ p = p'$, i.e. s.t. the following diagram commutes
			\begin{equation}
			\begin{tikzcd}
				A \arrow[r, "f", rightarrow, shift left=.5ex] \arrow[r, "g"', rightarrow, shift right=.5ex] & B \arrow[r, "P", rightarrow] \arrow[d, "p'", rightarrow] & C \arrow[dl, "\exists\, ! \gamma", dashrightarrow] \\
				    & C' & 
			\end{tikzcd}
			.\end{equation} 
		\end{description} 
	\end{itemize}
\end{defn}

\begin{rem}\leavevmode\vspace{-.2\baselineskip}
	\begin{itemize}
		\item The kernel of $A \xrightarrow{f} B$ is just the equalizer of $f$ and $0$, if it exists.
		\item The cokernel of $A \xrightarrow{f} B$ is just the coequalizer of $f$ and $0$, if it exists.
	\end{itemize}
\end{rem}

\begin{lem}
	Let $\mathsf{C}$ be a preadditive category with $0$ object.
	Let $f: A \to B$ in $\mathsf{C}$.
	\begin{itemize}
		\item $f$ is a mono (epi) iff $f \circ h = 0 \implies h = 0$ ($h \circ f = 0 \implies h = 0$),
		\item $f$ is a mono (epi) iff $0 \to A$ is a kernel of $f$ ($B \to 0$ is a cokernel of $f$),
		\item A kernel (cokernel) is mono (epi).
	\end{itemize}
\end{lem} 

\begin{defn}[Ker functor]
	Let $\mathsf{C}$ be a preadditive category admitting zero object.
	Consider $A \xrightarrow{f} B$ a morphism in $\mathsf{C}$.
	This induces a natural transformation 
	$f_*\colon h^A \to h^B$, given by the collection of maps
	\begin{align}
		f_*(X): h^A(X) = \mathrm{Hom}_{\mathsf{C}} \left( X, A \right) &\to \mathrm{Hom}_{\mathsf{C}} \left( X, B \right) = h^B(X) \\
		\alpha &\mapsto f \circ \alpha
	\end{align} 
	for $X \in \mathrm{Ob} \left(\mathsf{C}\right)$.
	For any $X \in \mathrm{Ob} \left(\mathsf{C}\right)$,
	$f_*(X)$ is a morphism of abelian groups, hence it admits a kernel.
	\begin{equation}
		\ker f_*(X) = \left\{ X \xrightarrow{\alpha} A  \ \middle|\ f \circ \alpha = 0\right\} \leq \mathrm{Hom}_{\mathsf{C}} \left( X, A \right)
	.\end{equation} 
	We can define the \textit{contravariant} functor
	\begin{equation}
	F := \ker \left[ f_*: \mathrm{Hom}_{\mathsf{C}} \left( -, A \right) \to \mathrm{Hom}_{\mathsf{C}} \left( -, B \right) \right]
	\end{equation} 
	That acts on a morphism $X \xrightarrow{h} Y$ as
	 \begin{align}
		 F(h): F(Y) &\to F(X) \\
		\beta &\mapsto \beta \circ f
	.\end{align} 
\end{defn}

\begin{prop}
	A morphism $A \xrightarrow{f} B$ in a preadditive category admitting zero object has a kernel iff the associated functor $F$ is representable.
	In this case a kernel of $f$ is given by $\left(K, \epsilon\right)$, as follows:
	Let $F \simeq_\eta \mathrm{Hom}_{\mathsf{C}} \left( -, K \right)$, for $K \in \mathrm{Ob} \left(\mathsf{C}\right)$ a representative of $F$.
	Then $\epsilon$ is given by
	\begin{align}
		\mathrm{Hom}_{\mathsf{C}} \left( K, K \right) &\xrightarrow{\eta_K} F(K) 
		\subset \mathrm{Hom}_{\mathsf{C}} \left( K, A \right)\\
		1_K &\mapsto \epsilon
	.\end{align} 
\end{prop} 

\begin{defn}[Coker functor]
	Let $\mathsf{C}$ be a preadditive category admitting zero object.
	Consider $A \xrightarrow{f} B$ a morphism in $\mathsf{C}$.
	This induces a natural transformation
	$f^*\colon h_B \to h_A$, given by the collection of maps
	\begin{align}
		f^*(X): h_B(X) = \mathrm{Hom}_{\mathsf{C}} \left( B, X \right) &\to
		\mathrm{Hom}_{\mathsf{C}} \left( A, X \right) = h_A(X) \\
		\beta &\mapsto \beta \circ f
	\end{align} 
	for $X \in \mathrm{Ob} \left(\mathsf{C}\right)$.
	For any $X \in \mathrm{Ob} \left(\mathsf{C}\right)$, $f^*(X)$ is a morphism of abelian groups, hence it admits a kernel in $\mathsf{Ab}$:
	\begin{equation}
		\ker f^*(X) = \left\{ B \xrightarrow{\beta} X \ \middle|\ \beta \circ f = 0 \right\}
	.\end{equation} 
	We can define a \textit{covariant} functor
	\begin{equation}
	F := \ker \left[ f^*: \mathrm{Hom}_{\mathsf{C}} \left( B, - \right) \to \mathrm{Hom}_{\mathsf{C}} \left( A, - \right) \right]
	\end{equation} 
	that acts on a morphism $X \xrightarrow{h} Y$ as
	\begin{align}
		F(h): F(X) &\to F(Y) \\
		\beta &\mapsto h \circ \beta
	.\end{align} 
\end{defn}

\begin{prop}
	Let $\mathsf{C}$ be a preadditive category admitting zero object.
	The morphism $A \xrightarrow{f} B$ has a cokernel iff $F$ is corepresentable.
	In other words, iff there exists $C \in \mathrm{Ob} \left(\mathsf{C}\right)$ and a natural isomorphism
	\begin{equation}
	F \simeq_\eta \mathrm{Hom}_{\mathsf{C}} \left( C, - \right)
	.\end{equation} 
	In this case a cokernel is given by $\left(C, p\right)$, with $C \in \mathrm{Ob} \left(\mathsf{C}\right)$ a representative of $F$ and $p$ given by
	\begin{align}
		\mathrm{Hom}_{\mathsf{C}} \left( C, C \right) &\to F(C) 
		\subset \mathrm{Hom}_{\mathsf{C}} \left( B, C \right)\\
		1_C &\mapsto p
	.\end{align} 
\end{prop} 

\begin{lem}
	Let $\mathsf{C}$ be a preadditive category with $0$ object.
	Let $A \xrightarrow{f} B$ be a kernel of some other morphism.
	Then, if $\mathrm{coker}\, f$ exists, we have
	\begin{equation}
		f = \ker \left( \mathrm{coker}\, f \right)
	.\end{equation} 
\end{lem} 

\begin{lem}
	Let $\mathsf{C}$ be a preadditive category with $0$ object.
	Let $A \xrightarrow{f} B$ be a cokernel of some morphism.
	Lat $f$ admit a kernel, then
	\begin{equation}
		f = \mathrm{coker}\, \left( \ker f \right)
	.\end{equation} 
\end{lem} 

\subsection{Product and Coproduct}

\begin{defn}[Product]
	Let $A, B \in \mathrm{Ob} \left(\mathsf{C}\right)$ for an arbitrary category $\mathsf{C}$.
	A \textbf{product} of $A$ and $B$, if it exists, is a triple $\left(A \prod B, \pi_A, \pi_B \right)$, where $A \prod B \in \mathrm{Ob} \left(\mathsf{C}\right)$, and the morphisms $\pi_A$ and $\pi_B$ in $\mathsf{C}$, called \textbf{projections}, 
	 \begin{equation}
	A \prod B \xrightarrow{\pi_A} A \quad \text{ and } \quad A \prod B \xrightarrow{\pi_B} B
	\end{equation} 
	satisfy the universal property:
	Given an arbitrary $\left(X, \alpha, \beta\right)$, with $X \in \mathrm{Ob} \left(\mathsf{C}\right)$, $X \xrightarrow{\alpha} A$ and $X \xrightarrow{\beta} B$ a pair of morphism, there exists a unique morphism $X \xrightarrow{\exists\, ! h} A \prod B$ s.t.
	\begin{equation}
	\begin{tikzcd}
		& X \arrow[ld, "\alpha"', rightarrow] \arrow[rd, "\beta", rightarrow] \arrow[dd, "h", "\exists\, !"', dashrightarrow] \\
		A & & B\\
		  & A \prod B \arrow[lu, "\pi_A", rightarrow] \arrow[ru, "\pi_B"', rightarrow] &
	\end{tikzcd}
	\end{equation} 
	the above diagram commutes.
	In other words, s.t. $\alpha = \pi_A \circ h$ and $\beta = \pi_B \circ h$.
\end{defn}

\begin{rem}
	If it exists, a product, is unique up to a unique isomorphism.
	This, as usual, is due to the universal property used to define the product.
\end{rem}

\begin{prop}
	Define the functor 
	\begin{equation}
	F := \mathrm{Hom}_{\mathsf{C}} \left( -, A \right) \cross \mathrm{Hom}_{\mathsf{C}} \left( -, B \right): \mathsf{C} \to \mathsf{Sets}
	\end{equation} 
	on objects as $F(X) := \mathrm{Hom}_{\mathsf{C}} \left( X, A \right) \cross \mathrm{Hom}_{\mathsf{C}} \left( X, B \right)$, and on morphisms $X \xrightarrow{f} Y$, for a couple of arrows $Y \xrightarrow{\alpha} A$ and $Y \xrightarrow{\beta} B$, as
	\begin{align}
		F(f): \mathrm{Hom}_{\mathsf{C}} \left( Y, A \right) \cross \mathrm{Hom}_{\mathsf{C}} \left( Y, B \right) &\to \mathrm{Hom}_{\mathsf{C}} \left( X, A \right) \cross \mathrm{Hom}_{\mathsf{C}} \left( X, B \right) \\
		\left(\alpha, \beta\right) &\mapsto \left(\alpha \circ f, \beta \circ f \right)
	.\end{align} 
	A product $\left(A \prod B, \pi_A, \pi_B \right)$ exists iff the functor $F$ is representable.
	In other words iff $F \simeq_\eta \mathrm{Hom}_{\mathsf{C}} \left( -, P \right)$
	for some $P \in \mathrm{Ob} \left(\mathsf{C}\right)$.
	In this case $\left(P, \pi_A, \pi_B\right)$ is a product of $A$ and $B$,
	where $\left(\pi_A, \pi_B\right)$ are given by
	\begin{align}
		\eta_P: \mathrm{Hom}_{\mathsf{C}} \left( P, P \right) &\to F(P) =
		\mathrm{Hom}_{\mathsf{C}} \left( P, A \right) \cross 
		\mathrm{Hom}_{\mathsf{C}} \left( P, B \right)  \\
		1_P &\mapsto \left(\pi_A, \pi_B\right)
	.\end{align} 
\end{prop} 

\begin{ex}\leavevmode\vspace{-.2\baselineskip}
	\begin{itemize}
		\item $\mathsf{C} = \mathsf{Sets}$, then $A \prod B = A \cross B$ is the cartesian product of sets, with $\pi_A$ and $\pi_B$ the projections.
		\item $\mathsf{C} = \mathsf{Mod}\text{-}R$, then $A \prod B = A \cross B$ is the set theoretic cartesian product, with componentwise operations. The projections are the set-theoretic projections.
		\item $\mathsf{C} = \mathsf{Rings}$, as above, $A \prod B = A \cross B$ is the set theoretic cartesian product, with componentwise operations. The projections are the set-theoretic projections.
	\end{itemize}
\end{ex} 

\begin{defn}[Coproduct]
	Let $A,B \in \mathrm{Ob} \left(\mathsf{C}\right)$ for an arbitrary category $\mathsf{C}$.
	A \textbf{coproduct} of $A$ and $B$, if it exists,
	is a triple $\left(A \coprod B, \epsilon_A, \epsilon_B \right)$,
	where $A \coprod B \in \mathrm{Ob} \left(\mathsf{C}\right)$
	and the morphisms $\epsilon_A$ and $\epsilon_B$, called \textbf{embeddings}, 
	\begin{equation}
	A \xrightarrow{\epsilon_A} A \coprod B \quad \text{ and } \quad B \xrightarrow{\epsilon_B} A \coprod B
	\end{equation} 
	satisfy the universal property:
	Given an arbitrary $\left(X, \alpha, \beta\right)$, with $X \in \mathrm{Ob} \left(\mathsf{C}\right)$, $A \xrightarrow{\alpha} X$ and $B \xrightarrow{\beta} X$ a pair of morphism, there exists a unique morphism $A \coprod B \xrightarrow{\exists\, ! h} X$ s.t.
	\begin{equation}
	\begin{tikzcd}
		& A \coprod B \arrow[ld, "\epsilon_A"', leftarrow] \arrow[rd, "\epsilon_B", leftarrow] \arrow[dd, "h", "\exists\, !"', dashrightarrow] \\
		A & & B\\
		  & X \arrow[lu, "\alpha", leftarrow] \arrow[ru, "\beta"', leftarrow] &
	\end{tikzcd}
	\end{equation} 
	the above diagram commutes.
	In other words, s.t. $h \circ \epsilon_A = \alpha$ and $h \circ \epsilon_B = \beta$.
\end{defn}

\begin{rem}
	A coproduct is a product in $\mathsf{C}^{op}$.
	Moreover, if it exists, then it is unique up to a unique isomorphism.
\end{rem}

\begin{prop}
	Define the functor 
	\begin{equation}
	F := \mathrm{Hom}_{\mathsf{C}} \left( A, - \right) \cross \mathrm{Hom}_{\mathsf{C}} \left( B, - \right): \mathsf{C} \to \mathsf{Sets}
	\end{equation} 
	on objects as $F(X) := \mathrm{Hom}_{\mathsf{C}} \left( A, X \right) \cross \mathrm{Hom}_{\mathsf{C}} \left( B, X \right)$, and on morphisms $X \xrightarrow{f} Y$, for a couple of arrows $Y \xrightarrow{\alpha} A$ and $Y \xrightarrow{\beta} B$, as
	\begin{align}
		F(f): \mathrm{Hom}_{\mathsf{C}} \left( A, X \right) \cross \mathrm{Hom}_{\mathsf{C}} \left( A, X \right) &\to \mathrm{Hom}_{\mathsf{C}} \left( A, Y \right) \cross \mathrm{Hom}_{\mathsf{C}} \left( A, Y \right) \\
		\left(\alpha, \beta\right) &\mapsto \left(f \circ \alpha, f \circ \beta \right)
	.\end{align} 
	A coproduct $\left(A \coprod B, \epsilon_A, \epsilon_B \right)$ exists iff the functor $F$ is corepresentable.
	In other words iff $F \simeq_\eta \mathrm{Hom}_{\mathsf{C}} \left( C, - \right)$
	for some $C \in \mathrm{Ob} \left(\mathsf{C}\right)$.
	In this case $\left(C, \epsilon_A, \epsilon_B\right)$ is a coproduct of $A$ and $B$,
	where $\left(\epsilon_A, \epsilon_B\right)$ are given by
	\begin{align}
		\eta_C: \mathrm{Hom}_{\mathsf{C}} \left( C, C \right) &\to F(C) = \mathrm{Hom}_{\mathsf{C}} \left( A, C \right) \cross \mathrm{Hom}_{\mathsf{C}} \left( B, C \right)  \\
		1_C &\mapsto \left(\epsilon_A, \epsilon_B\right)
	.\end{align} 
\end{prop} 

\begin{ex}\leavevmode\vspace{-.2\baselineskip}
	\begin{itemize}
		\item Let $\mathsf{C} = \mathsf{Sets}$, then $A \coprod B = A \sqcup B$, the disjoint union, with embeddings given by the inclusions.
		\item Let $\mathsf{C} = R\text{-}\mathsf{Mod}$, then ${}_RM \coprod {}_RN = \left(M \cross N, \epsilon_M, \epsilon_N\right)$, set-theoretically is the cartesian product, with componentwise operations and inclusions.
		\item Let $\mathsf{C} = \mathsf{CRings}$ the category of commutative rings.
			Then $R \coprod S = \left(R \otimes_{\Z} S, \epsilon_R, \epsilon_S\right)$ the coproduct of two commutative rings is given by their tensor product over $\Z$.
	\end{itemize}
\end{ex} 

\begin{defn}[Additive category]
	Let $\mathsf{C}$ be a preadditive category with $0$ object.
	$\mathsf{C}$ is said \textbf{additive} iff, given any pair (or finite family) of objects in $\mathsf{C}$, their product exists in $\mathsf{C}$.
\end{defn}

\begin{prop}
	Let $\mathsf{C}$ be a preadditive category with $0$ object.
	If product exist in $\mathsf{C}$, then coproduct exist and they are isomorphic.
	In particular we have the following for embeddings and projections:
	\begin{equation}
	\epsilon_A = 
	\begin{bmatrix}
		1_A \\ 0
	\end{bmatrix}, \quad
	\pi_A = 
	\begin{bmatrix}
		1_A & 0
	\end{bmatrix}, \quad
	\epsilon_B = 
	\begin{bmatrix}
		0 \\ 1_B
	\end{bmatrix}, \quad
	\pi_B = 
	\begin{bmatrix}
		0 & 1_B
	\end{bmatrix}
	.\end{equation} 
	This implies that these morphisms compose as
	\begin{equation}
	\pi_A \circ \epsilon_A = 1_A, \quad
	\pi_A \circ \epsilon_B = 0, \quad
	\pi_B \circ \epsilon_A = 0, \quad
	\pi_B \circ \epsilon_B = 1_B
	.\end{equation} 
\end{prop} 

\begin{defn}[(Co)product in preadditive categories]
	If $\mathsf{C}$ is a preadditive category, and the (co)product between $A, B \in \mathrm{Ob} \left(\mathsf{C}\right)$ exists in $\mathsf{C}$, they are denoted with 
	\begin{equation}
	A \oplus B
	.\end{equation} 
\end{defn}

\begin{prop}
	Let $\mathsf{C}$ be an \textbf{additive} category with $0$.
	Let $A, B \in \mathrm{Ob} \left(\mathsf{C}\right)$.
	The structure of abelian group of $\mathrm{Hom}_{\mathsf{C}} \left( A, B \right)$ is determined by $\mathsf{C}$.
\end{prop} 

\subsection{Infinite product and coproduct}

\begin{defn}[(Co)product of an arbitrary family of objects]
	Let $\left\{ A_i \right\}_{i \in I} \subset \mathrm{Ob} \left(\mathsf{C}\right)$ an arbitrary family of objects in the category $\mathsf{C}$.
	 \begin{itemize}
		 \item A \textbf{product} of the $A_i$s is the couple $\left(\prod_i A_i, (\pi_i)_{i \in I} \right)$, with $\prod_i A_i \in \mathrm{Ob} \left(\mathsf{C}\right)$, and morphisms $\pi_i: \prod_j A_j \to A_i$ for any $i \in I$, satisfying the universal property:
			 Given $X \in \mathrm{Ob} \left(\mathsf{C}\right)$ and a family of morphisms $X \xrightarrow{\alpha_i} A_i$, then $\exists\, !\, \alpha: X \to \prod_i A_i$ s.t. $\pi_i \circ \alpha = \alpha_i$ for all $i$.
		 \item A \textbf{coproduct} of the $A_i$s is the couple $\left(\coprod_i A_i, (\epsilon_i)_{i \in I} \right)$, with $\coprod_i A_i \in \mathrm{Ob} \left(\mathsf{C}\right)$, and morphisms $\epsilon_i: A_i \to \coprod_j A_j$ for any $i \in I$, satisfying the universal property:
			 Given $X \in \mathrm{Ob} \left(\mathsf{C}\right)$ and a family of morphisms $A_i \xrightarrow{\alpha_i} X$, then $\exists\, !\, \alpha: \coprod_i A_i \to X$ s.t. $\alpha \circ \epsilon_i = \alpha_i$ for all $i$.
			 In other words it is a product in $\mathsf{C}^{op}$.
	\end{itemize}
\end{defn}
 
\begin{ex}\leavevmode\vspace{-.2\baselineskip}
	\begin{itemize}
		\item Let $\mathsf{C} = \mathsf{Sets}$ and $\left\{ A_i \right\}_{i \in I} \subset \mathrm{Ob} \left(\mathsf{C}\right)$.
			The set $\prod_{i \in I} A_i$ (the infinite cartesian product), with usual projections, is a product in $\mathsf{Sets}$,
		\item Analogously, $\sqcup_{i \in I} A_i$ (the disjoint union), with the usual embeddings, is a coproduct in $\mathsf{Sets}$.
		\item Let $\mathsf{C} = \mathsf{Mod}\text{-}R$ and $\left\{ M_i \right\}_{i \in I} \subset \mathrm{Ob} \left(\mathsf{C}\right)$.
			The set 
			\begin{equation}
			\prod_{i \in I} M_i := \left\{ \left( x_i \right)_{i \in I} \ \middle|\ x_i \in M_i \,\forall\, i \in I \right\}	
			.\end{equation}
			(the infinite cartesian product), with componentwise operations and usual projections, is a product in $\mathsf{Mod}\text{-}R$.
			Clearly, given a family of morphisms $\alpha_i: X \to M_i$, one defines
			\begin{align}
				\alpha: X &\to \prod_{i \in I} M_i \\
				x &\mapsto \left( a_i(x) \right)_{i \in I}
			\end{align} 
			and easily checks the universal properties of products.
		\item Analogously the copruduct exists and is defined as follows
			\begin{equation}
				\coprod_{i \in I} M_i = \left\{ \left( x_i \right)_{i \in I} \ \middle|\ x_i \in M_i \,\forall\, i \in I \text{ and } x_i = 0  \text{ for almost all } i  \right\} \leq \prod_{i \in I} M_i
			.\end{equation} 
			with the embeddings
			\begin{align}
				\epsilon_i: M_i &\to \coprod_{i \in I} M_i \\
				x &\mapsto \left( \ldots, 0, x, 0, \ldots \right)
			,\end{align} 
			with nonzero entry only for the $i$-th component, is a coproduct in $\mathsf{Mod}\text{-}R$.
			In fact, given a family of morphisms $\alpha_i: M_i \to X$, the unique morphism is defined as
			\begin{align}
				\exists\, !\, \alpha: \coprod_{i \in I}M_i &\to X \\
				\left( x_i \right)_{i\in I} &\mapsto \sum_{i \in I}^{} \alpha_i(x_i)
			.\end{align} 
			It is important to remark that the sum makes sense, since $x_i \neq 0$ only for finitely many $i \in I$, hence it is a finite sum.
	\end{itemize}
\end{ex} 

\begin{prop}
	Let $\mathsf{C}$ be an arbitrary category.
	let $\left\{ A_i \right\}_{i \in I} \subset \mathrm{Ob} \left(\mathsf{C}\right)$ be an arbitrary family of objects.
	Assume that a product $\left(\prod_{i \in I}A_i, \pi_i\right)$ exists in $\mathsf{C}$, then
	given $X \in \mathrm{Ob} \left(\mathsf{C}\right)$, the map
	\begin{align}
		\mathrm{Hom}_{\mathsf{C}} \bigg( X, \prod_{i \in I}A_i \bigg)  &\xrightarrow{\phi_X} \prod_{i \in I} \mathrm{Hom}_{\mathsf{C}} \left( X, A_i \right) \\
		f &\mapsto \left( \pi_i \circ f \right)_{i \in I}
	\end{align} 
	is an isomorphism in $\mathsf{Sets}$ (by U.P.).
	Moreover the family $\left\{ \phi_X \right\}_{X \in \mathrm{Ob} \left(\mathsf{C}\right)}$ gives a natural isomorphism between the functors
	\begin{equation}
	F := \mathrm{Hom}_{\mathsf{C}} \bigg( -, \prod_{i \in I} A_i \bigg) \quad \text{ and } \quad G:= \prod_{i \in I} \mathrm{Hom}_{\mathsf{C}} \left( -, A_i \right)
	,\end{equation} 
	where $G$, on morphisms acts as: $G(f) = \prod_{i \in I} \mathrm{Hom}_{\mathsf{C}} \left( f, A_i \right)$.
\end{prop} 

\begin{prop}
	Let $\mathsf{C}$ be an arbitrary category.
	let $\left\{ A_i \right\}_{i \in I} \subset \mathrm{Ob} \left(\mathsf{C}\right)$ be an arbitrary family of objects.
	Assume that a coproduct $\left(\coprod_{i \in I}A_i, \epsilon_i\right)$ exists in $\mathsf{C}$, then
	given $X \in \mathrm{Ob} \left(\mathsf{C}\right)$, the map
	\begin{align}
		\mathrm{Hom}_{\mathsf{C}} \bigg( \coprod_{i \in I}A_i, X \bigg)  &\xrightarrow{\psi_X} \prod_{i \in I} \mathrm{Hom}_{\mathsf{C}} \left( A_i, X \right) \\
		f &\mapsto \left( f \circ \epsilon_i \right)_{i \in I}
	\end{align} 
	is an isomorphism in $\mathsf{Sets}$ (by U.P.).
	Moreover the family $\left\{ \psi_X \right\}_{X \in \mathrm{Ob} \left(\mathsf{C}\right)}$ gives a natural isomorphism between the functors
	\begin{equation}
	F := \mathrm{Hom}_{\mathsf{C}} \bigg( \coprod_{i \in I} A_i, - \bigg) \quad \text{ and } \quad G:= \prod_{i \in I} \mathrm{Hom}_{\mathsf{C}} \left( A_i, - \right)
	,\end{equation} 
	where $G$, on morphisms acts as: $G(f) = \prod_{i \in I} \mathrm{Hom}_{\mathsf{C}} \left( A_i, f \right)$.
\end{prop} 

\begin{rem}
	Notice that, if $\mathsf{C}$ is preadditive with $0$,
	then $\phi_X$ and $\psi_X$ are both isomorphisms of abelian groups.
	In particular $\left\{ \phi_X \right\}_{X \in \mathrm{Ob} \left(\mathsf{C}\right)}$ 
	and $\left\{ \psi_X \right\}_{X \in \mathrm{Ob} \left(\mathsf{C}\right)}$ are both natural
	isomorphisms of functors with values in $\mathsf{Ab}$.
\end{rem}

\begin{prop}
	Let $\mathsf{C}$ be an arbitrary category.
	Let $\left\{ A_i \right\}_{i \in I} \subset \mathrm{Ob} \left(\mathsf{C}\right),
	\left\{ B_i \right\}_{i \in I}
	\subset \mathrm{Ob} \left(\mathsf{C}\right)$.
	Let $\left\{ \alpha_i \right\}_{i \in I}$ a family of morphisms s.t. for each $i$ $\alpha_i: A_i \to B_i$.
	Assume that both products $\left(\prod_{i \in I} A_i , \pi_i\right)$ and $\left(\prod_{i \in I} B_i, p_i\right)$ exist in $\mathsf{C}$.
	Then 
	\begin{equation}
	\exists\, !\, \alpha: \prod_{i \in I} A_i  \to \prod_{i \in I} B_i
	\end{equation} 
	s.t. $p_i \circ \alpha = \alpha_i \circ \pi_i$.
	Moreover if, for all $i$, the morphism $\alpha_i$ is a monomorphism,
	then also $\alpha$ is a monomorphism.
\end{prop} 

\begin{prop}
	Let $\mathsf{C}$ be an arbitrary category.
	Let $\left\{ A_i \right\}_{i \in I} \subset \mathrm{Ob} \left(\mathsf{C}\right),
	\left\{ B_i \right\}_{i \in I}
	\subset \mathrm{Ob} \left(\mathsf{C}\right)$.
	Let $\left\{ \alpha_i \right\}_{i \in I}$ a family of morphisms s.t. for each $i$ $\alpha_i: A_i \to B_i$.
	Assume that both coproducts $\left(\coprod_{i \in I} A_i , \epsilon_i\right)$ and
	$\left(\coprod_{i \in I} B_i, \delta_i\right)$ exist in $\mathsf{C}$.
	Then 
	\begin{equation}
	\exists\, !\, \alpha: \coprod_{i \in I} A_i  \to \coprod_{i \in I} B_i
	\end{equation} 
	s.t. $\alpha \circ \epsilon_i = \delta_i \circ \alpha_i$.
	Moreover if, for all $i$, the morphism $\alpha_i$ is am epimorphism,
	then also $\alpha$ is a epimorphism.
\end{prop} 

\begin{prop}
	Let $\mathsf{C}$ be an arbitrary category.
	Consider an arbitrary family $\left\{ A_i \right\}_{i \in I} \subset \mathrm{Ob} \left(\mathsf{C}\right)$ s.t.
	the product $\left(\prod_{i \in I} A_i, \pi_i\right)$ (resp. the coproduct $\left(\coprod_{i \in I} A_i, \epsilon_i \right)$) exists in $\mathsf{C}$.
	Assume, moreover, that $\mathrm{Hom}_{\mathsf{C}} \left( A_i, A_j \right) \neq \emptyset$ for $i \neq j \in I$.
	It follows that $\pi_i$ (resp. $\epsilon_i$) is an epimorphism (resp. monomorphism) for all $i \in I$.
\end{prop} 

\begin{cor}
	In particular, if $\mathsf{C}$ is preadditive with $0$ object, then every $\mathrm{Hom}_{\mathsf{C}} \left( X, Y \right) \neq \emptyset$.
	This means that $\pi_i$ and $\epsilon_i$ in the above proposition are always respectively epi and mono.
	In particular, given $A, B \in \mathrm{Ob} \left(\mathsf{C}\right)$, then 
	$\pi_A: A \prod B \to A$ and $\pi_B: A \prod B \to B$ are epi, whereas
	$\epsilon_A: A \to A \prod B$ and $\epsilon_B: B \to A \prod B$ are mono.
\end{cor} 


\section{Abelian categories}

\begin{lem}[Parallel morphism]
	Let $\mathsf{C}$ be a preadditive category with $0$ object.
	Assume that every morphism in $\mathsf{C}$ admits kernel and cokernel, then
	\begin{equation}
	\begin{tikzcd}
		\ker f \arrow[r, "\epsilon", rightarrow] & A \arrow[r, "f", rightarrow] \arrow[d, "p"', rightarrow] \arrow[rd, "\beta", dashrightarrow] &
		B \arrow[r, "\pi", rightarrow] & \mathrm{coker}\, f \\
					       & \mathrm{coker}\, \epsilon \arrow[r, "\widetilde{f}"', dashrightarrow] & \ker \pi \arrow[u, "\mu"', rightarrow] &
	\end{tikzcd}
	\end{equation} 
	$\exists\, !\, \widetilde{f}\colon \mathrm{coker}\, \epsilon \to \ker \pi$ s.t. $\widetilde{f} \circ p = \beta$.
	$\widetilde{f}$ is called {\em parallel morphism} of $f$.
\end{lem} 

\begin{ex}
	Let $\mathsf{C} = \mathsf{Mod}\text{-}R$ and $A \xrightarrow{f} B$.
	Then $\mathrm{coker}\, (\ker f) = A / \ker f$ ad $\ker \left( \mathrm{coker}\, f \right) \simeq \ima f$.
	By the first isomorphism theorem we have
	\begin{equation}
	A/ \ker f \simeq_{\widetilde{f}} \ima f
	.\end{equation} 
\end{ex} 

\begin{defn}[Some notation]
	We denote the above objects as
	\begin{align}
		\coim f \coloneqq \coker \left( \ker f \right)\\
		\Ima f \coloneqq \ker \left( \coker f \right)	
	.\end{align} 
\end{defn}

\begin{defn}[Abelian category]
	A category $\mathsf{C}$ is said {\em abelian} iff it is additive and
	\begin{enumerate}
		\item every morphism has both kernel and cokernel,
		\item the parallel morphism $\widetilde{f}$ of $f$ is an isomorphism for any $f$.
	\end{enumerate}
	The second condition is equivalent to the following
	\begin{enumerate}
		\item[2'.] Every morphism $f$ in $\mathsf{C}$ factors as $\nu \beta$ with $\beta$ a cokernel and $\nu$ a kernel.
	\end{enumerate}
\end{defn}

\begin{lem}
	Let $f = \nu \beta$ in $\mathsf{C}$ a preadditive category with $0$ object.
	\begin{enumerate}
		\item If $\nu$ is a mono, then $\ker f = \ker \beta$, if they exist.
		\item If $\beta$ is epi, then $\coker f = \coker \nu$, if they exist.
	\end{enumerate}
\end{lem} 

\begin{lem}
	Assume that $\mathsf{C}$ is an abelian category.
	Let $A \xrightarrow{f} B$ be a morphism in $\mathsf{C}$, then
	\begin{enumerate}
		\item If $f$ is mono and epi, then $f$ is iso.
		\item If $f$ is mono, then $f = \ker \left( \coker f \right)$.
		\item If $f$ is epi, then $f = \coker \left( \ker f \right)$.
	\end{enumerate}
\end{lem} 

\begin{ex}
	Let $\mathsf{C} = \mathsf{Ab}$. We say that $G \in \mathsf{Ab}$ is torsion free iff $\,\forall\,  0 \neq x \in G$, for all $n \in \Z$ s.t. $nx = 0$, then $n = 0$.
	Instead $G$ is torsion iff $\,\forall\,  x \in G$ there exists $0 \neq n \in \Z$ s.t. $nx = 0$.
	Given $G \in \mathsf{Ab}$, we denote by $t(G) \leq G$ the torsion subgroup of $G$, i.e.
	\begin{equation}
		t(G) \coloneqq \left\{ x \in G \ \middle|\ \exists\, 0 \neq n \in \Z \text{ s.t. } nx = 0 \right\}
	.\end{equation} 
	Clearly, then, $G/t(G)$ is a torsion free group.

	Let's now see a few examples of abelian categories:
	\begin{itemize}
		\item Let $\mathsf{C} = \mathsf{Mod}\text{-}R$ the category of abelian groups.
			$\mathsf{C}$ is {\em abelian}: consider $A_R \xrightarrow{f} B_R$, then
			\begin{equation}
				\coker \left( \ker f \right) \simeq A/ \ker f \quad \text{ and } \quad
				\ker \left( \coker f \right) \simeq \ima f
			.\end{equation} 
			From the first isomorphism theorem we obtain an isomorphism of the two.
			Then one can show that this category is abelian.
		\item Let $\mathsf{T} \subset \mathsf{Ab}$  the full subcategory of abelian groups consisting of {\em torsion} abelian groups, then $\mathsf{T}$ is abelian.
			This is the case, since $\ker$ and $\coker$ in $\mathsf{T}$ correspond to the notions in $\mathsf{Ab}$, which is abelian.
	\end{itemize}
	The following, instead, are additive, with kernels and cokernels, but not abelian:
	\begin{itemize}
		\item Let $\mathsf{F} \subset \mathsf{Ab}$ be the full subcategory consisting of the torsion free abelian groups.
			Clearly $\mathsf{F}$ is closed under subgroups.
			Let $A \xrightarrow{f} B$ a morphism in $\mathsf{F}$.
			Let $K \xrightarrow{\epsilon} A$ a kernel of $f$ in $\mathsf{Ab}$, clearly $K \hookrightarrow A$, hence $K \in \mathrm{Ob} \left(\mathsf{F}\right)$ and $f$ admits kernel in $\mathsf{F}$.
			Let $\left(C, \pi\right)$ a cokernel in $\mathsf{Ab}$. It might not be in $\mathsf{F}$.
			Consider $C/t(C) \in \mathrm{Ob} \left(\mathsf{F}\right)$ and $B \xrightarrow{\pi} C \xrightarrow{q} C/t(C)$, then $q \circ \pi$ is a cokernel of $f$ in $\mathsf{F}$.
			It follows that $f$ admits also cokernel in $\mathsf{F}$.

			In other words we have just proved that $\mathsf{F}$ admits both kernels and cokernels.
			But $\mathsf{F}$ is not abelian.
			In order to show this we consider
			\begin{equation}
			\begin{tikzcd}
				\ker \dot{2} = 0 \arrow[r, "0", rightarrow] & \Z \arrow[r, "\dot{2}", rightarrow] \arrow[d, "1_\Z"', rightarrow] &
				\Z \arrow[r, "0", rightarrow] & 0 = \coker \dot{2}\\
				 & \Z \arrow[r, "\widetilde{\dot{2}}", rightarrow] & \Z \arrow[u, "1_\Z"', rightarrow] & 
			\end{tikzcd}
			,\end{equation} 
			where $\dot{2}\colon \Z \to \Z$ is the multiplication by $2$.
			In $\mathsf{F}$ we have $\coker \dot{2} = 0$, since in $\mathsf{Ab}$ $\coker \dot{2} = \Z/2\Z$, which is torsion.
			Then, in this example, $\widetilde{f} = \widetilde{\dot{2}}$, which is not an isomorphism in $\mathsf{F}$ 
			(nor in $\mathsf{Ab}$, and $\mathsf{F}$ is a full subcategory of $\mathsf{Ab}$).
			Also note that $\dot{2}$ is both mono and epi in $\mathsf{F}$, but not an iso.


		\item Let $G \in \mathrm{Ob} \left(\mathsf{Ab}\right)$ an abelian group. 
			We say that $G$ is {\em divisible} iff $\,\forall\, x \in G$ and $\,\forall\,  0 \neq n \in \Z$, $\exists\, y \in G$ s.t. $ny = x$.
			Instead an abelian group is called {\em reduced} iff it has no nonzero divisible subgroups.

			Let $\mathsf{D} \subset \mathsf{Ab}$ the full subcategory consisting of divisible abelian groups.
			Then $\mathsf{D}$ has kernels and cokernels, it is also additive, but not abelian.
	\end{itemize}
\end{ex} 

I'm actually not sure whether the following definition is correct, but I cannot find it on the internet and I really didn't understand what was part of the definition during the lecture.
\begin{defn}[Torsion pair of full subcategories]
	Let $\mathsf{C}$ be an abelian category, and $\mathsf{D} \subset \mathsf{C} \supset \mathsf{E}$ be two full subcategories.
	We say that the pair $\left(\mathsf{D}, \mathsf{E}\right)$ is a {\em torsion pair} iff given any $D \in \mathrm{Ob} \left(\mathsf{D}\right)$ and $E \in \mathrm{Ob} \left(\mathsf{E}\right)$ we have
	\begin{equation}
	\mathrm{Hom}_{\mathsf{C}} \left( D, E \right) = 0
	.\end{equation} 
\end{defn}

\begin{ex}\leavevmode\vspace{-.2\baselineskip}
	\begin{itemize}
		\item Consider the category $\mathsf{Ab}$ of abelian groups and $\mathsf{T} \subset \mathsf{Ab}$ the full subcategory of torsion abelian groups and $\mathsf{F} \subset \mathsf{Ab}$ the full subcategory of torsion-free abelian groups.
			The pair $\left(T, F\right)$ is a torsion pair, in fact, for any $T \in \mathrm{Ob} \left(\mathsf{T}\right)$ and $F \in \mathrm{Ob} \left(\mathsf{F}\right)$, we have
			\begin{equation}
			\mathrm{Hom}_{\mathsf{Ab}} \left( T, F \right) = 0
			.\end{equation} 
		\item Consider the full subcategories $\mathsf{D} \subset \mathsf{Ab}$ of all divisible groups and $\mathsf{R} \subset \mathsf{Ab}$ of all reduced groups.
			The pair $\left(\mathsf{D}, \mathsf{R}\right)$ is torsion, in fact, for any $D \in \mathrm{Ob} \left(\mathsf{D}\right)$ and $R \in \mathrm{Ob} \left(\mathsf{R}\right)$, we have
			\begin{equation}
			\mathrm{Hom}_{\mathsf{Ab}} \left( D, R \right) = 0
			.\end{equation} 
	\end{itemize}
\end{ex} 

\subsection{Pullback and Pushout}
\begin{defn}[Pullback]
	Let $\mathsf{C}$ be an arbitrary category.
	Let $A \xrightarrow{f} C$ and $B \xrightarrow{g} C$ be morphisms in $\mathsf{C}$.
	A {\em pullback} of $f$ and $g$ is a triple $\left(P, p_A, p_B\right)$, with $P \in \mathrm{Ob} \left(\mathsf{C}\right)$, $p_A\colon P \to A$ and $p_B\colon P \to B$ s.t. the following conditions are satisfied
	\begin{description}
		\item[PB1] The following square is commutative
			\begin{equation}
			\begin{tikzcd}
				P \arrow[r, "p_A", rightarrow] \arrow[d, "p_B"', rightarrow] & A \arrow[d, "f", rightarrow] \\
				B \arrow[r, "g"', rightarrow] & C
			\end{tikzcd}
			.\end{equation} 
			In other words we ask that $f \circ p_A = g \circ p_B$.
		\item[PB2] For any pair of morphisms $X \xrightarrow{\alpha} A$ and $X \xrightarrow{\beta} B$, from a fixed $X \in \mathrm{Ob} \left(\mathsf{C}\right)$, s.t. $f \circ \alpha = g \circ \beta$ then $\exists\, !\, X \xrightarrow{\gamma} P$ s.t. the following diagram commutes
			\begin{equation}
			\begin{tikzcd}
				X \arrow[rd, "\exists\, !\, \gamma", rightarrow] \arrow[rrd, "\alpha", rightarrow, bend left] \arrow[rdd, "\beta"', rightarrow, bend right] &  & \\
			   & P \arrow[r, "p_A", rightarrow] \arrow[d, "p_B"', rightarrow] & A \arrow[d, "f", rightarrow] \\
			   & B \arrow[r, "g"', rightarrow] & C
			\end{tikzcd}
			.\end{equation} 
			In other words, s.t. $p_B \circ \gamma = \beta$ and $p_A \circ \gamma = \alpha$.
	\end{description} 
\end{defn}
		
\begin{rem}
	Notice that {\em PB2} is a universal property.
	This means that, if a {\em pullback} of $f$ and $g$ exists, then it is unique up to a unique isomorphism.
\end{rem}

\begin{ex}
	Let $\mathsf{C}$ be a preadditive category with $0$ object.
	\begin{itemize}
	\item Consider $A \xrightarrow{f} C$ and $0 \xrightarrow{0} C$.
		A pullback of $f$ and $0$ exists iff $\ker f$ exists in $\mathsf{C}$.
		In particular $\left(P, p_A\right)$ is a kernel of $f$.
	\item Consider $A \xrightarrow{0} 0$ and $B \xrightarrow{0} 0$.
		The pullback of $0$ and $0$ exists iff the product of $A$ and $B$ exists, then the triple $\left(P, p_A, p_B\right)$ is a product of $A$ and $B$:
		\begin{equation}
		\begin{tikzcd}
			P \arrow[r, "p_A", rightarrow] \arrow[d, "p_B"', rightarrow] & A \arrow[d, "0", rightarrow] \\
			B \arrow[r, "0", rightarrow] & 0
		\end{tikzcd}
		.\end{equation} 
	\end{itemize}
\end{ex} 

\begin{prop}
	Let $\mathsf{C}$ be a preadditive category with $0$ object.
	If $\mathsf{C}$ admits kernel and finite products, then $\mathsf{C}$ has pullbacks.
	Moreover these are constructed by means of products and kernels.
\end{prop} 
\begin{proof}
	The construction via kernels and products goes as follows:
	Consider the morphisms $A \xrightarrow{f} C$ and $B \xrightarrow{g} C$.
	Let $\left(A \prod B, \pi_A, \pi_B\right)$ be a product.
	Let $\mu \coloneqq f \circ \pi_A - g \circ \pi_B\colon A \prod B \to C$.
	Finally, consider $\left(K, \epsilon\right)$ a kernel of $\mu$.
	Then $\left(K, p_A, p_B\right)$, with $p_A \coloneqq \pi_A \circ \epsilon$ and $p_B = \pi_B \circ \epsilon$, is a pullback of $f$ and $g$.
	The corresponding diagram is
	\begin{equation}
	\begin{tikzcd}
		K \arrow[rd, "\epsilon", rightarrow] \arrow[rrd, "p_A", rightarrow, bend left] \arrow[rdd, "p_B"', rightarrow, bend right] & & \\
	   & B \prod A \arrow[r, "\pi_A", rightarrow] \arrow[d, "\pi_B"', rightarrow] \arrow[rd, "\mu", rightarrow] & A \arrow[d, "f", rightarrow] \\
	   & B \arrow[r, "g"', rightarrow] & C
	\end{tikzcd}
	.\end{equation} 
\end{proof}

\begin{ex}
	Consider an abelian category $\mathsf{C}$, for example the category $\mathsf{Mod}\text{-}R$.
	Take two morphisms $A \xrightarrow{f} C$ and $B \xrightarrow{g} C$, then the pullback of $f$ and $g$ is a submodule $P \leq A \oplus B$, in particular it is
	\begin{align}
		P &= \left\{ \left( a,b \right) \in A \oplus B \ \middle|\ \mu \left(a, b\right) = 0 \right\}\\
		  &= \left\{ \left(a, b\right) \in A \oplus B \ \middle|\ f(a) = g(b) \right\}
	.\end{align} 
\end{ex} 

\begin{prop}
	Let $\mathsf{C}$ be preadditive with $0$ object.
	Let 
	\begin{equation}
	\begin{tikzcd}
		P \arrow[r, "p_A", rightarrow] \arrow[d, "p_B"', rightarrow] & A \arrow[d, "f", rightarrow] \\
		B \arrow[r, "g"', rightarrow] & C
	\end{tikzcd}
	\end{equation} 
	be a pullback diagram, then:
	\begin{itemize}
		\item If $g$ (resp. $f$) is mono, then $p_A$ (resp. $p_B$) [the parallel arrow] is mono.
		\item If $\mathsf{C}$ is abelian and $g$ (resp. $f$) is epi, then $p_A$ (resp. $p_B$) is epi.
		\item If $g$ (resp. $f$) is a kernel of $h$, then $p_A$ (resp. $p_B$) is a kernel of $h \circ f$ (resp. $h \circ g$).
	\end{itemize}
\end{prop} 

\begin{ex}[An application of the above result]
	Let $\mathsf{C}$ be an abelian category.
	Consider the following pullback diagram of the morphisms $f$ and $g$
	\begin{equation}
	\begin{tikzcd}
		P \arrow[r, "p_A", rightarrow] \arrow[d, "p_B"', rightarrow] &
		A \arrow[d, "f", rightarrow] \\
		B \arrow[r, "g"', rightarrow] &
		C
	\end{tikzcd}
	.\end{equation} 
	Assume that $g$ is epi.
	Take $\left(K, \epsilon\right)$ a kernel of $g$, then
	$\exists\, ! \delta\colon K \to P$ s.t. the following diagram commutes
	\begin{equation}
	\begin{tikzcd}
		K \arrow[r, "\delta", tail] & %\arrow[d, "", equal] &
		P \arrow[r, "p_A", twoheadrightarrow] \arrow[d, "p_B", rightarrow] &
		A \arrow[d, "f", rightarrow] \\
		K \arrow[r, "\epsilon"', tail] \arrow[u, "1_K", equal] &
		B \arrow[r, "g"', twoheadrightarrow] &
		C
	\end{tikzcd}
	.\end{equation} 
	Moreover $\delta$ is a kernel of $p_A$ (hence it is a monomorphism).
\end{ex} 

\begin{defn}[Pushout]
	Let $\mathsf{C}$ be an arbitrary category.
	A {\em pushout} of morphisms $C \xrightarrow{f} A$ and $C \xrightarrow{g} B$ in $\mathsf{C}$ is a {\em pullback} in $\mathsf{C}^{op}$.
	This means that we can dualize every result for the pullback.

	More explicitly, a pushout is a triple $ \left(P, \nu_A, \nu_B \right)$, with $P \in \mathrm{Ob} \left(\mathsf{C}\right)$, $\nu_A\colon A \to P$,  and $\nu_B\colon B \to P$ morphisms s.t. the following conditions are satisfied
	\begin{description}
		\item[PO1] The following square is commutative
			\begin{equation}
			\begin{tikzcd}
				C \arrow[r, "f", rightarrow] \arrow[d, "g"', rightarrow] &
				A \arrow[d, "\nu_A", rightarrow] \\
				B \arrow[r, "\nu_B"', rightarrow] &
				P
			\end{tikzcd}
			.\end{equation} 
			In other words we ask that $\nu_A \circ f = \nu_B \circ g$.
		\item[PO2] For any pair of morphisms $A \xrightarrow{\alpha} X$ and $B \xrightarrow{\beta} X$, into a fixed $X \in \mathrm{Ob} \left(\mathsf{C}\right)$, s.t. $\alpha \circ f = \beta \circ g$, then
			$\exists\, !\, P \xrightarrow{\gamma} X$ s.t. the following diagram commutes
			\begin{equation}
			\begin{tikzcd}
				C \arrow[r, "f", rightarrow] \arrow[d, "g"', rightarrow] &
				A \arrow[d, "\nu_A", rightarrow] \arrow[rdd, "\alpha", rightarrow, bend left] & \\
				B \arrow[r, "\nu_B"', rightarrow] \arrow[rrd, "\beta"', rightarrow, bend right] &
				P \arrow[rd, "\exists\, !\, \gamma", dashrightarrow] & \\
				& & X
			\end{tikzcd}
			.\end{equation} 			
			In other words, s.t. $\gamma \circ \nu_A = \alpha$ and $\gamma \circ \nu_B = \beta$.
	\end{description} 
\end{defn}
		
\begin{ex}
	Let $\mathsf{C}$ be a preadditive category with $0$ object.
	\begin{itemize}
		\item Consider $C \xrightarrow{f} A$ and $C \xrightarrow{0} 0$.
			A pushout of $f$ and $0$ exists iff $\coker f$ exists in $\mathsf{C}$.
			In partcular $\left(P, \nu_A \right)$ is a cokernel of $f$.
		\item Consider $0 \xrightarrow{0} A$ and $0 \xrightarrow{0} B$.
			The pushout diagram of $0$ and $0$ exists iff the coproduct of $A$ and $B$ exists.
			Then the triple $ \left(P, \nu_A, \nu_B\right)$ is a coproduct of $A$ and $B$:
			\begin{equation}
			\begin{tikzcd}
				0 \arrow[r, "0", rightarrow] \arrow[d, "0"', rightarrow] &
				A \arrow[d, "\nu_A", rightarrow] \\
				B \arrow[r, "\nu_B"', rightarrow] &
				P
			\end{tikzcd}
			.\end{equation} 
	\end{itemize}
\end{ex} 

\begin{prop}
	Let $\mathsf{C}$ be a preadditive category with $0$ object.
	If $\mathsf{C}$ admits cokernels and finite coproducts, then $C$ has pushouts.
	Moreover these are constructed by means of coproducts and cokernels.
\end{prop} 
\begin{proof}
	The construction goes as follows:
	Consider the morphisms $C \xrightarrow{f} A$ and $C \xrightarrow{g} B$.
	Let $\left(A \coprod B, \epsilon_A, \epsilon_B\right)$ be a coproduct.
	Let $\delta \coloneqq \epsilon_A \circ f - \epsilon_B \circ g:C \to A \coprod B$.
	Finally consider $\left(P, p\right)$ a cokernel of $\delta$.
	Then $\left(P, p \circ \epsilon_A, p \circ \epsilon_B\right)$ is a pushout of $f$ and $g$.
	The corresponding diagram is
	\begin{equation}
	\begin{tikzcd}
		C \arrow[rd, "\delta", rightarrow] \arrow[r, "f", rightarrow] \arrow[d, "g"', rightarrow] & 
		A \arrow[d, "\epsilon_A", rightarrow] \arrow[ddr, "", rightarrow, bend left] & \\
		B \arrow[r, "\epsilon_B"', rightarrow] \arrow[rrd, "", rightarrow, bend right] &
		A \coprod B \arrow[rd, "p", rightarrow] & \\
		& & P
	\end{tikzcd}
	.\end{equation} 
\end{proof}

\begin{prop}
	Let $\mathsf{C}$ be preadditive with $0$ object.
	Let 
	\begin{equation}
	\begin{tikzcd}
		C \arrow[r, "f", rightarrow] \arrow[d, "g"', rightarrow] &
		A \arrow[d, "\nu_A", rightarrow] \\
		B \arrow[r, "\nu_B"', rightarrow] & P
	\end{tikzcd}
	\end{equation} 
	be a pushout diagram, then:
	\begin{itemize}
		\item If $f$ (resp. $g$) is epi, then $\nu_B$ (resp. $\nu_A$) [the parallel arrow] is epi.
		\item If $\mathsf{C}$ is abelian and $f$ (resp. $g$) is mono, then $\nu_B$ (resp. $\nu_A$) is mono.
		\item If $f$ (resp. $g$) is a cokernel of $h$, then $\nu_B$ (resp. $\nu_A$) is a kernel of $g \circ h$ (resp. $f \circ h$).
	\end{itemize}
\end{prop} 

\begin{ex}
	Let $\mathsf{C} = \mathsf{Mod}\text{-}R$.
	Consider the morphisms $C \xrightarrow{f} A$ and $C \xrightarrow{g} B$.
	Then a pushout $P \simeq \frac{A \oplus B}{H}$, where $H$ is the image of $\delta$ as defined above, more explicitly
	\begin{equation}
		H \coloneqq \left\langle \left(f(c), -g(c)\right) \ \middle|\ c \in C \right\rangle
	.\end{equation} 
	More explicitly, the image of $C$ in $A$ and $B$ (resp. via $f$ and $g$) are glued together in $P$.
\end{ex} 

\begin{ex}[An application of the above result]
	Let $\mathsf{C}$ be an abelian category.
	Consider the following pushout diagram of the morphisms $f$ and $g$ 
	\begin{equation}
	\begin{tikzcd}
		C \arrow[r, "f", rightarrow] \arrow[d, "g"', rightarrow] &
		A \arrow[d, "\nu_A", rightarrow] \\
		B \arrow[r, "\nu_B"', rightarrow] &
		P
	\end{tikzcd}
	.\end{equation} 
	Assume that $f$ is mono.
	Take $\left(D, p\right)$ a cokernel of $f$, then $\exists\, !\, q\colon P \to D$ s.t. the following diagram commutes
	\begin{equation}
	\begin{tikzcd}
		C \arrow[r, "f", tail] \arrow[d, "g"', rightarrow] &
		A \arrow[r, "p", twoheadrightarrow] \arrow[d, "\nu_A", rightarrow] &
		D \arrow[d, "", equal] \\
		B \arrow[r, "\nu_B"', tail] &
		P \arrow[r, "q"', twoheadrightarrow] &
		D
	\end{tikzcd}
	.\end{equation} 
	Moreover $q$ is a cokernel of $\nu_B$ (hence it is an epimorphism).
\end{ex} 

\section{Exact categories}
\begin{rem}
	Let $\mathsf{C}$ be an abelian category and $A \xrightarrow{i} B \xrightarrow{d} C$ morphisms in $\mathsf{C}$, s.t.
	$i = \ker d$ and $d = \coker i$.
	Then $i$ is mono, $d$ is epi and $\ker d = \ima i$.
	In fact 
	\begin{equation}
		\begin{tikzcd}[column sep=small]
		0 \arrow[r, "", rightarrow] &
		A \arrow[r, "i", rightarrow] \arrow[d, "1_A"', equal] &
		B \arrow[r, "d", rightarrow] &
		C \\
		& \mathrm{coim}\, i = A \arrow[r, "\simeq", rightarrow] &
		\ker d = \ima i \arrow[u, "", rightarrow] & \\
	\end{tikzcd}
	.\end{equation} 
\end{rem}

\begin{defn}[Kernel-cokernel pair]
	Let $\mathsf{C}$ be an additive category.
	A \textbf{kernel cokernel pair} $\left(i, d\right)$ in $\mathsf{C}$ is a pair of composable morphisms
	\begin{equation}
	A \xrightarrow{i} B \xrightarrow{d} C
	\end{equation} 
	s.t. $i$ is a kernel of $d$ and $d$ is a cokernel of $i$.
\end{defn}

\begin{defn}[Inflation, deflation, conflation]
	Let $\mathcal{E}$ be a fixed class of kernel-cokernel pairs in $\mathsf{C}$.
	A sequence $E = \left(i, d\right) \in \mathcal{E}$
	\begin{equation}
	A \xrightarrow{i} B \xrightarrow{d} C
	\end{equation} 
	is called a \textbf{conflation}.
	A morphism $i: A \rightarrowtail B$ s.t. there exists a morphism $d$ with $\left(i, d\right) \in \mathcal{E}$ is called \textbf{inflation}.
	A morphism $d: B \twoheadrightarrow C$ s.t. there exists a morphism $i$ with $\left(i, d\right) \in \mathcal{E}$ is called \textbf{deflation}.
	Sometimes they are called admissible mono and admissible epi.
\end{defn}

\begin{defn}[Exact structure]
	Given an additive category $\mathsf{C}$, an \textbf{exact structure} in $\mathsf{C}$ is a class $\mathcal{E}$ of ker-coker pairs satisfying the following axioms and closed under isomorphisms, i.e.
	given a commutative diagram
	\begin{equation}
	\begin{tikzcd}
		A \arrow[r, "i", rightarrow] \arrow[d, "\alpha"', rightarrow] &
		B \arrow[r, "d", rightarrow] \arrow[d, "\beta", rightarrow] &
		C \arrow[d, "\gamma", rightarrow] \\
		A' \arrow[r, "i'", rightarrow] &
		B' \arrow[r, "d'", rightarrow] &
		C'
	\end{tikzcd}
	,\end{equation} 
	in which all the vertical arrows are isomorphisms, and $\left(i, d\right) \in \mathcal{E}$, then also $\left(i', d'\right) \in \mathcal{E}$.
	\begin{description}
		\item[Ex0] $1_0$ is a deflation,
		\item[Ex0$^{op}$] $1_0$ is an inflation,
		\item[Ex1] the class of deflations is closed under compositions,
		\item[Ex1$^{op}$] the class of inflations is closed under compositions,
		\item[Ex2] the pullback of a deflation along an arbitrary morphism exists and is a deflation,
		\item[Ex2$^{op}$] the pushout of an inflation along an arbitrary morphism exists and is an inflation.
	\end{description} 
	These last 2 axioms correspond to the following diagrams
	\begin{equation}
	\mathbf{Ex2}:
	\begin{tikzcd}
		Y' \arrow[r, "d'", twoheadrightarrow] \arrow[d, "f'"', rightarrow] &
		Z' \arrow[d, "f", rightarrow] \\
		Y \arrow[r, "d"', twoheadrightarrow] &
		Z
	\end{tikzcd}\quad\quad\quad\quad
	\mathbf{Ex2}^{op}:
	\begin{tikzcd}
		X \arrow[r, "i", tail] \arrow[d, "f"', rightarrow] &
		Y \arrow[d, "f'", rightarrow] \\
		X' \arrow[r, "i'", tail] &
		Y'
	\end{tikzcd}
	.\end{equation} 
	The interpetation is as follows (for the first diagram):
	given a deflation $d: Y \to Z$ and a morphism $f: Z' \to Z$, then,
	if the pullback of $\left(d, f\right)$ exists, let's denote it with $\left(Y', f', d'\right)$, also $d': Y' \to Z'$ is a deflation.
\end{defn}

\begin{defn}[Exact category]
	An \textbf{exact category} is a pair $\left(\mathsf{C}, \mathcal{E}\right)$, with $\mathsf{C}$ an additive category and $\mathcal{E}$ an exact structure on $\mathsf{C}$.
	Conflations in $\mathcal{E}$ are called \textbf{short exact sequences}.
\end{defn}

\begin{rem}
	$\mathcal{E}$ is an exact structure in $\mathsf{C}$ iff $\mathcal{E}^{op}$ is an exact structure in $\mathsf{C}^{op}$.
\end{rem}

\begin{rem}
	An abelian category $\mathsf{C}$ with $\mathcal{E}$ given by all of its ker-coker pairs is an exact category.
\end{rem}

\begin{defn}[Extensions closed subcategory]
	Given an abelian category $\mathsf{C}$, a full subcategory $\mathsf{C}' \subset \mathsf{C}$ is \textbf{extensions closed} iff, given a ker-coker pair
	$A \xrightarrow{i} B \xrightarrow{d} C$ with $A, C \in \mathrm{Ob} \left(\mathsf{C}'\right)$, then $B \in \mathrm{Ob} \left(\mathsf{C}'\right)$
\end{defn}

\begin{rem}
	An extensions closed subcategory of an abelian category is an exact category, but need not be abelian.
	In fact
	\begin{itemize}
		\item $\mathsf{F} \subset \mathsf{Ab}$ the full subcategory of torsion free abelian groups,
		\item $\mathsf{D} \subset \mathsf{Ab}$ the full subcategory of divisible abelian groups,
	\end{itemize}
	are both extensions closed in $\mathsf{Ab}$, but are not abelian.
	For the first, in fact, given
	\begin{equation}
	A \xrightarrow{i} B \xrightarrow{d} C
	,\end{equation} 
	with $A, C \in \mathrm{Ob} \left(\mathsf{F}\right)$, then $B/i(A), i(A) \in \mathrm{Ob} \left(\mathsf{F}\right)$.
	From this it can be easily proved that also $B \in \mathrm{Ob} \left(\mathsf{F}\right)$.
\end{rem}

The following proposition can be found in the paper \textit{Chain complexes and stable categories}, by B. Keller.
Also in the PhD thesis of  T. Bühler \textit{Exact categories}
(For more precise references see lecture 8-1, minute 20).
\begin{prop}[Keller]
	The axioms of exact categories are redundant.
	The following are enough \textbf{Ex0}, \textbf{Ex1}, \textbf{Ex2}, \textbf{Ex2}$^{op}$.
	They imply:
	\begin{description}
		\item[a] given $X, Y \in \mathrm{Ob} \left(\mathsf{C}\right)$, then the following is a conflation
			\begin{equation}
			X \xrightarrow{
				\begin{bmatrix}
					1 \\ 0
				\end{bmatrix} 
			} X \oplus Z
			\xrightarrow{
				\begin{bmatrix}
					0 & 1
				\end{bmatrix} 
			} Z
			.\end{equation} 
		\item[b] \textbf{Ex1}$^{op}$.
		\item[c] \textbf{Quillem's obscure axioms}: If a morphism $d$ has a kernel and if $d \circ e$ is a deflation for some morphism $e$, then also $d$ is a deflation.
		\item[c$^{op}$] \textbf{Quillem's obscure axioms}: If a morphism $i$ has a cokernel and if $k \circ i$ is an inflation for some morphism $k$, then also $i$ is an inflation.
	\end{description} 
\end{prop} 

\section{Limit}
We will concentrate on an arbitrary category $\mathsf{C}$, and on a small category $\mathsf{I}$, i.e. a category with $\mathrm{Ob} \left(\mathsf{I}\right)$ is a set.
Consider a functor 
\begin{equation}
F: \mathsf{I} \to \mathsf{C}
.\end{equation} 
Then $\,\forall\, i \in \mathrm{Ob} \left(\mathsf{I}\right)$, $F(i) \in \mathrm{Ob} \left(\mathsf{C}\right)$ and,
given a morphism $\lambda: i \to j$ in $\mathsf{I}$, then $F(\lambda): F(i) \to F(j)$.

\begin{defn}[Compatible family with respect to $F$]
	Consider a family $\left\{ \alpha_i \right\}_{i \in \mathrm{Ob} \left(\mathsf{I}\right)}$ of morphisms 
	$\alpha_i: X \to F(i)$ for a fixed $X \in \mathrm{Ob} \left(\mathsf{C}\right)$.
	It is said to be a \textbf{compatible family} with respect to $F$ iff
	given any morphism $\lambda: i \to j$ in $\mathsf{I}$, the following trangle commutes
	\begin{equation}
	\begin{tikzcd}
		X \arrow[r, "\alpha_i", rightarrow] \arrow[rd, "\alpha_j"', rightarrow] &
		F(i) \arrow[d, "F(\lambda)", rightarrow] \\
		&
		F(j)
	\end{tikzcd}
	.\end{equation} 
	In other words iff $\alpha_j = F(\lambda) \circ \alpha_i$ for every $i, j \in \mathrm{Ob} \left(\mathsf{I}\right)$ and every $\lambda: i \to j$.
\end{defn}

\begin{defn}[Projective (inverse) limit]
	A (projective/inverse) \textbf{limit} of $F$ is an object in $\mathsf{C}$, denoted with $\varprojlim F$, with morphisms
	$p_i: \varprojlim F \to F(i)$ for all $i \in \mathrm{Ob} \left(\mathsf{I}\right)$ stasfying the following conditions
	\begin{description}
		\item[LIM1] $\left\{ p_i \right\}_{i \in \mathrm{Ob} \left(\mathsf{I}\right)}$ is a compatible family of morphisms, i.e.
			\begin{equation}
			\begin{tikzcd}
				\varprojlim F \arrow[r, "p_i", rightarrow] \arrow[rd, "p_j"', rightarrow] &
				F(i) \arrow[d, "F(\lambda)", rightarrow] \\
				&
				F(j)
			\end{tikzcd}
			\end{equation} 
			the above diagram commutes for all $i, j \in \mathrm{Ob} \left(\mathsf{I}\right)$ and all $\lambda: i \to j$.
		\item[LIM2] For any $X \in \mathrm{Ob} \left(\mathsf{C}\right)$ and any compatible family of morphisms $\left\{ \alpha_i \right\}_{i \in \mathrm{Ob} \left(\mathsf{I}\right)}$, with $\alpha_i: X \to F(i)$, 
			$\exists\, !\, \alpha: X \to \varprojlim F$ s.t. $p_i \circ \alpha = \alpha_i$ $\,\forall\, i \in \mathrm{Ob} \left(\mathsf{I}\right)$, i.e.
			\begin{equation}
			\begin{tikzcd}
				X \arrow[r, "\alpha_i", rightarrow] \arrow[d, "\alpha"', rightarrow] &
				F(i)\\
				\varprojlim F \arrow[ru, "p_i"', rightarrow]  
			\end{tikzcd}
			.\end{equation} 
	\end{description} 
\end{defn}

\begin{rem}
	As always, since it is defined through a universal property, if $(\varprojlim F, p_i)$ exists, it is unique up to unique isomorphism.
\end{rem}

\begin{ex}
	Let $\mathsf{I}$ be a small discrete category, i.e. the morphisms in $\mathsf{I}$ are only the identities.
	Then, for any functor $F: \mathsf{I} \to \mathsf{C}$, $\varprojlim F$ exists iff 
	$\prod_{i \in \mathrm{Ob} \left(\mathsf{I}\right)} F(i)$ exists and they are isomorphic.
	In particular the $p_i$ s correspond with the projections of the product.
\end{ex} 

\begin{ex}
	Consider, in an arbitrary category $\mathsf{C}$, the following diagram
	\begin{equation}
	\begin{tikzcd}
		& A_1 \arrow[d, "f_1", rightarrow] \\
		A_2 \arrow[r, "f_2", rightarrow] &
		A_3
	\end{tikzcd}
	.\end{equation} 
	Consider the category $\mathsf{I}$ with $\mathrm{Ob} \left(\mathsf{I}\right) = \left\{ 1, 2, 3 \right\}$ and morphism (other than the identities) $1 \to 3$ and $2 \to 3$.
	Consider the functor $F: \mathsf{I} \to \mathsf{C}$ defined as follows:
	 \begin{equation}
		 \begin{matrix}
			 F(i) = A_i, &
			 F( 1 \to 3) = f_1, &
			 F( 2 \to 3) = f_2,
		 \end{matrix} 
	.\end{equation} 
	Then any $\varprojlim F$ corresponds with a pullback of the above diagram.
\end{ex} 

\begin{defn}[Complete category]
	A category $\mathsf{C}$ is called \textbf{complete} iff 
	every functor $F: \mathsf{I} \to \mathsf{C}$, from a small category $\mathsf{I}$, admits limit in $\mathsf{C}$.
\end{defn}

\begin{rem}[Some terminology]
	Assume that a preadditive category $\mathsf{C}$ has infinite products.
	Consider a functor $F: \mathsf{I} \to \mathsf{C}$, from a small category $\mathsf{I}$.
	For any morphism $\lambda: i \to j$ in $\mathsf{I}$, let's define
	\begin{equation}
		s(\lambda) := i \quad \text{ and } \quad t(\lambda) := j
	,\end{equation} 
	where $s$ denotes the source and $t$ the target of the morphism.
	Consider $\left(\prod_{i \in \mathrm{Ob} \left(\mathsf{I}\right)} F(i),  \pi_i \right)$ a product of $\left\{ F(i) \right\}_{i \in I}$ and the diagram
	\begin{equation}
	\begin{tikzcd}
		\prod_{i \in \mathrm{Ob} \left(\mathsf{I}\right)} F(i) \arrow[r, "\pi_{s(\lambda)}", rightarrow] 
		\arrow[rd, "\pi_{t(\lambda)}"', rightarrow] &
		F \left( s(\lambda) \right) \arrow[d, "F(\lambda)", rightarrow] \\
		& F \left( t(\lambda) \right)
	\end{tikzcd}
	.\end{equation} 
	In general it is not commutative, but we can define the morphism
	\begin{equation}
		\sigma_\lambda := F(\lambda) \circ \pi_{s(\lambda)} - \pi_{t(\lambda)}: \prod_{i \in \mathrm{Ob} \left(\mathsf{I}\right)} F(i) \to F \left( t(\lambda) \right)
	.\end{equation} 
	Let's now consider the product $\left(\prod_{\lambda \in \Lambda} F \left( t(\lambda) \right), q_\lambda\right)$, indexed by $\lambda \in \Lambda := \mathrm{Morph}\, \mathsf{I}$.
	By the universal property of products, the family $\left\{ \sigma_\lambda \right\}_{\lambda \in \Lambda}$ induces a unique morphism
	 \begin{equation}
		 \sigma: \prod_{i \in \mathrm{Ob} \left(\mathsf{I}\right)} F(i) \to \prod_{\lambda \in \Lambda} F \left( t(\lambda) \right)
	\end{equation} 
	s.t. $q_\lambda \circ\sigma = \sigma_\lambda$.
\end{rem}

\begin{prop}\label{prop:LimConstr}
	If a (preadditive) category $\mathsf{C}$ admits kernels and (infinite) products,
	then for every functor $F: \mathsf{I} \to \mathsf{C}$, from a small category $\mathsf{I}$, 
	$\mathsf{C}$ admits $\varprojlim F$.
	Moreover the limit is constructed by kernels and (infinite) products.
\end{prop} 
\begin{proof}
	The proof wants to show that the following construction actually is a limit for $F$.
	Consider $\left(K, \epsilon\right)$ a kernel for the above constructed morphism
	\begin{equation}
		\sigma: \prod_{i \in \mathrm{Ob} \left(\mathsf{I}\right)} F(i) \to \prod_{\lambda \in \Lambda} F \left( t(\lambda) \right)
	.\end{equation} 
	Then $\left(K, p_i\right)$, for $p_i := \pi_i \circ\epsilon$ is a projective limit of $F$.
\end{proof}

\begin{ex}
	Let $\mathsf{C} = \mathsf{Mod}\text{-}R$ and $\left( \mathsf{I}, \leq \right)$ a partially ordered set, viewed as a category.
	Consider a contravariant functor
	\begin{equation}
	F: \mathsf{I}^{op} \to \mathsf{Mod}\text{-}R
	.\end{equation} 
	This is equivalent to the data of $F(i) =: M_i \in \mathsf{Mod}\text{-}R$, and, for all $i \leq j$ of
	\begin{equation}
		F(i \to j) =: f_{ij}: M_j \to M_i
	.\end{equation} 
	Now, given $i \leq j \leq k$, then the following diagram commutes
	\begin{equation}
	\begin{tikzcd}
		M_k \arrow[r, "f_{jk}", rightarrow] \arrow[rd, "f_{ik}"', rightarrow] &
		M_j \arrow[d, "f_{ij}", rightarrow] \\
		& M_i
	\end{tikzcd}
	,\end{equation} 
	in other words $f_{ij} \circ f_{jk} = f_{ik}$.
	Moreover we require $f_{ii} = id_{M_i}$.

	We have, in fact, a correspondance, between contravariant functors from partially ordered sets and
	inverse systems of modules, which are families $\left\{ M_i, F_{ij} \right\}_{i \leq j}$ of modules $M_i$ and morphism $f_{ij}$ between them, satisfying the above compatibility conditions.

	Then, $\varprojlim F$ is called the \textbf{inverse limit} of $\left\{ M_i, f_{ij} \right\}_{i \leq j}$.
	The morphisms $f_{ij}$ are called the structural morphisms of the inverse system.
	Sometimes this is also denoted with $\varprojlim M_i$.
	
	Let's describe $\varprojlim M_i$ explicitly: for every $i \leq j$ we have the (not necessairily commutative) diagram
	\begin{equation}
	\begin{tikzcd}
		\prod_{i \in \mathrm{Ob} \left(\mathsf{I}\right)} M_i \arrow[r, "\pi_j", rightarrow] \arrow[rd, "\pi_i"', rightarrow] &
		M_j \arrow[d, "f_{ij}", rightarrow] \\
		& M_i
	\end{tikzcd} 
	.\end{equation}
	Let's define, for each $i \leq j$, $\sigma_{ij} := f_{ij} \circ\pi_j - \pi_i$.
	Then, by universal property of the product,
	$\exists\, !\, \sigma: \prod_{i \in \mathrm{Ob} \left(\mathsf{I}\right)} M_i \to \prod_{i \leq j} M_{ij}$,
	where $M_{ij} := M_i$ for every $i \leq j$.
	In the above construction $\pi_i \circ\sigma = \sigma_{ij}: \prod_{i \in \mathrm{Ob} \left(\mathsf{I}\right)} M_i \to M_{ij}$.
	Then we have
	\begin{align}
		\varprojlim M_i &\simeq \ker \sigma =
		\left\{ \mathbf{x} \in \prod_{i \in \mathrm{Ob} \left(\mathsf{I}\right)}
			M_i \ \middle|\ \sigma(\mathbf{x}) = 0
		\text{ i.e. } \sigma_{ij}(\mathbf{x}) =0 \,\forall\, i \leq j \right\} \\
				&=
		\left\{ \mathbf{x} = \left( x_i \right)_{i \in \mathrm{Ob} \left(\mathsf{C}\right)} \in
		\prod_{i \in \mathrm{Ob} \left(\mathsf{I}\right)} M_i \ \middle|\ 
		f_{ij}(x_j) = x_i, \ \,\forall\, i \leq j \right\}
	.\end{align} 
	It is a submodule of the product, in which, determined $x_j$, then $\,\forall\, i \leq j$, $x_i$ is determined by $x_j$, via the structural morphisms.
\end{ex} 

\begin{defn}[$I$-adic topology on a ring]
	Given a commutative ring $R$ and an ideal $I \triangleleft R$ of $R$.
	We define the \textbf{$\mathbf{I}$-adic topology on} $R$ as the linear topology determined by
	$\left\{ I^n \right\}_{n \in \N}$ as a basis for the neighbourhoods of $0$.
	The open subsets are generated by cosets of these ideals.
\end{defn}

\begin{rem}
	The \textbf{$\mathbf{I}$-adic topology} on $R$ is Hausdorff iff $\bigcap_{n \in \N} I^n = 0$.
\end{rem} 

\begin{ex}[Completion of a ring in the $i$-adic topology]
	Let $R$ be a commutative ring and $I \triangleleft R$ an ideal of $R$.
	For $n \leq m$, then $I^m \subset I^n$, hence the canonical projections
	\begin{align}
		\pi_{n,m}: R/I^m &\to R/I^n \\
		x + I^m &\mapsto x + I^n
	\end{align} 
	are well defined.
	We can check that $\left\{ R/I^n, \pi_{n,m} \right\}_{n \leq m}$ is a countable inverse system.
	\begin{align}
		\varprojlim R/I^n &=
		\left\{ \left( x_n + I^n \right)_{n \in \N} \in \prod_{n \in \N} R/I^n \ \middle|\ 
		\pi_{n,m}\left( x_m + I^m \right) = x_n + I^n \,\forall\, n \leq m \right\}\\
				  &=
		\left\{ \left( x_n + I^n \right)_{n \in \N} \in \prod_{n \in \N} R/I^n \ \middle|\ 
		x_m - x_n \in I^n \,\forall\, n \leq m \right\}
	.\end{align} 
	This is the \textbf{completion of} $R$ in the $I$-adic topology.
	It is called completion since, given $\left\{ x_n \right\}_{n \in \N}$ it is a \textit{Cauchy} sequence iff
	$\,\forall\, V$ neighbourhood of $0$, $\exists\, n_0 \in \N$ s.t. $x_n - x_m \in V$ for all $n,m \geq n_0$.
	Moreover we can define a \textit{neat Cauchy} sequence as a sequence $\left\{ x_n \right\}_{n \in \N}$ s.t.
	$\,\forall\, V_n := I^n$ m then $x_m - x_n \in V_n$ for all $m \geq n$.

	In particular an element $\left( x_n + I^n \right)_{n \in \N} \in \varprojlim R/I^n$ can be viewed as a limit
	of the Cauchy sequence $\left\{ x_n \right\}_{n \in \N}$.
	(This is the reason why it can be seen as the completion in the topology).

	Moreover we have a canonical projection
	\begin{align}
		\mu: R &\to \varprojlim R/I^n \\
		x &\mapsto \left( x + I^n \right)_{n \in \N}
	.\end{align} 
	Clearly $\ker \mu = \bigcap_{n \in \N} I^n$ (i.e. $\mu$ is injective iff
	$R$ is Hausdorff with the $I$-adic topology).
\end{ex} 

\begin{ex}[$p$-adic completion of the ring of integers]
	Let $R := \mathbb{Z}$ and $I := p  \mathbb{Z}$.
	\begin{equation}
		\hat{\mathbb{Z}}_p := \varprojlim \mathbb{Z}/p^n \mathbb{Z}
	\end{equation} 
	is the $p$-adic completion of the ring of integers.
	An element $\zeta \in \hat{\mathbb{Z}}$ can be written as
	\begin{equation}
	\zeta = a_0 + a_1 p + a_2 p^2 + \ldots
	,\end{equation} 
	with $0 \leq a_i < p$ for all  $i \geq 1$.
	In fact $x_0 + p \mathbb{Z} = a_0 + p\mathbb{Z}$, with $0 \leq a_0 < p$.
	Then $x_1 - x_0 \in p \mathbb{Z}$, hence $x_1 = a_0 + a_1 p$.
	Then, by induction, given  $x_n = a_0 + a_1 p + \ldots + a_{n} p^{n}$ and $x_{n+1} - x_n \in p^{n+1} \mathbb{Z}$, hence
	\begin{equation}
	x_{n+1} = a_0 + \ldots + a_{n+1} p^{n+1}
	.\end{equation} 
\end{ex} 

\subsection{The functor projective lim}
Fix $\mathsf{I}$ a small category and let $\mathsf{C}$ be a complete category.
Let $\mathsf{C}^{\mathsf{I}}$ be the functor category.

\begin{prop}
	\begin{align}
		\varprojlim: \mathsf{C}^{\mathsf{I}} &\to \mathsf{C} \\
		F &\mapsto \varprojlim F
	\end{align} 
	is a functor.
\end{prop} 	
\begin{proof}
	Given $\eta: F \to G$ a natural transformation between the functors
	$F, G \in \mathsf{C}^{\mathsf{I}}$, the functor associates it a morphism in the natural way
	\begin{equation}
	\varprojlim \eta: \varprojlim F \to \varprojlim G
	.\end{equation} 
\end{proof}

Let's study a little the category $\mathsf{C}^{\mathsf{I}}$, for a small category $\mathsf{I}$.
\begin{prop}
	$\mathsf{C}^{\mathsf{I}}$ inherits the properties of $\mathsf{C}$.
	More explicitly if $\mathsf{C}$ is preadditive/additive/abelian, then also
	$\mathsf{C}^{\mathsf{I}}$ is preadditive/additive/abelian.
	
	Morever construction in $\mathsf{C}$ can be done in $\mathsf{C}^{\mathsf{I}}$ locally, for every $i \in \mathrm{Ob} \left(\mathsf{I}\right)$.
	For instance:
	\begin{itemize}
		\item Given $\eta, \zeta \in \mathrm{Hom}_{\mathsf{C}^{\mathsf{I}}} \left( F, G \right)$, if $\mathsf{C}$ is preadditive, then
			$\left( \eta + \zeta \right)_i = \eta_i + \zeta_i$, for each object $i \in \mathrm{Ob} \left(\mathsf{I}\right)$.
		\item If $\mathsf{C}$ has products, then also $\mathsf{C}^{\mathsf{I}}$ has products.
			In particular, given two functors $F,G \in \mathrm{Ob} \left(\mathsf{C}^{\mathsf{I}}\right)$, we need to define the product
			$\left(F \Pi G, \pi_F, \pi_G\right)$, s.t. this is a product of $F$ and $G$ in $\mathsf{C}^{\mathsf{I}}$.
			On objects it is defined as expected
			\begin{equation}
				(F \prod G)(i) := F(i) \prod G(i)
			.\end{equation} 
			Moreover, on morphisms it is defined as follows: given $\lambda: i \to j$, then
			\begin{equation}
				\left( F \prod G \right)(\lambda) = 
				\begin{bmatrix}
					F(\lambda) & 0\\
					0 & G(\lambda)
				\end{bmatrix} 
			,\end{equation} 
			is our morphism $\left( F \Pi G \right)(\lambda): F(i) \Pi G(i) \to F(j) \Pi G(j)$.
			Finally we have to define the projections.
			They are constructed naturally as
			\begin{equation}
				(\pi_F)_i := \pi_{F(i)} \qquad \text{ and } \qquad \left( \pi_G \right)_i := \pi_{G(i)}
			.\end{equation} 
		\item If $\mathsf{C}$ has kernels, then also $\mathsf{C}^{\mathsf{I}}$ has kernels.
			Let $\eta: F \to G$ a natural transformation.
			Let's define $\ker \eta$ as an object of $\mathsf{C}^{\mathsf{I}}$.
			For every $i \in \mathrm{Ob} \left(\mathsf{I}\right)$ we define $K(i) := \ker \eta_i$ as an object in $\mathsf{C}$.
			This, for any morphism $\lambda: i \to j$, gives rise to the commutative diagram
			\begin{equation}
			\begin{tikzcd}
				K(i) \arrow[d, "\exists\, !\, \nu", dashrightarrow] \arrow[r, "\epsilon_i", rightarrow] &
				F(i) \arrow[r, "\eta_i", rightarrow] \arrow[d, "F(\lambda)", rightarrow] &
				G(i) \arrow[d, "G(\lambda)", rightarrow] \\
				K(j) \arrow[r, "\epsilon_j"', rightarrow] &
				F(j) \arrow[r, "\eta_j"', rightarrow] &
				G(j)
			\end{tikzcd}
			.\end{equation} 
			From this we define $K(\lambda) := \nu$.
			Then, the couple $\left(K, \left\{ \epsilon_i \right\}_{i \in \mathrm{Ob} \left(\mathsf{I}\right)} \right)$
			is the kernel of $\left\{ \eta_i \right\}_{i \in \mathrm{Ob} \left(\mathsf{I}\right)} $
		\item As an exercise to the writer: when you'll next read this line, please try to define the cokernel of a functor.
	\end{itemize}
\end{prop} 

\subsection{Characterization of projective limit}
Let, as before, $\mathsf{I}$ be a small category, and $\mathsf{C}^{\mathsf{I}}$ the category of functors $F: \mathsf{I} \to \mathsf{C}$.

\begin{defn}[Constant functor]
	Consider a fixed $X \in \mathrm{Ob} \left(\mathsf{C}\right)$.
	We define the constant functor
	\begin{equation}
	\Delta_X: \mathsf{I} \to \mathsf{C}
	.\end{equation} 
	On objects as $\Delta_X(i) = X$ for all $i \in \mathrm{Ob} \left(\mathsf{I}\right)$.
	On morphism $\Delta_X(\lambda) = id_X$ for all $\lambda: i \to j$.
\end{defn}

\begin{defn}[Diagonal functor]
	We define, in terms of the constant functor, the diagonal functor
	 \begin{equation}
	\Delta: \mathsf{C} \to \mathsf{C}^{\mathsf{I}}
	.\end{equation} 
	On objects as $\Delta(X) := \Delta_X$.
	On morphisms $f: X \to Y$, then
	\begin{equation}
		\Delta(f) := \bar{f}: \Delta_X \to \Delta_Y
	,\end{equation} 
	where $\bar{f}$ is a natural transformation s.t. for every $i \in \mathrm{Ob} \left(\mathsf{I}\right)$, $\bar{f}_i = f$.
\end{defn}

\begin{defn}[Some notation for the following proposition]
	Fix a functor $F \in \mathsf{C}^{\mathsf{I}}$.
	Let $H: \mathsf{C}^{op} \to \mathsf{Sets}$ be a contravariant functor from $\mathsf{C}$, defined as follows.
	On the objects $Y \in \mathrm{Ob} \left(\mathsf{C}\right)$, 
	$H(Y) := \mathrm{Nat} \left( \Delta_Y, F \right)$.
	On the morphisms, for $f: X \to Y$, 
	 \begin{align}
		 H(f): \mathrm{Nat} \left( \Delta_Y, F \right) &\to \mathrm{Nat} \left( \Delta_X, F \right) \\
		 \eta &\mapsto \eta \circ \bar{f}
	.\end{align} 
\end{defn}

\begin{prop}
	Given a functor $F \in \mathsf{C}^{\mathsf{I}}$, then
	$\varprojlim F$ exists iff the functor $H$ defined above is representable.
	In other words iff $\exists\, C \in \mathrm{Ob} \left(\mathsf{C}\right)$ s.t.
	the following two functors are naturally isomorphic
	\begin{equation}
	\mathrm{Hom}_{\mathsf{C}} \left( -, C \right) \simeq_{\varphi} H = \mathrm{Nat} \left( \Delta_{(-)}, F \right)
	.\end{equation} 
	In such case $C \simeq \varprojlim F$, and the compatible family is defined as
	\begin{align}
		\varphi_C: \mathrm{Hom}_{\mathsf{C}} \left( C, C \right) &\to \mathrm{Nat} \left( \Delta_C, F \right) \\
		1_C &\mapsto \bar{p} = \left\{ p_i \right\}_{i \in \mathrm{Ob} \left(\mathsf{I}\right)}
	.\end{align} 
\end{prop} 

\section{Colimit}
Let's dualize the notion of limit, to obtain the notion of colimit.
As usual we consider $\mathsf{I}$ a small category, and $F: \mathsf{I} \to \mathsf{C}$ a functor.

Before we introduce the notion of colimit let's dualize that of compatible family
\begin{defn}[Compatible family]
	Fix $X \in \mathrm{Ob} \left(\mathsf{C}\right)$ and
	consider a family $\left\{ \alpha_i \right\}_{i \in \mathrm{Ob} \left(\mathsf{I}\right)}$ of morphisms $\alpha_i: F(i) \to X$.
	This is said to be a \textbf{compatible family} with respect to $F$ iff, given any morphism $\lambda: i \to j$ in $\mathsf{I}$,
	the following triangle commutes
	\begin{equation}
	\begin{tikzcd}
		F(i) \arrow[r, "\alpha_i", rightarrow] \arrow[rd, "F(\lambda)"', rightarrow] &
		X \\
		&
		F(j) \arrow[u, "\alpha_j"', rightarrow] 
	\end{tikzcd}
	.\end{equation} 
	In other words iff $\alpha_i = \alpha_j \circ F(\lambda)$ for every $i, j \in \mathrm{Ob} \left(\mathsf{I}\right)$ and every $\lambda: i \to j$.
\end{defn}

\begin{defn}[Colimit/Injective (inverse) limit]
	A \textbf{colimit} of $F$, denoted with $\varinjlim F$ is a limit of $F$ in $\mathsf{C}^{op}$.
	More explicitly a colimit is an object in $\mathsf{C}$, still denoted with $\varinjlim F$, 
	with morphisms $\mu_i: F(i) \to \varinjlim F$ satisfying the following conditions
	\begin{description}
		\item[CoLIM1] $\left\{ \mu_i \right\}_{i \in \mathrm{Ob} \left(\mathsf{I}\right)}$ is a compatible family of morphisms, i.e.
			\begin{equation}
			\begin{tikzcd}
				F(i) \arrow[r, "\mu_i", rightarrow] \arrow[rd, "F(\lambda)"', rightarrow] &
				\varinjlim F \\
				&
				F(j) \arrow[u, "\mu_j"', rightarrow] 
			\end{tikzcd}
			\end{equation} 
			the above diagram commutes for all $i, j \in \mathrm{Ob} \left(\mathsf{I}\right)$ and all $\lambda: i \to j$.
		\item[CoLIM2] For any $X \in \mathrm{Ob} \left(\mathsf{C}\right)$ and any compatible family of morphisms
			$\left\{ \alpha_i \right\}_{i \in \mathrm{Ob} \left(\mathsf{I}\right)}$, 
			with $\alpha_i: F(i) \to X$, 
			$\exists\, !\, \alpha: \varinjlim F \to X$ s.t. 
			$\alpha \circ \mu_i = \alpha_i$ $\,\forall\, i \in \mathrm{Ob} \left(\mathsf{I}\right)$, i.e.
			\begin{equation}
			\begin{tikzcd}
				F(i) \arrow[r, "\alpha_i", rightarrow] \arrow[d, "\mu_i"', rightarrow] &
				X\\
				\varinjlim F \arrow[ru, "\alpha"', rightarrow]  
			\end{tikzcd}
			.\end{equation} 
	\end{description} 
\end{defn}

\begin{rem}
	As always, since it is defined through a universal property, if $\left(\varinjlim F, \mu_i \right)$ exists,
	it is unique up to a unique isomorphism.
\end{rem}

\begin{ex}
	Let $\mathsf{I}$ be a small discrete category, i.e. the morphisms in $\mathsf{I}$ are only the identities.
	Then, for any functor $F: \mathsf{I} \to \mathsf{C}$, $\varinjlim F$ exists iff
	$\coprod_{i \in \mathrm{Ob} \left(\mathsf{I}\right)} F(i)$ exists and they are isomorphic.
	In particular the $\mu_i$ s correspond with the embeddings of the coproduct.
\end{ex} 

\begin{ex}
	Consider, in an arbitrary category $\mathsf{C}$, the following diagram
	\begin{equation}
	\begin{tikzcd}
		A_3 \arrow[r, "f_1", rightarrow] \arrow[d, "f_2"', rightarrow] &
		A_1\\
		A_2
	\end{tikzcd}
	.\end{equation} 
	Consider the small category $\mathsf{I}$, with $\mathrm{Ob} \left(\mathsf{I}\right) := \left\{ 1, 2, 3 \right\}$ and morphisms, other than the identities, $3 \to 1$ and $3 \to 2$.
	Consider the functor $F: \mathsf{I} \to \mathsf{C}$ defined as follows:
	\begin{equation}
		F(i) = A_i, \qquad F( 3 \to 1 ) = f_1, \qquad
		F( 3 \to 2 ) = f_2
	.\end{equation} 
	Then any colimit of $F$ corresponds with a pushout of the above diagram.
\end{ex} 

\begin{defn}[Cocomplete category]
	A category $\mathsf{C}$ is called \textbf{cocomplete} iff every functor $F: \mathsf{I} \to \mathsf{C}$, from a small category $\mathsf{I}$, admits colimit in $\mathsf{C}$.
\end{defn}

\begin{prop}\label{prop:ColimConstr}
	If a (preadditive) category $\mathsf{C}$ admits cokernels and (infinite) coproducts, then
	for every functor $F: \mathsf{I} \to \mathsf{C}$, from a small category $\mathsf{I}$, $\mathsf{C}$ admits $\varinjlim F$.
	Moreover the colimit is constructed by cokernels and (infinite) coproducts.
\end{prop} 
\begin{proof}
	As above, the proof wants to show that the following construction asctually is a direct limit for $F$.
	Consider $\left(\coprod_{i \in \mathrm{Ob} \left(\mathsf{I}\right)} F(i), \epsilon_i\right)$ the coproduct of $F(i)$.
	Define
	\begin{equation}
		\psi_\lambda := \epsilon_{t(\lambda)} \circ F(\lambda) - \epsilon_{s(\lambda)}:
		F \left( s(\lambda)  \right) \to \coprod_{i \in \mathrm{Ob} \left(\mathsf{I}\right)} F(i)
	.\end{equation} 
	Then the family $\psi_\lambda$ induces a unique 
	\begin{equation}
		\psi: \coprod_{\lambda \in \Lambda} F \left( s(\lambda) \right) \to \coprod_{i \in \mathrm{Ob} \left(\mathsf{I}\right)} F(i)
	\end{equation} 
	s.t. $\psi \circ \epsilon_{s(\lambda)} = \psi_\lambda$.
	Moreover, we recall that $\Lambda := \mathrm{Morph}\, \mathsf{I}$.
	Then, denoted by $\left(C, p\right)$ a cokernel of $\psi$, $\left(C, \mu_i\right)$, where $\mu_i := p \circ\epsilon_i$, is an injective limit of $F$.
\end{proof}

Fix a small category $\mathsf{I}$ and let $\mathsf{C}$ be a cocomplete category.
\begin{prop}
	\begin{align}
		\varinjlim: \mathsf{C}^{\mathsf{I}} &\to \mathsf{C} \\
		F &\mapsto \varinjlim F
	\end{align} 
	is a functor.
\end{prop} 
\begin{proof}
	It is clear how the functor acts on objects.
	Let's define how it acts on $\eta: F \to G$ a natural transformation between the functors $F, G \in \mathsf{C}^{\mathsf{I}}$.
	It associates to $\eta$ a morphism in the natural way
	\begin{equation}
	\varinjlim \eta: \varinjlim F \to \varinjlim G
	.\end{equation} 
\end{proof}

\begin{prop}
	Given a functor $F \in \mathsf{C}^{\mathsf{I}}$, then $\varinjlim F$ exists iff the functor
	\begin{align}
		H: \mathsf{C} &\to \mathsf{Sets} \\
		Y &\mapsto \mathrm{Nat} \left( F, \Delta_Y \right)
	\end{align} 
	is corepresentable.
	In other words iff $\exists\, C \in \mathrm{Ob} \left(\mathsf{C}\right)$ s.t.
	the following two functors are naturally isomorphic
	\begin{equation}
		\mathrm{Hom}_{\mathsf{C}} \left( C, - \right) \simeq_{\varphi} H = \mathrm{Nat} \left( F, \Delta_{(-)} \right)
	.\end{equation} 
	In such case $C \simeq \varinjlim F$, and the compatible family is defined as
	\begin{align}
		\varphi_C: \mathrm{Hom}_{\mathsf{C}} \left( C, C \right) &\to \mathrm{Nat} \left( F, \Delta_C \right) \\
		1_C &\mapsto \bar{\mu} = \left\{ \mu_i \right\}_{i \in \mathrm{Ob} \left(\mathsf{I}\right)} 
	.\end{align} 
\end{prop} 

Let's now describe a particular case of colimits:
\begin{ex}
	Let $\mathsf{C} := \mathsf{Mod}\text{-}R$ and $\left( \mathsf{I}, \leq \right)$ a partially ordered set, viewed as a category.
	Consider a functor
	\begin{equation}
	F: \mathsf{I} \to \mathsf{Mod}\text{-}R
	.\end{equation} 
	This is equivalent to the data of $F(i) =: M_i \in \mathsf{Mod}\text{-}R$ and, for all $i \leq j$, of
	\begin{equation}
		F(i \to j) =: f_{ji}: M_i \to M_j
	.\end{equation} 
	Now, given $i \leq j \leq k$, then the following diagram commutes
	\begin{equation}
	\begin{tikzcd}
		M_i \arrow[r, "f_{ji}", rightarrow] \arrow[rd, "f_{ki}"', rightarrow] &
		M_j \arrow[d, "f_{kj}", rightarrow] \\
		&
		M_k
	\end{tikzcd}
	,\end{equation} 
	in other words $f_{kj} \circ f_{ji} = f_{ki}$. Moreover we require $f_{ii} = id_{M_i}$.

	We have, in fact, a correspondance, between functors from partially ordered sets and
	direct systems of modules, which are families $\left\{ M_i, f_{ij} \right\}_{i \leq j}$ of modules $M_i$ and morphism $f_{ij}$ between them, satisfying the above compatibility conditions.

	Then, $\varinjlim F$ is called the \textbf{direct limit} of $\left\{ M_i, f_{ij} \right\}_{i \leq j}$.
	The morphisms $f_{ij}$ are called the structural morphisms of the direct system.
	Sometimes this is also denoted with $\varinjlim M_i$.
	
	Let's describe $\varinjlim M_i$ explicitly: for every $i \leq j$ we have the (not necessairily commutative) diagram
	\begin{equation}
	\begin{tikzcd}
		M_i \arrow[r, "\epsilon_j", rightarrow] \arrow[d, "f_{ji}"', rightarrow] &
		\bigoplus_{i \in \mathrm{Ob} \left(\mathsf{I}\right)} M_i \\
		M_j \arrow[ru, "\epsilon_j"', rightarrow] & 
	\end{tikzcd} 
	.\end{equation}
	Let's define, for each $i \leq j$, $\psi_{ij} := \epsilon_j \circ f_{ji} - \epsilon_i$.
	Then, by universal property of the coproduct,
	$\exists\, !\, \psi: \bigoplus_{i \leq j} M_{ij} \to \bigoplus_{i \in \mathrm{Ob} \left(\mathsf{I}\right)} M_{i}$,
	where $M_{ij} := M_i$ for every $i \leq j$.
	In the above construction $\psi \circ \epsilon_i = \psi_{ij}: M_{ij} \to \bigoplus_{i \in \mathrm{Ob} \left(\mathsf{I}\right)} M_i$.
	Then we have
	\begin{align}
		\varinjlim M_i &\simeq \coker \psi =
		\frac{\bigoplus_{i \in \mathrm{Ob} \left(\mathsf{I}\right)} M_i}{\Ima \psi}
	,\end{align}
	where $\Ima \psi$ is generated by
	\begin{equation}
		\left\{ \epsilon_j \circ f_{ji}(x_i) - \epsilon_i(x_i) \ \middle|\ 
		x_i \in M_i,\ i \leq j \right\} \subset
		\bigoplus_{i \in \mathrm{Ob} \left(\mathsf{I}\right)} M_i
	.\end{equation} 
	In particualr the generators of $\Ima \psi$ are of the form
	\begin{equation}
		\left( \ldots, 0, \ldots, 
		0, x_i, 0, \ldots, 0, - f_{ji}(x_i),
		0, \ldots, 0, \ldots \right)
	.\end{equation} 	
	It is a submodule of the product, in which, $x_i$ in position $i$ and $f_{ji}(x_i)$ in position $j$ are identified.
\end{ex}

\begin{defn}[Directed poset]
	A poset $\left(I, \leq \right)$ is said \textbf{directed} (or \textbf{filtered}) iff
	\begin{equation}
	\,\forall\,  i, j \in I \ \exists\, k \in I \text{ s.t. } i \leq k \text{ and } i \leq k
	.\end{equation} 
\end{defn}
Morevoer, if $\left(M_i, f_{ji}\right)_{i \leq j}$, for $I$ filtered, then $\varinjlim M_i$ is called
directed (or filtered) limit.
In general it is easier to describe a colimit on a directed poset.

Before giving an example of one such limit, let's recall a definition:
\begin{defn}[Finitely generated module]
	A module $M_R$ is \textbf{finitely generated} iff there is an epimorphism
	$\phi: R^N := \bigoplus_{i=1}^N R \to M$, for some $N \in \N$.
	If we denote by $e_i$ the generators of $R^N$, then $\phi(e_i) = x_i$ are the generators of $M$.
	In other words we are saying that $\exists\, \left\{ x_1, \ldots, x_N \right\} \subset M$ a finite set of generators 
	s.t.
	\begin{equation}
	\,\forall\,  x \in R, \text{ then } x = \sum_{i=1}^{N} x_i r_i
	.\end{equation} 
\end{defn}

\begin{defn}[Finitely presented module]
	For a finitely generated module, with epimorphism $\phi: R^N \to M$,
	we denote by $K := \ker \phi$, the module of relations of $M$:
	\begin{equation}
	K = 
	\left\{ \left( r_1, \ldots, r_N \right) \in R^N \ \middle|\ \sum_{i=1}^{N} x_i r_i = 0 \right\}
	.\end{equation} 
	We say that $K$ is the module of relations of $M$ (also known as the first syzygy module).
	We say that $M$ is finitely presented if, being finitely generated, has also finitely generated first syzygy module.
\end{defn}

\begin{ex}
	Let $M \in \mathsf{Mod}\text{-}R$.
	Consider the family
	\begin{equation}
	\mathcal{F} := \left\{ N \leq M \ \middle|\ 
	N \text{ is finitely generated } \right\}
	.\end{equation} 
	Let's label the elements $N \in \mathcal{F}$ as $N= N_i$, for some index $i \in \mathsf{I}$, with $\mathsf{I}$ a set of indeces.
	Let's define a partial order on $\mathsf{I}$:
	$i \leq j$ iff $N_i \subset N_j$.
	Moreover, if $i, j \in \mathsf{I}$, then $N_i + N_j$ is finitely generated, hence $\exists\, k \in \mathsf{I}$ s.t.
	$N_i + N_j = N_k$, for some $k \in \mathsf{I}$.
	This makes $\left( \mathsf{I}, \leq \right)$ a filtered poset.
	We then label the inclusions as $\epsilon_{ji}: N_i \to N_j$ and $\epsilon_i: N_i \to M$.
	Clearly this makes $\left(N_i, \epsilon_{ji}\right)_{i \leq j}$ into a direct system.
\end{ex} 

\begin{prop}
	Every $R\text{-}\mathsf{Mod}$ $M$ is a directed limit of its finitely generated submodules.
	More explicitly, in the above notation,
	\begin{equation}
	\left(M, \epsilon_i\right) \simeq \varinjlim N_i
	.\end{equation} 
\end{prop} 

\begin{ex}[Prüfer group]
	An example of a direct limit construction in $\mathsf{C} = \mathsf{Ab}$.
	Let $M_n := \mathbb{Z}/p^n\mathbb{Z} = \left\langle c_n \right\rangle$, for $n \in \N$ and $p \in \N$ a prime number.
	Notice that $c_n$ ha order $p^n$, hence $p^n c_n = 0$.
	We define the structural morphisms of the direct system as
	\begin{align}
		f_{n+1, n}: \mathbb{Z}/p^n\mathbb{Z} &\to \mathbb{Z}/p^{n+1}\mathbb{Z} \\
		c_n &\mapsto p \cdot c_{n+1}
	\end{align} 
	extending it by linearity.
	Moreover, composing consecutive maps, we obtain
	\begin{align}
		f_{m, n}: \mathbb{Z}/p^n\mathbb{Z} &\to \mathbb{Z}/p^m\mathbb{Z} \\
		c_n &\mapsto p^{m-n} c_m
	\end{align} 
	and extending also this by linearity.
	Clearly $\left\{ \mathbb{Z}/p^n\mathbb{Z}, f_{m,n} \right\}_{n \leq m}$ is a direct system 
	(compatibility follows from the definition of $f_{m,n}$).
	We can consider the direct limit, denoted as follows, and called Prüfer group
	\begin{equation}
		\varinjlim \mathbb{Z}/p^n\mathbb{Z} = \Z (p^{\infty}) \simeq \bigcup_{n \in \N} \left\langle c_n \right\rangle
	,\end{equation} 
	where, in the last union, we consider the map $f_{n+1, n}$ as the inclusion of $\left\langle c_n \right\rangle$ in
	$\left\langle c_{n+1} \right\rangle$.
	Carrying out the construction described in the proposition we obtain that
	\begin{equation}
	\varinjlim \left\langle c_n \right\rangle \simeq
	\frac{\bigoplus_{n \in \N} \left\langle c_n \right\rangle}{\left\langle (c_n, -p \cdot c_{n+1}) \ \middle|\ n \in \N \right\rangle}
	.\end{equation} 
\end{ex} 

\subsection{Direct limit of modules}
\begin{lem}
	Let $\mathsf{C} = \mathsf{Mod}\text{-}R$ and $\left(\mathsf{I}, \leq\right)$ be a filtered poset.
	Let $\left\{ M_i, f_{ji} \right\}_{i \leq j}$ be a directed system of modules.
	In the notation of proposition \ref{prop:ColimConstr}, the direct limit $\left(\varinjlim M_i, \mu_i\right)$, has the
	compatible family of maps
	\begin{equation}
	\begin{tikzcd}
		M_i \arrow[r, "\mu_i", rightarrow] \arrow[d, "\epsilon_i", rightarrow] &
		\varinjlim M_i \\
		\bigoplus_{i \in \mathsf{I}} M_i \arrow[ru, "p"', rightarrow] &
	\end{tikzcd}
	.\end{equation} 
	Where $\mu_i := p \circ\epsilon_i$ and $p = \coker \psi$.
	If we denote with $D := \Ima \psi$, then
	every element $x \in \varinjlim M_i = (\bigoplus M_i)/D$ can be written as
	$\mu_i(x_i)$, for some $i \in \mathsf{I}$ and $x_i \in M_i$.

	(Then we can interpet $\varinjlim M_i = \sum_{i \in \mathsf{I}}^{} \mu_i (M_i)$).
\end{lem} 
\begin{proof}
	The idea is simply the fact that $\mathsf{I}$ is filtered 
	(hence for any finite set of indices we can find an index which is bigger than all of them).
	Given this one can easily use the relations to express any finite sum in terms of an element from a single $M_k$.
\end{proof}

\begin{lem}
	In the above notation and hypothesis, let $x = x_{i_1} + \ldots + x_{i_n} \in \bigoplus_{i \in \mathsf{I}} M_i$.
	$x \in D$ iff $\exists\, k \in \mathsf{I}$, $k \geq i_1, \ldots, i_n$ s.t.
	\begin{equation}
		f_{k, i_1}(x_{i_1}) + \ldots f_{k, i_n}(x_{i_n}) = 0 \in M_k
	.\end{equation} 
\end{lem} 

\begin{lem}
	In the above notation and hypothesis, let $x_i \in M_i$.
	Then 
	\begin{equation}
		\mu_i(x_i) = 0 \in \varinjlim M_i \iff \exists\, j \geq i \in \mathsf{I} \text{ s.t. } f_{ji}(x_i) = 0
	.\end{equation} 
\end{lem} 

\begin{prop}
	Let $M_R \in \mathsf{Mod}\text{-}R$, then $M_R$ is a direct limit of finitely presented modules.
\end{prop} 

\section{Exactness}
\subsection{Subobjects and quotients}
\begin{defn}[Subobject]
	Let $A$ be an object of a category (abelian) $\mathsf{C}$.
	Consider two monomorphism $f: B \to A$ and $g: C \to A$.
	We say that $f \sim g$ iff 
	$\exists\, \alpha: B \to C$ an isomorphism s.t. the following diagram commutes
	\begin{equation}
	\begin{tikzcd}
		B \arrow[r, "f", rightarrow] \arrow[rd, "\alpha"', rightarrow] &
		A\\
		&
		C \arrow[u, "g"', rightarrow] 
	\end{tikzcd}
	\end{equation} 
	i.e. s.t. $g \circ \alpha = f$.
	Clearly this is an equivalence relation.
	An equivalence class of monomorphisms ending in $A$ is called a \textbf{subobject} of $A$.
	Chosen a representative $f: B \to A$ we denote the corresponding subobject by $B \subseteq A$.

	Moreover, given $B_1$ and $B_2$ subobjects of $A$, we say that $B_1 \subseteq B_2$, $B_1$ is a subobject of $B_2$, iff
	$\exists\, \alpha: B_1 \to B_2$ a morphism s.t. the following diagram commutes
	\begin{equation}
	\begin{tikzcd}
		B_1 \arrow[r, "f_1", rightarrow] \arrow[rd, "\alpha"', rightarrow] &
		A\\
		&
		B_2 \arrow[u, "f_2"', rightarrow] 
	\end{tikzcd}
	\end{equation} 
	i.e. s.t. $f_2 \circ \alpha = f_1$.
	Notice that, in this case, $\alpha$ has to be mono.
\end{defn}

\begin{rem}
	If $B_1 \subseteq B_2 \subseteq A$ and $B_2 \subseteq B_1$, then $B_1$ and $B_2$ represent the same subobject of $A$.
\end{rem}

Let's now give the dual definition.
\begin{defn}[Quotient]
	Consider $f: A \to B$ and $g: A \to C$ two epimorphisms.
	We say that $f \sim g$ iff $\exists\, \alpha: B \to C$ an isomorphism s.t. the following diagram commutes
	\begin{equation}
	\begin{tikzcd}
		A \arrow[r, "f", rightarrow] \arrow[rd, "g"', rightarrow] &
		B \arrow[d, "\alpha", rightarrow] \\
		& C
	\end{tikzcd}
	\end{equation} 
	i.e. s.t. $\alpha \circ f = g$.
	Given one such morphism $f: A \to B$ we call the equivalence class a \textbf{quotient} of $A$.
\end{defn}
\begin{rem}[notation]
	Assume that $f: B \to A$ is a subobject of $A$ (i.e. $f$ is a mono).
	We write $A/B$ for the quotient object represented by $\coker f$.
\end{rem}

\begin{lem}
	Let $\mathsf{C}$ be an abelian category.
	Consider two composable morphisms $A \xrightarrow{f} B \xrightarrow{g} C$.
	Then $g \circ f = 0$ iff $\Ima f \subseteq \ker g$, viewed as subobjects of $B$.
\end{lem} 

\begin{lem}
	Let $\mathsf{C}$ be an abelian category.
	Consider two composable morphisms $A \xrightarrow{f} B \xrightarrow{g} C$.
	Then $\ker g \subseteq \Ima f$, viewed as subobjects of $B$, iff $\,\forall\, h: D \to B$ s.t.
	$g \circ h = 0$, $\exists\, !\, h': D \to \Ima f$ s.t. $\mu \circ h' = h$, where $\Ima f \xrightarrow{\mu} B$ is the natural morphism.
	In other words s.t. the following diagram commutes
	\begin{equation}
	\begin{tikzcd}
		A \arrow[r, "f", rightarrow] \arrow[d, "\beta"', rightarrow] &
		B\\
		\Ima f \arrow[ur, "\mu"', rightarrow] &
		D \arrow[u, "h"', rightarrow] \arrow[l, "\exists\, !\, h'", dashrightarrow] 
	\end{tikzcd}
	.\end{equation} 
\end{lem} 

\begin{defn}[Exact sequence]
	Let $\mathsf{C}$ be an abelian category.
	Consider a sequence of composable morphisms in $\mathsf{C}$
	\begin{equation}
	\ldots \to A_n \xrightarrow{f_n} 
	A_{n+1}  \xrightarrow{f_{n + 1}} 
	A_{n+2} \xrightarrow{f_{n + 2}} \ldots
	.\end{equation} 
	The sequence is \textbf{exact} at $n$ iff
	$\Ima f_n = \ker f_{n+1}$ as subobjects of $A_{n + 1}$.
	It is said to be \textbf{exact} iff it is exact at $n$ for every $n$.
\end{defn}

\begin{defn}[Short exact sequence]
	An \textbf{exact} sequence of the form
	\begin{equation}
	0 \to A_1 \xrightarrow{f_1} A_2 \xrightarrow{f_2} A_3 \to 0 
	\end{equation} 
	is called \textbf{short exact sequence}, abbreviated as s.e.q.
	In particular this sequence is exact iff
	$f_1$ is a mono, $f_2$ is an epi, and $\Ima f_1 = \ker f_2$.
\end{defn}

\begin{lem}
	Consider the following exact sequence
	\begin{equation}
	0 \to A \xrightarrow{f} B \xrightarrow{g} C
	.\end{equation} 
	Then $f = \ker g$.
\end{lem} 

\begin{lem}
	Consider the following exact sequence
	\begin{equation}
	A \xrightarrow{f} B \xrightarrow{g} C \to 0
	.\end{equation} 
	Then $g = \coker f$.
\end{lem} 
Let's combine the above lemmas

\begin{prop}
	Consider the following sequence
	\begin{equation}
	0 \to A \xrightarrow{f} B \xrightarrow{g} C \to 0
	.\end{equation} 
	This is exact (i.e. a s.e.q.) iff $f = \ker g$ and $g = \coker f$.
\end{prop} 

\subsection{Functors}
In this section we'll always work with abelian categories $\mathsf{C}$ and $\mathsf{D}$.

\begin{defn}[Exact functor]
	Let $F: \mathsf{C} \to \mathsf{D}$ be an additive functor.
	We say that $F$ is \textbf{exact} iff, for every exact sequence
	\begin{equation}
	A \xrightarrow{f} B \xrightarrow{g} C \quad \text{ in } \mathsf{C}
	,\end{equation} 
	then the image sequence is exact in $\mathsf{D}$ 
	\begin{equation}
	F(A) \xrightarrow{F(f)} F(B) \xrightarrow{F(g)} F(C)
	.\end{equation} 
	Equivalently $F$ is exact if given $\Ima f = \ker g$ in $\mathsf{C}$, then
	$\Ima F(f) = \ker F(g)$ in $\mathsf{D}$.
\end{defn}

\begin{defn}[Left (resp. right) exact functor]
	An additive functor $F: \mathsf{C} \to \mathsf{D}$ is \textbf{left} (resp. \textbf{right}) exact iff,
	for every exact sequence in $\mathsf{C}$
	\begin{equation}
		0 \to A \xrightarrow{f} B \xrightarrow{g} C \quad ( \text{resp. }
		A \xrightarrow{f} B \xrightarrow{g} C \to 0\, )
	,\end{equation} 
	then the image sequence is exact in $\mathsf{D}$ 
	\begin{equation}
		0 \to F(A) \xrightarrow{F(f)} F(B) \xrightarrow{F(g)} F(C) \quad ( \text{resp. }
		F(A) \xrightarrow{F(f)} F(B) \xrightarrow{F(g)} F(C) \to 0\, )
	.\end{equation} 
\end{defn}

\begin{prop}
	An additive functor $F: \mathsf{C} \to \mathsf{D}$ between abelian categories,
	is exact iff it is both left and right exact.
\end{prop} 

\begin{defn}[Split exact sequence]
	A short exact sequence in $\mathsf{C}$ (as usual an abelian category)
	\begin{equation}
	0 \to A \xrightarrow{f} B \xrightarrow{g} C \to 0 
	\end{equation} 
	is said \textbf{split exact} iff $\exists\, \alpha: B \to A \oplus C$ s.t. the following diagram is commutative
	\begin{equation}
	\begin{tikzcd}
		0 \arrow[r, "", rightarrow] &
		A \arrow[r, "f", rightarrow] \arrow[d, "", equal] &
		B \arrow[r, "g", rightarrow] \arrow[d, "\alpha", rightarrow] &
		C \arrow[r, "", rightarrow] \arrow[d, "", equal] &
		0\\
		0 \arrow[r, "", rightarrow] &
		A \arrow[r, "\epsilon_A", rightarrow] &
		A \oplus C \arrow[r, "\pi_C", rightarrow] &
		C \arrow[r, "", rightarrow] &
		0
	\end{tikzcd}
	.\end{equation} 
	Recall that, in matrix notation, the embedding and projection can be written as
	 \begin{equation}
	\epsilon_A = 
	\begin{bmatrix}
		1_A\\ 0
	\end{bmatrix} \qquad \text{ and } \qquad
	\pi_C = 
	\begin{bmatrix}
		0 & 1_c
	\end{bmatrix} 
	.\end{equation} 
\end{defn}

\begin{prop}
	Let $\mathsf{C}$ be an abelian category.
	TFAE
	\begin{enumerate}
		\item The sequence $0 \to A \xrightarrow{f} B \xrightarrow{g} C \to 0$ is split exact,
		\item $\exists\, f': B \to A$ s.t. $f' \circ f = 1_A$,
		\item $\exists\, g': C \to B$ s.t. $g \circ g' = 1_B$.
	\end{enumerate}
	In such a case $f'$ is called a section of $f$, and $g'$ a retraction of $g$.
\end{prop} 

\subsubsection{Some examples}
Recall that, given a category $\mathsf{C}$, we have the natural bifunctor
\begin{equation}
F = \mathrm{Hom}_{\mathsf{C}} \left( - , - \right): \mathsf{C}^{op} \cross \mathsf{C} \to \mathsf{Sets}
.\end{equation} 
Clearly, if $\mathsf{C}$ is preadditive, $F$ is an additive functor. Moreover

\begin{prop}
	Let $\mathsf{C}$ be an abelian category, then
	\begin{equation}
	\mathrm{Hom}_{\mathsf{C}} \left( -, - \right): \mathsf{C}^{op} \cross \mathsf{C} \to \mathsf{Ab}
	\end{equation} 
	is left exact in both variables.
\end{prop} 

\begin{rem}
	Recall that a contravariant functor $F: \mathsf{C} \to \mathsf{D}$, i.e. a covariant functor $F: \mathsf{C}^{op} \to \mathsf{D}$, 
	is left exact iff given any exact sequence in $\mathsf{C}$ 
	\begin{equation}
	A \to B \to C \to 0
	,\end{equation} 
	i.e. $0 \to C \to B \to A$ exact in $\mathsf{C}^{op}$, then the image sequence is exact in $\mathsf{D}$
	\begin{equation}
		0 \to F(A) \to F(B) \to F(C)
	.\end{equation} 
\end{rem}
\begin{rem}
Consider $\mathsf{C} = \mathsf{Mod}\text{-}R$ and $\left(\mathsf{I}, \leq \right)$ a filtered poset, then
the functors $F: \mathsf{I} \to \mathsf{Mod}\text{-}R$ are in correspondance with the direct systems of modules
$\left\{ M_i, f_{ji} \right\}_{i \leq j }$.
Consider the functors $F, G, L \in \mathsf{C}^{\mathsf{I}}$ and their corresponding
direct systems $\left\{ M_i, f_{ji} \right\}_{i \leq j}$, $\left\{ N_i, g_{ji} \right\}_{i \leq j}$ and $\left\{ L_i, l_{ji} \right\}_{i \leq j}$.
Then the the sequence 
\begin{equation}
0 \to F \xrightarrow{\eta} G \xrightarrow{\zeta} L \to 0
\end{equation} 
is exact iff the following diagram is commutative and has exact rows
\begin{equation}
\begin{tikzcd}
	0 \arrow[r, "", rightarrow] &
	M_i \arrow[r, "\eta_i", rightarrow] \arrow[d, "f_{ji}", rightarrow] &
	N_i \arrow[r, "\zeta_i", rightarrow] \arrow[d, "g_{ji}", rightarrow] &
	L_i \arrow[r, "", rightarrow] \arrow[d, "l_{ji}", rightarrow] &
	0 \\
	0 \arrow[r, "", rightarrow] &
	M_i \arrow[r, "\eta_j", rightarrow] &
	N_i \arrow[r, "\zeta_j", rightarrow] &
	L_i \arrow[r, "", rightarrow] &
	0
\end{tikzcd}
,\end{equation} 
for each $i \leq j$ in $\mathsf{I}$.
\end{rem} 

\begin{prop}
	Let $\mathsf{C} = \mathsf{Mod}\text{-}R$ and $\left(\mathsf{I}, \leq \right)$ be a filtered poset.
	Then the functor $\varinjlim: \mathsf{Mod}\text{-}R^{\mathsf{I}} \to \mathsf{Mod}\text{-}R$ is exact.
\end{prop} 

\begin{rem}
	Colimits, in general, are not exact, even in $\mathsf{Mod}\text{-}\Z = \mathsf{Ab}$.
	Consider, in fact, the category $\mathsf{I}$, characterized by
	$\mathrm{Ob} \left(\mathsf{I}\right) := \left\{ 1, 2, 3 \right\}$ and nontrivial arrows
	$1 \to 2$ and $1 \to 2$.
	Consider $F, G, H \in \mathsf{Ab}^{\mathsf{I}}$, defined as follows:
	\begin{equation}
	F : \ \, 
	\begin{tikzcd}
		\Z \arrow[r, "\dot{4}", rightarrow] \arrow[d, "0"', rightarrow] & \Z \\
		\Z &
	\end{tikzcd}\qquad
	G : \ \,
	\begin{tikzcd}
		\Z \arrow[r, "\dot{4}", rightarrow] \arrow[d, "0"', rightarrow] & \Z \\
		\Z &
	\end{tikzcd}\qquad
	H : \ \,
	\begin{tikzcd}
		\mathbb{Z}/2\mathbb{Z} \arrow[r, "0", rightarrow] \arrow[d, "0"', rightarrow] & \mathbb{Z}/2\mathbb{Z} \\
		\mathbb{Z}/2\mathbb{Z} &
	\end{tikzcd}
	.\end{equation} 
	Then $\varinjlim F = \coker \dot{4} = \varinjlim G$ and $\varinjlim H \simeq \mathbb{Z}/2\mathbb{Z}$.
	Consider the natural transofrmation $\dot{2}$ and $\pi$, that give rise to the sequence
	\begin{equation}
	\begin{tikzcd}
		0 \arrow[r, "", rightarrow] &
		F \arrow[r, "\dot{2}", rightarrow] &
		G \arrow[r, "\pi", rightarrow] &
		H \arrow[r, "", rightarrow] &
		0
	\end{tikzcd}
	,\end{equation} 
	which is exact in $\mathsf{Mod}\text{-}\Z^{\mathsf{I}}$.
	In fact this corresponds to
	\begin{equation}
	\begin{tikzcd}
		0 \arrow[r, "", rightarrow] &
		\Z \arrow[r, "\dot{2}", rightarrow] \arrow[d, "\dot{4}", rightarrow] &
		\Z \arrow[r, "\pi", rightarrow] \arrow[d, "\dot{4}", rightarrow] &
		\mathbb{Z}/2\mathbb{Z} \arrow[r, "", rightarrow] \arrow[d, "0", rightarrow] &
		0 \\
		0 \arrow[r, "", rightarrow] &
		\Z \arrow[r, "\dot{2}", rightarrow]&
		\Z \arrow[r, "\pi", rightarrow] &
		\mathbb{Z}/2\mathbb{Z} \arrow[r, "", rightarrow] &
		0
	\end{tikzcd}
	\end{equation} 
	\begin{equation}
	\begin{tikzcd}
		0 \arrow[r, "", rightarrow] &
		\Z \arrow[r, "\dot{2}", rightarrow] \arrow[d, "0", rightarrow] &
		\Z \arrow[r, "\pi", rightarrow] \arrow[d, "0", rightarrow] &
		\mathbb{Z}/2\mathbb{Z} \arrow[r, "", rightarrow] \arrow[d, "0", rightarrow] &
		0 \\
		0 \arrow[r, "", rightarrow] &
		\Z \arrow[r, "\dot{2}", rightarrow]&
		\Z \arrow[r, "\pi", rightarrow] &
		\mathbb{Z}/2\mathbb{Z} \arrow[r, "", rightarrow] &
		0
	\end{tikzcd}
	.\end{equation}
	And both are commutative with exact rows.
	Taking the image by $\varinjlim$ we obtain
	\begin{equation}
	\begin{tikzcd}
		0 \arrow[r, "", rightarrow] &
		\varinjlim F \simeq \mathbb{Z}/4\mathbb{Z} \arrow[r, "\dot{2}", rightarrow] &
		\varinjlim G \simeq \mathbb{Z}/4\mathbb{Z} \arrow[r, "\pi", rightarrow] &
		\varinjlim H \simeq \mathbb{Z}/2\mathbb{Z} \arrow[r, "", rightarrow] &
		0
	\end{tikzcd}
	\end{equation} 
	which is not exact, since $\dot{2}: \mathbb{Z}/4\mathbb{Z} \to \mathbb{Z}/4\mathbb{Z}$ is not injective.
\end{rem}

\begin{prop}
	Let $C := \mathsf{Mod}\text{-}R$ and $\left(\mathsf{I}, \leq\right)$ be a filtered poset.
	Then the functor $\varprojlim: \mathsf{Mod}\text{-}R^{\mathsf{I}} \to \mathsf{Mod}\text{-}R$ is left exact.
\end{prop} 

\begin{ex}
	In general, even in the category $\mathsf{Mod}\text{-}\Z = \mathsf{Ab}$, the functor $\varprojlim$ is not exact.
	It is enough to construct an epimorphism
	\begin{equation}
		\left\{ M_i, f_{ji} \right\}_{i \leq j} \xrightarrow{\zeta} \left\{ N_i, g_{ji} \right\}_{i \leq j} \to 0 
	\end{equation} 
	s.t. the induced $\varprojlim \zeta: \varprojlim M_i \to \varprojlim N_i$  is not epi.

	Let $\mathsf{I} = \N$, with the usual order.
	Let $M_n = \Z$ for every $n$, with structural morphisms $M_m \xrightarrow{3^{m-n}} M_n$, 
	that acts as $x \mapsto x \cdot 3^{m-n}$, for all $m \leq n$, for all $x \in \Z$.
	Let $N_n = \mathbb{Z}/2\mathbb{Z}$ for every $n$, with structural morphisms $id: \mathbb{Z}/2\mathbb{Z} \to \mathbb{Z}/2\mathbb{Z}$.
	We define
	\begin{equation}
	\left\{ M_n, 3^{m-n} \right\}_{ m \leq n} \xrightarrow{\pi} \left\{ \mathbb{Z}/2\mathbb{Z}, id \right\}_{ m \leq n}
	,\end{equation} 
	defined for all $n$ as the canonical projection.
	It clearly is both surjective for all $n$, and (as can be easily checked) it is a natural transformation, hence it is an epi in the category of functors.
	Notice that, given $\left( x_n \right)_{n \in \N} \in \varprojlim M_n$, then $x_1 = 3 \cdot x_2 = \ldots = 3^n x_{n+1}$,
	hence $x_1 \in \bigcap_{n \in \N} 3^n \Z = \emptyset$.
	In other words $\varprojlim M_n = \emptyset$.
	Instead, clearly, $\varprojlim N_n = \mathbb{Z}/2\mathbb{Z}$.
	Then, obviously, $\varprojlim \pi$ cannot be surjective.
\end{ex} 

\section{Injective and projective objects}
Let $\mathsf{C}$ be an \textit{arbitrary} category.

\begin{defn}[Projective object]
	Let $P \in \mathrm{Ob} \left(\mathsf{C}\right)$.
	$P$ is \textbf{projective} iff given any $\varphi: B \to C$ epimorphism in $\mathsf{C}$, 
	and any morphism $f: P \to C$, then there exists $g: P \to B$ s.t. $\varphi \circ g = f$, 
	i.e. s.t. the following diagram commutes
	\begin{equation}
	\begin{tikzcd}
		B \arrow[r, "\varphi", rightarrow] &
		C \arrow[r, "", rightarrow] &
		0 \\
		&
		P \arrow[lu, "\exists\, g", dashrightarrow] \arrow[u, "f"', rightarrow]  &
	\end{tikzcd}
	.\end{equation} 
	In such case $g$ is called a \textit{lift} of $f$.\newline
	Equivalently:
	$P$ is surjective iff
	\begin{equation}
	\mathrm{Hom}_{\mathsf{C}} \left( P, B \right) \xrightarrow{\mathrm{Hom}_{\mathsf{C}} \left( P, \varphi \right)} 
	\mathrm{Hom}_{\mathsf{C}} \left( P, C \right)
	\end{equation} 
	is an epimorphism (a surjection in $\mathsf{Sets}$) for every $\varphi$ epi.
\end{defn}

\begin{rem}
	If, moreover, $\mathsf{C}$ is abelian, then
	$P$ is \textbf{projective} iff $\mathrm{Hom}_{\mathsf{C}} \left( P, - \right)$ is exact.
	Hence $P$ is projective iff $\mathrm{Hom}_{\mathsf{C}} \left( P, -\right)$ is also \textbf{right} exact.
\end{rem}

\begin{defn}[Injective object]
	Let $I \in \mathrm{Ob} \left(\mathsf{C}\right)$.
	$I$ is \textbf{injective} iff $I$ is projective in $\mathsf{C}^{op}$.
	More explicitly, iff given any $\mu: A \to B$ mono in $\mathsf{C}$,
	and any morphism $f: A \to I$, then there exists $g: B \to I$ s.t. $g \circ \mu = f$,
	i.e. s.t. the following diagram commutes
	\begin{equation}
	\begin{tikzcd}
		0 \arrow[r, "", rightarrow]  &
		A \arrow[r, "\mu", rightarrow] \arrow[d, "f"', rightarrow] &
		B \arrow[dl, "\exists\, g", dashrightarrow] \\
		& I &
	\end{tikzcd}
	.\end{equation} 
	In such case $g$ is called an \textit{extension} of $f$.\newline
	Equivalently:
	$I$ is injective iff
	\begin{equation}
	\mathrm{Hom}_{\mathsf{C}} \left( B, I \right) \xrightarrow{\mathrm{Hom}_{\mathsf{C}} \left( \mu, I \right)} 
	\mathrm{Hom}_{\mathsf{C}} \left( A, I \right)
	\end{equation} 
	is an epimorphism (a surjection in $\mathsf{Sets}$) for every $\mu$ mono.
\end{defn}

\begin{rem}
	If, moreover, $\mathsf{C}$ is abelian, then
	$I$ is \textbf{injective} iff $\mathrm{Hom}_{\mathsf{C}} \left( -, I \right)$ is exact.
	Hence $I$ is injective iff $\mathrm{Hom}_{\mathsf{C}} \left( -, I\right)$ is also \textbf{right} exact.
\end{rem}

\begin{prop}
	Consider $\left\{ P_i \right\}_{i \in I}$ a family of objects in $\mathsf{C}$ an arbitrary category.
	Assume that $\coprod_{i \in I} P_i$ exists.
	Then $\coprod_{i \in I} P_i$ is a projective object in $\mathsf{C}$ 
	iff $P_i$ is projective $\,\forall\, i \in I$.
\end{prop} 

Dually we have the following

\begin{prop}
	Consider $\left\{ I_\lambda \right\}_{\lambda \in \Lambda}$ a family of objects in $\mathsf{C}$ an arbitrary category.
	Assume that $\prod_{\lambda \in \Lambda} I_\lambda$ exists.
	Then $\prod_{\lambda \in \Lambda} I_\lambda$ is an injective object in $\mathsf{C}$ 
	iff $I_\lambda$ is injective $\,\forall\, \lambda \in \Lambda$.
\end{prop} 

\begin{prop}[Baer's criterion for injectivity]
	Let $\mathsf{C} = \mathsf{Mod}\text{-}R$.
	$E_R \in \mathsf{Mod}\text{-}R$ is an injective module iff
	for any ideal $I_R \triangleleft R$ and every $f: I \to E$, there exists $g: R \to E$ s.t.
	the following diagram commutes
	 \begin{equation}
	\begin{tikzcd}
		I_R \arrow[r, "\mu", rightarrow] \arrow[d, "f"', rightarrow] &
		R \arrow[ld, "\exists\, g", dashrightarrow] \\
		E &
	\end{tikzcd}
	,\end{equation} 
	where $\mu: I_R \to R$ is the inclusion.
	In other words we ask $g \circ \mu = f$.
\end{prop} 

\begin{defn}[Enough projectives/injectives]
	Consider a category $\mathsf{C}$.
	\begin{itemize}
		\item We say that $\mathsf{C}$ has \textbf{enough projectives} iff,
	given any $C \in \mathrm{Ob} \left(\mathsf{C}\right)$, 
	there exists a projective object $P \in \mathrm{Ob} \left(\mathsf{C}\right)$ and an epimorphism $\varphi: P \to C$.
	\item We say that $\mathsf{C}$ has \textbf{enough injectives} iff, 
		given any object $C \in \mathrm{Ob} \left(\mathsf{C}\right)$,
		there exists an injective object $E \in \mathrm{Ob} \left(\mathsf{C}\right)$ and a monomorphism
		$\mu: C \to E$.
	\end{itemize}
\end{defn}

\begin{defn}[Free module]
	Let $\mathsf{C} := \mathsf{Mod}\text{-}R$.
	$M_R \in \mathsf{Mod}\text{-}R$ is \textbf{free} iff it has a free set of generators $\left\{ x_i \right\}_{i \in I}$, 
	with $x_i \in M$ for all $i$, s.t. $\,\forall\, x \in M$ it can be written in a unique way as a linear combination of the generators.
	More explicitly
	\begin{equation}
	x = \sum_{i \in I}^{} x_i r_i \qquad \text{with } r_i \text{ almost all zero}
	.\end{equation} 
	Clearly $M$ is free iff $M = \bigoplus_{i \in I}R_i$, with $R_i \simeq R$ for all $i$.
	In such case it has $\left\{ e_i \right\}_{i \in I}$ as a basis.
	Another notation for $\bigoplus_{i \in I} R_i$ is $R^{(I)}$.
\end{defn}

\begin{rem}
	It is easy to show that any free module is projective.
\end{rem}

\begin{prop}
	Let $\mathsf{C} = \mathsf{Mod}\text{-}R$.
	$P_R \in \mathsf{Mod}\text{-}R$ is projective iff 
	it is a direct summand of a free module.
\end{prop} 

\begin{rem}
	Projective modules are easy to describe.
	For injective ones we are able to do so only for a specific class of rings, for example for PIDs.

	For this purpose, recall that a module $M_R$ is \textbf{divisible} iff
	\begin{equation}
	\,\forall\, x \in M, \ \,\forall\, 0 \neq r \in R, \quad \exists\, y \in M \text{ s.t. } x = yr
	.\end{equation} 
\end{rem}

\begin{prop}
	Let $R$ be a PID, consider the category $\mathsf{C} := \mathsf{Mod}\text{-}R$.
	$E_R \in \mathsf{Mod}\text{-}R$ is injective iff it is divisible.
\end{prop} 
\begin{proof}
	It seems to me that any injective module is also divisible
	as soon as $xR \simeq R$ for any $x \in R$, i.e. I guess for integral domains
	(Baer's lemma still holds and we can check it on every principal ideal).
	The converse, however, requires that all ideals are principal.
\end{proof}

\begin{ex}[category with no nonzero projective objects]
	Let $\mathsf{C} := \mathsf{T}$ the full subcategory of $\mathsf{Ab}$ of torsion abelian groups.
	Then $\mathsf{T}$ has enough injectives, but no nonzero projective objects.
	\begin{itemize}
		\item Notice that $\mathsf{T} \subset \mathsf{Ab} = \mathsf{Mod}\text{-}\Z$,
			and $\Z$ is a PID.
			Then a torsion group is injective iff it is divisible.

			Consider an arbitrary $T \in \mathrm{Ob} \left(\mathsf{T}\right)$.
			Then $T$ has a set of generators $\left\{ x_i \right\}_{i \in I}$, each of order
			$o(x_i) = n_i \in \N$.
			Then we have an epimorphism
			\begin{equation}
			\varphi: \bigoplus_{i \in I} \mathbb{Z}/n_i\mathbb{Z} \twoheadrightarrow T
			.\end{equation} 
			Then an injective element $I \in \mathsf{T}$ containing $T$ is
			\begin{equation}
			\bigoplus_{i \in I} \mathbb{Q}/n_i \Z
			,\end{equation} 
			which is divisible, hence injective, and contains
			\begin{equation}
			\bigoplus_{i \in I} \mathbb{Z}/n_i\mathbb{Z}
			.\end{equation} 
			Finally we have an injection, given by the inclusion, which 
			states that $\mathsf{T}$ has enough injectives:
			\begin{equation}
			\frac{\bigoplus_{i \in I} \mathbb{Z}/n_i\mathbb{Z}}{\ker \varphi} \hookrightarrow 
			\frac{\bigoplus_{i \in I} \mathbb{Q}/n_i\Z}{\ker \varphi}
			.\end{equation} 
		\item There is a well-known fact saying that a subgroup of a direct sum
			of cyclic abelian groups is a direct sum of cyclic abelian groups.

			Consider $0 \neq T \in \mathrm{Ob} \left(\mathsf{T}\right)$,
			and assume it is projective.
			Then, for $\left\{ x_i \right\}_{i \in I}$ the generators of $T$, as above,
			\begin{equation}
			\begin{tikzcd}
				\bigoplus_{i \in I} \mathbb{Z}/n_i\mathbb{Z} \arrow[r, "\varphi", rightarrow] &
				T \arrow[r, "", rightarrow] &
				0 \\
				&
				T \arrow[ul, "\psi", dashrightarrow] \arrow[u, "1_T"', equals] &
			\end{tikzcd}
			.\end{equation} 
			Then $T$ is a subgroup of a direct sum of cyclic groups
			($1_T$ is injective, hence so has to be $\psi$).
			By the above remark
			\begin{equation}
			T \simeq \bigoplus_{j \in J} \mathbb{Z}/m_j\mathbb{Z}
			.\end{equation} 
			Since $T \neq 0$, then there exists $m_0$ s.t.
			$\mathbb{Z}/m_0\mathbb{Z} \neq 0$, and it is a projective object, since
			it is a direct summand of a projective object.
			Let's now consider the epimorphism
			\begin{equation}
			\mathbb{Z}/m_0^2\mathbb{Z} \twoheadrightarrow \mathbb{Z}/m_0\mathbb{Z}
			.\end{equation} 
			Reasoning as before we obtain that $\mathbb{Z}/m_0\mathbb{Z}$ is a direct
			summand of $\mathbb{Z}/m_0^2\mathbb{Z}$, which is a contradiction.
	\end{itemize}
\end{ex} 

\subsection{Functor categories}
\begin{rem}
	Let $\mathsf{I}$ be a small and preadditive category.
	Let $\mathsf{C}$ be an abelian category.
	Define $\mathrm{Hom}_{\mathsf{}} \left( \mathsf{I}, \mathsf{C} \right) \subset \mathsf{C}^{\mathsf{I}}$
	the subcategory of all additive functors $F: \mathsf{I} \to \mathsf{C}$.
	In this situation $\mathrm{Hom}_{\mathsf{}} \left( \mathsf{I}, \mathsf{C} \right)$ is abelian.
\end{rem}

\begin{lem}[Yoneda]
	Let $\mathsf{I}$ be as above.
	Let $\mathsf{C} := \mathsf{Ab}$.
	Fix $X \in \mathrm{Ob} \left(\mathsf{I}\right)$ and $F \in \mathrm{Hom}_{\mathsf{}} \left( \mathsf{I}^{op}, \mathsf{Ab} \right)$.
	There is an isomorphism
	\begin{equation}
	\mathrm{Nat} \left( h^X, F \right) \xrightarrow{\theta_{X,F}} 
	F(X)
	,\end{equation} 
	natural in $X$ and in $F$.
	Recall that $h^X := \mathrm{Hom}_{\mathsf{I}} \left( -, X \right)$.
\end{lem} 

\begin{rem}[An application of Yoned lemma]
	Consider $X, X' \in \mathrm{Ob} \left(\mathsf{I}\right)$. 
	Let $F := h^{X'}$, then \textit{Yoneda lemma} implies
	\begin{equation}
		\mathrm{Nat} \left( h^X, h^{X'} \right) \simeq h^{X'}(X) = \mathrm{Hom}_{\mathsf{I}} \left( X, X' \right)
	.\end{equation} 
\end{rem}

\begin{defn}[Yoneda embedding]
	Consider $\mathsf{I}$ small and preadditive, $\mathsf{C} = \mathsf{Ab}$.
	We define the \textbf{Yoneda embedding} as the functor
	$Y: \mathsf{I} \to \mathrm{Hom}_{\mathsf{}} \left( \mathsf{I}^{op}, \mathsf{Ab} \right)$, defined on objects as
	\begin{align}
		Y: \mathsf{I} &\to \mathrm{Hom}_{\mathsf{}} \left( \mathsf{I}^{op}, \mathsf{Ab} \right) \\
		X &\mapsto h^X
	\end{align} 
	and on morphisms, given $f: X \to X'$, by
	\begin{equation}
		Y(f):= \mathrm{Hom}_{\mathsf{I}} \left( -, f \right): h^X \to h^{X'}
	.\end{equation} 
\end{defn}

\begin{prop}
	The Yoneda embedding $Y$ is \textbf{fully faithful}.
	Moreover it sends distinct objects of $\mathsf{I}$ to distinct objects of $\mathrm{Hom}_{\mathsf{}} \left( \mathsf{I}^{op}, \mathsf{Ab} \right)$.
\end{prop} 

\begin{cor}
	Consider a small preadditive category $\mathsf{I}$.
	Then $\mathsf{I}$ is equivalent to the full subcategory of $\mathrm{Hom}_{\mathsf{}} \left( \mathsf{I}^{op}, \mathsf{Ab} \right)$ 
	consisting of the representable functors.
\end{cor} 

\begin{prop}
	For $\mathsf{I}$ as before (small and preadditive) and $X \in \mathrm{Ob} \left(\mathsf{I}\right)$, then
	$h^X$ is a projective object of $\mathrm{Hom}_{\mathsf{}} \left( \mathsf{I}^{op}, \mathsf{Ab} \right)$.
\end{prop} 

\begin{defn}[Generator of a category]
	Let $\mathsf{C}$ be a category.
	An object $G \in \mathrm{Ob} \left(\mathsf{C}\right)$ is a \textbf{generator} of $\mathsf{C}$ iff
	$\mathrm{Hom}_{\mathsf{C}} \left( G, - \right): \mathsf{C} \to \mathsf{Sets}$ is faithful.
	In other words iff the maps of sets
	 \begin{equation}
	\mathrm{Hom}_{\mathsf{C}} \left( C, D \right) \to
	\mathrm{Hom}_{\mathsf{Sets}} \left( \mathrm{Hom}_{\mathsf{C}} \left( G, C \right), 
	\mathrm{Hom}_{\mathsf{C}} \left( G, D \right) \right)
	\end{equation} 
	is injective for every $C, D \in \mathrm{Ob} \left(\mathsf{C}\right)$.
\end{defn}

\begin{rem}[Equivalent definition]
	$G$ is a generator, iff for every pair $f,g: C \to D$
	s.t. $\mathrm{Hom}_{\mathsf{C}} \left( G, f \right) = \mathrm{Hom}_{\mathsf{C}} \left( G, g \right)$,
	i.e. $g \circ \alpha = f \circ \alpha$ for all $\alpha: G \to C$, then $f = g$.

	In the case of a preadditive category $\mathsf{C}$, then $G$ is a generator iff
	for all morphisms $f$ in $\mathsf{C}$ s.t. $\mathrm{Hom}_{\mathsf{C}} \left( G, f \right) = 0$,
	i.e. s.t. $f \circ \alpha = 0$ (whenever admissible), then $f = 0$.
\end{rem}

\begin{defn}[Alternative notation for (co)products]
	Fix $X \in \mathrm{Ob} \left(\mathsf{C}\right)$ and $I$ a set.
	\begin{itemize}
		\item If $\prod_{i \in I} X_i$, with $X_i := X$ for all $i \in I$, exists we define the notation
			\begin{equation}
			X^I := \prod_{i \in I} X_i
			.\end{equation} 
		\item If $\coprod_{i \in I} X_i$, with $X_i := X$ for all $i \in I$, exists we define the notation
			\begin{equation}
				X^{(I)} := \coprod_{i \in I} X_i
			.\end{equation} 
	\end{itemize}
\end{defn}

\begin{prop}
	Assume that $\mathsf{C}$ has arbitrary coproducts.
	TFAE
	\begin{enumerate}
		\item $G$ is a generator of $\mathsf{C}$,
		\item $\,\forall\, X \in \mathrm{Ob} \left(\mathsf{C}\right)$, there is an epimorphism
			$G^{(I)} \to X$, for some set $I$.
	\end{enumerate}
\end{prop} 

\begin{defn}[Cogenerator of a category]
	Let $\mathsf{C}$ be a category.
	An object $C \in \mathrm{Ob} \left(\mathsf{C}\right)$ is a \textbf{cogenerator} of $\mathsf{C}$ iff
	$C$ is a generator in $\mathsf{C}^{op}$, i.e. iff
	$\mathrm{Hom}_{\mathsf{C}} \left( -, C \right): \mathsf{C}^{op} \to \mathsf{Sets}$ is faithful.
	In other words iff the maps of sets
	 \begin{equation}
	\mathrm{Hom}_{\mathsf{C}} \left( A, B \right) \to
	\mathrm{Hom}_{\mathsf{Sets}} \left( \mathrm{Hom}_{\mathsf{C}} \left( B, C \right), 
	\mathrm{Hom}_{\mathsf{C}} \left( A, C \right) \right)
	\end{equation} 
	is injective for every $A, B \in \mathrm{Ob} \left(\mathsf{C}\right)$.
\end{defn}

\begin{rem}[Equivalent definition]
	$C$ is a cogenerator, iff for every pair $f,g: A \to B$
	s.t. $\mathrm{Hom}_{\mathsf{C}} \left( f, C \right) = \mathrm{Hom}_{\mathsf{C}} \left( g, C \right)$,
	i.e. $\alpha \circ f = \alpha \circ g$ for all $\alpha: B \to C$, then $f = g$.

	In the case of a preadditive category $\mathsf{C}$, then $C$ is a generator iff
	for all morphisms $f$ in $\mathsf{C}$ s.t. $\mathrm{Hom}_{\mathsf{C}} \left( f, C \right) = 0$,
	i.e. s.t. $\alpha \circ f = 0$ (whenever admissible), then $f = 0$.
\end{rem}

\begin{prop}
	Assume that $\mathsf{C}$ has arbitrary products.
	TFAE
	\begin{enumerate}
		\item $C$ is a cogenerator of $\mathsf{C}$,
		\item $\,\forall\, X \in \mathrm{Ob} \left(\mathsf{C}\right)$, there is a monomorphism
			$\mu: X \to C^{I}$, for some set $I$.
	\end{enumerate}
\end{prop} 

\begin{ex}
	Let $\mathsf{C} := \mathsf{Mod}\text{-}R$.
	$R$ is a generator of $\mathsf{Mod}\text{-}R$:
	given a module $M_R$, and $\left\{ x_i \right\}_{i \in I}$ a set of generators for $M$, then
	\begin{equation}
		R^{(I)} = \bigoplus_{i \in I} R_i \xrightarrow{\phi} M \to 0
	,\end{equation} 
	in which $\phi(e_i) := x_i$.
	Moreover $R$ is projective, hence it is a projective generator.
\end{ex} 

\begin{rem}[A not-so-easy-to-prove fact about modules]
	Let $\mathsf{C} := \mathsf{Mod}\text{-}R$.
	Every module $M$ can be embedded in an injective module
	(i.e. $\mathsf{Mod}\text{-}R$ has enough injectives).
	Moreover every module $M$ admits an injective envelope, denoted $E(M)$,
	where the envelope is a minimal injective module containing $M$.
\end{rem}

\begin{ex}
	Let $\mathsf{C}:= \mathsf{Mod}\text{-}R$ as before.
	Let $\mathcal{S}$ be the set of simple modules $S \in \mathsf{Mod}\text{-}R$ (i.e. modules with no proper submodules).
	Recall that $S \in \mathcal{S}$ iff $S \simeq R/\mathfrak{m}_R$, for some maximal ideal $\mathfrak{m}_R \triangleleft R$.
	Given $S \in \mathcal{S}$, consider its injective envelope $E(S)$, and finally
	let's define
	\begin{equation}
		C := \prod_{S \in \mathcal{S}} E(S) \simeq
		\prod_{\mathfrak{m}_R \in \mathrm{Max}\, R} E \left( R / \mathfrak{m}_R \right)
		\in \mathrm{Ob} \left(\mathsf{C}\right)
	.\end{equation} 
	Then $C$ is an injective cogenerator of $\mathsf{Mod}\text{-}R$.
	In fact, consider $0 \neq X_R \in \mathsf{Mod}\text{-}R$, and $0 \neq x \in X_R$.
	Then $\left\langle x \right\rangle \simeq R/I$, 
	for $I = \left\{ r \in R \ \middle|\ xr = 0 \right\} \triangleleft R$.
	Consider any maximal ideal $\mathfrak{m}_R \triangleleft R$ s.t. $I \subset \mathfrak{m}_R$,
	then, since $E(R/\mathfrak{m}_R)$ is injective, we have the commutative diagram
	\begin{equation}
	\begin{tikzcd}
		0 \arrow[r, "", rightarrow] &
		\left\langle x \right\rangle \arrow[r, "", hookrightarrow] \arrow[d, "\pi"', rightarrow] &
		X \arrow[ldd, "\exists\, f_x \neq 0", rightarrow] \\
		& R/\mathfrak{m}_R \arrow[d, "", hookrightarrow] & \\
		& E(R/\mathfrak{m}_R) &
	\end{tikzcd}
	.\end{equation} 
	Then, for every $0 \neq x \in X$, we have the map
	\begin{equation}
		0 \neq f_x: X \to E \left( R/\mathfrak{m}_R \right) \hookrightarrow
		\prod_{\mathfrak{m}_R \in \mathrm{Max}\, R} E \left( R / \mathfrak{m}_R \right) =: C
	.\end{equation} 
	Then, by the universal property of products, viewing $X$ as a set,
	\begin{equation}
	\exists\, !\, f: X \hookrightarrow C^X
	\end{equation} 
	induced by the various $f_x$.
	Moreover this $f$ is mono, since for any $0 \neq x$ $f_x(x) \neq 0$.
\end{ex} 

\begin{rem}
	Notice that, if $\mathsf{C}$ has a projective generator, then $\mathsf{C}$
	has enough projectives.
	Analogously, if $\mathsf{C}$ has an injective cogenerator, then $\mathsf{C}$ 
	has enough injectives.
\end{rem} 

\subsection{Grothendieck categories}
\begin{defn}[Grothendieck category]
	An abelian category $\mathsf{C}$ is a \textbf{Grothendieck} category iff
	it is cocomplete, it has a generator, and filtered direct limits are exact in $\mathsf{C}$.
\end{defn}

\begin{rem}[Important fact]
	A Grothendieck category has injective envelopes, in particular injective cogenerators.
	Though it might have no nonzero projective objects.
\end{rem}

\begin{ex}\leavevmode\vspace{-.2\baselineskip}
	\begin{itemize}
		\item $\mathsf{Mod}\text{-}R$ and $R\text{-}\mathsf{Mod}$ are both Grothendieck categories.
		\item The category of coherent sheaves is Grothendieck, but has no nonzero projective objects.
		\item It can be shown that also the category of torsion abelian groups is Grothendieck.
	\end{itemize}
\end{ex} 

\section{Adjoint functors}
Let's introduce this topic with an example
\begin{ex}
	Let $\K$ be a field, and $\mathsf{C} := \mathsf{Vect-}\K$ the category of $\K$-Vector Spaces.
	Clearly we can define the forgetful functor, which acts on objects as
	\begin{align}
		\mathrm{For}: \mathsf{Vect-}\K &\to \mathsf{Sets} \\
		V_K &\mapsto V
	,\end{align} 
	forgetting about the structure of Vector Space.
	For a set $X$, moreover, we can construct the Vector Space $\left\langle X \right\rangle$,
	which is the Vector Space  for which $X$ is a basis.
	This induces a functor
	\begin{align}
		 \mathsf{Sets} &\to \mathsf{Vect-}\K \\
		 X&\mapsto \left\langle X \right\rangle =: V
	.\end{align} 
	Recall that, fixed $X \in \mathrm{Ob} \left(\mathsf{Sets}\right)$, and $W \in \mathrm{Ob} \left(\mathsf{Vect}-\K\right)$,
	for every map $\alpha: X \to \mathrm{for}\, W$, we can construct a unique linear map
	$f: \left\langle X \right\rangle \to W$ s.t. the diagram commutes
	\begin{equation}
	\begin{tikzcd}
		X \arrow[r, "\alpha", rightarrow] \arrow[d, "i", hookrightarrow] &
		W\\
		\left\langle X \right\rangle \arrow[ru, "\exists\, !\, f"', rightarrow] &
	\end{tikzcd}
	\end{equation} 
	i.e. s.t. $f(x) = \alpha(x)\ \,\forall\, x \in X$.
	In particualr we have a bijection
	\begin{equation}
	\begin{tikzcd}
		\mathrm{Hom}_{\mathsf{Sets}} \left( X, \mathrm{For}\, W \right) \arrow[r, "", leftrightarrow] &
		\mathrm{Hom}_{\K} \left( \left\langle X \right\rangle_{\K}, W_{\K} \right)
	\end{tikzcd}
	.\end{equation} 
\end{ex} 

\begin{defn}[Adjoint pair of functors]
	Let $\mathsf{C}$ and $\mathsf{D}$ be two categories.
	Consider two functors $L: \mathsf{C} \to \mathsf{D}$ and $R: \mathsf{D} \to \mathsf{C}$.
	The pair $\left(L, R\right)$ is called an \textbf{adjoint} pair iff there is
	\begin{equation}
	\begin{tikzcd}
		\mathrm{Hom}_{\mathsf{D}} \left( L(C), D \right) \arrow[r, "{\varphi(C,D)}", rightarrow] &
		\mathrm{Hom}_{\mathsf{C}} \left( C, R(D) \right)
	\end{tikzcd}
	\end{equation} 
	a bijection natural in $C$ and $D$.
	In particular $L$ is the left adjoint of $R$ and $R$ is the right adjoint of $L$.
	An adjoint pair is sometimes referred to as adjunction, and denoted by
	\begin{equation}
	\begin{tikzcd}
		\mathsf{C} \arrow[r, "L", rightarrow, shift left = .5ex] &
		\mathsf{D} \arrow[l, "R", rightarrow, shift left = .5ex] 
	\end{tikzcd}
	.\end{equation} 
\end{defn}

\begin{rem}
	In the above remark, the pair $\left(\left\langle - \right\rangle, \mathrm{For}\, \right)$ is an adjoint pair.
\end{rem} 

\begin{rem}
	Naturality of $\varphi$ in $C$ and $D$, more explicitly, means that the following diagrams commute
	\begin{equation}
	\begin{tikzcd}
		C \arrow[d, "f", rightarrow] &
		\mathrm{Hom}_{\mathsf{D}} \left( L(C), D \right) \arrow[r, "{\varphi(C,D)}", rightarrow] &
		\mathrm{Hom}_{\mathsf{C}} \left( C, R(D) \right)\\
		C' &
		\mathrm{Hom}_{\mathsf{D}} \left( L(C'), D \right) \arrow[r, "{\varphi(C',D)}"', rightarrow] 
		\arrow[u, "{\mathrm{Hom}_{\mathsf{D}} \left( L(f), D \right)}", rightarrow] &
		\mathrm{Hom}_{\mathsf{C}} \left( C', R(D) \right) 
		\arrow[u, "{\mathrm{Hom}_{\mathsf{C}} \left( f, R(D) \right)}"', rightarrow]
	\end{tikzcd}
	.\end{equation} 
	
	\begin{equation}
	\begin{tikzcd}
		D \arrow[d, "g", rightarrow] &
		\mathrm{Hom}_{\mathsf{D}} \left( L(C), D \right) \arrow[r, "{\varphi(C,D)}", rightarrow]
		\arrow[d, "{\mathrm{Hom}_{\mathsf{D}} \left( L(C), g \right)}"', rightarrow] &
		\mathrm{Hom}_{\mathsf{C}} \left( C, R(D) \right)
		\arrow[d, "{\mathrm{Hom}_{\mathsf{C}} \left( C, R(g) \right)}", rightarrow]\\
		D' &
		\mathrm{Hom}_{\mathsf{D}} \left( L(C), D' \right) \arrow[r, "{\varphi(C,D')}"', rightarrow] &
		\mathrm{Hom}_{\mathsf{C}} \left( C, R(D') \right) 
	\end{tikzcd}
	.\end{equation} 
\end{rem}

\begin{defn}[(Co)unit of an adjunction]
	Let $\mathsf{C}$ and $\mathsf{D}$ be two categories.
	Let $L: \mathsf{C} \to \mathsf{D}$ and $R: \mathsf{D} \to \mathsf{C}$ be functors
	s.t. $\left(L, R\right)$ is an adjoint pair.
	We define
	\begin{itemize}
		\item The \textbf{unit} of the adjunction, the natural transformation
			\begin{equation}
			\eta: id_{\mathsf{C}} \to R \circ L
			\end{equation} 
			defined, for every $C \in \mathrm{Ob} \left(\mathsf{C}\right)$, by
			\begin{equation}
				\eta_C := \varphi_{(C, L(C))} \left( 1_{LC} \right) \in \mathrm{Hom}_{\mathsf{C}} \left( C, RL(C) \right)
			.\end{equation} 
		\item The \textbf{counit} of the adjunction, the natural transformation
			\begin{equation}
			\zeta: L \circ R \to id_{\mathsf{D}}
			\end{equation} 
			defined, for every $D \in \mathrm{Ob} \left(\mathsf{D}\right)$, by
			\begin{equation}
				\zeta_D := \varphi_{(R(D), D)}^{-1} \left( 1_{RD} \right) \in \mathrm{Hom}_{\mathsf{C}} \left( LR(D), D \right)
			.\end{equation} 
	\end{itemize}
\end{defn}

\begin{rem}[]
	It is not obvious from the definition that the family of morphisms given by the unit and counit
	are natural transformation.
\end{rem}

\begin{prop}
	Given two right adjoints $R$ and $R'$ of the same functor $L$,
	then $R$ and $R'$ are naturally isomorphic.
\end{prop} 

\begin{prop}
	Given two left adjoints $L$ and $L'$ of the same functor $R$,
	then $L$ and $L'$ are naturally isomorphic.
\end{prop} 

\begin{prop}
	Let $F: \mathsf{C} \to \mathsf{D}$ and $R: \mathsf{D} \to \mathsf{C}$ be a pair of functors.
	TFAE
	\begin{itemize}
		\item $\left(L, R\right)$ is an adjoint pair,
		\item there exist natural transformations
			\begin{equation}
			\eta: id_{\mathsf{C}} \to R \circ L \qquad \text{ and }\qquad \zeta: L \circ R \to id_{\mathsf{D}}
			\end{equation} 
			s.t.
			\begin{align}
				\zeta_{L(C)} \circ L(\eta_C) &= id_{L(C)} \qquad
				\,\forall\, C \in \mathrm{Ob} \left(\mathsf{C}\right)\\
				R(\zeta_D) \circ \eta_{R(D)} &= id_{R(D)} \qquad
				\,\forall\, D \in \mathrm{Ob} \left(\mathsf{D}\right)
			.\end{align} 
	\end{itemize}
	In such case $\eta$ is the unit, and $\zeta$ the counit, of the adjunction.
\end{prop} 

\begin{rem}
	Let $\left(L, R\right)$, with $L: \mathsf{C} \to \mathsf{D}$ and $R: \mathsf{D} \to \mathsf{C}$, be an adjoint pair.
	Given an arbitrary morphism $\beta: C \to RD$, with $C \in \mathrm{Ob} \left(\mathsf{C}\right)$ and
	$D \in \mathrm{Ob} \left(\mathsf{D}\right)$.
	Let $\alpha: L(C) \to D$ the morphism s.t. $\varphi(C,D) = \beta$.
	Then there exists a commutative triangle, i.e. that $\beta = R(\alpha) \circ\eta_C$
	\begin{equation}
	\begin{tikzcd}
		C \arrow[r, "\beta", rightarrow] \arrow[d, "\eta_C"', rightarrow] &
		R(D)\\
		RL(C) \arrow[ru, "{R(\alpha)}"', rightarrow] &
	\end{tikzcd}
	.\end{equation} 
\end{rem} 

\begin{rem}[]
	Let $\left(L, R\right)$, with $L: \mathsf{C} \to \mathsf{D}$ and $R: \mathsf{D} \to \mathsf{C}$, be an adjoint pair.
	TFAE:
	\begin{enumerate}
		\item $R$ is faithful,
		\item $R$ reflects epimorphisms, i.e. if $Rf$ is an epi in $\mathsf{C}$,
			then $f$ is epi in $\mathsf{D}$,
		\item given $\beta: C \to R(D)$ epi, then $\alpha := \varphi^{-1}(C,D)(\beta)$ is epi,
		\item $\zeta_D: LR(D) \to D$ is epi for every $D \in \mathsf{D}$.
	\end{enumerate}
\end{rem}

\begin{rem}[]
	Given the definitions in the preliminaries, fix two rings $S$ and $R$, and an $R$-$S$ bimodule ${}_SM_R$,
	we can construct the functors (acting on the objects as)
	\begin{align}
		- \otimes_S M: \mathsf{Mod}\text{-}S &\to \mathsf{Mod}\text{-}R \\
		 N_S &\mapsto \left[ N \otimes_S M_R \right]_R
	,\end{align}
	\begin{align}
		\mathrm{Hom}_{R} \left( M_R, - \right): \mathsf{Mod}\text{-}R &\to \mathsf{Mod}\text{-}S \\
		L_R &\mapsto \left[ \mathrm{Hom}_{R}\left( {}_SM_R, L_R \right)\right]_S
	.\end{align} 	
\end{rem}

\begin{prop}
	The pair $\left(- \otimes_S M_R, \mathrm{Hom}_{R}\left( M_R, - \right)\right)$
	is an adjoint pair.
	Moreover also the pair $\left(M_R \otimes_R -, \mathrm{Hom}_{S}\left( M, - \right) \right)$ is an adjoint pair.
	Notice that, if the above functors are between categories of right modules,
	these are between categories of left modules.
\end{prop} 

\begin{ex}
	Let $\phi: R \to S$ be a ring homomorphism.
	Then any ${}_SN \in S\text{-}\mathsf{Mod}$ becomes also a left $R$-module via
	\begin{equation}
		r \cdot x := \phi(r) \cdot x \qquad \,\forall\, x \in N,\ \,\forall\, r \in R
	.\end{equation} 
	And analogously for any $N_S \in \mathsf{Mod}\text{-}S$.
	In particular $S$ becomes both a left and right $R$-module via $\phi$.
	We can then define the following functors:
	\begin{align}
		-\otimes_R S: \mathsf{Mod}\text{-}R &\to \mathsf{Mod}\text{-}S \\
		M_R &\mapsto M \otimes_R S
	\end{align} 
	called \textit{extension of scalars}.
	And also the \textit{restriction functor}
	\begin{align}
		\phi_*: \mathsf{Mod}\text{-}S &\to \mathsf{Mod}\text{-}S \\
		N_S &\mapsto N_R
	.\end{align} 
	Then the pair $\left(- \otimes_R S, \phi_*\right)$ is an adjoint pair.

	Analogously we can define the functor
	\begin{align}
		\mathrm{Hom}_{ R}\left( S_R, - \right): \mathsf{Mod}\text{-}R &\to \mathsf{Mod}\text{-}S\\
		M_R &\mapsto \left[ \mathrm{Hom}_{R}\left({}_SS_R, M_R \right) \right]_S
	,\end{align} 
	and the pair $\left(\phi_*, \mathrm{Hom}_{ R}\left( S, - \right)\right)$ is an adjoint pair.
\end{ex} 	

\begin{prop}
	Let $\mathsf{C}$ and $\mathsf{D}$ be arbitrary categories.
	Let $\left(L, R\right)$ be a pair of adjoint functors, $L: \mathsf{C} \to \mathsf{D}$ and
	$R: \mathsf{D} \to \mathsf{C}$.
	Then
	\begin{enumerate}
		\item $L$ preserves colimts, and in particular coproducts, pushouts, cokernels,
			when they exist,
		\item $R$ preserves limts, and in particular products, pullbacks, kernels,
			when they exist.
	\end{enumerate}
\end{prop} 

\begin{ex}
	If $\mathsf{C} := R\text{-}\mathsf{Mod}$ and $\mathsf{D}:= S\text{-}\mathsf{Mod}$, then
	$\left(M \otimes_R -, \mathrm{Hom}_{S}\left( M, - \right)\right)$, for ${}_SM_R$, is an adjoint pair.
	Then $M \otimes_R -$ preserves colimits.
	In fact, given a direct system $\left\{N_i, f_{ji}\right\}_{i, j \in \mathrm{Ob} \left(\mathsf{I}\right)}$, 
	for some small category $\mathsf{I}$, then
	\begin{equation}
		M \otimes_R \varinjlim_i N_i \simeq \varinjlim_i \left( M \otimes_R N_i \right)
	.\end{equation} 
	Analogously $\mathrm{Hom}_{S}\left( {}_S M, - \right)$ preserves limits.
	Then, given an inverse system $\left\{ L_i, f_{ij} \right\}_{i,j \in \mathrm{Ob} \left(\mathsf{I}\right)}$
	for some small category $\mathsf{I}$, then
	\begin{equation}
		\mathrm{Hom}_{S}\big( {}_S M, \varprojlim_i L_i \big) \simeq
		\varprojlim_i \mathrm{Hom}_{ S}\left( {}_S M, L_i \right)
	.\end{equation} 
\end{ex} 

\begin{rem}[Application of the proposition]
	Let $\mathsf{C}$ and $\mathsf{D}$ be abelian categories.
	Let $\left(L, R\right)$ be an adjoint pair, $L: \mathsf{C} \to \mathsf{D}$ and $R: \mathsf{D} \to \mathsf{C}$.
	Then $L$ is \textbf{right} exact, and $L$ is \textbf{left} exact.
\end{rem}

\begin{prop}
	Let $\mathsf{I}$ be a small category, and $\mathsf{C}$ be a cocomplete category.
	Then the colimit functor
	\begin{equation}
	\varinjlim: \mathsf{C}^{\mathsf{I}} \to \mathsf{C}
	\end{equation} 
	is a left adjoint.
	If, moreover, $\mathsf{C}$ is abelian, $\varinjlim$ is also right exact.

	Dually, if $\mathsf{C}$ is complete, then $\varprojlim$	is a right adjoint.
	Again, if $\mathsf{C}$ is abelian, then $\varprojlim$ is also left exaxct.
\end{prop} 

\section{Chain and cochain complexes}
Let, in the following, $\mathsf{A}$ be a {\em preadditive} category with $0$.

\begin{defn}[Chain complex over $\mathsf{A}$]
	We define $\mathrm{Ch}(\mathsf{A})$ the category of {\em chain complexes} over $\mathsf{A}$
	as the category whose objects are sequences
	\begin{equation}
	\ldots \to X_n \xrightarrow{d_n} X_{n-1}
	\xrightarrow{d_{n-1}} X_{n-2} \to \ldots
	\end{equation} 
	s.t. $X_i \in \mathrm{Ob} \left(\mathsf{A}\right)$, $d_i \circ d_{i+1} = 0$ for all $i \in \Z$.
	The morphisms $d_i$ are called {\em differentials} and the sequence is called {\em complex},
	denoted by $\left( X_{\bullet}, d^X \right)$, with $\left( d^X \right)^2 = 0$.

	Morphisms in $\mathrm{Ch}(\mathsf{A})$, denoted by
	$f\colon\left(X_{\bullet}, d^X\right) \to \left(Y_{\bullet}, d^Y\right)$,
	are a family of morphisms $\left\{ f_n \right\}_{n \in \Z}$, where
	$f_n \in \mathrm{Hom}_{\mathsf{A}} \left( X_n, Y_n \right)$,
	making the following diagram commute
	\begin{equation}
	\begin{tikzcd}
		\ldots \arrow[r, "", rightarrow] &
		X_n \arrow[r, "d^X_n", rightarrow] \arrow[d, "f_n", rightarrow] &
		X_{n-1} \arrow[r, "d^X_{n-1}", rightarrow] \arrow[d, "f_{n-1}", rightarrow] &
		X_{n-2} \arrow[r, "", rightarrow] \arrow[d, "f_{n-2}", rightarrow] &
		\ldots \\
		\ldots \arrow[r, "", rightarrow] &
		Y_n \arrow[r, "d^Y_n", rightarrow]&
		Y_{n-1} \arrow[r, "d^Y_{n-1}", rightarrow] &
		Y_{n-2} \arrow[r, "", rightarrow] &
		\ldots
	\end{tikzcd}
	,\end{equation} 
	i.e. such that $d^Y_n \circ f_N = f_{n-1} \circ d^X_n$ for all $n \in \Z$
	(more compactly $d^Y \circ f = f \circ d^X$).
\end{defn}

\begin{defn}[Cochain complex over $A$]
	We define $\mathrm{Cch}(\mathsf{A})$ the category of {\em cochain complexes} over $\mathsf{A}$
	as the category whose objects are sequences
	\begin{equation}
	\ldots \to X^n \xrightarrow{d^n} X^{n+1}
	\xrightarrow{d^{n+1}} X^{n+2} \to \ldots
	\end{equation} 
	s.t. $X^i \in \mathrm{Ob} \left(\mathsf{A}\right)$, $d^i \circ d^{i-1} = 0$ for all $i \in \Z$.
	The morphisms $d^i$ are called {\em differentials} and the sequence is called {\em complex},
	denoted by $\left( X^{\bullet}, d_X \right)$, with $\left( d_X \right)^2 = 0$.

	Morphisms in $\mathrm{Cch}(\mathsf{A})$, denoted by
	$f\colon\left(X^{\bullet}, d_X\right) \to \left(Y^{\bullet}, d_Y\right)$
	are a family of morphisms $\left\{ f^n \right\}_{n \in \Z}$, where
	$f^n \in \mathrm{Hom}_{\mathsf{A}} \left( X^n, Y^n \right)$,
	making the following diagram commute
	\begin{equation}
	\begin{tikzcd}
		\ldots \arrow[r, "", rightarrow] &
		X^n \arrow[r, "d_X^n", rightarrow] \arrow[d, "f^n", rightarrow] &
		X^{n+1} \arrow[r, "d_X^{n+1}", rightarrow] \arrow[d, "f^{n+1}", rightarrow] &
		X^{n+2} \arrow[r, "", rightarrow] \arrow[d, "f^{n+2}", rightarrow] &
		\ldots \\
		\ldots \arrow[r, "", rightarrow] &
		Y^n \arrow[r, "d_Y^n", rightarrow]&
		Y^{n+1} \arrow[r, "d_Y^{n+1}", rightarrow] &
		Y^{n+2} \arrow[r, "", rightarrow] &
		\ldots
	\end{tikzcd}
	,\end{equation} 
	i.e. such that $f^n \circ d_X^{n-1} = d_Y^{n-1} \circ f^{n-1}$ for all $n \in \Z$
	(more compactly $f \circ d_X = d_Y \circ f$).
\end{defn}

\begin{rem}[Additive categories]
	If $\mathsf{A}$ is additive, then also $\mathrm{Ch}(\mathsf{A})$ and $\mathrm{Cch}(\mathsf{A})$ are.
	In particular, given $\left(X^{\bullet}, d_X\right)$ and $\left(Y^{\bullet}, d_Y \right)$ two cochain complexes,
	then their coproduct $\left(X^{\bullet} \oplus Y^{\bullet}, d_X \oplus d_Y\right)$
	is given, degree wise, by
	\begin{equation}
	\left[ X^{\bullet} \oplus Y^{\bullet} \right]^n := X^n \oplus Y^n.
	\end{equation} 
	Analogously, degree wise, its differentials are defined by
	\begin{equation}
	d^n_{X^{\bullet} \oplus Y^{\bullet}} :=
	d^n_X \oplus d^n_Y =
	\begin{bmatrix}
		d^n_X & 0\\
		0 & d^n_Y
	\end{bmatrix} 
	.\end{equation} 
\end{rem}

\begin{defn}[Bounded (co)chain complex]
	A (co)chain complex $\left(X^{\bullet}, d_X\right)$ is {\em bounded} iff
	$\exists\, b \in \N$ s.t. $X^n = 0$ for all $\left| n \right| > b$.
	It is bounded {\em below}, resp. {\em above}, iff
	$\exists\, b \in \Z$ s.t. $X^n = 0$ for all
	$n < b$, resp. $n > b$.
	(Even though we used the notation for cochain complexes the definitions apply without
	modification to chain complexes).

	We denote respectively with $\mathrm{Ch}(\mathsf{A})^b$, $\mathrm{Ch}(\mathsf{A})^+$
	and $\mathrm{Ch}(\mathsf{A})^-$ the full subcategory of bounded, resp. above or below, chain complexes.
\end{defn}

\begin{defn}[Canonical functor]
	There is a canonical embedding
	\begin{align}
		\mathrm{can}: \mathsf{A} &\to \mathrm{Ch}(\mathsf{A}) \\
		A &\mapsto A^{\bullet} := \left[ 
		\ldots \to 0 \to (A^0 := A) \to 0 \to \ldots \right]
	.\end{align} 
	$A^{\bullet}$ is called complex concentrated in degree $0$.
	Clearly $\mathrm{can}$ is fully faithful, hence it is an embedding of $\mathsf{A}$ into $\mathrm{Ch}(\mathsf{A})$.
\end{defn}

\begin{defn}[Shift functor]
	Choose $p \in \Z$, then we can define the functor
	\begin{align}
		\left[ p \right]: \mathrm{Ch}(\mathsf{A}) &\to \mathrm{Ch}(\mathsf{A}) \\
		\left(X^{\bullet}, d_X\right) &\mapsto \left( X^{\bullet}[p], d_{X}[p] \right)
	,\end{align} 
	in which we define
	\begin{equation}
		\left( X^{\bullet} [p] \right)^n := X^{n+p} \qquad \text{ and } \qquad
		d^n_{X^{\bullet}[p]} := (-1)^{p} d_X^{n+p}
	.\end{equation} 
	More explicitly this functor shifts the objects in the (co)chain, by $p$ to the left.
	Analogously it acts on a morphism of complexes
	$f\colon \left( X_{\bullet}, d^{X} \right) \to \left( Y_{\bullet}, d^{Y} \right)$
	by shifting the morphisms of the family by $p$ to the left.
	More explicitly
	\begin{equation}
		\left( [p]f \right)^n := f^{n+p}
	.\end{equation} 
	Moreover we introduce the notation $f[p] := [p]f$.
\end{defn}

\begin{rem}[Shift functor]
	The above is called the {\em shift functor} if $p = 1$:
	\begin{equation}
		[1]: \mathrm{Ch}(\mathsf{A}) \to \mathrm{Ch}(\mathsf{A}).
	\end{equation} 
\end{rem}

\begin{rem}[]
	The functor $[p]: \mathrm{Ch}(\mathsf{A}) \to \mathrm{Ch}(\mathsf{A})$ is an automorphism of categories.
	In fact $[p] \circ [-p] = id_{\mathrm{Ch}(\mathsf{A})} = [-p] \circ [p]$.	
\end{rem}

\begin{rem}[Motivational remark]
	From algebraic topology.
	We define $\Delta_n$ the standard $n$-simplex.
	Given a topological space $X$ one wants to partition it into finitely many
	$n$-simplices.
	One can construct a chain (the simplicial chain complex) by considering $X_k$, for every $k \in \N$,
	the set of $k$-dimensional simplices appearing in the partition of $X$.
	Then one can create for each degree $k$ the free abelian group generated by $X_k$, we denote it by $(C_{\bullet})_k$.
	One also defines a differential $d_k: C_k \to C_{k-1}$, which gives rise to a chain complex.
\end{rem}

\begin{prop}
	Given an abelian category $\mathsf{A}$, then $\mathrm{Ch}(\mathsf{A})$ is abelian,
	i.e. it admits kernels, cokernles and $\mathrm{Coim}\, $ is canonically isomorphic to $\mathrm{Im}\, $.
\end{prop}

\begin{ex}
	Let's, for example, define the kernel of a morphism
	\begin{equation}
	f: \left( X^{\bullet}, d_{X} \right) \to \left( Y^{\bullet}, d_{Y} \right)
	.\end{equation} 
	Then, we denote by $K^{\bullet} := \ker f$ the cochain s.t. $K^n := \ker f^n$
	and with differential defined by:
	\begin{equation}
		\begin{tikzcd}[column sep=small, row sep=small]
		& X^n \arrow[rr, "d^n_X", rightarrow] \arrow[dd, "f^n"' near end, rightarrow] & &
		X^{n+1} \arrow[dd, "f^{n+1}" near end, rightarrow] \\
		\ker f^n \arrow[ru, "\epsilon^n", tail]
			\arrow[rr, "\exists\, ! d^n" near end, rightarrow, red, crossing over] & &
		\ker f^{n+1} \arrow[ru, "\epsilon^{n+1}"', tail] \\
		& Y^n \arrow[rr, "d^n_Y", rightarrow] & &
		Y^{n+1}
	\end{tikzcd}
	.\end{equation} 
	By the commutativity of the diagram we obtain
	\begin{equation}
	f^{n+1} \circ d^n_X \circ \epsilon^n = 
	d_Y^n \circ f^n \circ \epsilon^n = 0
	.\end{equation} 
	Then, by the second condition on kernels, we obtain $\exists\, !\, d^n: \ker f^n \to \ker f^{n+1}$ s.t.
	$d^n_X \circ \epsilon^n = \epsilon^{n+1} \circ d^n$.
\end{ex} 

\begin{defn}[Cohomology]
	Let $\mathsf{A}$ be an abelian category,
	and $\left( X^{\bullet}, d_{X} \right) \in \mathrm{Ch}(\mathsf{A})$.
	Then,  since $d^n_X \circ d^{n-1}_X = 0$, 
	as subobjects we have $\Ima d^{n-1}_X \subset \ker d_X^n$.
	Hence we can define, for all $n \in \Z$, the following quotient object
	\begin{equation}
		H^n(X) :=
		\frac{\ker d^n_X}{\Ima d_X^{n-1}} \in \mathrm{Ob} \left(\mathsf{A}\right)
	,\end{equation} 
	called the $n$-th cohomology of the cochain complex $\left( X^{\bullet}, d_{X} \right)$.
\end{defn}

\begin{ex}
	Let $\mathsf{A} = \mathsf{Ab}$ the category of abelian groups.
	Consider the following cochain
	\begin{equation}
	\ldots \to 0 \to \mathbb{Z}/4\mathbb{Z} \xrightarrow{\dot{2}} \mathbb{Z}/4\mathbb{Z}
	\xrightarrow{\dot{2}} \mathbb{Z}/4\mathbb{Z} \xrightarrow{\dot{2}} \ldots
	=: \left( X^{\bullet}, d_{X} \right)	
	.\end{equation} 
	Then $H^0(X) = 2\mathbb{Z}/4\mathbb{Z} \simeq \mathbb{Z}/2\mathbb{Z}$, whereas
	$H^n(X) = 0$ for all $n \neq 0$.
	If, instead, we considered the following object
	\begin{equation}
	\ldots \to \mathbb{Z}/4\mathbb{Z} \xrightarrow{\dot{2}} \mathbb{Z}/4\mathbb{Z}
	\xrightarrow{\dot{2}} \mathbb{Z}/4\mathbb{Z} \xrightarrow{\dot{2}} \ldots
	=: \left( X^{\bullet}, d_{X} \right)	
	.\end{equation} 
	Then $H^n(X) = 0$ for all $n \in \Z$ and we say that 
	$\left( X^{\bullet}, d_{X} \right)$ is {\em acyclic}.
\end{ex}

\begin{prop}
	Let $\mathsf{A}$ be an abelian category, then, for every $n \in \Z$, we can define 
	\begin{align}
		H^n: \mathrm{Ch}(\mathsf{A}) &\to \mathsf{A} \\
		\left( X^{\bullet}, d_{X} \right) &\mapsto H^n(X)
	.\end{align}
	In particular this is an additive functor.
\end{prop} 
\begin{proof}
	We need to construct, starting from a cochain map $f: X^{\bullet} \to Y^{\bullet}$, the associated cohomology morphism
	\begin{equation}
		H^n(f): H^n(X) \to H^n(Y)
	.\end{equation} 
\end{proof}

\begin{rem}[]
	If $\mathsf{A} := \mathsf{Mod}\text{-}R$, then every part of the above result can be
	checked by diagram chasing.
	In fact $z \in \ker d^n_X \iff d^n_X(z) = 0$, then
	\begin{equation}
		d^n_Y \circ f^n(z) = f^{n+1} \circ d^n_X(z) = 0
	,\end{equation} 
	hence $f^n(\ker d^n_X) \subset \ker d^n_Y$.
	Moreover, given $x \in \Ima d^{n-1}_X$, then $x = d^{n-1}_X(z)$, for some $z \in X^{n-1}$.
	Then
	\begin{equation}
		f^n(x) = f^n \circ d^{n-1}_X(z) =
		d^{n-1}_Y \circ f^{n-1} (z) \in \Ima d^{n-1}_Y
	.\end{equation} 
	Hence $f^n(\Ima d^{n-1}_X) \subset \Ima d^{n-1}_Y$.
	The $f$ induces a map on the quotient
	\begin{equation}
	\widetilde{f}: \frac{\ker d^n_X}{\Ima d_X^{n-1}} \to \frac{\ker d^n_Y}{\Ima d_Y^{n-1}}
	.\end{equation} 
\end{rem}

And now, a very important result!
\begin{thm}[Freyd-Mitchell embedding]
	Let $\mathsf{A}$ be a small, abelian category.
	Then there is a ring $R$ and a fully faithful exact functor
	\begin{equation}
	F: \mathsf{A} \to \mathsf{Mod}\text{-}R
	.\end{equation} 
\end{thm}

\begin{rem}[]
	The above theorem essentially states that we can consider objects of $\mathsf{A}$ as if they were modules.
	In particular any result in $\mathsf{Mod}\text{-}R$ involving only finitely many
	objects and morphisms (such as exactness, existence and vanishing of morphisms)
	holds in any abelian category $\mathsf{C}$.
	This is true, since we can always construct a small full subcategory $\mathsf{A}_0$ of $\mathsf{C}$,
	containing only the objects and morphism involved in the result (and, by a remark which will follow,
	an exact and fully faithful functor reflects exactness).

	Notice, however, that results for arbitrary family of objects do not translate so easily.
	For example the product of an arbitrary family of exact sequences in $\mathsf{Mod}\text{-}R$
	is still exact in $\mathsf{Mod}\text{-}R$, 
	but not in an arbitrary abelian category.
\end{rem}

\begin{proof}[Sketch of proof (Freyd-Mitchell)]
	Let $\underline{\mathrm{Hom}\left( \mathsf{A}^{op}, \mathsf{Ab} \right)}$
	be the category of the additive functors from $\mathsf{A}^{op}$ to $\mathsf{Ab}$.
	Then, by Yoneda lemma, the Yoneda embedding
	\begin{align}
		Y: \mathsf{A} &\to
		\underline{\mathrm{Hom}\left( \mathsf{A}^{op}, \mathsf{Ab} \right)} \\
		A &\mapsto h^A = \mathrm{Hom}_{\mathsf{A}} \left( -, A \right)
	\end{align} 
	is fully faithful.
	Moreover it is left exact, since, for every $A$ the functor $h^A$ is left exact.
	In fact
	\begin{equation}
		Y: \mathsf{A} \to \mathsf{L} := \mathrm{Lex} \left(\mathsf{A}^{op}, \mathsf{Ab}\right) \subset
	\underline{\mathrm{Hom}\left( \mathsf{A}^{op}, \mathsf{Ab} \right)}
	\end{equation} 
	takes values in the category $\mathrm{Lex}\left(\mathsf{A}^{op}, \mathsf{Ab}\right)$ of
	left exact functors from $\mathsf{A}^{op}$ to $\mathsf{Ab}$.
	We need some facts about $\mathsf{L}$ (which are not trivial to show):
	\begin{enumerate}
		\item $\mathsf{L}$ is an abelian category.
			In particular its kernels coincide with the ones in
			$\underline{\mathrm{Hom}_{ }\left( \mathsf{A}^{op}, \mathsf{Ab} \right)}$,
			whereas cokernels differ.
			This implies that the inclusion functor $\mathsf{L} \hookrightarrow
			\underline{\mathrm{Hom}\left( \mathsf{A}^{op}, \mathsf{Ab} \right)}$
			is only left exact.
		\item The Yoneda embedding $Y: \mathsf{A} \to \mathsf{L}$ is fully faithful and exact.
		\item $\mathsf{L}$ has arbitrary coproducts, i.e. $\mathsf{L}$ is cocomplete,
			and has a projective generator, which is faithful as a functor, 
			namely
			\begin{equation}
			P := \coprod_{A \in \mathrm{Ob} \left(\mathsf{A}\right)} h^A
			.\end{equation} 
			Recall that we can take this coproduct since $\mathsf{A}$ is a small category,
			hence $\mathrm{Ob} \left(\mathsf{A}\right)$ is a set.
	\end{enumerate}
	Summarizing: $\mathsf{A}$ is a small abelian full subcategory of $\mathsf{L}$, which is a
	cocomplete abelian category with a projective generator.
	Then Freyd-Mitchell follows from the following theorem.
\end{proof}

\begin{thm}[]
	Let $\mathsf{C}$ be a cocomplete abelian category with a projective generator.
	Then, for every small full abelian category $\mathsf{A} \subset \mathsf{C}$,
	there is a ring $R$ and a fully faithful exact functor
	\begin{equation}
	F: \mathsf{A} \to \mathsf{Mod}\text{-}R
	,\end{equation} 
	so that $\mathsf{A}$ is equivalent to a full subcategory of $\mathsf{Mod}\text{-}R$.
\end{thm}

\begin{defn}[Functor reflecting exactness]
	Let $\mathsf{C}$ and $\mathsf{D}$ be abelian categories, and
	$F\colon \mathsf{C} \to \mathsf{D}$ be an {\em additive functor}.
	We say that $F$ {\em reflects exactness} iff
	\begin{equation}
	A \to B \to C
	\end{equation} 
	is exact in $\mathsf{C}$, as soon as
	\begin{equation}
		F(A) \to F(B) \to F(C)
	\end{equation} 
	is exact in $\mathsf{D}$.
\end{defn}

\begin{lem}
	If $F$ is an exact and fully faithful functor,
	then $F$ reflects exactness.
	(you can simplify things if you prove it using Freyd-Mitchell)
\end{lem} 

\begin{prop}
	Let $\mathsf{A}$ be a small abelian category.
	The Yoneda embedding
	\begin{equation}
		Y\colon \mathsf{A} \to \underline{\mathrm{Hom}\left( \mathsf{A}^{op}, \mathsf{Ab} \right)}
	\end{equation} 
	reflects exactness.
\end{prop} 

\begin{defn}[Acyclic complex]
	A (co)chain complex $\left( X^{\bullet}, d_{X} \right)$ is {\em acyclic} iff 
	$H^n(X) = 0$ for all $n \in \Z$, i.e. as a sequence it is exact
	\begin{equation}
	\ldots \to X^{n-1} \xrightarrow{d_X^{n-1}} X^n \xrightarrow{d_X^n} 
	X^{n+1} \xrightarrow{d_X^{n+1}} X^{n+2} \to \ldots
	.\end{equation} 
\end{defn}

\subsection{Homotopy category}
Let $\mathsf{A}$ be an additive category, and $X^{\bullet}, Y^{\bullet} \in \mathrm{Ch}(\mathsf{A})$.

\begin{defn}[Nullhomotopic morphism]
	A morphism $f \in \mathrm{Hom}_{\mathrm{Ch}(\mathsf{A})} \left( X^{\bullet}, Y^{\bullet} \right)$ is
	{\em nullhomotopic}, or {\em homotopic to zero}, iff there exists
	a family of morphism $\left\{ s^n \right\}_{n \in \mathbb{N}}$, with
	$s^n\colon X^n \to Y^{n-1}$, in pictures
	\begin{equation}
	\begin{tikzcd}
		\ldots \arrow[r, "", rightarrow] &
		X^{n-1} \arrow[r, "d_X^{n-1}", rightarrow] \arrow[d, "f^{n-1}"', rightarrow] &
		X^n \arrow[r, "d^n_X", rightarrow] \arrow[d, "f^n"', rightarrow] \arrow[ld, "s^n"', rightarrow] &
		X^{n+1} \arrow[r, "", rightarrow] \arrow[d, "f^{n+1}", rightarrow] \arrow[dl, "s^{n+1}"', rightarrow] &
		\ldots \\
		\ldots \arrow[r, "", rightarrow] &
		Y^{n-1} \arrow[r, "d_Y^{n-1}"', rightarrow] &
		Y^n \arrow[r, "d^n_Y"', rightarrow] &
		Y^{n+1} \arrow[r, "", rightarrow] &
		\ldots
	\end{tikzcd}
	\end{equation}
	such that $f^n = s^{n+1} \circ d_X^{n} + d_Y^{n-1} \circ s^n$ for all $n \in \Z$.
	More compactly we write $f = s \circ d_X + d_Y \circ s$.
	The morphisms $s^n$ are called {\em homotopies} or {\em cochain contractions}.
	Moreover, if $f$ is nullhomotopic, we write $f \sim 0$.
\end{defn}

\begin{defn}[Homotopic morphisms]
	Two cochain maps $f,g\colon \left( X^{\bullet}, d_{X} \right) \to \left( Y^{\bullet}, d_{Y} \right)$
	are called {\em homotopic}, denoted by $f \sim g$, iff
	$f - g$ is nullhomotopic.
\end{defn}

\begin{rem}[]
	The relation $\sim$ is an equivalence relation.
\end{rem}

\begin{defn}[Homotopy category]
	Given, as before, an additive category $\mathsf{A}$, we define the homotopy category
	$K(\mathsf{A})$ as follows.
	Its objects are exactly the objects in $\mathrm{Ch}(\mathsf{A})$.
	Its morphisms, instead, are equivalence classes of (co)chain maps,
	under the homotopy relation $\sim$ we just defined.
	More explicitly
	\begin{align}
		\mathrm{Hom}_{K(\mathsf{A})} \left( X^\bullet, Y^\bullet \right)
		&\simeq \mathrm{Hom}_{\mathrm{Ch}(\mathsf{A})}\left( X^\bullet, Y^\bullet \right)/\sim\\
		g &\mapsto \left[ g \right]_{\sim}
	.\end{align} 
\end{defn}

\begin{rem}[]
	The homotopy relation $\sim$ is compatible with addition, hence it is a congruence.
	In particular, denoted with
	$\mathrm{Hom}_{t}\left( X^\bullet, Y^\bullet \right) \subset
	\mathrm{Hom}_{\mathrm{Ch}(\mathsf{A})}\left( X^\bullet, Y^\bullet \right)$ the subgroup of
	nullhomotopic (co)chain maps, then
	\begin{equation}
		\mathrm{Hom}_{K(\mathsf{A})}\left( X^\bullet, Y^\bullet \right) =
		\frac{\mathrm{Hom}_{\mathrm{Ch}(\mathsf{A})}\left( X^\bullet, Y^\bullet \right)}{
		\mathrm{Hom}_{t}\left( X^\bullet, Y^\bullet \right)}
	.\end{equation} 
	Moreover, let $f,g\colon X^\bullet \to Y^\bullet$ be homotopic cochain maps.
	Let $\alpha\colon Z^\bullet \to X^\bullet$ and $\beta\colon Y^\bullet \to W^\bullet$ be cochain maps,
	then, by linearity of composition, we obtain
	$\beta \circ f \circ \alpha \sim \beta \circ g \circ \alpha$.
\end{rem}

\begin{prop}
	$K(\mathsf{A})$ is an additive category, and the quotient functor,
	defined
	\begin{align}
		q\colon \mathrm{Ch}(\mathsf{A}) &\to K(\mathsf{A}) \\
		X^\bullet &\mapsto X^\bullet\\
		f &\mapsto [f]_{\sim}
	\end{align} 
	is an additive functor.
\end{prop} 

\begin{defn}[Homotopy equivalence]
	A cochain map $f\colon \left( X^{\bullet}, d_{X} \right) \to \left( Y^{\bullet}, d_{Y} \right)$
	is said to be a {\em homotopy equivalence} iff
	$\exists\, g\colon \left( Y^{\bullet}, d_{Y} \right) \to \left( X^{\bullet}, d_{X} \right)$
	s.t.
	$g \circ f \sim 1_X$ and $f \circ g \sim 1_Y$.
	In other words a homotopy equivalence is an isomorphism in $K(\mathsf{A})$.
\end{defn}

\begin{prop}
	Let $\mathsf{A}$ be an abelian category and
	$f\colon \left( X^{\bullet}, d_{X} \right) \to \left( Y^{\bullet}, d_{Y} \right)$ be a
	nullhomotopic cochain map.
	Then the induced cohomology map
	\begin{equation}
		H^n(f) =: \overline{f^n}\colon H^n(X) \to H^n(Y)
	\end{equation} 
	is the zero map for every $n \in \Z$.
\end{prop} 
\begin{proof}
	We can use Freyd-Mitchell
	(this proposition deals with a finite number of objects and morphisms).
	Then, by definition
	\begin{equation}
		H^n(f) \left( x + \ima d_X^{n-1} \right) = f^n(x) + \ima d_Y^{n-1}
	.\end{equation} 
	But $f^n(x) = d^{n-1}_Y \circ s^n(x) + s^{n+1} \circ d_X^{n}(x)$,
	then
	\begin{equation*}
		H^n(f)(x) = d^{n-1}_Y \circ s^n (x) + \ima d_Y^{n-1} = 0 + \ima d_Y^{n-1}.\qedhere
	\end{equation*} 
\end{proof}

\begin{cor}
	Let $f$ and $g$ be homotopic maps, then
	\begin{equation}
		H^n(f) = H^n(g) \qquad \,\forall\, n \in \Z
	.\end{equation} 
\end{cor} 
\begin{proof}
	$H^n$ is an additive functor for each $n \in \Z$.
\end{proof}

\begin{rem}[]
	In general $\mathsf{A}$ abelian implies $\mathrm{Ch}(\mathsf{A})$ abelian,
	but not $K(\mathsf{A})$ abelian.
\end{rem}

\begin{defn}[Semisimple ring]
	A ring $R$ is called {\em semisimple} iff every $R$-module is projective.
	Equivalently iff every short exact sequence splits.
\end{defn}

\begin{ex}
	Any field $\K$ is semisimple, but $\Z$ is not.
	As a consequence of the following proposition, we get that $K(\mathsf{Mod}\text{-}\Z)$
	is not abelian.
\end{ex}

\begin{prop}
	The following statement (and more importantly the proof)
	should be incorrect. 
	Here what should be the correct one
	(I'm not going to copy the proof again, though).
	Let $\mathsf{A} \coloneqq \mathsf{Mod}\text{-}R$.
	If $K(\mathsf{A})$ is abelian, then
	$R$ is semisimple.
\end{prop}
\begin{prop}
	Let $\mathsf{A} \coloneqq \mathsf{Mod}\text{-}R$.
	If $R$ is not semisimple, then $K(\mathsf{A})$ is not abelian.
\end{prop}
\begin{proof}
	Assume that $R$ is not semisimple, but $K(\mathsf{Mod}\text{-}R)$ is abelian.
	Since $R$ is not semisimple, then there exists a short exact sequence
	\begin{equation}\label{eqn:sesAbHomCat}
	0 \to X \xrightarrow{f} Y \xrightarrow{\pi}
	Z \to 0 \qquad \text{ in } \mathsf{Mod}\text{-}R
	\end{equation} 
	which does not split.
	Consider now $X^\bullet, Y^\bullet, Z^\bullet$ as complexes concentrated in degree $0$.
	Since $K(\mathsf{Mod}\text{-}R)$ is abelian, then  $f\coloneqq q(f)$ has a cokernel
	And, by uniqueness up to isomorphism of the cokernel,
	we can assume that $\pi$ is a cokernel of $f$ in $K(\mathsf{Mod}\text{-}R)$.

	Consider the complex $\mathrm{Cone}\, f$:
	\begin{equation}
		0 \to X \xrightarrow{f} Y \to 0
	,\end{equation} 
	where $Y$ is in degree $0$.
	Let's define the cochain map $\alpha\colon Y^\bullet \to \mathrm{Cone}\, f$
	defined by
	\begin{equation}
	\begin{tikzcd}
		\ldots \arrow[r, "", rightarrow] &
		0 \arrow[r, "", rightarrow] \arrow[d, "0"', rightarrow] &
		0 \arrow[r, "", rightarrow] \arrow[d, "0"', rightarrow] &
		Y \arrow[r, "", rightarrow] \arrow[d, "1_Y", rightarrow] &
		0 \arrow[r, "", rightarrow] \arrow[d, "0", rightarrow] &
		\ldots \\
		\ldots \arrow[r, "", rightarrow] &
		0 \arrow[r, "", rightarrow] &
		X \arrow[r, "f"', rightarrow] &
		Y \arrow[r, "", rightarrow] &
		0 \arrow[r, "", rightarrow] & \ldots
	\end{tikzcd}
	.\end{equation} 
	Then we claim that there exist $\gamma, \delta$ s.t. $\alpha = \gamma \circ \pi$ and $\delta \circ \alpha = \pi$,
	i.e. s.t. the following diagram commutes.
	\begin{equation}
		\begin{tikzcd}
		X^\bullet \arrow[r, "f", rightarrow] &
		Y^\bullet \arrow[r, "\pi", rightarrow] \arrow[dr, "\alpha"', rightarrow] &
		Z^\bullet \arrow[d, "\gamma"', dashrightarrow, shift right = 0.25em] \\
		& &
		\mathrm{Cone}\, f \arrow[u, "\delta"', dashrightarrow, shift right = 0.25em]
	\end{tikzcd}
	.\end{equation} 
	At first we notice that $\alpha \circ f = 0 $ in $K(\mathsf{Mod}\text{-}R)$, in fact:
	\begin{equation}
	\begin{tikzcd}
		&
		0 \arrow[r, "", rightarrow] &
		X \arrow[r, "", rightarrow] \arrow[d, "f", rightarrow] \arrow[ddl, "1_X"' near start, dashrightarrow, crossing over] &
		0 \arrow[ldd, "0" near end, dashrightarrow, crossing over]\\
		&
		0 \arrow[r, "", rightarrow] &
		Y \arrow[r, "", rightarrow] \arrow[d, "1_Y"', rightarrow] &
		0 \\
		0 \arrow[r, "", rightarrow] &
		X \arrow[r, "f"', rightarrow] &
		Y \arrow[r, "", rightarrow] &
		0
	\end{tikzcd}
	.\end{equation} 
	Since $\pi$ is a cokernel of $f$, then $\exists\, !\, \gamma\colon Z^\bullet \to \mathrm{Cone}\, f$ s.t. $\gamma \circ \pi = \alpha$.
	With regard to $\delta\colon \mathrm{Cone}\, f \to Z^\bullet$, instead, we define $(0, \pi)$, i.e.
	the family of maps which all correspond to zero, apart from degree $0$, in which it is $\pi$.
	Then $\delta \circ \alpha = \pi$, as described by the following diagram
	\begin{equation}
	\begin{tikzcd}
		0 \arrow[r, "", rightarrow] \arrow[d, "0", rightarrow] &
		Y \arrow[r, "", rightarrow] \arrow[d, "1_Y", rightarrow] &
		0 \arrow[d, "0", rightarrow] \\
		X \arrow[r, "f", rightarrow] \arrow[d, "0", rightarrow] &
		Y \arrow[r, "", rightarrow] \arrow[d, "\pi", rightarrow] &
		0 \arrow[d, "0", rightarrow] \\
		0 \arrow[r, "", rightarrow] &
		Z \arrow[r, "", rightarrow] &
		0
	\end{tikzcd}
	.\end{equation} 
	Then we have $\pi = \delta \circ \alpha = \delta \circ \gamma \circ \pi$.
	Since $\pi$ is epi (it is a cokernel), we obtain that
	$\delta \circ \gamma = id_Z$ in $K(\mathsf{Mod}\text{-}R)$.
	But then, if we denote by $\gamma_0\colon Z \to Y$ the morphism in degree $0$ of $\gamma$, we obtain
	that $\pi \circ \gamma_0 = 1_Z$, hence we have found a retraction of $\pi$ in \eqref{eqn:sesAbHomCat}.
	This is a contradiction, since we assumed it did not split.
\end{proof}

\subsection{Snake lemma and applications}
\begin{lem}
	Let $\mathsf{A}$ be an abelian category, and let
	\begin{equation}
	\begin{tikzcd}
		&
		A \arrow[r, "\alpha", rightarrow] \arrow[d, "f"', rightarrow] &
		B \arrow[r, "\beta", rightarrow] \arrow[d, "g"', rightarrow] &
		C \arrow[r, "", rightarrow] \arrow[d, "h"', rightarrow] &
		0 \\
		0 \arrow[r, "", rightarrow] &
		A' \arrow[r, "\alpha'", rightarrow] &
		B' \arrow[r, "\beta'", rightarrow] &
		C' &
	\end{tikzcd}
	\end{equation} 
	be a commutative diagram with exact rows.
	Then there is an exact sequence:
	\begin{equation}
	\ker f \xrightarrow{\underline{\alpha}} \ker g \xrightarrow{\underline{\beta}}
	\ker h \xrightarrow{\partial} \coker f \xrightarrow{\overline{\alpha'}} 
	\coker g \xrightarrow{\overline{\beta'}} \coker h
	,\end{equation} 
	in which $\partial$ is called the {\em connecting morphism}.
	Moreover $\alpha$ mono implies $\underline{\alpha}$ is mono,
	whereas $\beta'$ epi implies $\overline{\beta'}$ is epi.
\end{lem} 	

\begin{rem}[Short exact sequences in the category of complexes]
	Since the abelian structure of $\mathrm{Ch}(\mathsf{A})$ is defined degree wise, we have
	that a sequence in $\mathrm{Ch}(\mathsf{A})$
	\begin{equation}
	0 \to X^\bullet \xrightarrow{f} Y^\bullet \xrightarrow{g} 
	Z^\bullet \to 0
	\end{equation} 
	is exact in $\mathrm{Ch}(\mathsf{A})$ iff, for every $n \in \Z$, the corresponding
	\begin{equation}
	0 \to X^n \xrightarrow{f^n} Y^n \xrightarrow{g^n}
	Z^n \to 0
	\end{equation} 
	is exact in $\mathsf{A}$.
\end{rem}

\begin{thm}[Fundamental theorem in (co)homology]
	Consider a short exact sequence in $\mathrm{Ch}(\mathsf{A})$, for an abelian category $\mathsf{A}$,
	\begin{equation}
	0 \to X^\bullet \xrightarrow{f} Y^\bullet \xrightarrow{g} W^\bullet \to 0
	.\end{equation} 
	Then we can associate to it a long exact sequence in $\mathsf{A}$, called the
	{\em long exact sequence in (co)homology}, given as follows:
	\begin{equation}
		\ldots \to H^n(X^\bullet) \xrightarrow{H^n(f)} H^n(Y^\bullet) \xrightarrow{H^n(g)} 
		H^n (W^\bullet) \xrightarrow{\partial} H^{n+1}(X^\bullet) \to
		H^{n+1}(Y^\bullet) \to \ldots
	,\end{equation} 
\end{thm}
\begin{proof}
	The proof is essentially an application of the snake lemma.
	In particular we obtain that $\partial\colon H^n(W^\bullet) \to H^{n+1}(X^\bullet)$ acts as
	\begin{align}
		\partial\colon H^n(W^\bullet) &\to H^{n+1}(X^\bullet) \\
		[z^n] &\mapsto \left[ (f^{n+1})^{-1} \left( d_Y^n ((g^n)^{-1}(z^n)) \right) \right]
	.\end{align} 
	More visually it is defined by the following diagram chase:
	\begin{equation}
	\begin{tikzcd}
		&
		Y^n \arrow[r, "g^n", rightarrow] \arrow[d, "d_Y^n"', rightarrow] &
		Z^n \arrow[d, "0", rightarrow] \\
		X^{n+1} \arrow[r, "f^{n+1}", rightarrow] \arrow[d, "d_X^{n+1}"', rightarrow] &
		Y^{n+1} \arrow[r, "g^{n+1}", rightarrow] \arrow[d, "d_Y^{n+1}", rightarrow] &
		0 \\
		0 = X^{n+2} \arrow[r, "f^{n+2}", rightarrow] &
		0
	\end{tikzcd}
	.\end{equation} 
\end{proof}

\begin{rem}[Notation]
	We denote by $Z^n(X^\bullet) \coloneqq \ker d_X^n$, the $n$-cycles,
	and by $B^n(X^\bullet) \coloneqq \ima d_X^{n-1}$, the $n$-boundaries.
	Both clearly are subobjects of $X^n$.
\end{rem}

\begin{defn}[Long/short exact sequence category]
	Let $\mathsf{A}$ be an abelian category.
	\begin{itemize}
		\item We define $\mathsf{S}$, the category of short exact sequences
	in $\mathrm{Ch}(\mathsf{A})$, as the category whose objects
	are short exact sequences with objects in $\mathrm{Ob} \left(\mathrm{Ch}(\mathsf{A})\right)$
	and whose morphisms, called morphisms of short exact sequences,
	are triples $(f,g,h)$ of cochain maps such that
	the following diagram commutes
	\begin{equation}
	\begin{tikzcd}
		0 \arrow[r, "", rightarrow] &
		A^\bullet \arrow[r, "\alpha", rightarrow] \arrow[d, "f"', rightarrow] &
		B^\bullet \arrow[r, "\beta", rightarrow] \arrow[d, "g"', rightarrow] &
		C^\bullet \arrow[r, "", rightarrow] \arrow[d, "h"', rightarrow] &
		0 \\
		0 \arrow[r, "", rightarrow] &
		X^\bullet \arrow[r, "\alpha'"', rightarrow] &
		Y^\bullet \arrow[r, "\beta'"', rightarrow] &
		W^\bullet \arrow[r, "", rightarrow] &
		0 
	\end{tikzcd}
	.\end{equation} 
		\item We define $\mathsf{L}$, the category of long exact sequences
	in $\mathsf{A}$, as the category whose objects
	are exact sequences in $\mathrm{Ob} \left(\mathrm{Ch}\mathsf{A}\right)$
	and whose morphisms are morphisms of complexes, i.e.
	families of maps $\left\{ f^n \right\}_{n \in \Z}$ making
	the following diagram commute
	\begin{equation}
	\begin{tikzcd}
		\ldots \arrow[r, "", rightarrow] &
		A^n \arrow[r, "d_A^n", rightarrow] \arrow[d, "f^n"', rightarrow] &
		A^{n+1} \arrow[r, "d_A^{n+1}", rightarrow] \arrow[d, "f^{n+1}"', rightarrow] &
		A^{n+2} \arrow[r, "d_A^{n+2}", rightarrow] \arrow[d, "f^{n+2}"', rightarrow] &
		\ldots \\
		\ldots \arrow[r, "", rightarrow] &
		B^n \arrow[r, "d_B^n"', rightarrow] &
		B^{n+1} \arrow[r, "d_B^{n+1}"', rightarrow] &
		B^{n+2} \arrow[r, "d_B^{n+2}"', rightarrow] &
		\ldots 
	\end{tikzcd}
	.\end{equation} 
	\end{itemize}
\end{defn}

\begin{prop}
	Given an abelian category $\mathsf{A}$, then we can define a functor
	\begin{align}
		L\colon \mathsf{S} &\to \mathsf{L}
	,\end{align} 
	that, on objects, maps each short exact sequence of complexes to its corresponding exact
	sequence in (co)homology.
	In particular a given morphism in $\mathsf{S}$
	\begin{equation}
	\begin{tikzcd}
		0 \arrow[r, "", rightarrow] &
		A^\bullet \arrow[r, "\alpha", rightarrow] \arrow[d, "f"', rightarrow] &
		B^\bullet \arrow[r, "\beta", rightarrow] \arrow[d, "g"', rightarrow] &
		C^\bullet \arrow[r, "", rightarrow] \arrow[d, "h"', rightarrow] &
		0 \\
		0 \arrow[r, "", rightarrow] &
		X^\bullet \arrow[r, "\alpha'"', rightarrow] &
		Y^\bullet \arrow[r, "\beta'"', rightarrow] &
		W^\bullet \arrow[r, "", rightarrow] &
		0 
	\end{tikzcd}
	\end{equation} 
	gets mapped to the following morphism of long exact sequences, in $\mathsf{L}$ 
	\begin{equation*}
	\begin{tikzcd}[column sep = 2.4em]
		\ldots \arrow[r, "", rightarrow] &
%		H^n(A^\bullet) \arrow[r, "H^n(\alpha)", rightarrow] \arrow[d, "H^n(f)"', rightarrow] &
		H^n(B^\bullet) \arrow[r, "H^n(\beta)", rightarrow] \arrow[d, "H^n(g)"', rightarrow] &
		H^n(C^\bullet) \arrow[r, "\partial_1^n", rightarrow] \arrow[d, "H^n(h)"', rightarrow] 
		\arrow[rd, "\circlearrowright" description, phantom, rightarrow] &
		H^{n+1}(A^\bullet) \arrow[r, "H^{n+1}(\alpha)", rightarrow] \arrow[d, "H^{n+1}(f)", rightarrow] &
		H^{n+1}(B^\bullet) \arrow[r, "H^{n+1}(\beta)", rightarrow] \arrow[d, "H^{n+1}(g)", rightarrow] &
		\ldots \\
		\ldots \arrow[r, "", rightarrow] &
%		H^n(X^\bullet) \arrow[r, "H^n(\alpha')"', rightarrow] &
		H^n(Y^\bullet) \arrow[r, "H^n(\beta')"', rightarrow] &
		H^n(W^\bullet) \arrow[r, "\partial_2^n"', rightarrow] &
		H^{n+1}(X^\bullet) \arrow[r, "H^{n+1}(\alpha')"', rightarrow] &
		H^{n+1}(Y^\bullet) \arrow[r, "H^{n+1}(\beta')"', rightarrow] &
		\ldots 
	\end{tikzcd}
	.\end{equation*} 
	In particular also the squares involving the connecting morphisms $\partial^n$ commute,
	in other words we have $H^{n+1}(f) \circ \partial_1^n = \partial_2^n \circ H^n(h)$.
\end{prop} 

\begin{rem}[Notation]
	The long exact (co)homology sequence associated to
	\begin{equation}
	0 \to A^\bullet \to B^\bullet \to C^\bullet \to 0
	\end{equation} 
	can be visualized by the following diagram, called the exact triangle
	\begin{equation}
	\begin{tikzcd}[column sep=tiny]
		H^{\bullet}(A^\bullet) \arrow[rr, "", rightarrow] & &
		H^{\bullet}(B^\bullet) \arrow[dl, "", rightarrow] \\
		&
		H^{\bullet}(C^\bullet) \arrow[lu, "\partial", rightarrow] &
	\end{tikzcd}
	.\end{equation} 
\end{rem}

\begin{lem}[$3 \cross 3$ lemma]
	Let $\mathsf{A}$ be an abelian category.
	Consider the following commutative diagram with exact columns
	\begin{equation}
	\begin{tikzcd}
		0 \arrow[r, "", rightarrow] &
		A_1 \arrow[r, "", rightarrow] \arrow[d, "", tail] &
		B_1 \arrow[r, "", rightarrow] \arrow[d, "", tail] &
		C_1 \arrow[r, "", rightarrow] \arrow[d, "", tail] &
		0 \\
		0 \arrow[r, "", rightarrow] &
		A_2 \arrow[r, "", rightarrow] \arrow[d, "", two heads] &
		B_2 \arrow[r, "", rightarrow] \arrow[d, "", two heads] &
		C_2 \arrow[r, "", rightarrow] \arrow[d, "", two heads] &
		0 \\
		0 \arrow[r, "", rightarrow] &
		A_3 \arrow[r, "", rightarrow] &
		B_3 \arrow[r, "", rightarrow] &
		C_3 \arrow[r, "", rightarrow] &
		0 \\
	.\end{tikzcd}
	\end{equation} 
	\begin{enumerate}
		\item If the $2$nd and $3$rd rows are exact, then so is the $1$st.
		\item If the $1$st and $2$nd rows are exact, then so is the $3$rd.
		\item If the $1$st and $3$rd rows are exact, and the $2$nd is a complex,
			then the $2$nd is also exact.
	\end{enumerate}
\end{lem} 

\begin{lem}[$5$ lemma]
	Let $\mathsf{A}$ be an abelian category.
	Consider the following commutative diagram with exact rows
	\begin{equation}
	\begin{tikzcd}
		A_1 \arrow[r, "", rightarrow] \arrow[d, "a", rightarrow] &
		B_1 \arrow[r, "", rightarrow] \arrow[d, "b", rightarrow] &
		C_1 \arrow[r, "", rightarrow] \arrow[d, "c", rightarrow] &
		D_1 \arrow[r, "", rightarrow] \arrow[d, "d", rightarrow] &
		E_1 \arrow[d, "e", rightarrow] \\
		A_2 \arrow[r, "", rightarrow] &
		B_2 \arrow[r, "", rightarrow] &
		C_2 \arrow[r, "", rightarrow] &
		D_2 \arrow[r, "", rightarrow] &
		E_2 
	.\end{tikzcd}
	\end{equation} 
	\begin{enumerate}
		\item If $b$ and $d$ are mono and $a$ is epi, then $c$ is mono.
		\item If $b$ and $d$ are epi and $e$ is mono, then $c$ is epi.
	\end{enumerate}
\end{lem} 

\begin{defn}[quasi-isomorphism]
	Let $\mathsf{A}$ be an abelian category.
	Let $f\colon \left( X^{\bullet}, d_{X} \right) \to \left( Y^{\bullet}, d_{Y} \right)$ be a cochain map in $\mathrm{Ch}(\mathsf{A})$.
	We say that $f$ is a {\em quasi-isomorphism} iff the induced cohomology morphism
	\begin{equation}
		H^n(f)\colon H^n(X^\bullet) \to H^n(Y^\bullet)
	\end{equation} 
	is an isomorphism for every $n \in \Z$.
\end{defn}

\begin{lem}
	An homotopy equivalence $f\colon X^\bullet \to Y^\bullet$, i.e.
	an iso in $K(\mathsf{A})$, is a {\em quasi-isomorphism}.
\end{lem} 

\begin{lem}
	One can find examples of quasi-isomorphism, which is not an homotopy equivalence.
	(Look at morphisms of exact sequences).
\end{lem} 

\begin{lem}
	Let $\mathsf{A}$ be an abelian category and consider
	$\left( X^{\bullet}, d_{X} \right) \in \mathrm{Ch}(\mathsf{A})$.
	Define $\left( Z^{\bullet}, d_{Z} \right)$ by:
	\begin{equation}
		Z^n \coloneqq Z^n(X^\bullet) \coloneqq \ker d_X^n \qquad \text{ and } \qquad
		d^n_{Z} = 0 \qquad \,\forall\, n \in \Z
	.\end{equation} 
	Analogously define the complex $\left( B^{\bullet}, d_{B} \right)$ by
	\begin{equation}
		B^n \coloneqq B^n(X^\bullet) \coloneqq \ima d_X^{n-1} \qquad \text{ and } \qquad
		d^n_{B} = 0 \quad \,\forall\, n \in \Z
	.\end{equation} 
	Then there is a short exact sequence of complexes
	\begin{equation}
	0\to Z^\bullet \to X^\bullet \to
	B^\bullet[1] \to 0
	,\end{equation} 
	whose associated long exact sequence breaks into short exact sequences in $\mathsf{A}$.
\end{lem} 
\begin{proof}
	Apart from the exactness of the short exact sequence of complexes, notice that:
	$H^n(Z^\bullet) = Z^n$ and $H^n(B^\bullet[1]) = H^{n+1}(B^\bullet) = B^{n+1}$ for all $n \in \Z$.
	Then the associated long exact sequence is
	\begin{equation}
		\ldots B^n \xrightarrow{\partial} Z^n \to H^n(X^\bullet) \to B^{n+1} \xrightarrow{\partial}
		Z^{n+1} \to H^{n+1}(X^\bullet) \to \ldots
	.\end{equation} 
	But, for each $n$, the above breaks into the short exact sequences
	\begin{equation*}
		0 \to B^n \to Z^n \to H^n(X^\bullet) \to 0.\qedhere
	\end{equation*} 
\end{proof}

\begin{lem}
	Let $f\colon \left( X^{\bullet}, d_{X} \right) \to \left( Y^{\bullet}, d_{Y} \right)$
	be a cochain map in $\mathrm{Ch}(\mathsf{A})$,
	for an abelian category $\mathsf{A}$.
	Assume that $\left(\ker f^\bullet, d^\bullet\right)$ and $\left(\coker f^\bullet, d^\bullet\right)$
	are acyclic.
	Then $f$ is a quasi-isomorphism.
\end{lem} 

\begin{rem}[]
	Notice that the converse of the above lemma is not true:
	for example the complexes
	\begin{equation}
	X^\bullet = Y^\bullet = 0 \to \Z \xrightarrow{\dot{2}} 
	\Z \xrightarrow{\pi} \mathbb{Z}/2\mathbb{Z} \to 0
	\end{equation} 
	are both acyclic.
	Then the map $f = (\dot{4}, \dot{4}, 0)$, represented by
	\begin{equation}
	\begin{tikzcd}
		0 \arrow[r, "", rightarrow] &
		\Z \arrow[r, "\dot{2}", rightarrow] \arrow[d, "\dot{4}"', rightarrow] &
		\Z \arrow[r, "\pi", rightarrow] \arrow[d, "\dot{4}"', rightarrow] &
		\mathbb{Z}/2\mathbb{Z} \arrow[r, "", rightarrow] \arrow[d, "0"', rightarrow] &
		0 \\
		0 \arrow[r, "", rightarrow] &
		\Z \arrow[r, "\dot{2}"', rightarrow] &
		\Z \arrow[r, "\pi"', rightarrow] &
		\mathbb{Z}/2\mathbb{Z} \arrow[r, "", rightarrow] &
		0 
	\end{tikzcd}
	\end{equation} 
	is a quasi-isomorphism.
	Moreover the cochains $(\ker f)^\bullet$ and $(\coker f)^\bullet$ are
	\begin{align}
		(\ker f)^\bullet &= 0 \to 0 \to 0 \to \mathbb{Z}/2\mathbb{Z} \to 0\\
		(\coker f)^\bullet &= 0 \to \mathbb{Z}/4\mathbb{Z} \xrightarrow{\dot{2}} \mathbb{Z}/4\mathbb{Z}
		\xrightarrow{\pi} \mathbb{Z}/2\mathbb{Z} \to 0
	.\end{align} 
	Then we can compute that $H^2((\ker f)^\bullet) = \mathbb{Z}/2\mathbb{Z}$
	and $H^0((\coker f)^\bullet) = \mathbb{Z}/2\mathbb{Z}$.
\end{rem}

\subsection{Operation on complexes}
\begin{defn}[Canonical truncation]
	Let $\left( X^{\bullet}, d_{X} \right)$ be a cochain complex and $n \in \Z$.
	We define the \textbf{canonical truncation} of $\left( X^{\bullet}, d_{X} \right)$ to be the complex
	$\left( [\tau_{\leq n}(X^\bullet)]^{\bullet}, d_{[\tau_{\leq n}(X^\bullet)]} \right)$, whose objects are
	\begin{equation}
		[\tau_{\leq n}(X^\bullet)]^i := 
	\begin{cases}
		X^i & \text{ if } i < n\\
		\ker d_X^n & \text{ if } i = n\\
		0 & \text{ if } i > n\\
	\end{cases} 
	,\end{equation} 
	and differentials given by the induced ones.
	Denoted by $\epsilon^n: \ker d^n_X \to X^n$ the Kernel, then we have a
	natural cochain map $\epsilon: \tau_{\leq n}(X^\bullet) \to X^\bullet$, given by
	\begin{equation}
	\begin{tikzcd}
		\ldots \arrow[r, "", rightarrow] &
		X^{n-2} \arrow[r, "d^{n-2}", rightarrow] \arrow[d, "1_{X^{n-2}}"', rightarrow] &
		X^{n-1} \arrow[r, "d^{n-1}", rightarrow] \arrow[d, "1_{X^{n-1}}"', rightarrow] &
		Z^{n}(X^\bullet) \arrow[r, "0", rightarrow] \arrow[d, "\epsilon^{n}"', rightarrow] &
		0 \arrow[r, "0", rightarrow] \arrow[d, "0"', rightarrow] &
		\ldots \\
		\ldots \arrow[r, "", rightarrow] &
		X^{n-2} \arrow[r, "d^{n-2}"', rightarrow] &
		X^{n-1} \arrow[r, "d^{n-1}"', rightarrow] &
		X^{n} \arrow[r, "d^{n}"', rightarrow] &
		X^{n+1} \arrow[r, "d^{n+1}"', rightarrow] &
		\ldots \\
	\end{tikzcd}
	,\end{equation} 
	which is clearly a mono.
	Moreover we can compute the associated cohomology groups (assuming $\mathsf{A}$ is abelian, or that we can compute them)
	and they are
	\begin{equation}
		H^i \left( \tau_{\leq n}(X^\bullet) \right) =
		\begin{cases}
			0 & \text{ if } i > n\\
			H^i(X^\bullet) & \text{ if } i \leq n\\
		\end{cases} 
	.\end{equation} 
	Moreover, since $\epsilon$ is an ambedding, we can define the quotient complex
	$\left( [X^\bullet/\tau_{\leq n}(X^\bullet)]^{\bullet}, d_{[X^\bullet/\tau_{\leq n}(X^\bullet)]} \right)$,
	whose objects are
	\begin{equation}
		[X^\bullet/\tau_{\leq n}(X^\bullet)]^i := 
	\begin{cases}
		X^i & \text{ if } i > n\\
		X^n/\ker d_X^n & \text{ if } i = n\\
		0 & \text{ if } i < n\\
	\end{cases} 
	,\end{equation} 
	and differentials given by the induced one.
	Then, as expected
	\begin{equation}
		H^i \left( X^\bullet/\tau_{\leq n}(X^\bullet) \right) =
		\begin{cases}
			0 & \text{ if } i \leq n\\
			H^i(X^\bullet) & \text{ if } i > n\\
		\end{cases} 
	.\end{equation} 
	And we obtain a s.e.s. of complexes
	\begin{equation}
		0 \to\tau_{\leq n}(X^\bullet) \xrightarrow{\epsilon} X^\bullet
		\twoheadrightarrow X^\bullet/\tau_{\leq n}(X^\bullet) \to 0
	.\end{equation} 
\end{defn}

\begin{defn}[Stupid truncation]
	Given, as before, a cochain complex $\left( X^{\bullet}, d_{X} \right)$ and $n \in \Z$,
	one defines its	\textbf{stupid truncation} as the cochain complex with objects
	\begin{equation}
		[\sigma_{\leq n}(X^\bullet)]^i =
		\begin{cases}
			X^i & \text{ if } i \leq n\\
			0 & \text{ if } i > n
		\end{cases} 
	\end{equation} 
	and induced differentials.
	Then one can construct a canonical map $X^\bullet \to \sigma_{\leq n}(X^\bullet)$ as
	\begin{equation}
	\begin{tikzcd}
		\ldots \arrow[r, "", rightarrow] &
		X^{n-1} \arrow[r, "", rightarrow] \arrow[d, "1_{X^{n-1}}", rightarrow] &
		X^{n} \arrow[r, "", rightarrow] \arrow[d, "1_{X^{n}}", rightarrow] &
		X^{n+1} \arrow[r, "", rightarrow] \arrow[d, "0", rightarrow] &
		\ldots\\
		\ldots \arrow[r, "", rightarrow] &
		X^{n-1} \arrow[r, "", rightarrow] &
		X^{n} \arrow[r, "0"', rightarrow] &
		0 \arrow[r, "", rightarrow] &
		\ldots
	\end{tikzcd}
	.\end{equation} 
	Moreover we can compute its cohomology groups, and obtain that they are
	\begin{equation}
		H^i \left( \sigma_{\leq n}(X^\bullet) \right) =
		\begin{cases}
			0 & \text{ if } i > n\\
			X^n/\ima d_X^{n-1} & \text{ if } i = n\\
			H^i(X^\bullet) & \text{ if } i < n
		\end{cases} 
	.\end{equation} 
\end{defn}

\begin{defn}[Mapping cone]
	Let $f \in \mathrm{Hom}_{\mathrm{Ch}(\mathsf{A})} \left( X^\bullet, Y^\bullet \right)$ an arbitrary cochain map.
	We define the \textbf{mapping cone} of $f$ as the cochain complex,
	denoted by $(\mathrm{Cone}\, f)^\bullet$, whose objects are
	\begin{equation}
		[\mathrm{Cone}\, f]^n := Y^n \oplus X^{n+1}
	\end{equation} 
	and differentials $d^n_{\mathrm{Cone}\, f}: Y^n \oplus X^{n+1} \to Y^{n+1} \oplus X^{n+2}$ given by
	the following matrix
	\begin{equation}
	d^n_{\mathrm{Cone}\, f} :=
	\begin{bmatrix}
		d^n_Y & f^{n+1}\\
		0 & -d_X^{n+1}
	\end{bmatrix} 
	.\end{equation} 
	This really is a complex, since we have the identity
	\begin{equation}
	d^2_{\mathrm{Cone}\, f} =
	\begin{bmatrix}
		d^n_Y & f^{n+1}\\
		0 & -d_X^{n+1}
	\end{bmatrix} 
	\begin{bmatrix}
		d^{n-1}_Y & f^{n}\\
		0 & -d_X^{n}
	\end{bmatrix}  = 
	\begin{bmatrix}
		0 & d^n_Y f^n - f^{n+1} d^n_X\\
		0 & 0
	\end{bmatrix} 
	\end{equation} 
	and $f$ is a cochain map (hence the last matrix is zero).
\end{defn}

\begin{defn}[Cone of a complex]
	Given a complex $\left( X^{\bullet}, d_{X} \right)$, we define the cocahin $(\mathrm{Cone}\, X)^\bullet$ as the
	mapping cone of the cochain map $1_{X^\bullet}: X^\bullet \to X^\bullet$.
\end{defn}


\begin{rem}[]
	From the definition of mapping cone we obtain the short exact sequence of complexes
	\begin{equation}\label{eqn:ConeSes}
	0 \to Y^\bullet \xrightarrow{\alpha}
	(\mathrm{Cone}\, f)^\bullet \xrightarrow{\beta} X^\bullet[1] \to 0
	,\end{equation} 
	where the maps $\alpha$ and $\beta$ (check they are indeed cochain maps) are defined by the matrices
	\begin{equation}
	\alpha := 
	\begin{bmatrix}
		1_Y \\ 0
	\end{bmatrix} \qquad \text{ and } \qquad
	\beta := 
	\begin{bmatrix}
		0 & 1_{X^\bullet[1]}
	\end{bmatrix} 
	.\end{equation} 
	In particular, for each degree the s.e.s. splits, in fact it is
	\begin{equation}
	0 \to Y^n \to Y^n \oplus X^{n+1} \to X^{n+1} \to 0
	,\end{equation} 
	and the maps are induced by $\alpha$ and $\beta$ (hence the splitting).
\end{rem}

\begin{lem}
	Let the following be a s.e.s. sequence
	\begin{equation}
	0 \to Y^\bullet \to C^\bullet \to W^\bullet \to 0
	\end{equation} 
	s.t. it is degree-wise splitting.
	Then there is a cochain map $f: W^\bullet[-1] \to Y^\bullet$ s.t.
	$C^\bullet \simeq \mathrm{Cone}\, f$
\end{lem} 

\begin{lem}
	Let $f: X^\bullet \to Y^\bullet$ be a cochain map in $\mathrm{Ch}(\mathsf{A})$.
	Then $f$ is a quasi-isomorphism iff the complex
	$(\mathrm{Cone}\, f)^\bullet$ is acyclic.
\end{lem} 
\begin{proof}
	From the s.e.s. for the Cone of $f$, see \eqref{eqn:ConeSes}, and the 
	fundamental theorem in cohomology, one obtains the long exact cohomology sequence
	\begin{equation}
		\ldots \to H^{n-1}(X^\bullet[1]) \xrightarrow{\partial^n} H^n(Y^\bullet) \to
		H^n(\mathrm{Cone}\, f) \to H^n(X^\bullet[1]) \to \ldots
	.\end{equation} 
	One can show that $H^n(f) = \partial^n$, then
	$H^n(f)$ is an iso iff $H^n(\mathrm{Cone}\, f) = 0$.
\end{proof}

\begin{defn}[Split complex]
	A complex $\left( X^{\bullet}, d_{X} \right)$ is \textbf{split} iff there exist
	maps $s^n: X^{n+1} \to X^n$, for all $n \in \Z$, s.t.
	$d^n_X \circ s^n \circ d^n_X = d^n_X$ for all $n \in \Z$ (shortly $d = d \circ s \circ d$).
	The maps $s^n$ are called \textit{splitting maps}.
\end{defn}

\begin{lem}
	Let $\left( X^{\bullet}, d_{X} \right)$ be a complex, with cycles $Z^n$ and
	boundaries $B^n$.
	$X^\bullet$ is split iff, for every $n \in \Z$, there
	exist decompositions
	\begin{equation}
	X^n = Z^n \oplus C^n \qquad \text{ and } \qquad
	Z^n = B^n \oplus K^n
	,\end{equation} 
	with $K^n \simeq H^n(X^\bullet)$.
\end{lem} 
\begin{proof}
	In this proof we use the general fact, for $R$-modules, that given an idempotent
	endomorphism $e: M \to M$ (i.e. s.t. $e^2 = e$), then
	\begin{equation}
	M = \ker e \oplus \ima e
	.\end{equation} 
	In fact for any $x \in M$, then $x = e(x) + (x - e(x))$ and
	$e(x - e(x)) = 0$.
	Moreover, given $x \in \ker e \cap \ima e$, there exists $y$ s.t. $x = e(y)$, then
	\begin{equation}
		0 = e(x) = e(e(y)) = e(y) = x
	.\end{equation} 
\end{proof}

\begin{defn}[Split exact/contractible complex]
	A complex $\left( X^{\bullet}, d_{X} \right)$ is called \textbf{split exact} or \textbf{contractible} iff
	it is both \textit{split} and \textit{acyclic} (i.e. exact).
\end{defn}

\begin{rem}[]
	By the above lemma, the complex $\left( X^{\bullet}, d_{X} \right)$ is contractible iff
	there exist decompositions $X^n = Z^n \oplus C^n$ and $B^n = Z^n$, for every $n \in \Z$.
\end{rem}

\begin{lem}
	$(\mathrm{Cone}\, X)^\bullet$ is contractible.
\end{lem} 
\begin{proof}
	$\left( \mathrm{Cone}\, C \right)^\bullet$ is exact, since $1_{X^\bullet}$ is a quasi-isomorphism.
	Then we define the splitting maps by
	\begin{equation}
	s^n :=
	\begin{bmatrix}
		0 & 0\\
		1_{X^{n+1}} & 0
	\end{bmatrix} 
	.\end{equation} 
\end{proof}

\begin{lem}
	A complex $\left( X^{\bullet}, d_{X} \right)$ is contractible iff
	$1_{X^\bullet}$ is nullhomotopic.
\end{lem}
\begin{rem}[]
	This lemma can be stated as: any contractible complex is isomorphic to
	the $0$ complex in the homotopy category.
\end{rem}

\begin{lem}
	Let $f: X^\bullet \to Y^\bullet$ be a cochain map.
	Prove that $f \sim 0$ iff $f$ extends to
	\begin{equation}
		\begin{bmatrix}
			f & s
		\end{bmatrix} 
		: \mathrm{Cone}\, X \to Y
	,\end{equation} 
	where $\left\{ s^n \right\}_{n \in \Z}$ are the contractions.
\end{lem} 

\begin{lem}
	Let $\left( X^{\bullet}, d_{X} \right)$ be a split complex
	with splitting maps $\left\{ s^n \right\}_{n \in \Z} =: s$.
	Then $f = s \circ d + d \circ s$ is a cochain map (clearly, then $f \sim 0$).
\end{lem} 

\begin{rem}[]
	for all $A \in \mathrm{Ob} \left(\mathsf{A}\right)$, we define the following complex
	\begin{equation}
		D^n(A) := 0 \to A \xrightarrow{1_A} A \to 0
	,\end{equation} 
	where the non-zero elements are in degree $n$ and $n+1$.
	Clearly $D^n(A)$ contractible. In fact:
	\begin{equation}
	\begin{tikzcd}
		0 \arrow[r, "", rightarrow] &
		A \arrow[r, "1_A", rightarrow] \arrow[ld, "0"', rightarrow] \arrow[d, "1_A"', rightarrow] &
		A \arrow[r, "", rightarrow] \arrow[ld, "1_A"', rightarrow] \arrow[d, "1_A", rightarrow] &
		0 \arrow[ld, "0", rightarrow] \\
		0 \arrow[r, "", rightarrow] &
		A \arrow[r, "1_A"', rightarrow] &
		A \arrow[r, "", rightarrow] &
		0 
	\end{tikzcd}
	.\end{equation} 
\end{rem}

\begin{lem}
	for all $A \in \mathrm{Ob} \left(\mathsf{A}\right)$ and $X^\bullet \in \mathrm{Ch}(\mathsf{A})$ we have
	\begin{equation}
		\mathrm{Hom}_{\mathrm{Ch}(\mathsf{A})} \left( D^n(A), X^\bullet \right) \simeq
		\mathrm{Hom}_{\mathsf{A}} \left( A, X^n \right)
	.\end{equation} 
	In other words the pair $(D^n, (-)^n)$ is an adjoint pair for every $n \in \Z$, for the functors
	\begin{align}
		D^n: \mathsf{A} &\to \mathrm{Ch}(\mathsf{A}) \\
		A &\mapsto D^n(A)
	\end{align} 
	and
	\begin{align}
		(-)^n: \mathrm{Ch}(\mathsf{A}) &\to \mathsf{A} \\
		X^\bullet &\mapsto X^n
	.\end{align} 
\end{lem} 

\begin{prop}
	Let $\mathsf{A}$ be an abelian category.
	A complex $\left( P^{\bullet}, d_{P} \right)$ is a projective object of
	$\mathrm{Ch}(\mathsf{A})$ iff $P^i$ is projective in $\mathsf{A}$ for all
	$i \in \Z$ and $\left( P^{\bullet}, d_{P} \right)$ is contractible.

	A complex $\left( I^{\bullet}, d_{I} \right)$ is an injective object of
	$\mathrm{Ch}(\mathsf{A})$ iff $I^i$ is projective in $\mathsf{A}$ for all
	$i \in \Z$ and $\left( I^{\bullet}, d_{I} \right)$ is contractible.
\end{prop} 

\begin{lem}
	Assume that $\mathsf{A}$ is an abelian category, with enough
	projectives (i.e. $\,\forall\, A \in \mathrm{Ob} \left(\mathsf{A}\right)$ there is a
	projective object $P \in \mathrm{Ob} \left(\mathsf{A}\right)$, with an epi
	$P \xrightarrow{\varphi} A \to 0$).
	Then $\mathrm{Ch}(\mathsf{A})$ has enough projectives.
\end{lem} 

\begin{rem}[]
	Given a cochain complex $\left( X^{\bullet}, d_{X} \right)$, we can define an associated
	chain complex $\left( X_{\bullet}, d^{X} \right)$ by setting $X_n := X^{-n}$.
\end{rem}


\section{Derived functors}
\subsection{Resolutions}
\begin{defn}[(Co)homological $\partial$-functor]
	Let $\mathsf{A}, \mathsf{B}$ be abelian categories.
	A {\em (co)homological} $\partial${\em -functor} between $\mathsf{A}$ and $\mathsf{B}$
	is the data of a sequence of functors $\left\{ T^n \right\}_{n \in \Z}$, with
	$T^n\colon \mathsf{A} \to \mathsf{B}$ for every $n$,
	($\left\{ T_n \right\}_{n \in \Z}$ for the homological functors)
	s.t. $T^i = 0$ for all $i < 0$ ($T_i = 0$ for all $i > 0$)
	and for any short exact sequence $S \in \mathsf{S}(\mathsf{A})$
	\begin{equation}
	0 \to A \to B \to C \to 0
	\end{equation} 
	for all $n \in \Z$ there is a connecting morphism
	$\partial^n\colon T^n(C) \to T^{n+1}(A)$ (resp. $\partial_n\colon T_n(C) \to T_{n-1}(A)$)
	satisfying
	\begin{enumerate}
		\item there is a long exact sequence
			\begin{equation}
				\ldots \to T^{n-1}(C) \xrightarrow{\partial^{n-1}} T^n(A) \to
				T^n(B) \to T^n(C) \xrightarrow{\partial^n} T^{n+1}(A) \to
				\ldots =: T(S)
			,\end{equation} 
			respectively the long exact sequence
			\begin{equation}
				\ldots \to T_{n+1}(C) \xrightarrow{\partial_{n+1}} T_n(A) \to
				T_n(B) \to T_n(C) \xrightarrow{\partial_n} T_{n-1}(A) \to
				\ldots =: T(S)
			,\end{equation} 
		\item For any $S' \in \mathsf{S}(\mathsf{A})$ and any morphism
			$S \to S'$ in $\mathsf{S}(\mathsf{A})$, i.e.
			\begin{equation}
			\begin{tikzcd}
				0 \arrow[r, "", rightarrow] &
				A \arrow[r, "", rightarrow] \arrow[d, "f", rightarrow] &
				B \arrow[r, "", rightarrow] \arrow[d, "g", rightarrow] &
				C \arrow[r, "", rightarrow] \arrow[d, "h", rightarrow] &
				0 \\
				0 \arrow[r, "", rightarrow] &
				A' \arrow[r, "", rightarrow] &
				B' \arrow[r, "", rightarrow] &
				C' \arrow[r, "", rightarrow] &
				0,
			\end{tikzcd}
			\end{equation} 
			there is an associated morphism between the long exact sequences, i.e.
			a commutative diagram (with a clear dual for the homological case)
			\begin{equation}
			\begin{tikzcd}
				T^{n-1}(C) \arrow[r, "\partial^{n-1}", rightarrow] \arrow[d, "T^{n-1}(h)", rightarrow] &
				T^n(A) \arrow[r, "", rightarrow] \arrow[d, "T^n(f)", rightarrow] &
				T^n(B) \arrow[r, "", rightarrow] \arrow[d, "T^n(g)", rightarrow] &
				T^n(C) \arrow[r, "\partial^n", rightarrow] \arrow[d, "T^n(h)", rightarrow] &
				T^{n+1}(A) \arrow[d, "T^{n+1}(f)", rightarrow] \\
				T^{n-1}(C') \arrow[r, "\partial^{n-1}"', rightarrow] &
				T^n(A') \arrow[r, "", rightarrow] &
				T^n(B') \arrow[r, "", rightarrow] &
				T^n(C') \arrow[r, "\partial^n"', rightarrow] &
				T^{n+1}(A')
			\end{tikzcd}
			.\end{equation} 
	\end{enumerate}
	Then the family $T \coloneqq \left\{ T^n \right\}_{n \in \Z}\colon\mathsf{S}(\mathsf{A}) \to \mathsf{L}(\mathsf{B})$
	actually is a functor.
\end{defn}

\begin{rem}[]
	$T^0$ is always left exact, for a cohomological $\partial$-functor.
	In fact given a short exact sequence in $\mathsf{A}$
	\begin{equation}
		0 \to A \to B \to C \to 0
	,\end{equation} 
	the associated long exact sequence, since $T^{-1} = 0$, is
	\begin{equation}
		T^{-1}(C) = 0 \xrightarrow{\partial^{-1}} T^0(A) \to T^0(B) \to T^0(C) \to \ldots
	.\end{equation} 
	Analogously, one checks that $T_0$ is right exact, for any
	homological $\partial$-functor.
\end{rem}

\begin{ex}
	Consider $\mathsf{A} \coloneqq \mathsf{Mod}\text{-}R$, for a ring $R$.
	Consider $U^{-1}$ and $U^0$ free (in particular projective) $R$-modules
	and the morphism of modules $u\colon U^{-1} \to U^0$.
	Consider functor $T$, given by the family $\left\{ T^0, T^1 \right\}$ (i.e. $T^i = 0$ for all $i \neq 1,0$),
	where
	\begin{equation}
	T^0 \coloneqq \ker \mathrm{Hom}_{R} \left( u, - \right)
	\qquad \text{ and } \qquad
	T^1 \coloneqq \coker \mathrm{Hom}_{R}\left( u, - \right)
	.\end{equation} 
	Let's show that $T\colon\mathsf{Mod}\text{-}R \to \mathsf{Mod}\text{-}R$ is a
	cohomological $\partial$-function.
	Consider a short exact sequence of modules
	$0 \to A\to B \to C \to 0$.
	We need to show that the following is exact:
	\begin{equation}
		0 \to T^0(A) \to T^0(B) \to T^0(C) \to
		T^1(A) \to T^1(B) \to T^1(C) \to 0
	.\end{equation} 
	In fact we can apply the covariant hom functor $\mathrm{Hom}_{R}\left( u, - \right)$ and obtain
	\begin{equation}
	\begin{tikzcd}
		0 \arrow[r, "", rightarrow] &
		\mathrm{Hom}_{R}\left( U^0, A \right) \arrow[r, "", rightarrow]
		\arrow[d, "{\mathrm{Hom}_{R}\left( u , A \right)}"', rightarrow] &
		\mathrm{Hom}_{R}\left( U^0, B \right) \arrow[r, "", rightarrow]
		\arrow[d, "{\mathrm{Hom}_{R}\left( u , B \right)}"', rightarrow] &
		\mathrm{Hom}_{R}\left( U^0, C \right) \arrow[r, "", rightarrow]
		\arrow[d, "{\mathrm{Hom}_{R}\left( u , C \right)}"', rightarrow] &
		0 \\
		0 \arrow[r, "", rightarrow] &
		\mathrm{Hom}_{R}\left( U^{-1}, A \right) \arrow[r, "", rightarrow] &
		\mathrm{Hom}_{R}\left( U^{-1}, B \right) \arrow[r, "", rightarrow] &
		\mathrm{Hom}_{R}\left( U^{-1}, C \right) \arrow[r, "", rightarrow] &
		0 
	\end{tikzcd}
	\end{equation} 
	which clearly is commutative and with exact rows (both $U^0$ and $U^{-1}$ are free,
	hence projective, i.e. both $\mathrm{Hom}_{ R}\left( U^0, - \right)$ and
	$\mathrm{Hom}_{ R}\left( U^{-1}, - \right)$ are exact functors), then by the snake lemma
	we obtain exactness of the long sequence.

	Moreover consider $\mathsf{C} \coloneqq \left\{ X \in \mathrm{Ob} \left(\mathsf{Mod}\text{-}R\right) \ \middle|\ 
	T^0(X) = T^1(X) = 0\right\} \subset \mathsf{Mod}\text{-}R$.
	Then this subcategory of $\mathsf{Mod}\text{-}R$ is closed under
	kernel, cokernel, extension and products.
	In particular $C$ is an abelian full subcategory of $\mathsf{Mod}\text{-}R$.
	In fact, given $X, Y \in \mathrm{Ob} \left(\mathsf{C}\right)$, and a morphism $f\colon X \to Y$.
	Let $K \coloneqq \ker f$, $I \coloneqq \ima f$, and $C \coloneqq \coker f$.
	Then we have the short exact sequences
	$0 \to K \to X \to I \to 0$ and $0 \to I \to Y \to C \to 0$.
	The functor $T$ associates to them the long exact sequences
	\begin{equation}
		0 \to T^0(K) \to 0 \to T^0(I) \to
		T^1(K) \to 0 \to T^1(I) \to 0
	\end{equation} 
	\begin{equation}
		0 \to T^0(I) \to 0 \to T^0(C) \to
		T^1(I) \to 0 \to T^1(C) \to 0
	.\end{equation} 
	With simple computations one shows that $I, C, K \in \mathrm{Ob} \left(\mathsf{C}\right)$.
\end{ex} 

\begin{defn}[Left resolution]
	Let $\mathsf{A}$ be an abelian category, and $M \in \mathrm{Ob} \left(\mathsf{A}\right)$.
	A {\em left resolution} of $M$ is a chain-complex:
	\begin{equation}
		X_{\bullet} \coloneqq \ldots \to X_2 \xrightarrow{d_2} X_1 \xrightarrow{d_1} X_0 \to 0
	\end{equation} 
	s.t. there exists $\pi\colon X_0 \to M$ with which the augmented complex
	\begin{equation}
	X_{\bullet} \xrightarrow{\pi} M \to 0 \coloneqq \ldots \to X_2 \xrightarrow{d_2} 
	X_1 \xrightarrow{d_1} X_0 \xrightarrow{\pi} M \to 0
	\end{equation} 
	is exact.
	We can view these conditions as stating that $\pi$
	is a quasi-isomorphism between $X_{\bullet}$ and $M$,
	viewed as a complex concentrated in degree $0$.
	In fact $H_i(X_{\bullet}) = 0$ for all $i \neq 0$
	and $H_0(X_{\bullet}) \simeq M$.

	If, moreover, each $X_i$ is a projective object in $\mathsf{A}$, then
	the left resolution $X_{\bullet} \xrightarrow{\pi} M \to 0$ is called a
	{\em projective resolution} of $M$.
\end{defn}

\begin{lem}
	Let $\mathsf{A}$ be an abelian category, with enough projectives.
	Then every $M \in \mathrm{Ob} \left(\mathsf{A}\right)$ admits a
	projective resolution.
\end{lem} 
\begin{proof}
	One obtains an exact resolution by taking, each time, a projection onto the kernel of the previous map
	(it can be done, since $\mathsf{A}$ has enough projectives)
	\begin{equation}
	\ldots \to P_2 \xrightarrow{\pi_2} P_1 \xrightarrow{\pi_1} P_0 \xrightarrow{\pi_0} M \to 0
	.\end{equation} 
	We can define, for each $n \in \N$, $K_n \coloneqq \ker \pi_{n-1}$.
	Then $K_n$ is called $n$-th syzygy of $M$, sometimes
	denoted by $\Omega_n(M)$.

	Moreover the projective resolution is usually
	denoted by $P_{\bullet} \xrightarrow{\pi_0} M \to 0$.
\end{proof}

\begin{thm}[Comparison]
	Let $\mathsf{A}$ be an abelian category.
	Let $f_{-1}\colon M \to N$ be a morphism in $\mathsf{A}$.
	Consider the chain complex (not necessairily exact)
	\begin{equation}
	\ldots \to P_3 \to P_2 \to P_1 \to P_0 \xrightarrow{\pi} 
	M \to 0
	,\end{equation} 
	with $P_i$ projective for all $i \geq 0$.
	Let $Y_{\bullet} \xrightarrow{\sigma} N \to 0$ be a left resolution of $N$.
	Then there is a chain map $f\colon P_{\bullet} \to Y_{\bullet}$ lifting $f_{-1}$,
	i.e.
	\begin{equation}
	\begin{tikzcd}
		\ldots \arrow[r, "", rightarrow] &
		P_3 \arrow[r, "d_3^P", rightarrow] \arrow[d, "f_3"', rightarrow] &
		P_2 \arrow[r, "d_2^P", rightarrow] \arrow[d, "f_2"', rightarrow] &
		P_1 \arrow[r, "d_1^P", rightarrow] \arrow[d, "f_1"', rightarrow] &
		P_0 \arrow[r, "\pi", rightarrow] \arrow[d, "f_0"', rightarrow] &
		M \arrow[r, "", rightarrow] \arrow[d, "f_{-1}"', rightarrow] &
		0\\
		\ldots \arrow[r, "", rightarrow] &
		Y_3 \arrow[r, "d_3^Y"', rightarrow] &
		Y_2 \arrow[r, "d_2^Y"', rightarrow] &
		Y_1 \arrow[r, "d_1^Y"', rightarrow] &
		Y_0 \arrow[r, "\sigma"', rightarrow] &
		N \arrow[r, "", rightarrow] &
		0
	\end{tikzcd}
	.\end{equation} 
	Moreover given any other chain map $g \coloneqq \left\{ g_n \right\}_{n \geq 0}$
	lifting $f_{-1}$, then $f \sim g$ the two chain maps are homotopic.
	In other words the lift of $f_{-1}$ is unique up to homotopy.
\end{thm}

\begin{lem}
	Let $\mathsf{A}$ be an abelian category, and $P_{\bullet}$ an acyclic
	chain complex bounded below (i.e. s.t. $\exists\, m \in \Z$ for which
	$P_i = 0$ for all $i < m$) with projective components.
	Then $P_{\bullet}$ is contractible
	(hence it is projective in the category of complexes).
\end{lem} 

\begin{lem}[Horseshoe]
	Let $\mathsf{A}$ be an abelian category.
	Consider the short exact sequence
	\begin{equation}
	0 \to A \xrightarrow{f} B \xrightarrow{g} C \to 0
	,\end{equation} 
	and the projective resolutions $P_{\bullet} \to A \to 0$ and $Q_{\bullet} \to C \to 0$
	for $A$ and $C$.
	Then we can complete the diagram with the red arrows.
	\begin{equation}
	\begin{tikzcd}
		& & & 0 \arrow[d, "", rightarrow] & \\
		\ldots \arrow[r, "", rightarrow] &
		P_1 \arrow[r, "", rightarrow] &
		P_0 \arrow[r, "", rightarrow] &
		A \arrow[r, "", rightarrow] \arrow[d, "f", rightarrow] &
		0\\
		{\color{red}\ldots} \arrow[r, red, "", rightarrow] &
		{\color{red} P_1 \oplus Q_1} \arrow[r, red, "", rightarrow] &
		{\color{red} P_0 \oplus Q_0} \arrow[r, red, "", rightarrow] &
		B \arrow[r, red, "", rightarrow] \arrow[d, "g", rightarrow] &
		{\color{red}0}\\
		\ldots \arrow[r, "", rightarrow] &
		Q_1 \arrow[r, "", rightarrow] &
		Q_0 \arrow[r, "", rightarrow] &
		C \arrow[r, "", rightarrow] \arrow[d, "", rightarrow] &
		0\\
		& & & 0 &
	\end{tikzcd}
	.\end{equation} 
	In particular $\left( P_{\bullet} \oplus Q_{\bullet}, d^P_{\bullet} \oplus d^Q_{\bullet} \right)$
	gives a projective resolution of $B$, completing the diagram in the second row.
\end{lem} 

Let's now dualize everything we obtained up to now:
\begin{defn}[Right coresolution]
	Let $\mathsf{A}$ be an abelian category and $M \in \mathrm{Ob} \left(\mathsf{A}\right)$.
	A {\em right coresolution} of $M$ is a cochain complex
	\begin{equation}
	Y^\bullet \coloneqq 0 \to Y_0 \xrightarrow{d^0} Y^1 \xrightarrow{d^1} Y^2 \to\ldots	
	\end{equation} 
	s.t. there exists a morphism $\delta^0\colon M \to Y^0$ with which the augmented complex
	\begin{equation}
	0 \to M \xrightarrow{\delta^0} Y^\bullet \coloneqq 
	0 \to M \xrightarrow{\delta^0} Y_0 \xrightarrow{d^0} Y^1 \xrightarrow{d^1} Y^2 \to\ldots	
	\end{equation} 
	is exact.
	In other words we ask that $\delta^0$ is a quasi-isomorphism between
	$Y^\bullet$ and $M$ concentrated in degree $0$.
	Then $H^i(Y^\bullet) = 0$ for all $i \neq 0$ and $H^0(Y^\bullet) \simeq M$.

	If, moreover, each $Y^i$ is an injective object in $\mathsf{A}$,
	then the right coresolution $0 \to M \xrightarrow{\delta^0} Y^\bullet$ is called an
	{\em injective coresolution} of $M$.
\end{defn}

\begin{lem}
	Let $\mathsf{A}$ be an abelian category, with enough injectives.
	Then every object $M \in \mathrm{Ob} \left(\mathsf{A}\right)$ admits
	an injective coresolution.
\end{lem} 
\begin{proof}
	One obtains an exact resolution by taking, each time, the cokernel of the previous map
	\begin{equation}
		0 \to M \xrightarrow{\delta^0} I^0 \xrightarrow{\delta^1} I^1 \xrightarrow{\delta^1} I^2 \to \ldots
	.\end{equation} 
	We can define, for each $n \in \N$, $C^n \coloneqq \coker \delta^{n-1}$.
	Then $C^n$ is called $n$-th cosyzygy of $M$, sometimes
	denoted by $\Omega^n(M)$.

	Moreover the projective resolution is usually
	denoted by $0 \to M \xrightarrow{\delta^0} I^{\bullet}$.
\end{proof}

\begin{thm}[Comparison]
	Le $\mathsf{A}$ be an abelian category.
	Let $f^{-1}\colon M \to N$ a morphism in $\mathsf{A}$.
	Consider the cochain complex (not necessairily exact)
	\begin{equation}
	0 \to N \xrightarrow{\eta} I^0 \to I^1 \to I^2 \to I^3 \to \ldots
	,\end{equation} 
	with $I^i$ injective for all $i \geq 0$.
	Let $0 \to M \xrightarrow{\delta^0} Y^{\bullet}$ a right coresolution of $M$.
	Then there exists a cochain map $f\colon Y^\bullet \to I^\bullet$
	extending $f^{-1}$, i.e.
	\begin{equation}
	\begin{tikzcd}
		0 \arrow[r, "", rightarrow] &
		M \arrow[r, "\delta^0", rightarrow] \arrow[d, "f^{-1}"', rightarrow] &
		Y^0 \arrow[r, "d^0_Y", rightarrow] \arrow[d, "f^0"', rightarrow] &
		Y^1 \arrow[r, "d^1_Y", rightarrow] \arrow[d, "f^1"', rightarrow] &
		Y^2 \arrow[r, "d^2_Y", rightarrow] \arrow[d, "f^2"', rightarrow] &
		Y^3 \arrow[r, "d^3_Y", rightarrow] \arrow[d, "f^3"', rightarrow] &
		\ldots\\
		0 \arrow[r, "", rightarrow] &
		N \arrow[r, "\eta"', rightarrow] &
		I^0 \arrow[r, "d^0_I"', rightarrow] &
		I^1 \arrow[r, "d^1_I"', rightarrow] &
		I^2 \arrow[r, "d^2_I"', rightarrow] &
		I^3 \arrow[r, "d^3_I"', rightarrow] &
		\ldots\\
	\end{tikzcd}
	.\end{equation} 
	Moreover, given any other cochain map $g \coloneqq \left\{ g^n \right\}_{n \geq 0}$ extending
	$f^{-1}$, then $f \sim g$, the two cochain maps are homotopic.
	In other words the extension of $f^{-1}$ is unique up to homotopy.
\end{thm}

\section{Derived functors}

\subsection{Left derived functors}
\begin{rem}[]
	A functor $F: \mathsf{A} \to \mathsf{B}$ between additive categories
	induces a functor, again denoted by $F$, 
	\begin{align}
		F: \mathrm{Ch}(\mathsf{A}) &\to \mathrm{Ch}(\mathsf{A}) \\
		\left( X^{\bullet}, d_{X} \right) &\mapsto \left( F(X^{\bullet}), d_{F(X^\bullet)} \right)
	,\end{align} 
	where $\left[ F(X^\bullet) \right]^n := F(X^n)$ and $d_{F(X^\bullet)}^n := f(d^n_{X^\bullet})$.
	Given a morphism $f: X^\bullet \to Y^\bullet$ in $\mathrm{Ch}(\mathsf{A})$, then
	$d_Y \circ f = f \circ d_X$.
	Then $F(f) \circ F(d_X) = f(d_Y) \circ F(f)$, i.e. $F(f)$ is a morphism in $\mathrm{Ch}(\mathsf{B})$.

	Moreover, if $F$ is an additive functor, then
	$f = s \circ d_X + d_Y \circ s$ implies
	$F(f) = F(s) \circ F(d_X) + F(d_Y) \circ F(s)$,
	hence $F$ induces a functor
	\begin{equation}
		F: K(\mathsf{A}) \to K(\mathsf{B})
	.\end{equation} 
\end{rem}

\begin{defn}[Left derived functors]
	Let $\mathsf{A}$ and $\mathsf{B}$ be abelian categories.
	Assume that $\mathsf{A}$ has enough projectives and $F: \mathsf{A} \to \mathsf{B}$
	is a right exact functor.
	We define the left derived functor $L_iF: \mathsf{A} \to \mathsf{B}$
	s.t. $L_iF(A) := H_i \left( F(P_{\bullet}) \right)$, 
	for $P_{\bullet} \to A \to 0$ a projective resolution of $A$
	and $i \geq 0$.
\end{defn}

\begin{rem}[]
	One actually needs to prove that the above is a good definition, i.e.
	that $L_i F$ does not depend on the projective resolution $P_{\bullet} \to A \to 0$.

	Moreover one can prove that $L_0F \simeq F$ as functors.
	In fact consider any projective resolution $P_{\bullet} \to A \to 0$ of $A$. Then
	\begin{equation}
	\ldots \to P_2 \to P_1 \xrightarrow{d_1} P_0 \to A \to 0
	\end{equation} 
	is exact, with $F$ right exact.
	Then 
	\begin{equation}
		F(P_1) \xrightarrow{F(d_1)} F(P_0) \to F(A) \to 0
	\end{equation} 
	is also exact.
	In particular $\coker F(d_1) \simeq F(A)$.
	But then $L_0 F(A) = H_0 \left( F(P_{\bullet}) \right)$.
	We know that
	\begin{equation}
		F(P_{\bullet}) = \ldots \to F(P_1) \xrightarrow{F(d_1)} F(P_0) \to 0
	\end{equation} 
	hence that $H_0 \left( F(P_{\bullet}) \right) = \coker F(d_1) = F(A)$.
\end{rem}

\begin{lem}\leavevmode\vspace{-.2\baselineskip}
	\begin{enumerate}[label=(\alph*)]
		\item For each $i \in \N$ $L_i F$ is well defined, up to natural isomorphism.
		\item Let $\alpha: A \to C$ be a morphism in $\mathsf{A}$.
			Then there are natural maps
			\begin{equation}
				L_i F(\alpha): L_iF(A) \to L_iF(C)
			.\end{equation} 
		\item For any $i \geq 0$, the functor $L_i F$ is additive.
	\end{enumerate}
\end{lem} 

\begin{lem}
	Let $f: A \to C$ be a morphism in $\mathsf{A}$.
	Then $L_0F(f) = F(f)$.
\end{lem} 

\begin{prop}
	Let $F$ and $L_i F$ be as in the above definition.
	If $A \in \mathsf{A}$ is projective, then $L_iF(A) = 0$
	for all $i > 0$ (recall that $L_0F(A) \simeq F(A)$).
\end{prop} 

\begin{defn}[F-acyclic object]
	Let $\mathsf{A}$ be an abelian category with enough projectives
	and $F: \mathsf{A} \to \mathsf{B}$ be a right exact functor.
	An object $A \in \mathrm{Ob} \left(\mathsf{A}\right)$ is called
	$F$-\textbf{acyclic} iff $L_iF(A) = 0$ for all $i > 0$
\end{defn}

\begin{defn}[F-acyclic resolution]
	Let $\mathsf{A}$ be an abelian category with enough projectives,
	$F: \mathsf{A} \to \mathsf{B}$ be a right exact functor and $A \in \mathrm{Ob} \left(\mathsf{A}\right)$.
	A left resolution $Q_{\bullet} \to A \to 0$ of $A$ is called an
	$F$-\textbf{acyclic resolution} iff $Q_i$ are $F$-acyclic
	for all $i \geq 0$.	
\end{defn}

\begin{rem}[]
	Any projective object $A \in \mathrm{Ob} \left(\mathsf{A}\right)$ is $F$-acyclic
	for any right exact functor $F$.
\end{rem}

\begin{thm}[]
	Let $\mathsf{A}$ and $\mathsf{B}$ be abelian categories.
	Assume that $\mathsf{A}$ has enough projectives and $F: \mathsf{A} \to \mathsf{B}$
	is a right exact functor.
	Then the left derived functors $\left\{ L_i F \right\}_{i \geq 0}$
	form a homological $\partial$-functor.
\end{thm}

\begin{defn}[Morphism of (co)homological $\partial$-functor]
	Ler $S,T: \mathsf{A} \to \mathsf{B}$ be cohomological $\partial$-functors.
	A morphism $S \to T$ is a sequence of natural transformations
	$\eta^n: S^n \to T^n$ commuting with $\partial$.
	More explicitly, given any s.e.s. $0 \to A \to B \to C \to 0$ in $\mathsf{A}$,
	the following diagram commutes
	\begin{equation}
	\begin{tikzcd}
		S^n(C) \arrow[r, "\partial^n_S", rightarrow] \arrow[d, "\eta^n_C"', rightarrow] &
		S^{n+1}(A) \arrow[d, "\eta^{n+1}_A", rightarrow] \\
		T^n(C) \arrow[r, "\partial^n_T"', rightarrow] &
		T^{n+1}(A)
	\end{tikzcd}
	.\end{equation} 
	(Clearly for homological $\partial$-functors one only has to dualize).
\end{defn}

\begin{defn}[Universal cohomological $\partial$-functor]
	A cohomological $\partial$-functor $T$ is called \textbf{universal} iff
	given any cohomological $\partial$-functor $S$, and any natural
	transformation $\eta^0: T^0 \to S^0$, then
	$\exists\, !\, \left\{ \eta^n: T^n \to S^n \right\}_{n \geq 0}$	a natural transformation
	of $\partial$-functors extending $\eta^0$.
	(Analogously of homological $\partial$-functors).
\end{defn}

\begin{lem}
	Consider an exact functor $F: \mathsf{A} \to \mathsf{B}$.
	Show that $T^0 := F$ and $T^n := 0$ for all $n > 0$
	define  a universal cohomological $\partial$-functor $\left\{ T^n \right\}_{n \in \N}$.
	(Analogously setting $T_0 := F$ and $T_n := 0$, for a universal homological
	$\partial$-functor).
\end{lem}  

\begin{thm}[]
	Let $\mathsf{A}$ and $\mathsf{B}$ be abelian categories.
	Assume that $\mathsf{A}$ has enough projectives and $F: \mathsf{A} \to \mathsf{B}$
	is a right exact functor.
	Then the left derived functors $\left\{ L_i F \right\}_{i \geq 0}$
	form a universal homological $\partial$-functor.
\end{thm}

\begin{lem}
	Let $\mathsf{A}$ and $\mathsf{B}$ be abelian categories.
	Assume that $\mathsf{A}$ has enough projectives and $F: \mathsf{A} \to \mathsf{B}$
	is a right exact functor.
	Consider $G: \mathsf{B} \to \mathsf{C}$	an exact funcotr, then:
	\begin{equation}
		L_i \left( G \circ F \right) \simeq_{\text{nat.}} G \circ L_i F
		\qquad \,\forall\, i \geq 0
	.\end{equation} 
\end{lem} 

\begin{lem}
	Consider $G: \mathsf{A} \to \mathsf{B}$ an exact functor between abelian categories.
	Consider $X_{\bullet} \in \mathrm{Ch}(\mathsf{A})$, then for every $i \in \Z$
	\begin{equation}
		G \left( H_{i}\left( X_{\bullet} \right) \right) =
		H_{i}\left( G(X_{\bullet}) \right)
	.\end{equation} 
\end{lem} 

\begin{lem}[Dimension shifting]
	Let $\mathsf{A}$ and $\mathsf{B}$ be abelian categories.
	Assume that $\mathsf{A}$ has enough projectives and $F: \mathsf{A} \to \mathsf{B}$
	is a right exact functor.
	Consider a s.e.s. $0 \to K \to Q \to A \to 0$ in $\mathsf{A}$,
	with $Q$ an $F$-acyclic object (e.g. if $Q$ is projective).
	Then
	\begin{enumerate}
		\item $L_1 F(A) = \ker \left( F(K) \to F(Q) \right)$,
		\item $L_iF(A) \simeq L_{i-1} F(K)$ for all $i \geq 2$.
	\end{enumerate}
\end{lem} 

\begin{rem}[]
	Let $\mathsf{A}$ be an abelian category. We define
	\begin{equation}
	\mathrm{Ch}_{\geq 0}(\mathsf{A}) :=
	\left\{ \left( X_{\bullet}, d^{X} \right) \in \mathrm{Ch}(\mathsf{A}) \ \middle|\ 
	X_n = 0 \,\forall\, n < 0\right\}
	.\end{equation} 
	By the fundamental theorem on homology, we know that $\left\{ H_n \right\}_{n \in \Z}$,
	for $H_n: \mathrm{Ch}_{\geq 0}(\mathsf{A}) \to \mathsf{A}$,
	is a homological $\partial$-functor
\end{rem}

\begin{lem}
	Moreover one can prove that $\left\{ H_n \right\}_{n \in \Z}$ is a \textbf{universal}
	homological $\partial$-functor.
\end{lem} 

\begin{lem}
	Let $\mathsf{A}$ and $\mathsf{B}$ be abelian categories, s.t. $\mathsf{A}$ has enough projectives.
	Consider $F: \mathsf{A} \to \mathsf{B}$ an exact functor, then
	\begin{equation}
		L_i F(A) = 0 \qquad
		\,\forall\, A \in \mathrm{Ob} \left(\mathsf{A}\right), \,\forall\, i > 0
	.\end{equation} 
	Moreover we also know that $L_0 F \simeq F$.
\end{lem} 

\subsection{Right derived functors}
\begin{rem}[Standard assumption]\label{rem:RDFStdAssumption}
	In the following section we will assume the following:
	$\mathsf{A}$ and $\mathsf{B}$ are abelian categories.
	Moreover we assume that $\mathsf{A}$ has enough injectives, and
	$F: \mathsf{A} \to \mathsf{B}$ is a left exact functor.
\end{rem}

\begin{defn}[Right derived funtors]
	Let $\mathsf{A}$, $\mathsf{B}$ and $F$ be as in remark \ref{rem:RDFStdAssumption}.
	We define the right derived functors
	$R^i F: \mathsf{A} \to \mathsf{B}$ s.t.
	$R^iF(A) := H^i \left( F(I^\bullet) \right)$,
	for $0 \to A \to I^\bullet$ an injective coresolution of $A$, and $i \geq 0$.
\end{defn}

\begin{rem}[Important!]
	Recall that $A \in \mathrm{Ob} \left(\mathsf{A}\right)$ is injective
	iff $A$ is projective in $\mathsf{A}^{op}$.
	Then, given an injective coresolution $0 \to A \to I^\bullet$ for $A$,
	then $I_{\bullet} \to A \to 0$ becomes a projective resolution in $\mathsf{A}^{op}$.

	Then, given $F: \mathsf{A} \to \mathsf{B}$ a left exact functor, 
	we define $F^{op}: \mathsf{A}^{op} \to \mathsf{B}^{op}$ a covariant functor.
	Clearly $\mathsf{A}^{op}$ has enough projectives, moreover $F^{op}$ is right exact
	(in fact $F$ is left exact iff $F^{op}$ is right exact).
	Then we can define the left derived functor $L_iF^{op}(A)$.
	Finally we have the equality
	\begin{equation}
		\left( L_i F^{op} \right)^{op} (A) = R^i F(A)
	.\end{equation} 
	In particular $\left\{ R^iF \right\}_{i \geq 0}$ form a universal cohomological $\partial$-functor.
	Moreover, dualizing the previous results, we obtain
	\begin{itemize}
		\item $R^0F \simeq F$,
		\item Given a s.e.s. in $\mathsf{A}$
			\begin{equation}
			0 \to A \to B \to C \to 0
			\end{equation} 
			there is an associated long exact sequence
			\begin{equation}
				0 \to F(A) \to F(B) \to F(C) \xrightarrow{\partial^0} 
				R^1F(A) \to R^1F(B) \to R^iF(C) \xrightarrow{\partial^1} 
				\ldots
			.\end{equation} 
	\end{itemize}
\end{rem}

\begin{defn}[$F$-acyclic objects]
	Let $\mathsf{A}$, $\mathsf{B}$ and $F$ be as in remark \ref{rem:RDFStdAssumption}.
	An object $A \in \mathrm{Ob} \left(\mathsf{A}\right)$ is
	$F$-\textbf{acyclic} iff
	\begin{equation}
		R^iF(A) = 0 \qquad \,\forall\, i > 0
	.\end{equation} 
\end{defn}

\begin{rem}[]
	Any injective object $Q \in \mathrm{Ob} \left(\mathsf{A}\right)$ is 
	$F$-acyclic for any left-exact functor $F$.
\end{rem}

\begin{lem}
	Consider $\mathsf{A}$ an abelian category with enough injectives, and
	$F: \mathsf{A} \to \mathsf{B}$ an exact functor, then
	$R^iF = 0$ for all $i > 0$.
\end{lem} 

\begin{ex}
	Let $\mathsf{A}$ an abelian category with enough injectives.
	Fix $M \in \mathrm{Ob} \left(\mathsf{A}\right)$, then consider
	\begin{equation}
	H_M := \mathrm{Hom}_{\mathsf{A}} \left( M, - \right): \mathsf{A} \to \mathsf{Ab}
	\end{equation} 
	the covariant Hom functor.
	We know that $H_M$ is left exact.
	Then we can define the right derived functors of $H_M$.
	In particular they are defined as:
	For an object $A \in \mathrm{Ob} \left(\mathsf{A}\right)$,
	take an injective coresolution of $A$: $0 \to A \to I^\bullet$, then
	\begin{equation}
		R^iH_M(A) = H^i \left( \mathrm{Hom}_{\mathsf{A}} \left( M, I^\bullet \right) \right)
	.\end{equation} 
	Moreover one introduces the notation (which is especially useful in the category of modules)
	\begin{equation}
		\mathrm{Ext}_{\mathsf{A}}^i \left( M,A \right) := R^i H_M (A)
	.\end{equation} 
\end{ex} 

\begin{prop}
	Let $\mathsf{A}$ abelian with enough injectives
	(e.g. $\mathsf{A} = \mathsf{Mod}\text{-}R$).
	Fix $A \in \mathrm{Ob} \left(\mathsf{A}\right)$, TFAE:
	\begin{enumerate}
		\item $A$ is injective,
			i.e. $\mathrm{Hom}_{\mathsf{A}} \left( -, A \right)$ is exact.
		\item $\mathrm{Ext}^i_{\mathsf{A}}(M,A) = 0$ for all $M \in \mathrm{Ob} \left(\mathsf{A}\right)$
			and for all $i \geq 0$.
		\item $\mathrm{Ext}^1_{\mathsf{A}}(M,A) = 0$ for all $M \in \mathrm{Ob} \left(\mathsf{A}\right)$.
	\end{enumerate}
\end{prop} 
We can dualize the above proposition and obtain
\begin{prop}
	Let $\mathsf{A}$ abelian with enough injectives
	(e.g. $\mathsf{A} = \mathsf{Mod}\text{-}R$).
	Fix $M \in \mathrm{Ob} \left(\mathsf{A}\right)$, TFAE:
	\begin{enumerate}
		\item $M$ is projective,
			i.e. $\mathrm{Hom}_{\mathsf{A}} \left( M, - \right)$ is exact.
		\item $\mathrm{Ext}^i_{\mathsf{A}}(M,A) = 0$ for all $A \in \mathrm{Ob} \left(\mathsf{A}\right)$
			and for all $i \geq 0$.
		\item $\mathrm{Ext}^1_{\mathsf{A}}(M,A) = 0$ for all $A \in \mathrm{Ob} \left(\mathsf{A}\right)$.
	\end{enumerate}
\end{prop} 

\subsection{Derived functors of contravariant functors}
\begin{rem}[Right derived functors of a contravariant functor]
	Let $\mathsf{A}$ and $\mathsf{B}$ be abelian categories and $F: \mathsf{A} \to \mathsf{B}$
	a contravariant left-exact functor (e.g. $F = H^M := \mathrm{Hom}_{\mathsf{A}} \left( -, M \right)$
	for $M \in \mathrm{Ob} \left(\mathsf{A}\right)$).
	Then $F: \mathsf{A}^{op} \to \mathsf{B}$ is covariant and, still, left-exact.
	If $\mathsf{A}^{op}$ has enough injectives (iff $\mathsf{A}$ has enough projectives)
	we can define the right derived functors $R^iF: \mathsf{A}^{op} \to \mathsf{B}$, for $i \geq 0$.
	In particular this is computed by taking a projective resolution of $A \in \mathrm{Ob} \left(\mathsf{A}\right)$:
	$P_{\bullet} \to A \to 0$, which gives an injective coresolution
	$0 \to A \to P^\bullet$ of $A$ in $\mathsf{A}^{op}$.
	Then we define
	\begin{equation}
		R^iF(A) := H^i(F(P_\bullet))
	.\end{equation} 
	Notice that given a chain complex $P_{\bullet}$, then $F(P_{\bullet})$ is a 
	cochain complex.
\end{rem}

\begin{rem}[]
	Let $\mathsf{A}$ be an abelian category with enough injectives and projectives (e.g. for
	$\mathsf{A} = \mathsf{Mod}\text{-}R$).
	Then, fixed $M \in \mathrm{Ob} \left(\mathsf{A}\right)$, $H_M := \mathrm{Hom}_{\mathsf{A}} \left( M, - \right)$
	is a covariant, left-exact, functor.
	In particular it admits right-derived funtors
	\begin{equation}
		R^iH_M(A) = \mathrm{Ext}^i_{\mathsf{A}} \left( M, A \right) =
		H^i \left( H_M (I^\bullet) \right)
	,\end{equation} 
	for an injective coresolution $0 \to A \to I^\bullet$ of $A$.
	Moreover we can consider $H^A := \mathrm{Hom}_{\mathsf{A}} \left( -, A \right)$, which
	is a contravariant, left-exact, functor.
	Also this admits right-derived functors
	\begin{equation}
		R^iH^A(M) = H^i \left( H^A(P_{\bullet}) \right)
	,\end{equation} 
	for $P_{\bullet} \to M \to 0$ a projective resolution of $M$.
\end{rem}

\begin{thm}[Balancing of Ext]
	\begin{equation}
		R^i H_M(A) = \mathrm{Ext}^i_{\mathsf{A}}(M,A)\simeq R^i H^A (M)
	.\end{equation} 
\end{thm}

\begin{rem}[Consequence]
	This theorem means that $\mathrm{Ext}^i_{\mathsf{A}}(M,A)$
	can be computed in two equivalent ways:
	We can consider $0 \to A \to I^\bullet$ an injective coresolution of $A$,
	or $P_{\bullet} \to M \to 0$ a projective resolution of $M$
	and
	\begin{equation}
		H^i \left( \mathrm{Hom}_{\mathsf{A}} \left( M, I^\bullet \right) \right) \simeq
		\mathrm{Ext}^i_{\mathsf{A}} (M,A) \simeq
		H^i \left( \mathrm{Hom}_{\mathsf{A}} \left( P_{\bullet}, A \right) \right)
	.\end{equation} 
\end{rem}

\begin{rem}[]
	Let $\mathsf{A}$ be an abelian category with arbitrary coproducts.
	Consider $\left( X_i^{\bullet}, d_{X_i} \right)_{i \in I}$ a family
	of cochain complexes in $\mathrm{Ch}(\mathsf{A})$.
	Then the cochain complex $( \widetilde{X}^{\bullet}, d_{\widetilde{X}} )$, with
	objects $( \widetilde{X}^\bullet )^n := \coprod_{i \in I} X_i^n$
	and differentials $d^n_{\widetilde{X}} := \coprod_{i \in I} d^n_{X_i}$,
	is a coproduct of $X_i^\bullet$ in $\mathrm{Ch}(\mathsf{A})$.
	Then one checks that
	\begin{equation}
	H^{n}( \widetilde{X} ) = 
	\coprod_{i \in I} H^{n}\left( X_i \right)
	\qquad \,\forall\, n \in \Z
	.\end{equation} 
	Analogously, if $\mathsf{A}$ admits arbitrary products,
	consider $( X_i^{\bullet}, d_{X_i} )_{i \in I}$ a family
	of cochain complexes in $\mathrm{Ch}(\mathsf{A})$.
	Then the cochain complex $( \widetilde{X}^{\bullet}, d_{\widetilde{X}} )$, with
	objects $( \widetilde{X}^\bullet )^n := \prod_{i \in I} X_i^n$
	and differentials $d^n_{\widetilde{X}} := \prod_{i \in I} d^n_{X_i}$,
	is a product of $X_i^\bullet$ in $\mathrm{Ch}(\mathsf{A})$.
	Then one checks that
	\begin{equation}
	H^{n}( \widetilde{X} ) = 
	\prod_{i \in I} H^{n}\left( X_i \right)
	\qquad \,\forall\, n \in \Z
	.\end{equation} 
\end{rem}

\begin{lem}
	Let $\left(L, R\right)$ be an adjoint pair of functors
	$L: \mathsf{A} \to \mathsf{B}$ and $R: \mathsf{B} \to \mathsf{A}$,
	between additive categories.
	Then $\left(L, R\right)$ induces an adjoint pair of morphisms
	\begin{equation}
	L: \mathrm{Ch}(\mathsf{A}) \to \mathrm{Ch}(\mathsf{B})
	\qquad \text{ and } \qquad
	R: \mathrm{Ch}(\mathsf{B}) \to \mathrm{Ch}(\mathsf{A})
	.\end{equation} 
\end{lem} 

\begin{prop}
	Let $\mathsf{A}$ and $\mathsf{B}$ be abelian categories.
	Consider an adjoint pair of functors $\left(F, G\right)$,
	for $F: \mathsf{A} \to \mathsf{B}$ and $G: \mathsf{B} \to \mathsf{A}$.
	Assume that $\mathsf{A}$ has enough projectives and arbitrary coproducts,
	whereas $\mathsf{B}$ has enough injectives and arbitrary products.
	Let $\left\{ A_\alpha \right\}_{\alpha \in \mathcal{A}} \subset \mathrm{Ob} \left(\mathsf{A}\right)$
	be a family of objects of $\mathsf{A}$
	and $\left\{ B_\beta \right\}_{\beta \in \mathcal{B}} \subset \mathrm{Ob} \left(\mathsf{B}\right)$
	be a family of objects of $\mathsf{B}$.
	Then
	\begin{equation}
		L_iF \bigg( \coprod_{\alpha \in \mathcal{A}} A_\alpha \bigg) \simeq
		\coprod_{\alpha \in \mathcal{A}} L_i F(A_\alpha)
	\end{equation} 
	and
	\begin{equation}
		R^iF \bigg( \prod_{\beta \in \mathcal{B}} B_\beta \bigg) \simeq
		\prod_{\beta \in \mathcal{B}} R^iG \left( B_\beta \right)
	.\end{equation} 
\end{prop} 

\subsection{Derived functors of tensor product functors}
Recall that, for a ring $R$, and $M_R \in \mathsf{Mod}\text{-}R$,
then 
\begin{equation}
M_R \otimes_R - : R\text{-}\mathsf{Mod} \to \mathsf{Ab}
\qquad \text{ and } \qquad
\mathrm{Hom}_{\Z}\left( M, - \right): R\text{-}\mathsf{Mod} \to \mathsf{Ab}
\end{equation} 
constitute an adjoint pair $\left(M_R \otimes_R -, \mathrm{Hom}_{ \Z}\left( M, - \right)\right)$.

As a consequence $T_M := M_R \otimes_R -$ is a left adjoint, hence it is
right exact, preserves coproducts, $\varinjlim$.

\begin{defn}[Flat module]
	Consider $M_R \in \mathsf{Mod}\text{-}R$.
	We say that $M_R$ if \textbf{flat} iff $T_M := M_R \otimes_R -$ is exact
	(i.e. iff $T_M$ is also left exact).
	Simmetrically ${}_RN \in R\text{-}\mathsf{Mod}$ is flat iff $- \otimes_R N$ is exact.
\end{defn}

\begin{prop}
	Let $M_R \in \mathsf{Mod}\text{-}R$.
	TFAE:
	\begin{enumerate}
		\item $M_R$ is flat,
		\item for every mono $0 \to {}_RA \xrightarrow{\mu} {}_RB$ of left $R$-modules, then
			$M \otimes A \xrightarrow{id_M \otimes \mu} M \otimes B$ is mono (in $\mathsf{Ab}$),
		\item $L_i \left( M \otimes_R - \right) (N) = 0$ for all $i \geq 1$ and for all $N \in R\text{-}\mathsf{Mod}$,
		\item $L_1 \left( M \otimes_R - \right) (N) = 0$ for all $N \in R\text{-}\mathsf{Mod}$.
	\end{enumerate}
\end{prop} 
Dually:
\begin{prop}
	Let ${}_RN \in R \text{-}\mathsf{Mod}$.
	TFAE:
	\begin{enumerate}
		\item ${}_RN$ is flat,
		\item for every mono $0 \to A_R \xrightarrow{\mu} B_R$ of right $R$-modules, then
			$A \otimes N \xrightarrow{\mu \otimes id_N} B \otimes N$ is mono (in $\mathsf{Ab}$),
		\item $L_i \left( - \otimes_R N\right) (M) = 0$ for all $i \geq 1$ and for all $M \in \mathsf{Mod}\text{-}R$,
		\item $L_1 \left( - \otimes_R N \right) (M) = 0$ for all $M \in \mathsf{Mod}\text{-}R$.
	\end{enumerate}
\end{prop} 

\begin{rem}[]
	Combining the above propositions we obtain that
	$M_R$ is flat iff $M_R$ is $\left( - \otimes_R N \right)$-acyclic for all ${}_R N$ left $R$-modules.
	Analogously ${}_RN$ is flat iff ${}_RN$ is $\left( M \otimes_R - \right)$-acyclic for all $M_R$ right $R$-modules.
\end{rem}

\begin{defn}[Notation]
	Called $T_M := M \otimes_R -$, then we define
	\begin{equation}
		\mathrm{Tor}^R_i (M,N) := L_i \left( M \otimes_R - \right) (N)
	.\end{equation} 
\end{defn}

\begin{thm}[Balancing of Tor]
	\begin{equation}
		\mathrm{Tor}_i^R(M,N) = L_i \left( M \otimes_R - \right)(N) =
		L_i \left( - \otimes_R N \right)(M)
	\end{equation}
	for all $i \geq 0$, all $M \in \mathsf{Mod}\text{-}R$ and all $N \in R\text{-}\mathsf{Mod}$.
\end{thm}

\begin{rem}[Consequence]
	The above theorem means that $\mathrm{Tor}_i^R(M,N)$ can be computed in two equivalent ways:
	Consider $P_{\bullet} \to {}_RN \to 0$ a projective resolution of $N$
	or $Q_{\bullet} \to M_R \to 0$ a projective resolution of $M_R$, then
	\begin{equation}
		H_i \left( M \otimes_R P_{\bullet} \right) \simeq
		\mathrm{Tor}_i^R (M,N) \simeq
		H_i \left( Q_{\bullet} \otimes_R N \right)
	.\end{equation} 
\end{rem}

\begin{prop}
	Let $\left\{ M_i \right\}_{i \in I}$ be a family of right $R$-modules.
	Then
	\begin{enumerate}
		\item $\bigoplus_{i \in I} M_i$ is flat iff $M_i$ is flat for all $i \in I$,
		\item If $\left\{ M_i \right\}_{i \in I}$ is a direct system of flat $R$-modules, 
			then the filtered direct limit $\varinjlim_{i \in I} M_i$ is flat.
	\end{enumerate}
\end{prop} 

\begin{rem}[]
	For every ${}_RN$ $T_N := - \otimes_R N$ is a left adjoint.
	This means that $T_N$ preserves colimts, 
	in particular, for every direct system $\left\{ M_i, F_{ij} \right\}_{i \leq j}$, then
	\begin{equation}
		\big( \varinjlim_{i \in I} M_i \big) \otimes_R N \simeq
		\varinjlim_{i \in I} \left( M_i \otimes_R N \right)
	.\end{equation} 
\end{rem}

\begin{rem}[]
	\begin{equation}
	\varinjlim_{i \in I} M_i \text{ flat } \centernot\implies M_i \text{ flat}
	.\end{equation} 
	In fact every module is the filtered direct limit of its finitely generated
	submodules.
	Though it is not true that, given $M$ flat, then its finitely generated submodules are flat.

	As an example any ring $R$ is a free, hence flat, $R$-module.
	Though this doesn't imply that its (finitely generated) ideals are flat.
	For instance, take $R := \K[x,y]$, for a field $\K$.
	Consider $\mathfrak{m} := (x,y)$ the maximal ideal generated by $x$ and $y$.
	Consider the mono $0 \to \mathfrak{m} \xrightarrow{\epsilon} R$, and
	\begin{align}
		\mathfrak{m} \otimes_R \mathfrak{m} &\to \mathfrak{m} \otimes_R R \simeq \mathfrak{m} \\
		a \otimes b &\mapsto a \cdot b
	.\end{align} 
	In fact $0 \neq x \otimes y - y \otimes x \mapsto xy - yx = 0$,
	then $\mathfrak{m}$ is finitely generated, but not flat.
\end{rem}

\begin{lem}
	Let $\mathsf{A}$ and $\mathsf{B}$ be abelian categories with enough projectives.
	Consider $F: \mathsf{A} \to \mathsf{B}$ a right exact functor.
	Then $L_iF$ can be computed using $F$-acyclic resolutions, instead of projective resolutions.
	More explicitly, given
	$Q_{\bullet} \to A \to 0$ a resolution of $A$ s.t. $Q_i$ is $F$-acyclic for each $i$, then
	\begin{equation}
		L_iF(A) \simeq H_i \left( F(Q_{\bullet}) \right)
	.\end{equation} 
\end{lem} 

\begin{rem}[]
	In particular $\mathrm{Tor}_i^R(-,-)$ can be computed using flat resolutions.
\end{rem}

\begin{ex}[Flat modules]
	Clearly any projective $P_R$ right $R$-module is flat, since it is $- \otimes_R N$-acyclic for
	all ${}_RN$ modules.
	Analogously a a projective left $R$-module ${}_RP$ is flat.
	In particular any free module is flat.
\end{ex}

\begin{ex}[Flat modules]
	Recall the definition of localization:
	given a commutative ring $R$ and a multiplicatively closed subset $S \subset R$, 
	i.e. s.t. $0 \notin S, 1 \in S$ and $s,t \in S \implies st \in S$, 
	we can consider the localization
	\begin{equation}
	R_S = R \left[ S^{-1} \right] :=
	\left\{ \frac{r}{s} \ \middle|\ r \in R, \, s \in S \text{ and } \frac{r}{s} = \frac{r'}{s'}
	\iff \exists\, t \in S \text{ s.t. } t \left( rs' - r's \right) = 0 \right\}
	.\end{equation} 
	Notice, moreover, that given any module $M$, then
	\begin{equation}
	M \otimes_R R_S =: M_S =
	\left\{ \frac{x}{s} \ \middle|\ x \in M, \, s \in S \right\}
	\end{equation} 
	and $\frac{x}{s} = \frac{x'}{s'}$ iff there exists $t \in S \text{ s.t. } t \left( xs' - x's \right) = 0$.
	In particular $\frac{x}{1} = 0$ iff $\exists\, t \in S$ s.t. $tx = 0$.
	Moreover any element $\zeta \in M_R \otimes_R R_S$ can be represented as $y \otimes \frac{1}{s}$,
	for $s \in S$ and $y \in M$.

	Let's prove that $R_S$ is a flat $R$-module.
	Consider a mono $0 \to A_R \xrightarrow{\mu} B_R$, we have to prove that
	\begin{equation}
	A_R \otimes_R R_S \xrightarrow{\mu \otimes 1_{R_S}} B_R \otimes_R R_S
	\end{equation}
	is still mono.
	Let's consider $x \otimes \frac{1}{t} \in A_R \otimes_R R_S$, then 
	$\frac{x}{t} \xmapsto{\mu} \frac{\mu(x)}{t}$.
	Assume $\frac{\mu(x)}{t} = 0$, i.e. there exists $s \in S$ s.t. $s \mu(x) = 0$,
	which means $\mu(sx) = 0$, hence $sx = 0$, since $\mu$ is mono.
	But this means that $\frac{x}{t} = 0$.
\end{ex}

\begin{thm}[Lazard]
	A module is flat iff
	it is a filtered direct limit of projective modules,
	or a direct limit of finitely generated free modules.
	(It can be specialized to left or right modules, then every module in the
	statement has to be either left or right, accordingly).
\end{thm}

\begin{lem}
	Let $\mathsf{C}$ and $\mathsf{D}$ be abelian categories, and $L: \mathsf{C} \to \mathsf{D}$ and
	$R: \mathsf{D} \to \mathsf{C}$ be an adjoint pair $\left(L, R\right)$.
	Assume that $L$ is an exact functor.
	Then, if $I$ is an injective object of $\mathsf{D}$,
	then $R(I)$ is injective in $\mathsf{C}$.
	Dually, if $R$ is exact, and $P$ is a projective object of $\mathsf{C}$, then
	$L(P)$ is a projective object of $\mathsf{D}$.
\end{lem} 

\begin{prop}
	Let ${}_SF_R$ be an $S$-$R$-bimodule and ${}_SE$ be an injective left $S$-module, then
	\begin{itemize}
		\item If $F_R$ is flat, then $\mathrm{Hom}_{S}\left( {}_SF_R, {}_SE \right)$ is 
			an injective left $R$-module.
		\item Conversely, if ${}_SE$ is an injective cogenerator of $S$-Mod
			and $\mathrm{Hom}_{S}\left( {}_SF_R, {}_SE \right)$
			is an injective left $R$-module, then $F_R$ is flat.
	\end{itemize}
\end{prop} 

\begin{cor}
	Since $\mathbb{Q}/\Z$ is an injective cogenerator in the category $\mathsf{Ab} = \mathsf{Mod}\text{-}\Z$:
	it is the direct sum of the injective envelopes of the simple modules $\mathbb{Z}/p\mathbb{Z}$:
	\begin{equation}
		\mathbb{Q}/\Z = \bigoplus_{p \in P} E \left( \mathbb{Z}/p\mathbb{Z} \right)
	.\end{equation} 
	The module ${}_{\Z}F_R$ is flat iff
	$\mathrm{Hom}_{\Z}\left( F, \mathbb{Q}/\Z \right)$ is an injective left
	$R$-module.
	Moreover we use the notation for the above, which is also called the \textbf{character module}
	\begin{equation}
	F_R^*:= \mathrm{Hom}_{\Z}\left( F_R, \mathbb{Q}/\Z \right)
	.\end{equation} 
\end{cor} 

\begin{thm}[Dimension shifting for right derived functors]\leavevmode\vspace{-.2\baselineskip}
	\begin{enumerate}
		\item Let $F: \mathsf{A} \to \mathsf{B}$ be a covariant left-exact functor
			between abelian categories, with $\mathsf{A}$ having enough injectives.
			Let $Q$ be an $F$-acyclic object (e.g. $Q$ injective) and
			\begin{equation}
			0 \to K \to Q \to A \to 0
			\end{equation} 
			be a s.e.s.
			Then, for all $i \geq 1$,
			\begin{equation}
				R^iF(A) \simeq R^{i+1}F(K)
			.\end{equation} 
		\item Let $F: \mathsf{A} \to \mathsf{B}$ be a contravariant left-exact functor
			between abelian categories, with $\mathsf{A}$ having enough projectives.
			Let $Q$ be an $F$-acyclic object (e.g. $Q$ projective) and
			\begin{equation}
			0 \to K \to Q \to A \to 0
			\end{equation} 
			be a s.e.s.
			Then, for all $i \geq 1$,
			\begin{equation}
				R^iF(K) \simeq R^{i+1}F(A)
			.\end{equation} 
	\end{enumerate}
\end{thm}
\begin{proof}\leavevmode\vspace{-.2\baselineskip}
	\begin{enumerate}
		\item Consider the long exact sequence
			\begin{equation}
				R^1F(K) \to R^1F(Q) = 0 \to R^1F(A) \to
				R^2F(K) \to R^2F(Q) = 0 \to \ldots
			.\end{equation} 
			Since $R^iF(Q) = 0$ for all $i$, we have our thesis.
		\item Consider the long exact sequence
			\begin{equation}
				R^1F(A) \to R^1F(Q) = 0 \to R^1F(K) \to
				R^2F(A) \to R^2F(Q) = 0 \to \ldots
			.\end{equation} 
			Since $R^iF(Q) = 0$ for all $i$, we have our thesis.\qedhere
	\end{enumerate}
\end{proof}

\begin{rem}[]
	Assume that $\mathsf{A}$ is an abelian category with enough projectives.
	Consider $M \in \mathrm{Ob} \left(\mathsf{A}\right)$ s.t. for all $N \in \mathrm{Ob} \left(\mathsf{A}\right)$
	\begin{equation}
		\mathrm{Ext}^{n+i}_{\mathsf{A}}(M,N) = 0 \qquad \,\forall\, i \geq 0
	.\end{equation} 
	By dimension shifting $\mathrm{Ext}^1_{\mathsf{A}}(K_n,N) = \mathrm{Ext}^2_{\mathsf{A}}(K_{n-1},N) =
	\ldots = \mathrm{Ext}^{n+1}_{\mathsf{A}}(M,N)$ for all $N \in \mathrm{Ob} \left(\mathsf{A}\right)$.
	And, moreover, for $K_n = \Omega_n(M)$, the $n$-th syzygy of $M$, we have
	\begin{equation}
		\mathrm{Ext}^{n+1}_{\mathsf{A}}(M,N) \simeq
		\mathrm{Ext}^1_{\mathsf{A}}(K_n,N)
	.\end{equation} 
	In particular the above condition holds iff $K_n$ is projective.

	Analogously, if $\mathsf{A}$ has enough injectives we obtain:
	By dimension shifting $\mathrm{Ext}^1_{\mathsf{A}}(M,C_n) = \mathrm{Ext}^2_{\mathsf{A}}(M,C_{n-1}) =
	\ldots = \mathrm{Ext}^{n+1}_{\mathsf{A}}(M,N)$ for all $N \in \mathrm{Ob} \left(\mathsf{A}\right)$.
	And, moreover, for $C_n = \Omega^n(N)$, the $n$-th cosyzygy of $N$, we have
	\begin{equation}
		\mathrm{Ext}^{n+1}_{\mathsf{A}}(M,N) \simeq
		\mathrm{Ext}^1_{\mathsf{A}}(M,C_n)
	.\end{equation} 
	In particular $\mathrm{Ext}^{n+i}_{\mathsf{A}}(M,N) = 0 \,\forall\, i \geq 0$
	iff $C_n$ is injective.
\end{rem}

\begin{lem}[Schanuel]
	Let $\mathsf{A}$ be an abelian category. Let $P,Q \in \mathrm{Ob} \left(\mathsf{A}\right)$
	be projective objects.
	Assume that the following are s.e.s.
	\begin{equation}
	0\to K\to P \to M \to 0
	\qquad \text{ and } \qquad
	0 \to H \to Q \to M \to 0
	.\end{equation} 
	Then $K \oplus Q \simeq H \oplus P$.
	In particular $K$ is projective iff $H$ is projective.
\end{lem} 
\begin{cor}
	Consider the two long exact sequences with $P_{i}, Q_{i}$ projective
	\begin{equation}
	0 \to K_n \to P_{n-1} \to P_{n-2} \to \ldots \to P_1 \to P_0 \to M \to 0
	\end{equation} 
	and
	\begin{equation}
	0 \to H_n \to Q_{n-1} \to Q_{n-2} \to \ldots \to Q_1 \to Q_0 \to M \to 0
	.\end{equation} 
	Then
	\begin{equation}
	K_n \oplus Q_{n-1} \oplus P_{n-2} \oplus \ldots \simeq
	K_n \oplus P_{n-1} \oplus Q_{n-2} \oplus \ldots
	.\end{equation} 
	In particular $K_n$ is projective iff $H_n$ is projective.
\end{cor} 

\begin{defn}[Projective dimension]
	Let $\mathsf{A}$ be an abelian category, with enough proectives.
	Consider $M \in \mathrm{Ob} \left(\mathsf{A}\right)$.
	We define the \textbf{projective dimension} of $M$, denoted by $\mathrm{p.d.}(M)$, 
	as the smallest integer $n \in \N$ s.t. there exist $P_i \in \mathrm{Ob} \left(\mathsf{A}\right)$
	projective and an exact sequence
	\begin{equation}
	0 \to P_n \to P_{n-1} \to \ldots \to P_1 \to P_0 \to M \to 0
	,\end{equation} 
	i.e. it is the minimal length of a projective resolution of $M$.
	Equivalently $n$ is the minimal index s.t. the $n$-th syzygy of $M$ is
	already a projective object.
	If no finite resolution exists, we define $\mathrm{p.d.}\, M = \infty$.
\end{defn}

\begin{rem}[]
	The projective dimension is well defined thanks to Schanuel lemma.
\end{rem}

\begin{ex}[Infinite projective dimension]
	Let $R := \mathbb{Z}/2\mathbb{Z}$ and $M := \mathbb{Z}/2\mathbb{Z}$ as an $R$-module.
	Then
	\begin{equation}
	0 \to \mathbb{Z}/2\mathbb{Z} \to \mathbb{Z}/4\mathbb{Z} \xrightarrow{\pi} \mathbb{Z}/2\mathbb{Z} \to 0
	\end{equation} 
	is exact.
	This means that $\Omega_1(M) = M$, hence $\mathrm{p.d.}\, M = \infty$.

	Analogouusly for $R := \mathbb{K}[x]/ \left( x^2 \right)$, for a field $\K$,
	$R$ is called the ring of dual numbers.
	Then $M_R := (x) / (x^2)$ has infinite projective dimension.
\end{ex} 

\begin{prop}
	Let $\mathsf{A}$ be an abelian category with enough projectives.
	Let $M \in \mathrm{Ob} \left(\mathsf{A}\right)$, then TFAE
	\begin{enumerate}
		\item $\mathrm{p.d.}\, M \leq n$,
		\item $\mathrm{Ext}^{n+i}_{\mathsf{A}} (M,N) = 0$ for all $N \in \mathsf{A}$, and all $i \geq 1$,
		\item $\mathrm{Ext}^{n+1}_{\mathsf{A}} (M,N) = 0$ for all $N \in \mathsf{A}$.
	\end{enumerate}
\end{prop} 
\begin{cor}
	If $M \in \mathrm{Ob} \left(\mathsf{A}\right)$ (for $\mathsf{A}$ as above) 
	has $\mathrm{p.d.}\, M = n$, then
	$\mathrm{Ext}^{n+1}_{\mathsf{A}}(M,N) = 0$ for all $N \in \mathrm{Ob} \left(\mathsf{A}\right)$
	and $\exists\, N_0 \in \mathrm{Ob} \left(\mathsf{A}\right)$ s.t.
	$\mathrm{Ext}^n_{\mathsf{A}}(M,N_0) \neq 0$.
\end{cor} 
Let's now dualize everything for injectives
\begin{lem}[Schanuel for injectives]
	Let $\mathsf{A}$ an abelian category and $M \in \mathrm{Ob} \left(\mathsf{A}\right)$.
	Let $I, E \in \mathrm{Ob} \left(\mathsf{A}\right)$ be injective objects.
	Assume the following are s.e.s.
	\begin{equation}
	0 \to M \to I \to C \to 0
	\qquad \text{ and } \qquad
	0 \to M \to E \to D \to 0
	.\end{equation} 
	Then $C \oplus E \simeq I \oplus D$.
	In particular $D$ is injective iff $C$ is injective.
\end{lem} 

\begin{cor}
	Consider the two long exact sequences with $I^n, E^n$ injective
	\begin{equation}
	0 \to M \to I^0 \to I^1 \to \ldots \to I^{n-1} \to C \to 0
	\end{equation} 
	and
	\begin{equation}
	0 \to M \to E^0 \to E^1 \to \ldots \to E^{n-1} \to D \to 0
	.\end{equation} 
	Then 
	\begin{equation}
	C \oplus E^{n-1} \oplus I^{n-2} \oplus \ldots \simeq
	D \oplus I^{n-1} \oplus E^{n-2} \oplus \ldots
	.\end{equation} 
	In particular $C$ is injective iff $D$ is injective.
\end{cor} 

\begin{defn}[Injective dimension]
	Let $\mathsf{A}$ be an abelian category with enough injectives.
	Consider $M \in \mathrm{Ob} \left(\mathsf{A}\right)$.
	We define the \textbf{injective dimension} of $M$, 
	denoted by $\mathrm{i.d.}\, M$, as the smallest integer $n \in \N$ s.t.
	there exist $I^n \in \mathrm{Ob} \left(\mathsf{A}\right)$ injective and an exact sequence
	\begin{equation}
	0 \to M \to I^0 \to I^1 \to \ldots \to
	I^{n-1} \to I^n \to 0
	,\end{equation} 
	i.e. $n$ is the minimal legth of an injective coresolution of $M$.
	Equivalently $n$ is the minimal index s.t. the $n$-th cosyzygy of $M$ is already
	an injective object.
	If no finite resolution exists, we define $\mathrm{i.d.}\, M = \infty$.
\end{defn}

\begin{ex}[Infinite injective dimension]
	Consider $R := \mathbb{Z}/4\mathbb{Z}$.
	Prove that $R$ is self injective, i.e. $R$ is injective as an $R$-module
	(you can prove using Baer's criterion).
	Let $M_R := \mathbb{Z}/2\mathbb{Z}$.
	Prove that $\mathrm{i.d.}\, M_R = \infty$.
\end{ex} 

\begin{prop}
	Let $\mathsf{A}$ be an abelian category with enough projectives.
	Let $M \in \mathrm{Ob} \left(\mathsf{A}\right)$, then TFAE
	\begin{enumerate}
		\item $\mathrm{i.d.}\, M \leq n$,
		\item $\mathrm{Ext}^{n+i}_{\mathsf{A}} (N,M) = 0$ for all $N \in \mathsf{A}$, and all $i \geq 1$,
		\item $\mathrm{Ext}^{n+1}_{\mathsf{A}} (N,M) = 0$ for all $N \in \mathsf{A}$.
	\end{enumerate}
\end{prop} 

\begin{defn}[Right global dimension of $R$]
	Let $R$ be a ring.
	We define the \textbf{right global dimension} of $R$, denoted by
	$\mathrm{r.gld}\, R$, as
	\begin{equation}
	\mathrm{r.gld}\, R := \sup \left\{ \mathrm{p.d.}\, M_R \ \middle|\ M_R \in \mathsf{Mod}\text{-}R \right\}
	.\end{equation} 
\end{defn}

\begin{thm}[Global dimension]
	Consider the following numbers:
	\begin{align}
		(2) &:= \sup \left\{ \mathrm{i.d.}\, M_R \ \middle|\ M_R \in \mathsf{Mod}\text{-}R \right\}\\
		(3) &:= \sup \left\{ \mathrm{p.d.}\, R/I_R \ \middle|\ I_R \triangleleft R 
		\text{ is a right ideal}\, \right\}\\
		(4) &:= \sup \left\{n \in \N \ \middle|\ \mathrm{Ext}^n_{\mathsf{A}}(M,N) \neq 0
		\text{ for some } M_R, N_R \in \mathsf{Mod}\text{-}R \right\}
	.\end{align} 
	Let's call $(1) := \mathrm{r.gld}\, R$.
	Then, if finite, $(1) = (2) = (3) = (4)$.
	Moreover, if any is infinite, also all the others are.
\end{thm}

\begin{lem}
	Let $R$ be a ring.
	Consider the s.e.s.
	\begin{equation}
	0 \to K \to F \to M \to 0
	\qquad \text{ and } \qquad
	0 \to H \to G \to M \to 0
	,\end{equation} 
	with $F, G$ flat.
	Then $K$ is flat iff $H$ is flat.
\end{lem} 

\begin{defn}[Flat (weak) dimension]
	Let $R$ be a ring and $M_R \in \mathsf{Mod}\text{-}R$.
	We define the \textbf{flat (or weak) dimension} of $M_R$,
	denoted by $\mathrm{f.d}_R\, M_R$ or $\mathrm{w.d.}\, M_R$,
	as the minimum length of a flat resolution of $M$.
\end{defn}
\begin{rem}[]
	Clearly, by the above lemma, this is a good definition.
\end{rem}

\begin{prop}
	For $M_R \in \mathsf{Mod}\text{-}R$, the following are equivalent:
	\begin{enumerate}
		\item $\mathrm{w.d.}\, M \leq n$,
		\item $\mathrm{Tor}_{n+i}^R (N,M) = 0$ for all $N \in \mathsf{A}$, and all $i \geq 1$,
		\item $\mathrm{Tor}_{n+1}^R (N,M) = 0$ for all $N \in \mathsf{A}$.
	\end{enumerate}
\end{prop} 

\begin{defn}[Right weak-global dimension]
	Let $R$ be a ring.
	We define the \textbf{right weeak global dimension} of $R$,
	denoted by $\mathrm{r.w.gld}\, R$,
	as
	\begin{equation}
	\mathrm{r.w.gl.dim}\, R := \sup \left\{ \mathrm{w.d.}\, M_R \ \middle|\ M_R \in \mathsf{Mod}\text{-}R \right\}
	.\end{equation} 
\end{defn}

\begin{rem}[]
	Notice that $\mathrm{r.w.gl.dim}\, R = \sup \left\{ \mathrm{w.d.}\, {}_RN \ \middle|\ {}_RN \in R\text{-}\mathsf{Mod} \right\}$.
	Then we can anlogoulsy define the \textbf{left weak global dimension} of $R$
	and $\mathrm{r.w.gl.dim}\, R = \mathrm{l.w.gl.dim}\, R$.
\end{rem}

\begin{rem}[]
	Consider $M_R \in \mathsf{Mod}\text{-}R$.
	We defined its character module $M^* := \mathrm{Hom}_{\Z}\left( M, \mathbb{Q}/\Z \right)$
	and proved that $M_R$ is flat iff $M^*$ is injective.
	Moreover we can define a canonical map $\mu: M \to M^{**}$ that
	acts as: given $x \in M_R$, $\mu(x) \in \mathrm{Hom}_{ \Z}\left( M^*, \mathbb{Q}/\Z \right)$
	s.t. for $f \in M^* = \mathrm{Hom}_{\Z}\left( M, \mathbb{Q}/\Z \right)$,
	we define $\mu(x) (f) := f(x)$.
	Notice that $\mu$ is mono, since $\mathbb{Q}/\Z$ is an injective cogenerator of $\mathsf{Ab}$.
	Moreover, for any $0 \neq x \in M$, we can define a nonzero map
	$g: \left\langle x \right\rangle_{\Z} \to \mathbb{Q}/\Z$
	which can be extended to the whole $M$.
\end{rem}

\begin{prop}
	Let $M_R$ be a right $R$-module, then TFAE
	\begin{enumerate}
		\item $M_R$ is flat,
		\item $M^*$ is an injective left $R$-module,
		\item for all ${}_RI \triangleleft {}_RR$ (a left ideal)
			\begin{equation}
			M_R \otimes_R I \simeq MI =
			\left\{ \sum_{i=1}^{n} x_i a_i \ \middle|\ x_i \in M_R,\, a_i \in {}_RI,\, n \in \N \right\}
			,\end{equation} 
		\item $\mathrm{Tor}^R_1(M, R/{}_RI) = 0$ for all ${}_RI \triangleleft {}_RR$.
	\end{enumerate}
	(Clearly all of the above holds true even for left $R$-modules).
\end{prop} 

\begin{rem}[]
	Consider the embedding $0 \to I \xrightarrow{\epsilon} R$ of $I$ into $R$ and $M_R \in \mathsf{Mod}\text{-}R$,
	then we can take the tensor
	$M \otimes_R I \xrightarrow{id_M \otimes \epsilon} M \otimes_R R \simeq M$ acting as
	$x \otimes a \mapsto xa$.
	Then
	\begin{equation}
		\ima (id_M \otimes \epsilon) = \left\{ \sum_{i=1}^{n} x_ia_i \ \middle|\ x_i \in M, a_i \in I \right\} =
		MI
	.\end{equation} 
	Thus $M \otimes I \simeq MI$ iff $id_M \otimes \epsilon$ is mono.
\end{rem}

\begin{lem}
	Consider $f: M_R \to N_R$ a morphism in the catogory of right $R$-modules.
	Let $f^* := \mathrm{Hom}_{\Z}\left( f, \mathbb{Q}/\Z \right)$.
	\begin{itemize}
		\item $f$ is mono iff $f^*$ is epi,
		\item $f$ is epi iff $f^*$ is mono.
	\end{itemize}
\end{lem} 

\begin{lem}
	Consider $M_R$ and ${}_RN$. Then there are canonical isomorphisms:
	\begin{itemize}
		\item $M_R \otimes_R R \simeq R$ as right $R$-modules 
			(resp. $R \otimes_R N \simeq {}_RN$ as left $R$-modules),
		\item $\mathrm{Hom}_{ R}\left( R, M \right) \simeq M$ as right $R$-modules
			(resp. $\mathrm{Hom}_{ R}\left( R, N \right) \simeq N$ as left $R$-modules),
		\item $M \otimes_R R/{}_RI \simeq M/MI$ as abelian groups
			(resp. $R/I_R \otimes N \simeq N/IN$ as abelian groups).
	\end{itemize}
\end{lem} 

\begin{lem}
	Let $\K$ be a field.
	Consider $R := \K[x,y]$ and $\mathfrak{m} := (x,y)$.
	Then $R/\mathfrak{m} \simeq \K$.
	\begin{itemize}
		\item Show that $\K$ has a projective resolution
			\begin{equation}
			0 \to R \xrightarrow{\beta} R \oplus R \xrightarrow{\alpha} R \to
			R/\mathfrak{m} \simeq \K \to 0
			,\end{equation} 
			where $\beta = \begin{bmatrix}-y \\ x\end{bmatrix}$ and $\alpha(e_1) = x$, $\alpha(e_2) = y$.
		\item Show that $\mathrm{Tor}^R_2(\K,\K) \simeq \mathrm{Tor}^R_1(\mathfrak{m}, \K) \simeq \K$,
			sot that $\mathfrak{m}$ is torsion-free and not flat.
		\item $\mathrm{p.d.}\, \mathfrak{m} = 1$, $\mathrm{p.d.}\, \K = 2$
			and $\mathrm{w.d.}\, \K = 2$.
	\end{itemize}
\end{lem} 

\begin{prop}\leavevmode\vspace{-.2\baselineskip}
	\begin{enumerate}
		\item $\mathrm{Tor}^R_n \left( \bigoplus_{i \in I} M_i, N \right) \simeq
			\bigoplus_{i \in I} \mathrm{Tor}^R_n(M_i,R)$

		\item For a (from the proof I guess it is filtered)
			direct system of modules $\left\{ M_i, f_{ji} \right\}_{i \leq j}$
			\begin{equation}
				\mathrm{Tor}^R_n \big( \varinjlim_{i \in I} M_I, N \big) \simeq
				\varinjlim_{i \in I} \mathrm{Tor}^R_n(M_i, N)
			.\end{equation} 
	\end{enumerate}
	This theorem holds also for the second component of $\mathrm{Tor}$.
\end{prop}

\begin{lem}
	Let $M_R \in \mathsf{Mod}\text{-}R$.
	$M$ is flat iff 
	\begin{equation}
		\mathrm{Tor}^R_1 (M, R/I) = 0
	\end{equation} 
	for all ${}_RI \triangleleft R$ finitely generated left ideal.
\end{lem} 

\begin{prop}
	Let $\mathsf{A}$ be an abelian category with products, coproducts,
	enough injectives and projectives.
	For any family $\left\{ M_i \right\}_{i \in I}$, $\left\{ N_i \right\}_{i \in I}$,
	any object $M,N \in \mathrm{Ob} \left(\mathsf{A}\right)$ and any $n \in \N$:
	\begin{enumerate}
		\item $\mathrm{Ext}^n_{\mathsf{A}} \left( M, \prod_{i \in I} N_i\right) \simeq
			\prod_{i \in I} \mathrm{Tor}^n_{\mathsf{A}}(M,N_i)$
		\item $\mathrm{Ext}^n_{\mathsf{A}} \left( \bigoplus_{i \in I}M_i, N \right) \simeq
			\prod_{i \in I} \mathrm{Tor}^n_{\mathsf{A}}(M_i,N)$
	\end{enumerate}
	This theorem holds also for the second component of $\mathrm{Tor}$.
\end{prop}

\begin{rem}[]
	Let $\left\{ M_i, f_{ji} \right\}_{i \leq j}$ be a directed system of modules.
	In general
	\begin{equation}
		\mathrm{Ext}^R_n \big( \varinjlim_{i \in I} M_i, N\big)
		\not\simeq \varinjlim_{i \in I} \mathrm{Ext}^R_n (M_i, N)
	.\end{equation} 
\end{rem}

\begin{ex}
	Let $F_R$ be a flat, but not projective right $R$-module.
	Then there exists a module $N$ s.t. $\mathrm{Ext}^1_R(F,N) \neq 0$.
	Moreover $F = \varinjlim_{i \in I} G_i$,
	for $G_i$ finitely generated free modules (by Lazard theorem).
	Then, for all $i \in I$, $\mathrm{Ext}^1_R(G_i, N) = 0$.
	In other words we have a counterexample to the adove "equality".

	(Notice that there exist module such as $F_R$, in fact $\mathbb{Q}$
	is a flat, but not projective, module;
	in particular it is flat, since it is a localization of $\Z$).
\end{ex} 

\begin{defn}[Right (left) hereditary ring]
	A ring $R$ is right (resp left) \textbf{hereditary} iff every submodule of a projective right
	(resp left) $R$-module is projective.
\end{defn}

\begin{prop}[Characterization of hereditary rings]
	Let $R$ be a ring, TFAE:
	\begin{enumerate}
		\item $R$ is right hereditary,
		\item $\mathrm{r.gl.dim}\, R \leq 1$,
		\item $\mathrm{Ext}^2_R(M,N) = 0$ for all $M, N \in \mathsf{Mod}\text{-}R$,
		\item $\mathrm{Ext}^2_R(R/I,N) = 0$ for all $N_R \in \mathsf{Mod}\text{-}R$ and $I_R \triangleleft R$
			right ideal,
		\item $I_R$ is projective for every right ideal $I_R \triangleleft R$.
	\end{enumerate}
	Clearly there exists also a left version of this proposition
	(instead of the right global dimension one checks left global dimension).
\end{prop} 

\begin{ex}
	Being right or left hereditary is not symmetrical
	In particular Kaplansky constructed an example of a ring $R$ which is right hereditary,
	but has $\mathrm{l.gl.dim}\, R = 2$, i.e. it is not left hereditary.
	Small gave another example of a right hereditary ring $R$, 
	with $\mathrm{l.gl.dim}\, R = 3$.
\end{ex} 

Recall the definition of weak global dimension of a ring $R$:
\begin{equation}
\mathrm{w.gl.dim}\, R = \sup \left\{ \mathrm{w.dim.}\, M_R \ \middle|\ M_R \in \mathsf{Mod}\text{-}R \right\}
.\end{equation} 
(By symmetry of $\mathrm{Tor}$ functor this coincides with the left weak global dimension).

\begin{prop}
	Let $R$ be a ring. TFAE:
	\begin{enumerate}
		\item $\mathrm{w.gl.dim}\, R \leq 1$,
		\item $\mathrm{Tor}^R_2(M,N) = 0$ for all $M \in \mathsf{Mod}\text{-}R$ and $N \in R\text{-}\mathsf{Mod}$,
		\item Every submodule of a flat module is flat,
		\item $\mathrm{Tor}^2_R(R/I_R, N) = 0$ for all $I_R \triangleleft R$ right ideals
			and $N \in R\text{-}\mathsf{Mod}$,
		\item Every right ideal $I_R \triangleleft R$ is a flat $R$-module.
	\end{enumerate}
\end{prop} 

\begin{rem}[]
	Recall that we have the following implications:
	choose a ring $R$, and an $R$-module $M$, then
	\begin{itemize}
		\item $M$ free $\implies$ $M$ projective,
		\item $M$ projective $\implies$ $M$ flat,
		\item direct limits of projective modules are flat,
		\item in general, it is not true that any flat module is projective.
	\end{itemize}
	We now want to show that in the particular case where $M$ is finitely presented, then
	it is projective as soon as it is flat.
\end{rem}

\begin{defn}[Finitely presented module]
	Recall that $M \in R\text{-}\mathsf{Mod}$ is finitely presented iff there is a s.e.s.
	\begin{equation}
	0 \to K \to R^n \to M \to 0
	,\end{equation} 
	with $n \in \N$ and $K$ a finitely generated $R$-module.
\end{defn}

\begin{lem}
	Let $M_R$ be a finitely presented module and consider the s.e.s.
	\begin{equation}
	0 \to H \to P \to M \to 0
	,\end{equation} 
	with $P$ projective and finitely generated.
	Then $H$ is finitely generated.
\end{lem} 
\begin{rem}[]
	The above implies that $M_R$ is finitely presented iff there is an exact sequence
	\begin{equation}
	R^m \to R^n \to M \to 0
	,\end{equation} 
	for some $m,n \in \N$.
\end{rem}

\begin{rem}[Pano's guess]
	I guess that any finitely generated projective $R$-module $P$ is also finitely presented:
	\begin{equation}
		0 \to K \to R^n \to P \to 0
	\end{equation} 
	given such an exact sequence, by projectivity it splits.
	Then $K$ is a direct summand of a free module, hence it is finitely generated.
\end{rem}


\begin{rem}[]
	Recall that, fixed a pair of right $R$-modules $M_R, N_R \in \mathsf{Mod}\text{-}R$,
	then $\mathrm{Hom}_{\Z}\left( N, \mathbb{Q}/\Z \right) = 
	\left( N_R \right)^* \in R\text{-}\mathsf{Mod}$ is a left $R$-module.
	Moreover there exists a morphism in $\mathsf{Ab}$
	\begin{align}
		\sigma_{M,N}: M \otimes N^* &\to \left[ \mathrm{Hom}_{R}\left( M, N \right) \right]^* 
		= \mathrm{Hom}_{\Z}\left( \mathrm{Hom}_{ R}\left( M, N \right), \mathbb{Q}/\Z \right)\\
		x \otimes f &\mapsto g
	,\end{align} 
	where $g$ acts as follows on $\alpha \in \mathrm{Hom}_{R}\left( M, N \right)$
	\begin{equation}
		g(\alpha) := f \left( \alpha(x) \right)
	.\end{equation} 
\end{rem}

\begin{lem}
	Let $M_R$ be a finitely presented $R$-module,
	then $\sigma_{M,N}$, as defined above, is an isomorphism for all $N \in \mathsf{Mod}\text{-}R$.
\end{lem} 

\begin{thm}[]
	A finitely presented flat module $M_R$ is projective.
\end{thm}

\subsubsection{Tor under change of rings}
Let $R,S$ be rings, and $f: R \to S$ a ring homomorphism.
Then $S$ is an $R$-$R$ bimodule via $f$ and every $S$-module
is an $R$-module via restriction of scalars.
Moreover, given any $M_R \in \mathsf{Mod}\text{-}R$, then
$M_R \otimes_R S$ is a right $S$-module via extension of scalars.

\begin{prop}
	Let $f: R \to S$ be a ring homomorphism.
	Assume that ${}_RS$ is a flat left $R$-module.
	Then for all $M_R \in \mathsf{Mod}\text{-}R$, $n \in \N$ and ${}_SC \in S\text{-}\mathsf{Mod}$
	(hence we also have ${}_SC \in R\text{-}\mathsf{Mod}$)
	\begin{equation}
		\mathrm{Tor}^R_n (M_R, {}_SC) \simeq
		\mathrm{Tor}^S_n (M \otimes_R S, {}_SC)
	.\end{equation} 
\end{prop} 

\begin{prop}
	Let $f: R \to S$ be a ring homomorphism.
	Assume that ${}_RS$ is a flat left $R$-module.
	Then, for all $M_R \in \mathsf{Mod}\text{-}R$, $C_S \in \mathsf{Mod}\text{-}S$ and $n \in \N$
	\begin{equation}
		\mathrm{Ext}^n_R (M, C) \simeq
		\mathrm{Ext}^n_S (M \otimes_R S, C)
	.\end{equation} 
\end{prop} 

\begin{prop}
	Let $R, S$ be commutative rings, and $f: R \to S$ a ring homomorphism.
	Assume $S$ is a flat $R$-module.
	Then for all modules $M$ and $N$, and $n \in \N$
	\begin{equation}
		\mathrm{Tor}^R_n (M,N) \otimes_R S \simeq
		\mathrm{Tor}^S_n (M \otimes_R S, N \otimes_R S)
	.\end{equation} 
\end{prop} 


\begin{cor}
	Let $R$ be a commutative ring, $M$ and $N$ be $R$-modules and $n \in \N$.
	TFAE:
	\begin{enumerate}
		\item $\mathrm{Tor}^R_n (M,N) = 0$,
		\item $\mathrm{Tor}_n^{R_P}(M_P, N_P) = 0$ for all $P \in \mathrm{Spec}\, R$,
		\item $\mathrm{Tor}_n^{R_{\mathfrak{m}}} (M_{\mathfrak{m}}, N_{\mathfrak{m}})$
			for all $\mathfrak{m} \in \mathrm{MaxSpec}\, R$.
	\end{enumerate}
\end{cor} 

\subsubsection{Hom and Ext with finitely presented modules}
Let $R, S$ be commutative rings, $\varphi: R \to S$ be a ring homomorphism
and $S$ be flat (e.g. $T$ a multiplicatively closed subset of $R$
and $S := R_T = RT^{-1}$).

\begin{prop}
	Let $R,S$ be commutative rings, $\varphi: R \to S$ be a ring homomorphism.
	Assume $S$ is a flat $R$-module
	and consider $M_R$ a finitely presented $R$-module.
	Then, for any $N \in \mathsf{Mod}\text{-}R$
	\begin{equation}
	\mathrm{Hom}_{ S}\left( M \otimes_R S, N \otimes_R S \right) \simeq
	\mathrm{Hom}_{ R}\left( M, N \right) \otimes_R S
	.\end{equation} 	
\end{prop} 

Let's now extend this result for the $\mathrm{Ext}$ functor:
\begin{defn}
	We denote by $\mathsf{mod}\text{-}R$ (using the lowercase m to differentiate from the
	bigger category) the category of right $R$-modules $M$ with a projective
	resolution of finitely generated projective modules
	(i.e. all the syzygys $\Omega_n(M)$ are finitely generated:
	at each point the kernel [i.e. the syzygy] is an epimorphic image of a finitely generated module).
\end{defn}

\begin{rem}[]
	If $R$ is a right Noetherian ring, then
	the objects of $\mathrm{mod}\text{-}R$ are exactly the finitely generated $R$-modules.
\end{rem}

\begin{defn}[Right coherent ring]
	a ring $R$ is \textbf{right coherent} iff every finitely generated right ideal
	is also finitely presented.
	Equivalently iff every finitely generated submodule of a finitely
	presented right $R$-module is finitely presented.
\end{defn}
\begin{rem}[]
	Let $R$ be right coherent, then
	$\mathsf{mod}\text{-}R$ is the category of finitely presented right $R$-modules.
\end{rem}

\begin{prop}
	Let $R,S$ be commutative rings, $\varphi: R \to S$ a ring homomorphism
	and assume that $S$ is a flat $R$-module.
	Consider $M \in \mathsf{mod}\text{-}R$ and $N \in \mathsf{Mod}\text{-}R$,
	then for all $n \in \N$
	\begin{equation}
		\mathrm{Ext}^n_S \left( M \otimes_R S, N \otimes_R S \right) \simeq
		\mathrm{Ext}^n_R \left( M,N \right) \otimes_R S
	.\end{equation} 
\end{prop} 

\begin{cor}
	Let $R$ be a commutative ring,
	$M_R \in \mathsf{mod}\text{-}R$, $N \in \mathsf{Mod}\text{-}R$ and $n \in \N$.
	TFAE:
	\begin{enumerate}
		\item $\mathrm{Ext}^n_R = 0$,
		\item $\mathrm{Ext}^{ n}_{ R_P} \left( M_P, N_P \right) = 0$ for all $P \in \mathrm{Spec}\, R$,
		\item $\mathrm{Ext}^{ n}_{ R_{\mathfrak{m}}} \left( M_{\mathfrak{m}}, N_{\mathfrak{m}} \right) = 0$
			for all $\mathfrak{m} \in \mathrm{MaxSpec}\, R$.
	\end{enumerate}
\end{cor} 

\subsection{Homological formulas relating Ext and Hom}
\begin{prop}
	Consider $R,S$ rings.
	Let ${}_RN_S$ be an $S$-$R$ bimodule and $M_R \in \mathsf{Mod}\text{-}R$.
	Consider $C_S$ an injective right $S$-module.
	Then for all $n \geq 0$
	\begin{equation}
	\mathrm{Ext}^{ n}_{ R} \left( M_R, \mathrm{Hom}_{ S}\left( N_S, C_S \right)_R \right) \simeq
	\mathrm{Hom}_{ S}\left( \mathrm{Tor}^{ R}_{ n} \left( M, N \right)_S, C_S  \right)
	.\end{equation} 
	In particular, if $S := \Z$ and $C := \mathbb{Q}/\Z$, then
	\begin{equation}
	\mathrm{Ext}^{ n}_{ R} \left( M_R, N^* \right) \simeq
	\left[ \mathrm{Tor}^{ R}_{ n} \left( M, N \right)\right]^*
	.\end{equation} 
\end{prop} 

\begin{prop}
	Consider $R,S$ rings.
	Let ${}_RN_S$ be an $S$-$R$ bimodule and $M_R \in \mathsf{mod}\text{-}R$.
	Consider ${}_SC$ an injective left $S$-module.
	Then for all $n \geq 0$
	\begin{equation}
		\mathrm{Tor}^{ R}_{ n} \left( M_R, \mathrm{Hom}_{ S}\left( {}_SN_R, {}_SC \right) \right) \simeq
		\mathrm{Hom}_{ S}\left( \mathrm{Ext}^{ n}_{ R} \left( M_R, {}_SN_R \right), {}_SC \right)
	.\end{equation} 
	In particular, if $S := \Z$ and $C := \mathbb{Q}/\Z$, then
	\begin{equation}
		\mathrm{Tor}^{ R}_{ n} \left( M, N^* \right) \simeq
		\left[ \mathrm{Ext}^{ n}_{ R} \left( M, N \right) \right]^*
	.\end{equation} 
\end{prop} 

\begin{ex}
	Let $M_R \in \mathsf{Mod}\text{-}R$,
	${}_RG_S$ and $R$-$S$ bimodule and $C_S \in \mathsf{Mod}\text{-}S$.
	Assume that $\mathrm{Tor}^{ R}_{ 1} \left( M, G \right) = 0$, then
	there is a monomorphism (of abelian groups)
	\begin{equation}
		\mathrm{Ext}^{ 1}_{ R} \left( M_R, \mathrm{Hom}_{ S}\left( {}_RG_S, C_S \right) \right) \hookrightarrow 
		\mathrm{Ext}^{ 1}_{ S} \left( M \otimes_R G, C_S \right)
	.\end{equation}
\end{ex} 

\subsubsection{Yoneda extension}
Our next aim is, given an abelian category $\mathsf{A}$ and objects $A,B \in \mathrm{Ob} \left(\mathsf{A}\right)$,
to define $\mathrm{Ext}^{ }_{ \mathsf{A}} \left( A, B \right)$ even though
$\mathsf{A}$ might not have enough injectives nor projectives.
\begin{defn}[Extension]
	Let $\mathsf{A}$ be an abelian category, $A,B \in \mathrm{Ob} \left(\mathsf{A}\right)$.
	An extension of $A$ by $B$ is a s.e.s.
	\begin{equation}
		\zeta := 0 \to B \to  X \to A \to 0
	.\end{equation} 
	We say that two extensions $\zeta$ and $\zeta'$ are equivalent, denoted by $\zeta \sim \zeta'$,
	iff there is a commutative diagram s.t. the nontrivial vertical arrow is an isomorphism
	\begin{equation}
	\begin{tikzcd}
		\zeta : &
		0 \arrow[r, "", rightarrow] &
		B \arrow[r, "", rightarrow] \arrow[d, "", equal] &
		X \arrow[r, "", rightarrow] \arrow[d, "\simeq", rightarrow] &
		A \arrow[r, "", rightarrow] \arrow[d, "", equal] &
		0\\
		\zeta' : &
		0 \arrow[r, "", rightarrow] &
		B \arrow[r, "", rightarrow] &
		X' \arrow[r, "", rightarrow] &
		A \arrow[r, "", rightarrow] &
		0
	\end{tikzcd}
	.\end{equation} 
\end{defn}
\begin{rem}[Split extensions]
	Recall the characterization of splitting s.e.s.s: an extension
	\begin{equation}
	\zeta := 0 \to B \xrightarrow{\mu} X \xrightarrow{p} A \to 0
	\end{equation} 
	splits iff it is equivalent to the following extension of $A$ by $B$:
	\begin{equation}
	0 \to B \xrightarrow{\epsilon_B} A \oplus B \xrightarrow{\pi_A} A \to 0
	.\end{equation} 
	Equivalently iff there is $f: X \to B$ s.t. $f \circ \mu = 1_B$
	iff there is $g: A \to X$ s.t. $p \circ g = 1_A$.
\end{rem}

\begin{rem}[Class of extensions and Ext]
	Denote by $\mathrm{E}(A,B)$ the class of all extensions of $A$ by $B$.
	If we denote by $\sim$ the above equivalence relation, and we can
	define $\mathrm{Ext}^{ 1}_{ \mathsf{A}} \left( A, B \right)$ (i.e. if $\mathsf{A}$
	has enough injectives of projectives), then we want to construct an isomorphism
	$\theta$ of abelian groups
	\begin{equation}
	\mathrm{Ext}^{ 1}_{ \mathsf{A}} \left( A, B \right) \simeq_{\theta}
	\frac{\mathrm{E}(A,B)}{\sim}
	.\end{equation} 
	Let's define $\theta$:
	Fix $A,B \in \mathrm{Ob} \left(\mathsf{A}\right)$, and consider an extension of $A$ by $B$
	\begin{equation}
	\zeta := 0 \to B \to X \to A \to 0
	.\end{equation} 
	Assume that $\mathrm{Ext}^{ n}_{ \mathsf{A}} \left( -, B \right)$ exist,
	forming a cohomological $\partial$-functor (e.g. if $\mathsf{A}$ has enough
	projectives).
	Apply $\mathrm{Hom}_{\mathsf{A}} \left( -, B \right)$ to $\zeta$, and obtain
	\begin{equation}
	0 \to \mathrm{Hom}_{\mathsf{A}} \left( A, B \right) \to \mathrm{Hom}_{\mathsf{A}} \left( X, B \right) \to 
	\mathrm{Hom}_{\mathsf{A}} \left( B, B \right) \xrightarrow{\partial} \mathrm{Ext}^{ 1}_{ \mathsf{A}} \left( A, B \right)
	.\end{equation} 
	Finally we define $\theta(\zeta) := \partial(1_B) \in \mathrm{Ext}^{ 1}_{ \mathsf{A}} \left( A, B \right)$.
\end{rem}

\begin{lem}
	Fixed $\mathsf{A}$ and $A,B \in \mathrm{Ob} \left(\mathsf{A}\right)$ as before, 
	if $\zeta \sim \zeta' \in \mathrm{E}(A,B)$, then $\theta(\zeta) = \theta(\zeta')$.

	In other words $\theta: \mathrm{E}(A,B) \to \mathrm{Ext}^{ 1}_{ \mathsf{A}} \left( A, B \right)$
	induces a map on the quotient $\frac{\mathrm{E}(A,B)}{\sim}$.
\end{lem} 

\begin{thm}[]
	Given $\mathsf{A}$, $A,B \in \mathrm{Ob} \left(\mathsf{A}\right)$ as before,
	$\theta$ gives a bijective correspondance
	\begin{equation}
	\begin{tikzcd}[column sep=small]
		\frac{\mathrm{E}(A,B)}{\sim} \arrow[r, "\theta", leftrightarrow] &
		\mathrm{Ext}^{ 1}_{ \mathsf{A}} \left( A, B \right)
	\end{tikzcd}
	.\end{equation} 
\end{thm}

\begin{lem}
	Let $\mathsf{A}$ be an abelian category.
	Conider a commutative diagram
	\begin{equation}
	\begin{tikzcd}
		0 \arrow[r, "", rightarrow] &
		K \arrow[r, "\nu", rightarrow] \arrow[d, "\beta"', rightarrow] &
		M \arrow[r, "", rightarrow] \arrow[d, "h", rightarrow] \arrow[ld, "\zeta"', dashrightarrow] &
		A \arrow[r, "", rightarrow] \arrow[d, "", equal] &
		0\\
		0 \arrow[r, "", rightarrow] &
		B \arrow[r, "", rightarrow] &
		Y \arrow[r, "", rightarrow] &
		A \arrow[r, "", rightarrow] &
		0
	\end{tikzcd}
	,\end{equation} 
	with exact rows and where the leftmost square is a pushout.
	Show that there is $g: M \to B$ s.t. $g \circ \nu = \beta$
	iff the second row splits.
\end{lem} 

\begin{lem}
	Let $\mathsf{A}$, $A,B \in \mathrm{Ob} \left(\mathsf{A}\right)$ as before,
	then $\mathrm{Ext}^{ 1}_{ \mathsf{A}} \left( A, B \right) = 0$ (as abelian groups)
	iff every extension of $A$ by $B$ splits.
\end{lem} 

\begin{rem}[]
	Consider $\zeta \in \mathrm{E}(A,B)$,
	then $\gamma \in \mathrm{Hom}_{\mathsf{A}} \left( A', A \right)$
	gives $\zeta\gamma \in \mathrm{E}(A',B)$
	\begin{equation}
	\begin{tikzcd}
		\zeta\gamma: &
		0 \arrow[r, "", rightarrow] &
		B \arrow[r, "", rightarrow] \arrow[d, "", equal] &
		X' \arrow[r, "", rightarrow] \arrow[d, "", rightarrow] &
		A' \arrow[r, "", rightarrow] \arrow[d, "\pi", rightarrow] &
		0\\
		\zeta: &
		0 \arrow[r, "", rightarrow] &
		B \arrow[r, "", rightarrow] &
		X \arrow[r, "", rightarrow] &
		A \arrow[r, "", rightarrow] &
		0
	\end{tikzcd}
	,\end{equation} 
	where $X'$ is a pullback of the diagram
	\begin{equation}
	\begin{tikzcd}
		& A' \arrow[d, "\pi", rightarrow] \\
		X \arrow[r, "", rightarrow] & A
	\end{tikzcd}
	.\end{equation} 
	Analogously $\beta \in \mathrm{Hom}_{\mathsf{A}} \left( B, B' \right)$ gives
	$\beta\zeta \in \mathrm{E}(A,B')$:
	\begin{equation}
	\begin{tikzcd}
		\zeta: &
		0 \arrow[r, "", rightarrow] &
		B \arrow[r, "", rightarrow] \arrow[d, "\beta", rightarrow] &
		X \arrow[r, "", rightarrow] \arrow[d, "", rightarrow] &
		A \arrow[r, "", rightarrow] \arrow[d, "", equal] &
		0\\
		\beta\zeta: &
		0 \arrow[r, "", rightarrow] &
		B' \arrow[r, "", rightarrow] &
		X' \arrow[r, "", rightarrow] &
		A \arrow[r, "", rightarrow] &
		0
	\end{tikzcd}
	,\end{equation} 
	where $X'$ is a pushout of the diagram
	\begin{equation}
	\begin{tikzcd}
		B \arrow[r, "", rightarrow] \arrow[d, "p"', rightarrow] & X\\
		B' &
	\end{tikzcd}
	.\end{equation} 
\end{rem}

\begin{defn}[Baer sum in $\mathrm{E}(A,B)/\sim$]
	Consider $\mathsf{A}$, $A,B \in \mathrm{Ob} \left(\mathsf{A}\right)$ as before.
	Let $\left[ \zeta \right], \left[ \zeta' \right]$ be the equivalence classes, wrt $\sim$,
	of $\zeta,\zeta' \in \mathrm{E}(A,B)$:
	\begin{equation}
	\zeta := 0 \to B \xrightarrow{i} X \xrightarrow{\pi} A \to 0
	\qquad \text{ and } \qquad
	\zeta' := 0 \to B \xrightarrow{i'} X' \xrightarrow{\pi'} A \to 0
	.\end{equation} 
	Now consider the extension of $A \oplus A$ by $B \oplus B$, given by the direct sum
	\begin{equation}
	\zeta \oplus \zeta' :=
	0 \to B \oplus B \xrightarrow{i \oplus i'} X \oplus X' \xrightarrow{\pi \oplus \pi'} 
	A \oplus A \to 0
	\end{equation} 
	and $\Delta_A: A \to A \oplus A$ the diagonal map, i.e. $\Delta_A = \begin{bmatrix} 1_A \\ 1_A \end{bmatrix}$,
	and $\nabla_B: B \to B \oplus B$ the codiagonal map, i.e. $\nabla_B = \begin{bmatrix} 1_B & 1_B \end{bmatrix}$.
	By the above remark we can construct
	\begin{equation}
	\begin{tikzcd}
		\nabla_B \left( \zeta \oplus \zeta' \right) \Delta_A: &
		0 \arrow[r, "", rightarrow] &
		B \arrow[r, "", rightarrow] &
		Z \arrow[r, "", rightarrow] &
		A \arrow[r, "", rightarrow] \arrow[d, "", equal] &
		0\\
		\left( \zeta \oplus \zeta' \right) \Delta_A : &
		0 \arrow[r, "", rightarrow] &
		B \oplus B \arrow[r, "", rightarrow] \arrow[d, "", equal] \arrow[u, "\nabla_B", rightarrow] &
		Y \arrow[r, "", rightarrow] \arrow[d, "", rightarrow] \arrow[u, "", rightarrow] &
		A \arrow[r, "", rightarrow] \arrow[d, "\Delta_A", rightarrow] &
		0\\
		\zeta \oplus \zeta' : &
		0 \arrow[r, "", rightarrow] &
		B \oplus B \arrow[r, "", rightarrow] &
		X \oplus X' \arrow[r, "", rightarrow] &
		A \oplus A \arrow[r, "", rightarrow] &
		0
	\end{tikzcd}
	.\end{equation} 
	Finally we define $\left[ \zeta \right] + \left[ \zeta' \right] :=
	\left[ \nabla_B \left( \zeta \oplus \zeta' \right)\Delta_A \right]$.
	We can also see that this definition is independent of the choice
	of representatives.
\end{defn}

\begin{prop}
	$\mathrm{E}(A,B)/\sim$, with the just defined Baer sum, is an abelian group,
	and 
	\begin{equation}
		\theta: \frac{\mathrm{E}(A,B)}{\sim} \to \mathrm{Ext}^{ 1}_{ \mathsf{A}} \left( A, B \right)
	\end{equation} 
	is a group homomorphism.
\end{prop} 
By this proposition, one can also define $\mathrm{Ext}^1_R$
to be $\mathrm{E}(A,B)/\sim$, and this definition does not require the 
abelian category $\mathsf{A}$ to have enough injectives/projectives.

Analogously this construction can be carried out for every $n \in \N$:
\begin{defn}[$n$ extension]
	Let $\mathsf{A}$ be an abelian category, $A,B \in \mathrm{Ob} \left(\mathsf{A}\right)$.
	An $n$ extension of $A$ by $B$, denoted by $\zeta \in \mathrm{E}_n(A,B)$, is an exact sequence
	\begin{equation}
		\zeta := 0 \to B \to X_n \to X_{n-1} \to \ldots \to X_1 \to A \to 0
	.\end{equation} 
	We say that two extensions $\zeta$ and $\zeta'$ are equivalent, denoted by $\zeta \sim \zeta'$,
	iff there is a commutative diagram s.t. the nontrivial vertical arrows are all isomorphisms
	\begin{equation}
	\begin{tikzcd}
		\zeta : &
		0 \arrow[r, "", rightarrow] &
		B \arrow[r, "", rightarrow] \arrow[d, "", equal] &
		X_n \arrow[r, "", rightarrow] \arrow[d, "f_n", rightarrow] &
		X_{n-1} \arrow[r, "", rightarrow] \arrow[d, "f_{n-1}", rightarrow] &
		\ldots \arrow[r, "", rightarrow] &
		X_1 \arrow[r, "", rightarrow] \arrow[d, "f_1", rightarrow] &
		A \arrow[r, "", rightarrow] \arrow[d, "", equal] &
		0\\
		\zeta' : &
		0 \arrow[r, "", rightarrow] &
		B \arrow[r, "", rightarrow] &
		X'_n \arrow[r, "", rightarrow] &
		X'_{n-1} \arrow[r, "", rightarrow] &
		\ldots \arrow[r, "", rightarrow] &
		X'_1 \arrow[r, "", rightarrow] &
		A \arrow[r, "", rightarrow] &
		0
	\end{tikzcd}
	.\end{equation} 
\end{defn}

\begin{prop}
	For every $n \in \N$, if $\mathrm{Ext}^{ n}_{ \mathsf{A}} \left( A, B \right)$ is defined
	in $\mathsf{A}$ abelian,
	then we have the isomorphism of abelian groups
	\begin{equation}
		\frac{\mathrm{E}_n(A,B)}{\sim} \simeq
		\mathrm{Ext}^{ n}_{ \mathsf{A}} \left( A, B \right)
	.\end{equation} 
\end{prop} 

%\include{Sections/Sheaves}

\end{document}
