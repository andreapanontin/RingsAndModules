\section{Chain and cochain complexes}
Let, in the following, $\mathsf{A}$ be a \textit{preadditive} category with $0$.

\begin{defn}[Chain complex over $\mathsf{A}$]
	We define $\mathrm{Ch}(\mathsf{A})$ the category of \textbf{chain complexes} over $\mathsf{A}$
	as the category whose objects are sequences
	\begin{equation}
	\ldots \to X_n \xrightarrow{d_n} X_{n-1}
	\xrightarrow{d_{n-1}} X_{n-2} \to \ldots
	\end{equation} 
	s.t. $X_i \in \mathrm{Ob} \left(\mathsf{A}\right)$, $d_i \circ d_{i+1} = 0$ for all $i \in \Z$.
	The morphisms $d_i$ are called \textit{differentials} and the sequence is called \textit{complex},
	denoted by $\left( X_{\bullet}, d^X_{\bullet} \right)$, with $\left( d^X \right)^2 = 0$.

	Morphisms in $\mathrm{Ch}(\mathsf{A})$, denoted by
	$f\colon\left(X_{\bullet}, d^X_{\bullet}\right) \to \left(Y_{\bullet}, d^Y_{\bullet}\right)$,
	are a family of morphisms $\left\{ f_n \right\}_{n \in \Z}$, where
	$f_n \in \mathrm{Hom}_{\mathsf{A}} \left( X_n, Y_n \right)$,
	making the following diagram commute
	\begin{equation}
	\begin{tikzcd}
		\ldots \arrow[r, "", rightarrow] &
		X_n \arrow[r, "d^X_n", rightarrow] \arrow[d, "f_n", rightarrow] &
		X_{n-1} \arrow[r, "d^X_{n-1}", rightarrow] \arrow[d, "f_{n-1}", rightarrow] &
		X_{n-2} \arrow[r, "", rightarrow] \arrow[d, "f_{n-2}", rightarrow] &
		\ldots \\
		\ldots \arrow[r, "", rightarrow] &
		Y_n \arrow[r, "d^Y_n", rightarrow]&
		Y_{n-1} \arrow[r, "d^Y_{n-1}", rightarrow] &
		Y_{n-2} \arrow[r, "", rightarrow] &
		\ldots
	\end{tikzcd}
	,\end{equation} 
	i.e. such that $d^Y_n \circ f_N = f_{n-1} \circ d^X_n$ for all $n \in \Z$
	(more compactly $d^Y \circ f = f \circ d^X$).
\end{defn}

\begin{defn}[Cochain complex over $A$]
	We define $\mathrm{Cch}(\mathsf{A})$ the category of \textbf{cochain complexes} over $\mathsf{A}$
	as the category whose objects are sequences
	\begin{equation}
	\ldots \to X^n \xrightarrow{d^n} X^{n+1}
	\xrightarrow{d^{n+1}} X^{n+2} \to \ldots
	\end{equation} 
	s.t. $X^i \in \mathrm{Ob} \left(\mathsf{A}\right)$, $d^i \circ d^{i-1} = 0$ for all $i \in \Z$.
	The morphisms $d^i$ are called \textit{differentials} and the sequence is called \textit{complex},
	denoted by $\left( X^{\bullet}, d_X^{\bullet} \right)$, with $\left( d_X \right)^2 = 0$.

	Morphisms in $\mathrm{Cch}(\mathsf{A})$, denoted by
	$f\colon\left(X^{\bullet}, d_X^{\bullet}\right) \to \left(Y^{\bullet}, d_Y^{\bullet}\right)$
	are a family of morphisms $\left\{ f^n \right\}_{n \in \Z}$, where
	$f^n \in \mathrm{Hom}_{\mathsf{A}} \left( X^n, Y^n \right)$,
	making the following diagram commute
	\begin{equation}
	\begin{tikzcd}
		\ldots \arrow[r, "", rightarrow] &
		X^n \arrow[r, "d_X^n", rightarrow] \arrow[d, "f^n", rightarrow] &
		X^{n+1} \arrow[r, "d_X^{n+1}", rightarrow] \arrow[d, "f^{n+1}", rightarrow] &
		X^{n+2} \arrow[r, "", rightarrow] \arrow[d, "f^{n+2}", rightarrow] &
		\ldots \\
		\ldots \arrow[r, "", rightarrow] &
		Y^n \arrow[r, "d_Y^n", rightarrow]&
		Y^{n+1} \arrow[r, "d_Y^{n+1}", rightarrow] &
		Y^{n+2} \arrow[r, "", rightarrow] &
		\ldots
	\end{tikzcd}
	,\end{equation} 
	i.e. such that $f^n \circ d_X^{n-1} = d_Y^{n-1} \circ f^{n-1}$ for all $n \in \Z$
	(more compactly $f \circ d_X = d_Y \circ f$).
\end{defn}

\begin{rem}[Additive categories]
	If $\mathsf{A}$ is additive, then also $\mathrm{Ch}(\mathsf{A})$ and $\mathrm{Cch}(\mathsf{A})$ are.
	In particular, given $\left(X^{\bullet}, d_X\right)$ and $\left(Y^{\bullet}, d_Y \right)$ two cochain complexes,
	then their coproduct $\left(X^{\bullet} \oplus Y^{\bullet}, d_X \oplus d_Y\right)$
	is given, degree wise, by
	\begin{equation}
	\left[ X^{\bullet} \oplus Y^{\bullet} \right]^n := X^n \oplus Y^n.
	\end{equation} 
	Analogously, degree wise, its differentials are defined by
	\begin{equation}
	d^n_{X^{\bullet} \oplus Y^{\bullet}} :=
	d^n_X \oplus d^n_Y =
	\begin{bmatrix}
		d^n_X & 0\\
		0 & d^n_Y
	\end{bmatrix} 
	.\end{equation} 
\end{rem}

\begin{defn}[Bounded (co)chain complex]
	A (co)chain complex $\left(X^{\bullet}, d_X\right)$ is \textbf{bounded} iff
	$\exists\, b \in \N$ s.t. $X^n = 0$ for all $\left| n \right| > b$.
	It is bounded \textbf{below}, resp. \textbf{above}, iff
	$\exists\, b \in \Z$ s.t. $X^n = 0$ for all
	$n < b$, resp. $n > b$.
	(Even though we used the notation for cochain complexes the definitions apply without
	modification to chain complexes).

	We denote respectively with $\mathrm{Ch}(\mathsf{A})^b$, $\mathrm{Ch}(\mathsf{A})^+$
	and $\mathrm{Ch}(\mathsf{A})^-$ the full subcategory of bounded, resp. above or below, chain complexes.
\end{defn}

\begin{defn}[Canonical functor]
	There is a canonical embedding
	\begin{align}
		\mathrm{can}: \mathsf{A} &\to \mathrm{Ch}(\mathsf{A}) \\
		A &\mapsto A^{\bullet} := \left[ 
		\ldots \to 0 \to (A^0 := A) \to 0 \to \ldots \right]
	.\end{align} 
	$A^{\bullet}$ is called complex concentrated in degree $0$.
	Clearly $\mathrm{can}$ is fully faithful, hence it is an embedding of $\mathsf{A}$ into $\mathrm{Ch}(\mathsf{A})$.
\end{defn}

\begin{defn}[Shift functor]
	Choose $p \in \Z$, then we can define the functor
	\begin{align}
		\left[ p \right]: \mathrm{Ch}(\mathsf{A}) &\to \mathrm{Ch}(\mathsf{A}) \\
		\left(X^{\bullet}, d_X\right) &\mapsto \left( X^{\bullet}[p], d_{X}[p] \right)
	,\end{align} 
	in which we define
	\begin{equation}
		\left( X^{\bullet} [p] \right)^n := X^{n+p} \qquad \text{ and } \qquad
		d^n_{X^{\bullet}[p]} := (-1)^{p} d_X^{n+p}
	.\end{equation} 
	More explicitly this functor shifts the objects in the (co)chain, by $p$ to the left.
	Analogously it acts on a morphism of complexes
	$f\colon \left( X_{\bullet}, d^{X} \right) \to \left( Y_{\bullet}, d^{Y} \right)$
	by shifting the morphisms of the family by $p$ to the left.
	More explicitly
	\begin{equation}
		\left( [p]f \right)^n := f^{n+p}
	.\end{equation} 
	Moreover we introduce the notation $f[p] := [p]f$.
\end{defn}

\begin{rem}[Shift functor]
	The above is called the {\em shift functor} if $p = 1$:
	\begin{equation}
		[1]: \mathrm{Ch}(\mathsf{A}) \to \mathrm{Ch}(\mathsf{A}).
	\end{equation} 
\end{rem}

\begin{rem}[]
	The functor $[p]: \mathrm{Ch}(\mathsf{A}) \to \mathrm{Ch}(\mathsf{A})$ is an automorphism of categories.
	In fact $[p] \circ [-p] = id_{\mathrm{Ch}(\mathsf{A})} = [-p] \circ [p]$.	
\end{rem}

\begin{rem}[Motivational remark]
	From algebraic topology.
	We define $\Delta_n$ the standard $n$-simplex.
	Given a topological space $X$ one wants to partition it into finitely many
	$n$-simplices.
	One can construct a chain (the simplicial chain complex) by considering $X_k$, for every $k \in \N$,
	the set of $k$-dimensional simplices appearing in the partition of $X$.
	Then one can create for each degree $k$ the free abelian group generated by $X_k$, we denote it by $(C_{\bullet})_k$.
	One also defines a differential $d_k: C_k \to C_{k-1}$, which gives rise to a chain complex.
\end{rem}

\begin{prop}
	Given an abelian category $\mathsf{A}$, then $\mathrm{Ch}(\mathsf{A})$ is abelian,
	i.e. it admits kernels, cokernles and $\mathrm{Coim}\, $ is canonically isomorphic to $\mathrm{Im}\, $.
\end{prop}

\begin{ex}
	Let's, for example, define the kernel of a morphism
	\begin{equation}
	f: \left( X^{\bullet}, d_{X} \right) \to \left( Y^{\bullet}, d_{Y} \right)
	.\end{equation} 
	Then, we denote by $K^{\bullet} := \ker f$ the cochain s.t. $K^n := \ker f^n$
	and with differential defined by:
	\begin{equation}
		\begin{tikzcd}[column sep=small, row sep=small]
		& X^n \arrow[rr, "d^n_X", rightarrow] \arrow[dd, "f^n"' near end, rightarrow] & &
		X^{n+1} \arrow[dd, "f^{n+1}" near end, rightarrow] \\
		\ker f^n \arrow[ru, "\epsilon^n", tail]
			\arrow[rr, "\exists\, ! d^n" near end, rightarrow, red, crossing over] & &
		\ker f^{n+1} \arrow[ru, "\epsilon^{n+1}"', tail] \\
		& Y^n \arrow[rr, "d^n_Y", rightarrow] & &
		Y^{n+1}
	\end{tikzcd}
	.\end{equation} 
	By the commutativity of the diagram we obtain
	\begin{equation}
	f^{n+1} \circ d^n_X \circ \epsilon^n = 
	d_Y^n \circ f^n \circ \epsilon^n = 0
	.\end{equation} 
	Then, by the second condition on kernels, we obtain $\exists\, !\, d^n: \ker f^n \to \ker f^{n+1}$ s.t.
	$d^n_X \circ \epsilon^n = \epsilon^{n+1} \circ d^n$.
\end{ex} 

\begin{defn}[Cohomology]
	Let $\mathsf{A}$ be an abelian category,
	and $\left( X^{\bullet}, d_{X} \right) \in \mathrm{Ch}(\mathsf{A})$.
	Then,  since $d^n_X \circ d^{n-1}_X = 0$, 
	as subobjects we have $\Ima d^{n-1}_X \subset \ker d_X^n$.
	Hence we can define, for all $n \in \Z$, the following quotient object
	\begin{equation}
		H^n(X) :=
		\frac{\ker d^n_X}{\Ima d_X^{n-1}} \in \mathrm{Ob} \left(\mathsf{A}\right)
	,\end{equation} 
	called the $n$-th cohomology of the cochain complex $\left( X^{\bullet}, d_{X} \right)$.
\end{defn}

\begin{ex}
	Let $\mathsf{A} = \mathsf{Ab}$ the category of abelian groups.
	Consider the following cochain
	\begin{equation}
	\ldots \to 0 \to \mathbb{Z}/4\mathbb{Z} \xrightarrow{\dot{2}} \mathbb{Z}/4\mathbb{Z}
	\xrightarrow{\dot{2}} \mathbb{Z}/4\mathbb{Z} \xrightarrow{\dot{2}} \ldots
	=: \left( X^{\bullet}, d_{X} \right)	
	.\end{equation} 
	Then $H^0(X) = 2\mathbb{Z}/4\mathbb{Z} \simeq \mathbb{Z}/2\mathbb{Z}$, whereas
	$H^n(X) = 0$ for all $n \neq 0$.
	If, instead, we considered the following object
	\begin{equation}
	\ldots \to \mathbb{Z}/4\mathbb{Z} \xrightarrow{\dot{2}} \mathbb{Z}/4\mathbb{Z}
	\xrightarrow{\dot{2}} \mathbb{Z}/4\mathbb{Z} \xrightarrow{\dot{2}} \ldots
	=: \left( X^{\bullet}, d_{X} \right)	
	.\end{equation} 
	Then $H^n(X) = 0$ for all $n \in \Z$ and we say that 
	$\left( X^{\bullet}, d_{X} \right)$ is \textit{acyclic}.
\end{ex}

\begin{prop}
	Let $\mathsf{A}$ be an abelian category, then, for every $n \in \Z$, we can define 
	\begin{align}
		H^n: \mathrm{Ch}(\mathsf{A}) &\to \mathsf{A} \\
		\left( X^{\bullet}, d_{X} \right) &\mapsto H^n(X)
	.\end{align}
	In particular this is an additive functor.
\end{prop} 
\begin{proof}
	We need to construct, starting from a cochain map $f: X^{\bullet} \to Y^{\bullet}$, the associated cohomology morphism
	\begin{equation}
		H^n(f): H^n(X) \to H^n(Y)
	.\end{equation} 
\end{proof}

\begin{rem}[]
	If $\mathsf{A} := \mathsf{Mod}\text{-}R$, then every part of the above result can be
	checked by diagram chasing.
	In fact $z \in \ker d^n_X \iff d^n_X(z) = 0$, then
	\begin{equation}
		d^n_Y \circ f^n(z) = f^{n+1} \circ d^n_X(z) = 0
	,\end{equation} 
	hence $f^n(\ker d^n_X) \subset \ker d^n_Y$.
	Moreover, given $x \in \Ima d^{n-1}_X$, then $x = d^{n-1}_X(z)$, for some $z \in X^{n-1}$.
	Then
	\begin{equation}
		f^n(x) = f^n \circ d^{n-1}_X(z) =
		d^{n-1}_Y \circ f^{n-1} (z) \in \Ima d^{n-1}_Y
	.\end{equation} 
	Hence $f^n(\Ima d^{n-1}_X) \subset \Ima d^{n-1}_Y$.
	The $f$ induces a map on the quotient
	\begin{equation}
	\widetilde{f}: \frac{\ker d^n_X}{\Ima d_X^{n-1}} \to \frac{\ker d^n_Y}{\Ima d_Y^{n-1}}
	.\end{equation} 
\end{rem}

And now, a very important result!
\begin{thm}[Freyd-Mitchell embedding]
	Let $\mathsf{A}$ be a small, abelian category.
	Then there is a ring $R$ and a fully faithful exact functor
	\begin{equation}
	F: \mathsf{A} \to \mathsf{Mod}\text{-}R
	.\end{equation} 
\end{thm}

\begin{rem}[]
	The above theorem essentially states that we can consider objects of $\mathsf{A}$ as if they were modules.
	In particular any result in $\mathsf{Mod}\text{-}R$ involving only finitely many
	objects and morphisms (such as exactness, existence and vanishing of morphisms)
	holds in any abelian category $\mathsf{C}$.
	This is true, since we can always construct a small full subcategory $\mathsf{A}_0$ of $\mathsf{C}$,
	containing only the objects and morphism involved in the result (and, by a remark which will follow,
	an exact and fully faithful functor reflects exactness).

	Notice, however, that results for arbitrary family of objects do not translate so easily.
	For example the product of an arbitrary family of exact sequences in $\mathsf{Mod}\text{-}R$
	is still exact in $\mathsf{Mod}\text{-}R$, 
	but not in an arbitrary abelian category.
\end{rem}

\begin{proof}[Sketch of proof (Freyd-Mitchell)]
	Let $\underline{\mathrm{Hom}\left( \mathsf{A}^{op}, \mathsf{Ab} \right)}$
	be the category of the additive functors from $\mathsf{A}^{op}$ to $\mathsf{Ab}$.
	Then, by Yoneda lemma, the Yoneda embedding
	\begin{align}
		Y: \mathsf{A} &\to
		\underline{\mathrm{Hom}\left( \mathsf{A}^{op}, \mathsf{Ab} \right)} \\
		A &\mapsto h^A = \mathrm{Hom}_{\mathsf{A}} \left( -, A \right)
	\end{align} 
	is fully faithful.
	Moreover it is left exact, since, for every $A$ the functor $h^A$ is left exact.
	In fact
	\begin{equation}
		Y: \mathsf{A} \to \mathsf{L} := \mathrm{Lex} \left(\mathsf{A}^{op}, \mathsf{Ab}\right) \subset
	\underline{\mathrm{Hom}\left( \mathsf{A}^{op}, \mathsf{Ab} \right)}
	\end{equation} 
	takes values in the category $\mathrm{Lex}\left(\mathsf{A}^{op}, \mathsf{Ab}\right)$ of
	left exact functors from $\mathsf{A}^{op}$ to $\mathsf{Ab}$.
	We need some facts about $\mathsf{L}$ (which are not trivial to show):
	\begin{enumerate}
		\item $\mathsf{L}$ is an abelian category.
			In particular its kernels coincide with the ones in
			$\underline{\mathrm{Hom}_{ }\left( \mathsf{A}^{op}, \mathsf{Ab} \right)}$,
			whereas cokernels differ.
			This implies that the inclusion functor $\mathsf{L} \hookrightarrow
			\underline{\mathrm{Hom}\left( \mathsf{A}^{op}, \mathsf{Ab} \right)}$
			is only left exact.
		\item The Yoneda embedding $Y: \mathsf{A} \to \mathsf{L}$ is fully faithful and exact.
		\item $\mathsf{L}$ has arbitrary coproducts, i.e. $\mathsf{L}$ is cocomplete,
			and has a projective generator, which is faithful as a functor, 
			namely
			\begin{equation}
			P := \coprod_{A \in \mathrm{Ob} \left(\mathsf{A}\right)} h^A
			.\end{equation} 
			Recall that we can take this coproduct since $\mathsf{A}$ is a small category,
			hence $\mathrm{Ob} \left(\mathsf{A}\right)$ is a set.
	\end{enumerate}
	Summarizing: $\mathsf{A}$ is a small abelian full subcategory of $\mathsf{L}$, which is a
	cocomplete abelian category with a projective generator.
	Then Freyd-Mitchell follows from the following theorem.
\end{proof}

\begin{thm}[]
	Let $\mathsf{C}$ be a cocomplete abelian category with a projective generator.
	Then, for every small full abelian category $\mathsf{A} \subset \mathsf{C}$,
	there is a ring $R$ and a fully faithful exact functor
	\begin{equation}
	F: \mathsf{A} \to \mathsf{Mod}\text{-}R
	,\end{equation} 
	so that $\mathsf{A}$ is equivalent to a full subcategory of $\mathsf{Mod}\text{-}R$.
\end{thm}

\begin{defn}[Functor reflecting exactness]
	Let $\mathsf{C}$ and $\mathsf{D}$ be abelian categories, and
	$F\colon \mathsf{C} \to \mathsf{D}$ be an {\em additive functor}.
	We say that $F$ \textbf{reflects exactness} iff
	\begin{equation}
	A \to B \to C
	\end{equation} 
	is exact in $\mathsf{C}$, as soon as
	\begin{equation}
		F(A) \to F(B) \to F(C)
	\end{equation} 
	is exact in $\mathsf{D}$.
\end{defn}

\begin{lem}
	If $F$ is an exact and fully faithful functor,
	then $F$ reflects exactness.
	(you can simplify things if you prove it using Freyd-Mitchell)
\end{lem} 

\begin{prop}
	Let $\mathsf{A}$ be a small abelian category.
	The Yoneda embedding
	\begin{equation}
		Y\colon \mathsf{A} \to \underline{\mathrm{Hom}\left( \mathsf{A}^{op}, \mathsf{Ab} \right)}
	\end{equation} 
	reflects exactness.
\end{prop} 

\begin{defn}[Acyclic complex]
	A (co)chain complex $\left( X^{\bullet}, d_{X} \right)$ is \textbf{acyclic} iff 
	$H^n(X) = 0$ for all $n \in \Z$, i.e. as a sequence it is exact
	\begin{equation}
	\ldots \to X^{n-1} \xrightarrow{d_X^{n-1}} X^n \xrightarrow{d_X^n} 
	X^{n+1} \xrightarrow{d_X^{n+1}} X^{n+2} \to \ldots
	.\end{equation} 
\end{defn}
