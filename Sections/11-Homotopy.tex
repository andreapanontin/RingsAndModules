\subsection{Homotopy category}
Let $\mathsf{A}$ be an additive category, and $X^{\bullet}, Y^{\bullet} \in \mathrm{Ch}(\mathsf{A})$.

\begin{defn}[Nullhomotopic morphism]
	A morphism $f \in \mathrm{Hom}_{\mathrm{Ch}(\mathsf{A})} \left( X^{\bullet}, Y^{\bullet} \right)$ is
	{\em nullhomotopic}, or {\em homotopic to zero}, iff there exists
	a family of morphism $\left\{ s^n \right\}_{n \in \mathbb{N}}$, with
	$s^n\colon X^n \to Y^{n-1}$, in pictures
	\begin{equation}
	\begin{tikzcd}
		\ldots \arrow[r, "", rightarrow] &
		X^{n-1} \arrow[r, "d_X^{n-1}", rightarrow] \arrow[d, "f^{n-1}"', rightarrow] &
		X^n \arrow[r, "d^n_X", rightarrow] \arrow[d, "f^n"', rightarrow] \arrow[ld, "s^n"', rightarrow] &
		X^{n+1} \arrow[r, "", rightarrow] \arrow[d, "f^{n+1}", rightarrow] \arrow[dl, "s^{n+1}"', rightarrow] &
		\ldots \\
		\ldots \arrow[r, "", rightarrow] &
		Y^{n-1} \arrow[r, "d_Y^{n-1}"', rightarrow] &
		Y^n \arrow[r, "d^n_Y"', rightarrow] &
		Y^{n+1} \arrow[r, "", rightarrow] &
		\ldots
	\end{tikzcd}
	\end{equation}
	such that $f^n = s^{n+1} \circ d_X^{n} + d_Y^{n-1} \circ s^n$ for all $n \in \Z$.
	More compactly we write $f = s \circ d_X + d_Y \circ s$.
	The morphisms $s^n$ are called {\em homotopies} or {\em cochain contractions}.
	Moreover, if $f$ is nullhomotopic, we write $f \sim 0$.
\end{defn}

\begin{defn}[Homotopic morphisms]
	Two cochain maps $f,g: \left( X^{\bullet}, d_{X} \right) \to \left( Y^{\bullet}, d_{Y} \right)$
	are called {\em homotopic}, denoted by $f \sim g$, iff
	$f - g$ is nullhomotopic.
\end{defn}

\begin{rem}[]
	The relation $\sim$ is an equivalence relation.
\end{rem}

\begin{defn}[Homotopy category]
	Given, as before, an additive category $\mathsf{A}$, we define the homotopy category
	$K(\mathsf{A})$ as follows.
	Its objects are exactly the objects in $\mathrm{Ch}(\mathsf{A})$.
	Its morphisms, instead, are equivalence classes of (co)chain maps,
	under the homotopy relation $\sim$ we just defined.
	More explicitly
	\begin{align}
		\mathrm{Hom}_{K(\mathsf{A})} \left( X^\bullet, Y^\bullet \right)
		&\simeq \mathrm{Hom}_{\mathrm{Ch}(\mathsf{A})}\left( X^\bullet, Y^\bullet \right)/\sim\\
		g &\mapsto \left[ g \right]_{\sim}
	.\end{align} 
\end{defn}

\begin{rem}[]
	The homotopy relation $\sim$ is compatible with addition, hence it is a congruence.
	In particular, denoted with
	$\mathrm{Hom}_{t}\left( X^\bullet, Y^\bullet \right) \subset
	\mathrm{Hom}_{\mathrm{Ch}(\mathsf{A})}\left( X^\bullet, Y^\bullet \right)$ the subgroup of
	nullhomotopic (co)chain maps, then
	\begin{equation}
		\mathrm{Hom}_{K(\mathsf{A})}\left( X^\bullet, Y^\bullet \right) =
		\frac{\mathrm{Hom}_{\mathrm{Ch}(\mathsf{A})}\left( X^\bullet, Y^\bullet \right)}{
		\mathrm{Hom}_{t}\left( X^\bullet, Y^\bullet \right)}
	.\end{equation} 
	Moreover, let $f,g: X^\bullet \to Y^\bullet$ be homotopic cochain maps.
	Let $\alpha: Z^\bullet \to X^\bullet$ and $\beta: Y^\bullet \to W^\bullet$ be cochain maps,
	then, by linearity of composition, we obtain
	$\beta \circ f \circ \alpha \sim \beta \circ g \circ \alpha$.
\end{rem}

\begin{prop}
	$K(\mathsf{A})$ is an additive category, and the quotient functor,
	defined
	\begin{align}
		q: \mathrm{Ch}(\mathsf{A}) &\to K(\mathsf{A}) \\
		X^\bullet &\mapsto X^\bullet\\
		f &\mapsto [f]_{\sim}
	\end{align} 
	is an additive functor.
\end{prop} 

\begin{defn}[Homotopy equivalence]
	A cochain map $f: \left( X^{\bullet}, d_{X} \right) \to \left( Y^{\bullet}, d_{Y} \right)$
	is said to be a {\em homotopy equivalence} iff
	$\exists\, g: \left( Y^{\bullet}, d_{Y} \right) \to \left( X^{\bullet}, d_{X} \right)$
	s.t.
	$g \circ f \sim 1_X$ and $f \circ g \sim 1_Y$.
	In other words a homotopy equivalence is an isomorphism in $K(\mathsf{A})$.
\end{defn}

\begin{prop}
	Let $\mathsf{A}$ be an abelian category and
	$f: \left( X^{\bullet}, d_{X} \right) \to \left( Y^{\bullet}, d_{Y} \right)$ be a
	nullhomotopic cochain map.
	Then the induced cohomology map
	\begin{equation}
		H^n(f) =: \overline{f^n}: H^n(X) \to H^n(Y)
	\end{equation} 
	is the zero map for every $n \in \Z$.
\end{prop} 
\begin{proof}
	We can use Freyd-Mitchell
	(this proposition deals with a finite number of objects and morphisms).
	Then, by definition
	\begin{equation}
		H^n(f) \left( x + \ima d_X^{n-1} \right) = f^n(x) + \ima d_Y^{n-1}
	.\end{equation} 
	But $f^n(x) = d^{n-1}_Y \circ s^n(x) + s^{n+1} \circ d_X^{n}(x)$,
	then
	\begin{equation*}
		H^n(f)(x) = d^{n-1}_Y \circ s^n (x) + \ima d_Y^{n-1} = 0 + \ima d_Y^{n-1}.\qedhere
	\end{equation*} 
\end{proof}

\begin{cor}
	Let $f$ and $g$ be homotopic maps, then
	\begin{equation}
		H^n(f) = H^n(g) \qquad \,\forall\, n \in \Z
	.\end{equation} 
\end{cor} 
\begin{proof}
	$H^n$ is an additive functor for each $n \in \Z$.
\end{proof}

\begin{rem}[]
	In general $\mathsf{A}$ abelian implies $\mathrm{Ch}(\mathsf{A})$ abelian,
	but not $K(\mathsf{A})$ abelian.
\end{rem}

\begin{defn}[Semisimple ring]
	A ring $R$ is called {\em semisimple} iff every $R$-module is projective.
	Equivalently iff every short exact sequence splits.
\end{defn}

\begin{ex}
	Any field $\K$ is semisimple, but $\Z$ is not.
	As a consequence of the following proposition, we get that $K(\mathsf{Mod}\text{-}\Z)$
	is not abelian.
\end{ex}

\begin{prop}
	The following statement (and more importantly the proof)
	should be incorrect. 
	Here what should be the correct one
	(I'm not going to copy the proof again, though).
	Let $\mathsf{A} := \mathsf{Mod}\text{-}R$.
	If $K(\mathsf{A})$ is abelian, then
	$R$ is semisimple.
\end{prop}
\begin{prop}
	Let $\mathsf{A} := \mathsf{Mod}\text{-}R$.
	If $R$ is not semisimple, then $K(\mathsf{A})$ is not abelian.
\end{prop}
\begin{proof}
	Assume that $R$ is not semisimple, but $K(\mathsf{Mod}\text{-}R)$ is abelian.
	Since $R$ is not semisimple, then there exists a s.e.s.
	\begin{equation}\label{eqn:sesAbHomCat}
	0 \to X \xrightarrow{f} Y \xrightarrow{\pi}
	Z \to 0 \qquad \text{ in } \mathsf{Mod}\text{-}R
	\end{equation} 
	which does not split.
	Consider now $X^\bullet, Y^\bullet, Z^\bullet$ as complexes concentrated in degree $0$.
	Since $K(\mathsf{Mod}\text{-}R)$ is abelian, then  $f:= q(f)$ has a cokernel
	And, by uniqueness up to isomorphism of the cokernel,
	we can assume that $\pi$ is a cokernel of $f$ in $K(\mathsf{Mod}\text{-}R)$.

	Consider the complex $\mathrm{Cone}\, f$:
	\begin{equation}
		0 \to X \xrightarrow{f} Y \to 0
	,\end{equation} 
	where $Y$ is in degree $0$.
	Let's define the cochain map $\alpha: Y^\bullet \to \mathrm{Cone}\, f$
	defined by
	\begin{equation}
	\begin{tikzcd}
		\ldots \arrow[r, "", rightarrow] &
		0 \arrow[r, "", rightarrow] \arrow[d, "0"', rightarrow] &
		0 \arrow[r, "", rightarrow] \arrow[d, "0"', rightarrow] &
		Y \arrow[r, "", rightarrow] \arrow[d, "1_Y", rightarrow] &
		0 \arrow[r, "", rightarrow] \arrow[d, "0", rightarrow] &
		\ldots \\
		\ldots \arrow[r, "", rightarrow] &
		0 \arrow[r, "", rightarrow] &
		X \arrow[r, "f"', rightarrow] &
		Y \arrow[r, "", rightarrow] &
		0 \arrow[r, "", rightarrow] & \ldots
	\end{tikzcd}
	.\end{equation} 
	Then we claim that there exist $\gamma, \delta$ s.t. $\alpha = \gamma \circ \pi$ and $\delta \circ \alpha = \pi$,
	i.e. s.t. the following diagram commutes.
	\begin{equation}
		\begin{tikzcd}
		X^\bullet \arrow[r, "f", rightarrow] &
		Y^\bullet \arrow[r, "\pi", rightarrow] \arrow[dr, "\alpha"', rightarrow] &
		Z^\bullet \arrow[d, "\gamma"', dashrightarrow, shift right = 0.25em] \\
		& &
		\mathrm{Cone}\, f \arrow[u, "\delta"', dashrightarrow, shift right = 0.25em]
	\end{tikzcd}
	.\end{equation} 
	At first we notice that $\alpha \circ f = 0 $ in $K(\mathsf{Mod}\text{-}R)$, in fact:
	\begin{equation}
	\begin{tikzcd}
		&
		0 \arrow[r, "", rightarrow] &
		X \arrow[r, "", rightarrow] \arrow[d, "f", rightarrow] \arrow[ddl, "1_X"' near start, dashrightarrow, crossing over] &
		0 \arrow[ldd, "0" near end, dashrightarrow, crossing over]\\
		&
		0 \arrow[r, "", rightarrow] &
		Y \arrow[r, "", rightarrow] \arrow[d, "1_Y"', rightarrow] &
		0 \\
		0 \arrow[r, "", rightarrow] &
		X \arrow[r, "f"', rightarrow] &
		Y \arrow[r, "", rightarrow] &
		0
	\end{tikzcd}
	.\end{equation} 
	Since $\pi$ is a cokernel of $f$, then $\exists\, !\, \gamma: Z^\bullet \to \mathrm{Cone}\, f$ s.t. $\gamma \circ \pi = \alpha$.
	With regard to $\delta: \mathrm{Cone}\, f \to Z^\bullet$, instead, we define $(0, \pi)$, i.e.
	the family of maps which all correspond to zero, apart from degree $0$, in which it is $\pi$.
	Then $\delta \circ \alpha = \pi$, as described by the following diagram
	\begin{equation}
	\begin{tikzcd}
		0 \arrow[r, "", rightarrow] \arrow[d, "0", rightarrow] &
		Y \arrow[r, "", rightarrow] \arrow[d, "1_Y", rightarrow] &
		0 \arrow[d, "0", rightarrow] \\
		X \arrow[r, "f", rightarrow] \arrow[d, "0", rightarrow] &
		Y \arrow[r, "", rightarrow] \arrow[d, "\pi", rightarrow] &
		0 \arrow[d, "0", rightarrow] \\
		0 \arrow[r, "", rightarrow] &
		Z \arrow[r, "", rightarrow] &
		0
	\end{tikzcd}
	.\end{equation} 
	Then we have $\pi = \delta \circ \alpha = \delta \circ \gamma \circ \pi$.
	Since $\pi$ is epi (it is a cokernel), we obtain that
	$\delta \circ \gamma = id_Z$ in $K(\mathsf{Mod}\text{-}R)$.
	But then, if we denote by $\gamma_0: Z \to Y$ the morphism in degree $0$ of $\gamma$, we obtain
	that $\pi \circ \gamma_0 = 1_Z$, hence we have found a retraction of $\pi$ in \eqref{eqn:sesAbHomCat}.
	This is a contradiction, since we assumed it did not split.
\end{proof}

\subsection{Snake lemma and applications}
\begin{lem}
	Let $\mathsf{A}$ be an abelian category, and let
	\begin{equation}
	\begin{tikzcd}
		&
		A \arrow[r, "\alpha", rightarrow] \arrow[d, "f"', rightarrow] &
		B \arrow[r, "\beta", rightarrow] \arrow[d, "g"', rightarrow] &
		C \arrow[r, "", rightarrow] \arrow[d, "h"', rightarrow] &
		0 \\
		0 \arrow[r, "", rightarrow] &
		A' \arrow[r, "\alpha'", rightarrow] &
		B' \arrow[r, "\beta'", rightarrow] &
		C' &
	\end{tikzcd}
	\end{equation} 
	be a commutative diagram with exact rows.
	Then there is an exact sequence:
	\begin{equation}
	\ker f \xrightarrow{\underline{\alpha}} \ker g \xrightarrow{\underline{\beta}}
	\ker h \xrightarrow{\partial} \coker f \xrightarrow{\overline{\alpha'}} 
	\coker g \xrightarrow{\overline{\beta'}} \coker h
	,\end{equation} 
	in which $\partial$ is called the {\em connecting morphism}.
	Moreover $\alpha$ mono implies $\underline{\alpha}$ is mono,
	whereas $\beta'$ epi implies $\overline{\beta'}$ is epi.
\end{lem} 	

\begin{rem}[Short exact sequences in the category of complexes]
	Since the abelian structure of $\mathrm{Ch}(\mathsf{A})$ is defined degree wise, we have
	that a sequence in $\mathrm{Ch}(\mathsf{A})$
	\begin{equation}
	0 \to X^\bullet \xrightarrow{f} Y^\bullet \xrightarrow{g} 
	Z^\bullet \to 0
	\end{equation} 
	is exact in $\mathrm{Ch}(\mathsf{A})$ iff, for every $n \in \Z$, the corresponding
	\begin{equation}
	0 \to X^n \xrightarrow{f^n} Y^n \xrightarrow{g^n}
	Z^n \to 0
	\end{equation} 
	is exact in $\mathsf{A}$.
\end{rem}

\begin{thm}[Fundamental theorem in (co)homology]
	Consider a short exact sequence in $\mathrm{Ch}(\mathsf{A})$, for an abelian category $\mathsf{A}$,
	\begin{equation}
	0 \to X^\bullet \xrightarrow{f} Y^\bullet \xrightarrow{g} W^\bullet \to 0
	.\end{equation} 
	Then we can associate to it a long exact sequence in $\mathsf{A}$, called the
	{\em long exact sequence in (co)homology}, given as follows:
	\begin{equation}
		\ldots \to H^n(X^\bullet) \xrightarrow{H^n(f)} H^n(Y^\bullet) \xrightarrow{H^n(g)} 
		H^n (W^\bullet) \xrightarrow{\partial} H^{n+1}(X^\bullet) \to
		H^{n+1}(Y^\bullet) \to \ldots
	,\end{equation} 
\end{thm}
\begin{proof}
	The proof is essentially an application of the snake lemma.
	In particular we obtain that $\partial: H^n(W^\bullet) \to H^{n+1}(X^\bullet)$ acts as
	\begin{align}
		\partial: H^n(W^\bullet) &\to H^{n+1}(X^\bullet) \\
		[z^n] &\mapsto \left[ (f^{n+1})^{-1} \left( d_Y^n ((g^n)^{-1}(z^n)) \right) \right]
	.\end{align} 
	More visually it is defined by the following diagram chase:
	\begin{equation}
	\begin{tikzcd}
		&
		Y^n \arrow[r, "g^n", rightarrow] \arrow[d, "d_Y^n"', rightarrow] &
		Z^n \arrow[d, "0", rightarrow] \\
		X^{n+1} \arrow[r, "f^{n+1}", rightarrow] \arrow[d, "d_X^{n+1}"', rightarrow] &
		Y^{n+1} \arrow[r, "g^{n+1}", rightarrow] \arrow[d, "d_Y^{n+1}", rightarrow] &
		0 \\
		0 = X^{n+2} \arrow[r, "f^{n+2}", rightarrow] &
		0
	\end{tikzcd}
	.\end{equation} 
\end{proof}

\begin{rem}[Notation]
	We denote by $Z^n(X^\bullet) := \ker d_X^n$, the $n$-cycles,
	and by $B^n(X^\bullet) := \ima d_X^{n-1}$, the $n$-boundaries.
	Both clearly are subobjects of $X^n$.
\end{rem}

\begin{defn}[Long/short exact sequence category]
	Let $\mathsf{A}$ be an abelian category.
	\begin{itemize}
		\item We define $\mathsf{S}$, the category of short exact sequences
	in $\mathrm{Ch}(\mathsf{A})$, as the category whose objects
	are short exact sequences with objects in $\mathrm{Ob} \left(\mathrm{Ch}(\mathsf{A})\right)$
	and whose morphisms, called morphisms of short exact sequences,
	are triples $(f,g,h)$ of cochain maps such that
	the following diagram commutes
	\begin{equation}
	\begin{tikzcd}
		0 \arrow[r, "", rightarrow] &
		A^\bullet \arrow[r, "\alpha", rightarrow] \arrow[d, "f"', rightarrow] &
		B^\bullet \arrow[r, "\beta", rightarrow] \arrow[d, "g"', rightarrow] &
		C^\bullet \arrow[r, "", rightarrow] \arrow[d, "h"', rightarrow] &
		0 \\
		0 \arrow[r, "", rightarrow] &
		X^\bullet \arrow[r, "\alpha'"', rightarrow] &
		Y^\bullet \arrow[r, "\beta'"', rightarrow] &
		W^\bullet \arrow[r, "", rightarrow] &
		0 
	\end{tikzcd}
	.\end{equation} 
		\item We define $\mathsf{L}$, the category of long exact sequences
	in $\mathsf{A}$, as the category whose objects
	are exact sequences in $\mathrm{Ob} \left(\mathrm{Ch}\mathsf{A}\right)$
	and whose morphisms are morphisms of complexes, i.e.
	families of maps $\left\{ f^n \right\}_{n \in \Z}$ making
	the following diagram commute
	\begin{equation}
	\begin{tikzcd}
		\ldots \arrow[r, "", rightarrow] &
		A^n \arrow[r, "d_A^n", rightarrow] \arrow[d, "f^n"', rightarrow] &
		A^{n+1} \arrow[r, "d_A^{n+1}", rightarrow] \arrow[d, "f^{n+1}"', rightarrow] &
		A^{n+2} \arrow[r, "d_A^{n+2}", rightarrow] \arrow[d, "f^{n+2}"', rightarrow] &
		\ldots \\
		\ldots \arrow[r, "", rightarrow] &
		B^n \arrow[r, "d_B^n"', rightarrow] &
		B^{n+1} \arrow[r, "d_B^{n+1}"', rightarrow] &
		B^{n+2} \arrow[r, "d_B^{n+2}"', rightarrow] &
		\ldots 
	\end{tikzcd}
	.\end{equation} 
	\end{itemize}
\end{defn}

\begin{prop}
	Given an abelian category $\mathsf{A}$, then we can define a functor
	\begin{align}
		L: \mathsf{S} &\to \mathsf{L}
	,\end{align} 
	that, on objects, maps each short exact sequence of complexes to its corresponding exact
	sequence in (co)homology.
	In particular a given morphism in $\mathsf{S}$
	\begin{equation}
	\begin{tikzcd}
		0 \arrow[r, "", rightarrow] &
		A^\bullet \arrow[r, "\alpha", rightarrow] \arrow[d, "f"', rightarrow] &
		B^\bullet \arrow[r, "\beta", rightarrow] \arrow[d, "g"', rightarrow] &
		C^\bullet \arrow[r, "", rightarrow] \arrow[d, "h"', rightarrow] &
		0 \\
		0 \arrow[r, "", rightarrow] &
		X^\bullet \arrow[r, "\alpha'"', rightarrow] &
		Y^\bullet \arrow[r, "\beta'"', rightarrow] &
		W^\bullet \arrow[r, "", rightarrow] &
		0 
	\end{tikzcd}
	\end{equation} 
	gets mapped to the following morphism of long exact sequences, in $\mathsf{L}$ 
	\begin{equation*}
	\begin{tikzcd}[column sep = 2.4em]
		\ldots \arrow[r, "", rightarrow] &
%		H^n(A^\bullet) \arrow[r, "H^n(\alpha)", rightarrow] \arrow[d, "H^n(f)"', rightarrow] &
		H^n(B^\bullet) \arrow[r, "H^n(\beta)", rightarrow] \arrow[d, "H^n(g)"', rightarrow] &
		H^n(C^\bullet) \arrow[r, "\partial_1^n", rightarrow] \arrow[d, "H^n(h)"', rightarrow] 
		\arrow[rd, "\circlearrowright" description, phantom, rightarrow] &
		H^{n+1}(A^\bullet) \arrow[r, "H^{n+1}(\alpha)", rightarrow] \arrow[d, "H^{n+1}(f)", rightarrow] &
		H^{n+1}(B^\bullet) \arrow[r, "H^{n+1}(\beta)", rightarrow] \arrow[d, "H^{n+1}(g)", rightarrow] &
		\ldots \\
		\ldots \arrow[r, "", rightarrow] &
%		H^n(X^\bullet) \arrow[r, "H^n(\alpha')"', rightarrow] &
		H^n(Y^\bullet) \arrow[r, "H^n(\beta')"', rightarrow] &
		H^n(W^\bullet) \arrow[r, "\partial_2^n"', rightarrow] &
		H^{n+1}(X^\bullet) \arrow[r, "H^{n+1}(\alpha')"', rightarrow] &
		H^{n+1}(Y^\bullet) \arrow[r, "H^{n+1}(\beta')"', rightarrow] &
		\ldots 
	\end{tikzcd}
	.\end{equation*} 
	In particular also the squares involving the connecting morphisms $\partial^n$ commute,
	in other words we have $H^{n+1}(f) \circ \partial_1^n = \partial_2^n \circ H^n(h)$.
\end{prop} 

\begin{rem}[Notation]
	The long exact (co)homology sequence associated to
	\begin{equation}
	0 \to A^\bullet \to B^\bullet \to C^\bullet \to 0
	\end{equation} 
	can be visualized by the following diagram, called the exact triangle
	\begin{equation}
	\begin{tikzcd}[column sep=tiny]
		H^{\bullet}(A^\bullet) \arrow[rr, "", rightarrow] & &
		H^{\bullet}(B^\bullet) \arrow[dl, "", rightarrow] \\
		&
		H^{\bullet}(C^\bullet) \arrow[lu, "\partial", rightarrow] &
	\end{tikzcd}
	.\end{equation} 
\end{rem}

\begin{lem}[$3 \cross 3$ lemma]
	Let $\mathsf{A}$ be an abelian category.
	Consider the following commutative diagram with exact columns
	\begin{equation}
	\begin{tikzcd}
		0 \arrow[r, "", rightarrow] &
		A_1 \arrow[r, "", rightarrow] \arrow[d, "", tail] &
		B_1 \arrow[r, "", rightarrow] \arrow[d, "", tail] &
		C_1 \arrow[r, "", rightarrow] \arrow[d, "", tail] &
		0 \\
		0 \arrow[r, "", rightarrow] &
		A_2 \arrow[r, "", rightarrow] \arrow[d, "", two heads] &
		B_2 \arrow[r, "", rightarrow] \arrow[d, "", two heads] &
		C_2 \arrow[r, "", rightarrow] \arrow[d, "", two heads] &
		0 \\
		0 \arrow[r, "", rightarrow] &
		A_3 \arrow[r, "", rightarrow] &
		B_3 \arrow[r, "", rightarrow] &
		C_3 \arrow[r, "", rightarrow] &
		0 \\
	.\end{tikzcd}
	\end{equation} 
	\begin{enumerate}
		\item If the $2$nd and $3$rd rows are exact, then so is the $1$st.
		\item If the $1$st and $2$nd rows are exact, then so is the $3$rd.
		\item If the $1$st and $3$rd rows are exact, and the $2$nd is a complex,
			then the $2$nd is also exact.
	\end{enumerate}
\end{lem} 

\begin{lem}[$5$ lemma]
	Let $\mathsf{A}$ be an abelian category.
	Consider the following commutative diagram with exact rows
	\begin{equation}
	\begin{tikzcd}
		A_1 \arrow[r, "", rightarrow] \arrow[d, "a", rightarrow] &
		B_1 \arrow[r, "", rightarrow] \arrow[d, "b", rightarrow] &
		C_1 \arrow[r, "", rightarrow] \arrow[d, "c", rightarrow] &
		D_1 \arrow[r, "", rightarrow] \arrow[d, "d", rightarrow] &
		E_1 \arrow[d, "e", rightarrow] \\
		A_2 \arrow[r, "", rightarrow] &
		B_2 \arrow[r, "", rightarrow] &
		C_2 \arrow[r, "", rightarrow] &
		D_2 \arrow[r, "", rightarrow] &
		E_2 
	.\end{tikzcd}
	\end{equation} 
	\begin{enumerate}
		\item If $b$ and $d$ are mono and $a$ is epi, then $c$ is mono.
		\item If $b$ and $d$ are epi and $e$ is mono, then $c$ is epi.
	\end{enumerate}
\end{lem} 

\begin{defn}[quasi-isomorphism]
	Let $\mathsf{A}$ be an abelian category.
	Let $f: \left( X^{\bullet}, d_{X} \right) \to \left( Y^{\bullet}, d_{Y} \right)$ be a cochain map in $\mathrm{Ch}(\mathsf{A})$.
	We say that $f$ is a {\em quasi-isomorphism} iff the induced cohomology morphism
	\begin{equation}
		H^n(f): H^n(X^\bullet) \to H^n(Y^\bullet)
	\end{equation} 
	is an isomorphism for every $n \in \Z$.
\end{defn}

\begin{lem}
	An homotopy equivalence $f\colon X^\bullet \to Y^\bullet$, i.e.
	an iso in $K(\mathsf{A})$, is a {\em quasi-isomorphism}.
\end{lem} 

\begin{lem}
	One can find examples of quasi-isomorphism, which is not an homotopy equivalence.
	(Look at morphisms of exact sequences).
\end{lem} 

\begin{lem}
	Let $\mathsf{A}$ be an abelian category and consider
	$\left( X^{\bullet}, d_{X} \right) \in \mathrm{Ch}(\mathsf{A})$.
	Define $\left( Z^{\bullet}, d_{Z} \right)$ by:
	\begin{equation}
		Z^n := Z^n(X^\bullet) := \ker d_X^n \qquad \text{ and } \qquad
		d^n_{Z} = 0 \qquad \,\forall\, n \in \Z
	.\end{equation} 
	Analogously define the complex $\left( B^{\bullet}, d_{B} \right)$ by
	\begin{equation}
		B^n := B^n(X^\bullet) := \ima d_X^{n-1} \qquad \text{ and } \qquad
		d^n_{B} = 0 \quad \,\forall\, n \in \Z
	.\end{equation} 
	Then there is a short exact sequence of complexes
	\begin{equation}
	0\to Z^\bullet \to X^\bullet \to
	B^\bullet[1] \to 0
	,\end{equation} 
	whose associated long exact sequence breaks into short exact sequences in $\mathsf{A}$.
\end{lem} 
\begin{proof}
	Apart from the exactness of the short exact sequence of complexes, notice that:
	$H^n(Z^\bullet) = Z^n$ and $H^n(B^\bullet[1]) = H^{n+1}(B^\bullet) = B^{n+1}$ for all $n \in \Z$.
	Then the associated long exact sequence is
	\begin{equation}
		\ldots B^n \xrightarrow{\partial} Z^n \to H^n(X^\bullet) \to B^{n+1} \xrightarrow{\partial}
		Z^{n+1} \to H^{n+1}(X^\bullet) \to \ldots
	.\end{equation} 
	But, for each $n$, the above breaks into the short exact sequences
	\begin{equation*}
		0 \to B^n \to Z^n \to H^n(X^\bullet) \to 0.\qedhere
	\end{equation*} 
\end{proof}

\begin{lem}
	Let $f: \left( X^{\bullet}, d_{X} \right) \to \left( Y^{\bullet}, d_{Y} \right)$
	be a cochain map in $\mathrm{Ch}(\mathsf{A})$,
	for an abelian category $\mathsf{A}$.
	Assume that $\left(\ker f^\bullet, d^\bullet\right)$ and $\left(\coker f^\bullet, d^\bullet\right)$
	are acyclic.
	Then $f$ is a quasi-isomorphism.
\end{lem} 

\begin{rem}[]
	Notice that the converse of the above lemma is not true:
	for example the complexes
	\begin{equation}
	X^\bullet = Y^\bullet = 0 \to \Z \xrightarrow{\dot{2}} 
	\Z \xrightarrow{\pi} \mathbb{Z}/2\mathbb{Z} \to 0
	\end{equation} 
	are both acyclic.
	Then the map $f = (\dot{4}, \dot{4}, 0)$, represented by
	\begin{equation}
	\begin{tikzcd}
		0 \arrow[r, "", rightarrow] &
		\Z \arrow[r, "\dot{2}", rightarrow] \arrow[d, "\dot{4}"', rightarrow] &
		\Z \arrow[r, "\pi", rightarrow] \arrow[d, "\dot{4}"', rightarrow] &
		\mathbb{Z}/2\mathbb{Z} \arrow[r, "", rightarrow] \arrow[d, "0"', rightarrow] &
		0 \\
		0 \arrow[r, "", rightarrow] &
		\Z \arrow[r, "\dot{2}"', rightarrow] &
		\Z \arrow[r, "\pi"', rightarrow] &
		\mathbb{Z}/2\mathbb{Z} \arrow[r, "", rightarrow] &
		0 
	\end{tikzcd}
	\end{equation} 
	is a quasi-isomorphism.
	Moreover the cochains $(\ker f)^\bullet$ and $(\coker f)^\bullet$ are
	\begin{align}
		(\ker f)^\bullet &= 0 \to 0 \to 0 \to \mathbb{Z}/2\mathbb{Z} \to 0\\
		(\coker f)^\bullet &= 0 \to \mathbb{Z}/4\mathbb{Z} \xrightarrow{\dot{2}} \mathbb{Z}/4\mathbb{Z}
		\xrightarrow{\pi} \mathbb{Z}/2\mathbb{Z} \to 0
	.\end{align} 
	Then we can compute that $H^2((\ker f)^\bullet) = \mathbb{Z}/2\mathbb{Z}$
	and $H^0((\coker f)^\bullet) = \mathbb{Z}/2\mathbb{Z}$.
\end{rem}
