\subsection{Operations on complexes}
\begin{defn}[Canonical truncation]
	Let $\left( X^{\bullet}, d_{X} \right)$ be a cochain complex and $n \in \Z$.
	We define the {\em canonical truncation} of $\left( X^{\bullet}, d_{X} \right)$ to be the complex
	$\left( [\tau_{\leq n}(X^\bullet)]^{\bullet}, d_{[\tau_{\leq n}(X^\bullet)]} \right)$, whose objects are
	\begin{equation}
		[\tau_{\leq n}(X^\bullet)]^i := 
	\begin{cases}
		X^i & \text{ if } i < n\\
		\ker d_X^n & \text{ if } i = n\\
		0 & \text{ if } i > n\\
	\end{cases} 
	,\end{equation} 
	and differentials given by the induced ones.
	Denoted by $\epsilon^n: \ker d^n_X \to X^n$ the Kernel, then we have a
	natural cochain map $\epsilon: \tau_{\leq n}(X^\bullet) \to X^\bullet$, given by
	\begin{equation}
	\begin{tikzcd}
		\ldots \arrow[r, "", rightarrow] &
		X^{n-2} \arrow[r, "d^{n-2}", rightarrow] \arrow[d, "1_{X^{n-2}}"', rightarrow] &
		X^{n-1} \arrow[r, "d^{n-1}", rightarrow] \arrow[d, "1_{X^{n-1}}"', rightarrow] &
		Z^{n}(X^\bullet) \arrow[r, "0", rightarrow] \arrow[d, "\epsilon^{n}"', rightarrow] &
		0 \arrow[r, "0", rightarrow] \arrow[d, "0"', rightarrow] &
		\ldots \\
		\ldots \arrow[r, "", rightarrow] &
		X^{n-2} \arrow[r, "d^{n-2}"', rightarrow] &
		X^{n-1} \arrow[r, "d^{n-1}"', rightarrow] &
		X^{n} \arrow[r, "d^{n}"', rightarrow] &
		X^{n+1} \arrow[r, "d^{n+1}"', rightarrow] &
		\ldots \\
	\end{tikzcd}
	,\end{equation} 
	which is clearly a mono.
	Moreover we can compute the associated cohomology groups
	(assuming $\mathsf{A}$ is abelian, or that we can compute them)
	and they are
	\begin{equation}
		H^i \left( \tau_{\leq n}(X^\bullet) \right) =
		\begin{cases}
			0 & \text{ if } i > n\\
			H^i(X^\bullet) & \text{ if } i \leq n\\
		\end{cases} 
	.\end{equation} 
	Moreover, since $\epsilon$ is an embedding, we can define the quotient complex,
	which we denote by
	$\left( [X^\bullet/\tau_{\leq n}(X^\bullet)]^{\bullet}, d_{[X^\bullet/\tau_{\leq n}(X^\bullet)]} \right)$,
	whose objects are
	\begin{equation}
		[X^\bullet/\tau_{\leq n}(X^\bullet)]^i := 
	\begin{cases}
		X^i & \text{ if } i > n\\
		X^n/\ker d_X^n & \text{ if } i = n\\
		0 & \text{ if } i < n\\
	\end{cases} 
	,\end{equation} 
	and differentials given by the induced one.
	Then, as expected
	\begin{equation}
		H^i \left( X^\bullet/\tau_{\leq n}(X^\bullet) \right) =
		\begin{cases}
			0 & \text{ if } i \leq n\\
			H^i(X^\bullet) & \text{ if } i > n\\
		\end{cases} 
	.\end{equation} 
	And we obtain a short exact sequence of complexes
	\begin{equation*}
	\begin{tikzcd}
		0 \arrow[r, "", rightarrow] &
		\tau_{\leq n}(X^\bullet) \arrow[r, "\epsilon", rightarrow] &
		X^\bullet \arrow[r, "", rightarrow] &
		X^\bullet/ \tau_{\leq n}(X^\bullet) \arrow[r, "", rightarrow] &
		0
	.\end{tikzcd}
	\end{equation*}
\end{defn}

\begin{defn}[Stupid truncation]
	Given, as before, a cochain complex $\left( X^{\bullet}, d_{X} \right)$ and $n \in \Z$,
	one defines its	{\em stupid truncation} as the cochain complex with objects
	\begin{equation}
		[\sigma_{\leq n}(X^\bullet)]^i =
		\begin{cases}
			X^i & \text{ if } i \leq n\\
			0 & \text{ if } i > n
		\end{cases} 
	\end{equation} 
	and induced differentials.
	Then one can construct a canonical map $X^\bullet \to \sigma_{\leq n}(X^\bullet)$ as
	\begin{equation}
	\begin{tikzcd}
		\ldots \arrow[r, "", rightarrow] &
		X^{n-1} \arrow[r, "", rightarrow] \arrow[d, "1_{X^{n-1}}", rightarrow] &
		X^{n} \arrow[r, "", rightarrow] \arrow[d, "1_{X^{n}}", rightarrow] &
		X^{n+1} \arrow[r, "", rightarrow] \arrow[d, "0", rightarrow] &
		\ldots\\
		\ldots \arrow[r, "", rightarrow] &
		X^{n-1} \arrow[r, "", rightarrow] &
		X^{n} \arrow[r, "0"', rightarrow] &
		0 \arrow[r, "", rightarrow] &
		\ldots
	\end{tikzcd}
	.\end{equation} 
	Moreover we can compute its cohomology groups, and obtain that they are
	\begin{equation}
		H^i \left( \sigma_{\leq n}(X^\bullet) \right) =
		\begin{cases}
			0 & \text{ if } i > n\\
			X^n/\ima d_X^{n-1} & \text{ if } i = n\\
			H^i(X^\bullet) & \text{ if } i < n
		\end{cases} 
	.\end{equation} 
\end{defn}

\begin{defn}[Mapping cone]
	Let $f \in \mathrm{Hom}_{\mathrm{Ch}(\mathsf{A})} \left( X^\bullet, Y^\bullet \right)$ an arbitrary cochain map.
	We define the {\em mapping cone} of $f$ as the cochain complex,
	denoted by $(\mathrm{Cone}\, f)^\bullet$, whose objects are
	\begin{equation}
		[\mathrm{Cone}\, f]^n := Y^n \oplus X^{n+1}
	\end{equation} 
	and differentials $d^n_{\mathrm{Cone}\, f}: Y^n \oplus X^{n+1} \to Y^{n+1} \oplus X^{n+2}$ given by
	the following matrix
	\begin{equation}
	d^n_{\mathrm{Cone}\, f} :=
	\begin{bmatrix}
		d^n_Y & f^{n+1}\\
		0 & -d_X^{n+1}
	\end{bmatrix} 
	.\end{equation} 
	This really is a complex, since we have the identity
	\begin{equation}
	d^2_{\mathrm{Cone}\, f} =
	\begin{bmatrix}
		d^n_Y & f^{n+1}\\
		0 & -d_X^{n+1}
	\end{bmatrix} 
	\begin{bmatrix}
		d^{n-1}_Y & f^{n}\\
		0 & -d_X^{n}
	\end{bmatrix}  = 
	\begin{bmatrix}
		0 & d^n_Y f^n - f^{n+1} d^n_X\\
		0 & 0
	\end{bmatrix} 
	\end{equation} 
	and $f$ is a cochain map (hence the last matrix is zero).
\end{defn}

\begin{defn}[Cone of a complex]
	Given a complex $\left( X^{\bullet}, d_{X} \right)$, we define the cocahin $(\mathrm{Cone}\, X)^\bullet$ as the
	mapping cone of the cochain map $1_{X^\bullet}: X^\bullet \to X^\bullet$.
\end{defn}


\begin{rem}[]
	From the definition of mapping cone we obtain the short exact sequence of complexes
	\begin{equation}\label{eqn:ConeSes}
	\begin{tikzcd}
		0 \arrow[r, "", rightarrow] &
		Y^\bullet \arrow[r, "\alpha", rightarrow] &
		\left( \mathrm{Cone}\, f \right)^\bullet \arrow[r, "\beta", rightarrow] &
		X^\bullet[1] \arrow[r, "", rightarrow] &
		0
	,\end{tikzcd}
	\end{equation}
	where the maps $\alpha$ and $\beta$ (check they are indeed cochain maps) are defined by the matrices
	\begin{equation}
	\alpha := 
	\begin{bmatrix}
		1_Y \\ 0
	\end{bmatrix} \qquad \text{ and } \qquad
	\beta := 
	\begin{bmatrix}
		0 & 1_{X^\bullet[1]}
	\end{bmatrix} 
	.\end{equation} 
	In particular, for each degree, the short exact sequence splits, in fact it is
	\begin{equation}
	0 \to Y^n \to Y^n \oplus X^{n+1} \to X^{n+1} \to 0
	,\end{equation} 
	and the maps are induced by $\alpha$ and $\beta$ (hence the splitting).
\end{rem}

\begin{lem}
	Let the following be a degree-wise splitting short exact sequence
	\begin{equation*}
	\begin{tikzcd}
		0 \arrow[r, "", rightarrow] &
		Y^\bullet \arrow[r, "", rightarrow] &
		C^\bullet \arrow[r, "", rightarrow] &
		W^\bullet \arrow[r, "", rightarrow] &
		0
	.\end{tikzcd}
	\end{equation*}
	Then there is a cochain map $f: W^\bullet[-1] \to Y^\bullet$ s.t.
	$C^\bullet \simeq \mathrm{Cone}\, f$.
\end{lem} 

\begin{lem}
	Let $f\colon X^\bullet \to Y^\bullet$ be a cochain map in $\mathrm{Ch}(\mathsf{A})$.
	Then $f$ is a quasi-isomorphism iff the complex
	$(\mathrm{Cone}\, f)^\bullet$ is acyclic.
\end{lem} 
\begin{proof}
	From the short exact sequence for the Cone of $f$, see \eqref{eqn:ConeSes}, and the 
	fundamental theorem in cohomology, one obtains the long exact cohomology sequence
	\begin{equation}
		\ldots \to H^{n-1}(X^\bullet[1]) \xrightarrow{\partial^n} H^n(Y^\bullet) \to
		H^n(\mathrm{Cone}\, f) \to H^n(X^\bullet[1]) \to \ldots
	.\end{equation} 
	One can show that $H^n(f) = \partial^n$, then
	$H^n(f)$ is an isomorphism iff $H^n(\mathrm{Cone}\, f) = 0$.
\end{proof}

\begin{defn}[Split complex]
	A complex $\left( X^{\bullet}, d_{X} \right)$ is {\em split} iff there exist
	maps $s^n: X^{n+1} \to X^n$, for all $n \in \Z$, s.t.
	$d^n_X \circ s^n \circ d^n_X = d^n_X$ for all $n \in \Z$ (shortly $d = d \circ s \circ d$).
	The maps $s^n$ are called {\em splitting maps}.
\end{defn}

\begin{lem}
	Let $\left( X^{\bullet}, d_{X} \right)$ be a complex, with cycles $Z^n$ and
	boundaries $B^n$.
	$X^\bullet$ is split iff, for every $n \in \Z$, there
	exist decompositions
	\begin{equation}
	X^n = Z^n \oplus C^n \qquad \text{ and } \qquad
	Z^n = B^n \oplus K^n
	,\end{equation} 
	with $K^n \simeq H^n(X^\bullet)$.
\end{lem} 
\begin{proof}
	In this proof we use the general fact, for $R$-modules, that given an idempotent
	endomorphism $e: M \to M$ (i.e. s.t. $e^2 = e$), then
	\begin{equation}
	M = \ker e \oplus \ima e
	.\end{equation} 
	In fact for any $x \in M$, then $x = e(x) + (x - e(x))$ and
	$e(x - e(x)) = 0$.
	Moreover, given $x \in \ker e \cap \ima e$, there exists $y$ s.t. $x = e(y)$, then
	\begin{equation*}
		0 = e(x) = e(e(y)) = e(y) = x.\qedhere
	\end{equation*} 
\end{proof}

\begin{defn}[Split exact/contractible complex]
	A complex $\left( X^{\bullet}, d_{X} \right)$ is called {\em split exact} or {\em contractible} iff
	it is both {\em split} and {\em acyclic} (i.e. exact).
\end{defn}

\begin{rem}[]
	By the above lemma, the complex $\left( X^{\bullet}, d_{X} \right)$ is contractible iff
	there exist decompositions $X^n = Z^n \oplus C^n$ and $B^n = Z^n$, for every $n \in \Z$.
\end{rem}

\begin{lem}
	$(\mathrm{Cone}\, X)^\bullet$ is contractible.
\end{lem} 
\begin{proof}
	$\left( \mathrm{Cone}\, C \right)^\bullet$ is exact, since $1_{X^\bullet}$ is a quasi-isomorphism.
	Then we define the splitting maps by
	\begin{equation*}
	s^n :=
	\begin{bmatrix}
		0 & 0\\
		1_{X^{n+1}} & 0.
	\end{bmatrix} \qedhere
	\end{equation*} 
\end{proof}

\begin{lem}
	A complex $\left( X^{\bullet}, d_{X} \right)$ is contractible iff
	$1_{X^\bullet}$ is nullhomotopic.
\end{lem}
\begin{rem}[]
	This lemma can be stated as: any contractible complex is isomorphic to
	the $0$ complex in the homotopy category.
\end{rem}

\begin{lem}
	Let $f: X^\bullet \to Y^\bullet$ be a cochain map.
	Then $f \sim 0$ iff $f$ extends to
	\begin{equation}
		\begin{bmatrix}
			f & s
		\end{bmatrix} 
		\colon \mathrm{Cone}\, X \to Y
	,\end{equation} 
	where $\left\{ s^n \right\}_{n \in \Z}$ are the contractions.
\end{lem} 

\begin{lem}
	Let $\left( X^{\bullet}, d_{X} \right)$ be a split complex
	with splitting maps $s \coloneqq \left\{ s^n \right\}_{n \in \Z}$.
	Then $f = s \circ d + d \circ s$ is a cochain map (clearly, then $f \sim 0$).
\end{lem} 

\begin{rem}[]
	for all $A \in \mathrm{Ob} \left(\mathsf{A}\right)$, we define the following complex
	\begin{equation}
		D^n(A) \coloneqq 
		\ldots \to 0 \to A \xrightarrow{1_A} A \to 0 \to \ldots
	,\end{equation} 
	where the non-zero elements are in degree $n$ and $n+1$.
	Clearly $D^n(A)$ contractible. In fact:
	\begin{equation}
	\begin{tikzcd}
		0 \arrow[r, "", rightarrow] &
		A \arrow[r, "1_A", rightarrow] \arrow[ld, "0"', rightarrow] \arrow[d, "1_A"', rightarrow] &
		A \arrow[r, "", rightarrow] \arrow[ld, "1_A"', rightarrow] \arrow[d, "1_A", rightarrow] &
		0 \arrow[ld, "0", rightarrow] \\
		0 \arrow[r, "", rightarrow] &
		A \arrow[r, "1_A"', rightarrow] &
		A \arrow[r, "", rightarrow] &
		0 
	\end{tikzcd}
	.\end{equation} 
\end{rem}

\begin{lem}
	for all $A \in \mathrm{Ob} \left(\mathsf{A}\right)$ and $X^\bullet \in \mathrm{Ch}(\mathsf{A})$ we have,
	naturally in both components,
	\begin{equation}
		\mathrm{Hom}_{\mathrm{Ch}(\mathsf{A})} \left( D^n(A), X^\bullet \right) \simeq
		\mathrm{Hom}_{\mathsf{A}} \left( A, X^n \right)
	.\end{equation} 
	In other words the pair $(D^n, (-)^n)$ is an adjoint pair for every $n \in \Z$, for the functors
	\begin{equation*}
	\begin{tikzcd}[row sep = 0ex
		,/tikz/column 1/.append style={anchor=base east}
		,/tikz/column 2/.append style={anchor=base west}]
		D^n\colon \mathsf{A} \arrow[r, "", rightarrow] &
		\mathrm{Ch}(\mathsf{A}) \\
		A \arrow[r, "", mapsto] & D^n(A)
	\end{tikzcd}
	\qquad \text{ and } \qquad
	\begin{tikzcd}[row sep = 0ex
		,/tikz/column 1/.append style={anchor=base east}
		,/tikz/column 2/.append style={anchor=base west}]
		(-)^n\colon \mathrm{Ch}(\mathsf{A}) \arrow[r, "", rightarrow] &
		\mathsf{A} \\
		X^\bullet \arrow[r, "", mapsto] & x^n
	.\end{tikzcd}
	\end{equation*} 
\end{lem} 

\begin{prop}
	Let $\mathsf{A}$ be an abelian category.
	A complex $\left( P^{\bullet}, d_{P} \right)$ is a projective object of
	$\mathrm{Ch}(\mathsf{A})$ iff $P^i$ is projective in $\mathsf{A}$ for all
	$i \in \Z$ and $\left( P^{\bullet}, d_{P} \right)$ is contractible.

	A complex $\left( I^{\bullet}, d_{I} \right)$ is an injective object of
	$\mathrm{Ch}(\mathsf{A})$ iff $I^i$ is injective in $\mathsf{A}$ for all
	$i \in \Z$ and $\left( I^{\bullet}, d_{I} \right)$ is contractible.
\end{prop} 

\begin{lem}
	Assume that $\mathsf{A}$ is an abelian category, with enough
	projectives (i.e. $\,\forall\, A \in \mathrm{Ob} \left(\mathsf{A}\right)$ there is a
	projective object $P \in \mathrm{Ob} \left(\mathsf{A}\right)$, with an epi
	$P \xrightarrow{\varphi} A \to 0$).
	Then $\mathrm{Ch}(\mathsf{A})$ has enough projectives.
\end{lem} 

\begin{rem}[]
	Given a cochain complex $\left( X^{\bullet}, d_{X} \right)$, we can define an associated
	chain complex $\left( X_{\bullet}, d^{X} \right)$ by setting $X_n := X^{-n}$.
\end{rem}
