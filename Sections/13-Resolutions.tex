\section{Derived functors}
\subsection{Resolutions}
\begin{defn}[(Co)homological $\partial$-functor]
	Let $\mathsf{A}, \mathsf{B}$ be abelian categories.
	A {\em (co)homological} $\partial${\em -functor} between $\mathsf{A}$ and $\mathsf{B}$
	is the data of a sequence of functors $\left\{ T^n \right\}_{n \in \Z}$, with
	$T^n\colon \mathsf{A} \to \mathsf{B}$ for every $n$,
	($\left\{ T_n \right\}_{n \in \Z}$ for the homological functors)
	s.t. $T^i = 0$ for all $i < 0$ ($T_i = 0$ for all $i > 0$)
	and for any short exact sequence $S \in \mathsf{S}(\mathsf{A})$
	\begin{equation}
	0 \to A \to B \to C \to 0
	\end{equation} 
	for all $n \in \Z$ there is a connecting morphism
	$\partial^n\colon T^n(C) \to T^{n+1}(A)$ (resp. $\partial_n\colon T_n(C) \to T_{n-1}(A)$)
	satisfying
	\begin{enumerate}
		\item there is a long exact sequence
			\begin{equation}
				\ldots \to T^{n-1}(C) \xrightarrow{\partial^{n-1}} T^n(A) \to
				T^n(B) \to T^n(C) \xrightarrow{\partial^n} T^{n+1}(A) \to
				\ldots =: T(S)
			,\end{equation} 
			respectively the long exact sequence
			\begin{equation}
				\ldots \to T_{n+1}(C) \xrightarrow{\partial_{n+1}} T_n(A) \to
				T_n(B) \to T_n(C) \xrightarrow{\partial_n} T_{n-1}(A) \to
				\ldots =: T(S)
			,\end{equation} 
		\item For any $S' \in \mathsf{S}(\mathsf{A})$ and any morphism
			$S \to S'$ in $\mathsf{S}(\mathsf{A})$, i.e.
			\begin{equation}
			\begin{tikzcd}
				0 \arrow[r, "", rightarrow] &
				A \arrow[r, "", rightarrow] \arrow[d, "f", rightarrow] &
				B \arrow[r, "", rightarrow] \arrow[d, "g", rightarrow] &
				C \arrow[r, "", rightarrow] \arrow[d, "h", rightarrow] &
				0 \\
				0 \arrow[r, "", rightarrow] &
				A' \arrow[r, "", rightarrow] &
				B' \arrow[r, "", rightarrow] &
				C' \arrow[r, "", rightarrow] &
				0,
			\end{tikzcd}
			\end{equation} 
			there is an associated morphism between the long exact sequences, i.e.
			a commutative diagram (with a clear dual for the homological case)
			\begin{equation}
			\begin{tikzcd}
				T^{n-1}(C) \arrow[r, "\partial^{n-1}", rightarrow] \arrow[d, "T^{n-1}(h)", rightarrow] &
				T^n(A) \arrow[r, "", rightarrow] \arrow[d, "T^n(f)", rightarrow] &
				T^n(B) \arrow[r, "", rightarrow] \arrow[d, "T^n(g)", rightarrow] &
				T^n(C) \arrow[r, "\partial^n", rightarrow] \arrow[d, "T^n(h)", rightarrow] &
				T^{n+1}(A) \arrow[d, "T^{n+1}(f)", rightarrow] \\
				T^{n-1}(C') \arrow[r, "\partial^{n-1}"', rightarrow] &
				T^n(A') \arrow[r, "", rightarrow] &
				T^n(B') \arrow[r, "", rightarrow] &
				T^n(C') \arrow[r, "\partial^n"', rightarrow] &
				T^{n+1}(A')
			\end{tikzcd}
			.\end{equation} 
	\end{enumerate}
	Then the family $T \coloneqq \left\{ T^n \right\}_{n \in \Z}\colon\mathsf{S}(\mathsf{A}) \to \mathsf{L}(\mathsf{B})$
	actually is a functor.
\end{defn}

\begin{rem}[]
	$T^0$ is always left exact, for a cohomological $\partial$-functor.
	In fact given a short exact sequence in $\mathsf{A}$
	\begin{equation}
		0 \to A \to B \to C \to 0
	,\end{equation} 
	the associated long exact sequence, since $T^{-1} = 0$, is
	\begin{equation}
		T^{-1}(C) = 0 \xrightarrow{\partial^{-1}} T^0(A) \to T^0(B) \to T^0(C) \to \ldots
	.\end{equation} 
	Analogously, one checks that $T_0$ is right exact, for any
	homological $\partial$-functor.
\end{rem}

\begin{ex}
	Consider $\mathsf{A} \coloneqq \mathsf{Mod}\text{-}R$, for a ring $R$.
	Consider $U^{-1}$ and $U^0$ free (in particular projective) $R$-modules
	and the morphism of modules $u\colon U^{-1} \to U^0$.
	Consider functor $T$, given by the family $\left\{ T^0, T^1 \right\}$ (i.e. $T^i = 0$ for all $i \neq 1,0$),
	where
	\begin{equation}
	T^0 \coloneqq \ker \mathrm{Hom}_{R} \left( u, - \right)
	\qquad \text{ and } \qquad
	T^1 \coloneqq \coker \mathrm{Hom}_{R}\left( u, - \right)
	.\end{equation} 
	Let's show that $T\colon\mathsf{Mod}\text{-}R \to \mathsf{Mod}\text{-}R$ is a
	cohomological $\partial$-function.
	Consider a short exact sequence of modules
	$0 \to A\to B \to C \to 0$.
	We need to show that the following is exact:
	\begin{equation}
		0 \to T^0(A) \to T^0(B) \to T^0(C) \to
		T^1(A) \to T^1(B) \to T^1(C) \to 0
	.\end{equation} 
	In fact we can apply the covariant hom functor $\mathrm{Hom}_{R}\left( u, - \right)$ and obtain
	\begin{equation}
	\begin{tikzcd}
		0 \arrow[r, "", rightarrow] &
		\mathrm{Hom}_{R}\left( U^0, A \right) \arrow[r, "", rightarrow]
		\arrow[d, "{\mathrm{Hom}_{R}\left( u , A \right)}"', rightarrow] &
		\mathrm{Hom}_{R}\left( U^0, B \right) \arrow[r, "", rightarrow]
		\arrow[d, "{\mathrm{Hom}_{R}\left( u , B \right)}"', rightarrow] &
		\mathrm{Hom}_{R}\left( U^0, C \right) \arrow[r, "", rightarrow]
		\arrow[d, "{\mathrm{Hom}_{R}\left( u , C \right)}"', rightarrow] &
		0 \\
		0 \arrow[r, "", rightarrow] &
		\mathrm{Hom}_{R}\left( U^{-1}, A \right) \arrow[r, "", rightarrow] &
		\mathrm{Hom}_{R}\left( U^{-1}, B \right) \arrow[r, "", rightarrow] &
		\mathrm{Hom}_{R}\left( U^{-1}, C \right) \arrow[r, "", rightarrow] &
		0 
	\end{tikzcd}
	\end{equation} 
	which clearly is commutative and with exact rows (both $U^0$ and $U^{-1}$ are free,
	hence projective, i.e. both $\mathrm{Hom}_{ R}\left( U^0, - \right)$ and
	$\mathrm{Hom}_{ R}\left( U^{-1}, - \right)$ are exact functors), then by the snake lemma
	we obtain exactness of the long sequence.

	Moreover consider $\mathsf{C} \coloneqq \left\{ X \in \mathrm{Ob} \left(\mathsf{Mod}\text{-}R\right) \ \middle|\ 
	T^0(X) = T^1(X) = 0\right\} \subset \mathsf{Mod}\text{-}R$.
	Then this subcategory of $\mathsf{Mod}\text{-}R$ is closed under
	kernel, cokernel, extension and products.
	In particular $C$ is an abelian full subcategory of $\mathsf{Mod}\text{-}R$.
	In fact, given $X, Y \in \mathrm{Ob} \left(\mathsf{C}\right)$, and a morphism $f\colon X \to Y$.
	Let $K \coloneqq \ker f$, $I \coloneqq \ima f$, and $C \coloneqq \coker f$.
	Then we have the short exact sequences
	$0 \to K \to X \to I \to 0$ and $0 \to I \to Y \to C \to 0$.
	The functor $T$ associates to them the long exact sequences
	\begin{equation}
		0 \to T^0(K) \to 0 \to T^0(I) \to
		T^1(K) \to 0 \to T^1(I) \to 0
	\end{equation} 
	\begin{equation}
		0 \to T^0(I) \to 0 \to T^0(C) \to
		T^1(I) \to 0 \to T^1(C) \to 0
	.\end{equation} 
	With simple computations one shows that $I, C, K \in \mathrm{Ob} \left(\mathsf{C}\right)$.
\end{ex} 

\begin{defn}[Left resolution]
	Let $\mathsf{A}$ be an abelian category, and $M \in \mathrm{Ob} \left(\mathsf{A}\right)$.
	A {\em left resolution} of $M$ is a chain-complex:
	\begin{equation}
		X_{\bullet} \coloneqq \ldots \to X_2 \xrightarrow{d_2} X_1 \xrightarrow{d_1} X_0 \to 0
	\end{equation} 
	s.t. there exists $\pi\colon X_0 \to M$ with which the augmented complex
	\begin{equation}
	X_{\bullet} \xrightarrow{\pi} M \to 0 \coloneqq \ldots \to X_2 \xrightarrow{d_2} 
	X_1 \xrightarrow{d_1} X_0 \xrightarrow{\pi} M \to 0
	\end{equation} 
	is exact.
	We can view these conditions as stating that $\pi$
	is a quasi-isomorphism between $X_{\bullet}$ and $M$,
	viewed as a complex concentrated in degree $0$.
	In fact $H_i(X_{\bullet}) = 0$ for all $i \neq 0$
	and $H_0(X_{\bullet}) \simeq M$.

	If, moreover, each $X_i$ is a projective object in $\mathsf{A}$, then
	the left resolution $X_{\bullet} \xrightarrow{\pi} M \to 0$ is called a
	{\em projective resolution} of $M$.
\end{defn}

\begin{lem}
	Let $\mathsf{A}$ be an abelian category, with enough projectives.
	Then every $M \in \mathrm{Ob} \left(\mathsf{A}\right)$ admits a
	projective resolution.
\end{lem} 
\begin{proof}
	One obtains an exact resolution by taking, each time, a projection onto the kernel of the previous map
	(it can be done, since $\mathsf{A}$ has enough projectives)
	\begin{equation}
	\ldots \to P_2 \xrightarrow{\pi_2} P_1 \xrightarrow{\pi_1} P_0 \xrightarrow{\pi_0} M \to 0
	.\end{equation} 
	We can define, for each $n \in \N$, $K_n \coloneqq \ker \pi_{n-1}$.
	Then $K_n$ is called $n$-th syzygy of $M$, sometimes
	denoted by $\Omega_n(M)$.

	Moreover the projective resolution is usually
	denoted by $P_{\bullet} \xrightarrow{\pi_0} M \to 0$.
\end{proof}

\begin{thm}[Comparison]
	Let $\mathsf{A}$ be an abelian category.
	Let $f_{-1}\colon M \to N$ be a morphism in $\mathsf{A}$.
	Consider the chain complex (not necessairily exact)
	\begin{equation}
	\ldots \to P_3 \to P_2 \to P_1 \to P_0 \xrightarrow{\pi} 
	M \to 0
	,\end{equation} 
	with $P_i$ projective for all $i \geq 0$.
	Let $Y_{\bullet} \xrightarrow{\sigma} N \to 0$ be a left resolution of $N$.
	Then there is a chain map $f\colon P_{\bullet} \to Y_{\bullet}$ lifting $f_{-1}$,
	i.e.
	\begin{equation}
	\begin{tikzcd}
		\ldots \arrow[r, "", rightarrow] &
		P_3 \arrow[r, "d_3^P", rightarrow] \arrow[d, "f_3"', rightarrow] &
		P_2 \arrow[r, "d_2^P", rightarrow] \arrow[d, "f_2"', rightarrow] &
		P_1 \arrow[r, "d_1^P", rightarrow] \arrow[d, "f_1"', rightarrow] &
		P_0 \arrow[r, "\pi", rightarrow] \arrow[d, "f_0"', rightarrow] &
		M \arrow[r, "", rightarrow] \arrow[d, "f_{-1}"', rightarrow] &
		0\\
		\ldots \arrow[r, "", rightarrow] &
		Y_3 \arrow[r, "d_3^Y"', rightarrow] &
		Y_2 \arrow[r, "d_2^Y"', rightarrow] &
		Y_1 \arrow[r, "d_1^Y"', rightarrow] &
		Y_0 \arrow[r, "\sigma"', rightarrow] &
		N \arrow[r, "", rightarrow] &
		0
	\end{tikzcd}
	.\end{equation} 
	Moreover given any other chain map $g \coloneqq \left\{ g_n \right\}_{n \geq 0}$
	lifting $f_{-1}$, then $f \sim g$ the two chain maps are homotopic.
	In other words the lift of $f_{-1}$ is unique up to homotopy.
\end{thm}

\begin{lem}
	Let $\mathsf{A}$ be an abelian category, and $P_{\bullet}$ an acyclic
	chain complex bounded below (i.e. s.t. $\exists\, m \in \Z$ for which
	$P_i = 0$ for all $i < m$) with projective components.
	Then $P_{\bullet}$ is contractible
	(hence it is projective in the category of complexes).
\end{lem} 

\begin{lem}[Horseshoe]
	Let $\mathsf{A}$ be an abelian category.
	Consider the short exact sequence
	\begin{equation}
	0 \to A \xrightarrow{f} B \xrightarrow{g} C \to 0
	,\end{equation} 
	and the projective resolutions $P_{\bullet} \to A \to 0$ and $Q_{\bullet} \to C \to 0$
	for $A$ and $C$.
	Then we can complete the diagram with the red arrows.
	\begin{equation}
	\begin{tikzcd}
		& & & 0 \arrow[d, "", rightarrow] & \\
		\ldots \arrow[r, "", rightarrow] &
		P_1 \arrow[r, "", rightarrow] &
		P_0 \arrow[r, "", rightarrow] &
		A \arrow[r, "", rightarrow] \arrow[d, "f", rightarrow] &
		0\\
		{\color{red}\ldots} \arrow[r, red, "", rightarrow] &
		{\color{red} P_1 \oplus Q_1} \arrow[r, red, "", rightarrow] &
		{\color{red} P_0 \oplus Q_0} \arrow[r, red, "", rightarrow] &
		B \arrow[r, red, "", rightarrow] \arrow[d, "g", rightarrow] &
		{\color{red}0}\\
		\ldots \arrow[r, "", rightarrow] &
		Q_1 \arrow[r, "", rightarrow] &
		Q_0 \arrow[r, "", rightarrow] &
		C \arrow[r, "", rightarrow] \arrow[d, "", rightarrow] &
		0\\
		& & & 0 &
	\end{tikzcd}
	.\end{equation} 
	In particular $\left( P_{\bullet} \oplus Q_{\bullet}, d^P_{\bullet} \oplus d^Q_{\bullet} \right)$
	gives a projective resolution of $B$, completing the diagram in the second row.
\end{lem} 

Let's now dualize everything we obtained up to now:
\begin{defn}[Right coresolution]
	Let $\mathsf{A}$ be an abelian category and $M \in \mathrm{Ob} \left(\mathsf{A}\right)$.
	A {\em right coresolution} of $M$ is a cochain complex
	\begin{equation}
	Y^\bullet \coloneqq 0 \to Y_0 \xrightarrow{d^0} Y^1 \xrightarrow{d^1} Y^2 \to\ldots	
	\end{equation} 
	s.t. there exists a morphism $\delta^0\colon M \to Y^0$ with which the augmented complex
	\begin{equation}
	0 \to M \xrightarrow{\delta^0} Y^\bullet \coloneqq 
	0 \to M \xrightarrow{\delta^0} Y_0 \xrightarrow{d^0} Y^1 \xrightarrow{d^1} Y^2 \to\ldots	
	\end{equation} 
	is exact.
	In other words we ask that $\delta^0$ is a quasi-isomorphism between
	$Y^\bullet$ and $M$ concentrated in degree $0$.
	Then $H^i(Y^\bullet) = 0$ for all $i \neq 0$ and $H^0(Y^\bullet) \simeq M$.

	If, moreover, each $Y^i$ is an injective object in $\mathsf{A}$,
	then the right coresolution $0 \to M \xrightarrow{\delta^0} Y^\bullet$ is called an
	{\em injective coresolution} of $M$.
\end{defn}

\begin{lem}
	Let $\mathsf{A}$ be an abelian category, with enough injectives.
	Then every object $M \in \mathrm{Ob} \left(\mathsf{A}\right)$ admits
	an injective coresolution.
\end{lem} 
\begin{proof}
	One obtains an exact resolution by taking, each time, the cokernel of the previous map
	\begin{equation}
		0 \to M \xrightarrow{\delta^0} I^0 \xrightarrow{\delta^1} I^1 \xrightarrow{\delta^1} I^2 \to \ldots
	.\end{equation} 
	We can define, for each $n \in \N$, $C^n \coloneqq \coker \delta^{n-1}$.
	Then $C^n$ is called $n$-th cosyzygy of $M$, sometimes
	denoted by $\Omega^n(M)$.

	Moreover the projective resolution is usually
	denoted by $0 \to M \xrightarrow{\delta^0} I^{\bullet}$.
\end{proof}

\begin{thm}[Comparison]
	Le $\mathsf{A}$ be an abelian category.
	Let $f^{-1}\colon M \to N$ a morphism in $\mathsf{A}$.
	Consider the cochain complex (not necessairily exact)
	\begin{equation}
	0 \to N \xrightarrow{\eta} I^0 \to I^1 \to I^2 \to I^3 \to \ldots
	,\end{equation} 
	with $I^i$ injective for all $i \geq 0$.
	Let $0 \to M \xrightarrow{\delta^0} Y^{\bullet}$ a right coresolution of $M$.
	Then there exists a cochain map $f\colon Y^\bullet \to I^\bullet$
	extending $f^{-1}$, i.e.
	\begin{equation}
	\begin{tikzcd}
		0 \arrow[r, "", rightarrow] &
		M \arrow[r, "\delta^0", rightarrow] \arrow[d, "f^{-1}"', rightarrow] &
		Y^0 \arrow[r, "d^0_Y", rightarrow] \arrow[d, "f^0"', rightarrow] &
		Y^1 \arrow[r, "d^1_Y", rightarrow] \arrow[d, "f^1"', rightarrow] &
		Y^2 \arrow[r, "d^2_Y", rightarrow] \arrow[d, "f^2"', rightarrow] &
		Y^3 \arrow[r, "d^3_Y", rightarrow] \arrow[d, "f^3"', rightarrow] &
		\ldots\\
		0 \arrow[r, "", rightarrow] &
		N \arrow[r, "\eta"', rightarrow] &
		I^0 \arrow[r, "d^0_I"', rightarrow] &
		I^1 \arrow[r, "d^1_I"', rightarrow] &
		I^2 \arrow[r, "d^2_I"', rightarrow] &
		I^3 \arrow[r, "d^3_I"', rightarrow] &
		\ldots\\
	\end{tikzcd}
	.\end{equation} 
	Moreover, given any other cochain map $g \coloneqq \left\{ g^n \right\}_{n \geq 0}$ extending
	$f^{-1}$, then $f \sim g$, the two cochain maps are homotopic.
	In other words the extension of $f^{-1}$ is unique up to homotopy.
\end{thm}
