\subsection{Left derived functors}
\begin{rem}[]
	A functor $F\colon\mathsf{A} \to \mathsf{B}$ between additive categories
	induces a functor, again denoted by $F$, 
	\begin{align}
		F\colon\mathrm{Ch}(\mathsf{A}) &\to \mathrm{Ch}(\mathsf{A}) \\
		\left( X^{\bullet}, d_{X} \right) &\mapsto \left( F(X^{\bullet}), d_{F(X^\bullet)} \right)
	,\end{align} 
	where $\left[ F(X^\bullet) \right]^n \coloneqq F(X^n)$ and $d_{F(X^\bullet)}^n \coloneqq F(d^n_{X^\bullet})$.
	Given a morphism $f\colon X^\bullet \to Y^\bullet$ in $\mathrm{Ch}(\mathsf{A})$, one has
	$d_Y \circ f = f \circ d_X$.
	Then $F(f) \circ F(d_X) = f(d_Y) \circ F(f)$, i.e. $F(f)$ is a morphism in $\mathrm{Ch}(\mathsf{B})$.

	Moreover, if $F$ is an additive functor, then
	$f = s \circ d_X + d_Y \circ s$ implies
	$F(f) = F(s) \circ F(d_X) + F(d_Y) \circ F(s)$,
	hence $F$ induces a functor
	\begin{equation}
		F\colon K(\mathsf{A}) \to K(\mathsf{B})
	.\end{equation} 
\end{rem}

\begin{defn}[Left derived functors]
	Let $\mathsf{A}$ and $\mathsf{B}$ be abelian categories.
	Assume that $\mathsf{A}$ has enough projectives and $F\colon \mathsf{A} \to \mathsf{B}$
	is a right exact functor.
	We define the left derived functor $L_iF\colon\mathsf{A} \to \mathsf{B}$
	s.t. $L_iF(A) \coloneqq H_i \left( F(P_{\bullet}) \right)$, 
	for $P_{\bullet} \to A \to 0$ a projective resolution of $A$
	and $i \geq 0$.
\end{defn}

\begin{rem}[]
	One actually needs to prove that the above is a good definition, i.e.
	that $L_i F$ does not depend on the projective resolution $P_{\bullet} \to A \to 0$.

	Moreover one can prove that $L_0F \simeq F$ as functors.
	In fact consider any projective resolution $P_{\bullet} \to A \to 0$ of $A$. Then
	\begin{equation}
	\ldots \to P_2 \to P_1 \xrightarrow{d_1} P_0 \to A \to 0
	\end{equation} 
	is exact, with $F$ right exact.
	Then 
	\begin{equation}
		F(P_1) \xrightarrow{F(d_1)} F(P_0) \to F(A) \to 0
	\end{equation} 
	is also exact.
	In particular $\coker F(d_1) \simeq F(A)$.
	But then $L_0 F(A) = H_0 \left( F(P_{\bullet}) \right)$.
	We know that
	\begin{equation}
		F(P_{\bullet}) = \ldots \to F(P_1) \xrightarrow{F(d_1)} F(P_0) \to 0
	\end{equation} 
	hence that $H_0 \left( F(P_{\bullet}) \right) = \coker F(d_1) = F(A)$.
\end{rem}

\begin{lem}\leavevmode\vspace{-.2\baselineskip}
	\begin{enumerate}[label=(\alph*)]
		\item For each $i \in \N$, $L_i F$ is well defined, up to natural isomorphism.
		\item Let $\alpha\colon A \to C$ be a morphism in $\mathsf{A}$.
			Then there are natural maps
			\begin{equation}
				L_i F(\alpha)\colon L_iF(A) \to L_iF(C)
			.\end{equation} 
		\item For any $i \geq 0$, the functor $L_i F$ is additive.
	\end{enumerate}
\end{lem} 

\begin{lem}
	Let $f\colon A \to C$ be a morphism in $\mathsf{A}$.
	Then $L_0F(f) = F(f)$.
\end{lem} 

\begin{prop}
	Let $F$ and $L_i F$ be as in the above definition.
	If $A \in \mathsf{A}$ is projective, then $L_iF(A) = 0$
	for all $i > 0$ (recall that $L_0F(A) \simeq F(A)$).
\end{prop} 

\begin{defn}[F-acyclic object]
	Let $\mathsf{A}$ be an abelian category with enough projectives
	and $F\colon\mathsf{A} \to \mathsf{B}$ be a right exact functor.
	An object $A \in \mathrm{Ob} \left(\mathsf{A}\right)$ is called
	$F$-{\em acyclic} iff $L_iF(A) = 0$ for all $i > 0$.
\end{defn}

\begin{defn}[F-acyclic resolution]
	Let $\mathsf{A}$ be an abelian category with enough projectives,
	$F\colon\mathsf{A} \to \mathsf{B}$ be a right exact functor and $A \in \mathrm{Ob} \left(\mathsf{A}\right)$.
	A left resolution $Q_{\bullet} \to A \to 0$ of $A$ is called an
	$F$-{\em acyclic resolution} iff $Q_i$ are $F$-acyclic
	for all $i \geq 0$.	
\end{defn}

\begin{rem}[]
	Any projective object $A \in \mathrm{Ob} \left(\mathsf{A}\right)$ is $F$-acyclic
	for any right exact functor $F$.
\end{rem}

\begin{thm}[]
	Let $\mathsf{A}$ and $\mathsf{B}$ be abelian categories.
	Assume that $\mathsf{A}$ has enough projectives and $F\colon\mathsf{A} \to \mathsf{B}$
	is a right exact functor.
	Then the left derived functors $\left\{ L_i F \right\}_{i \geq 0}$
	form a homological $\partial$-functor.
\end{thm}

\begin{defn}[Morphism of (co)homological $\partial$-functor]
	Let $S,T\colon\mathsf{A} \to \mathsf{B}$ be cohomological $\partial$-functors.
	A morphism $S \to T$ is a sequence of natural transformations
	$\eta^n\colon S^n \to T^n$ commuting with $\partial$.
	More explicitly, given any short exact sequence $0 \to A \to B \to C \to 0$ in $\mathsf{A}$,
	the following diagram commutes
	\begin{equation}
	\begin{tikzcd}
		S^n(C) \arrow[r, "\partial^n_S", rightarrow] \arrow[d, "\eta^n_C"', rightarrow] &
		S^{n+1}(A) \arrow[d, "\eta^{n+1}_A", rightarrow] \\
		T^n(C) \arrow[r, "\partial^n_T"', rightarrow] &
		T^{n+1}(A)
	\end{tikzcd}
	.\end{equation} 
	(Clearly for homological $\partial$-functors one only has to dualize).
\end{defn}

\begin{defn}[Universal cohomological $\partial$-functor]
	A cohomological $\partial$-functor $T$ is called {\em universal} iff
	given any cohomological $\partial$-functor $S$, and any natural
	transformation $\eta^0\colon T^0 \to S^0$, then
	$\exists\, !\, \left\{ \eta^n\colon T^n \to S^n \right\}_{n \geq 0}$ a natural transformation
	of $\partial$-functors extending $\eta^0$.
	(Analogously of homological $\partial$-functors).
\end{defn}

\begin{lem}
	Consider an exact functor $F\colon\mathsf{A} \to \mathsf{B}$.
	Setting $T^0 \coloneqq F$ and $T^n \coloneqq 0$ for all $n > 0$
	defines a universal cohomological $\partial$-functor $\left\{ T^n \right\}_{n \in \N}$.
	(Analogously setting $T_0 \coloneqq F$ and $T_n \coloneqq 0$, for a universal homological
	$\partial$-functor).
\end{lem}  

\begin{thm}[]
	Let $\mathsf{A}$ and $\mathsf{B}$ be abelian categories.
	Assume that $\mathsf{A}$ has enough projectives and $F\colon\mathsf{A} \to \mathsf{B}$
	is a right exact functor.
	Then the left derived functors $\left\{ L_i F \right\}_{i \geq 0}$
	form a universal homological $\partial$-functor.
\end{thm}

\begin{lem}
	Let $\mathsf{A}$ and $\mathsf{B}$ be abelian categories.
	Assume that $\mathsf{A}$ has enough projectives and $F\colon\mathsf{A} \to \mathsf{B}$
	is a right exact functor.
	Consider $G\colon\mathsf{B} \to \mathsf{C}$ an exact functor, then:
	\begin{equation}
		L_i \left( G \circ F \right) \simeq_{\text{nat.}} G \circ L_i F
		\qquad \,\forall\, i \geq 0
	.\end{equation} 
\end{lem} 

\begin{lem}
	Consider $G\colon\mathsf{A} \to \mathsf{B}$ an exact functor between abelian categories.
	Consider $X_{\bullet} \in \mathrm{Ch}(\mathsf{A})$, then for every $i \in \Z$
	\begin{equation}
		G \left( H_{i}\left( X_{\bullet} \right) \right) =
		H_{i}\left( G(X_{\bullet}) \right)
	.\end{equation} 
\end{lem} 

\begin{lem}[Dimension shifting]
	Let $\mathsf{A}$ and $\mathsf{B}$ be abelian categories.
	Assume that $\mathsf{A}$ has enough projectives and $F\colon\mathsf{A} \to \mathsf{B}$
	is a right exact functor.
	Consider a short exact sequence $0 \to K \to Q \to A \to 0$ in $\mathsf{A}$,
	with $Q$ an $F$-acyclic object (e.g. if $Q$ is projective).
	Then
	\begin{enumerate}
		\item $L_1 F(A) = \ker \left( F(K) \to F(Q) \right)$,
		\item $L_iF(A) \simeq L_{i-1} F(K)$ for all $i \geq 2$.
	\end{enumerate}
\end{lem} 

\begin{rem}[]
	Let $\mathsf{A}$ be an abelian category. We define
	\begin{equation}
	\mathrm{Ch}_{\geq 0}(\mathsf{A}) \coloneqq
	\left\{ \left( X_{\bullet}, d^{X} \right) \in \mathrm{Ch}(\mathsf{A}) \ \middle|\ 
	X_n = 0 \,\forall\, n < 0\right\}
	.\end{equation} 
	By the fundamental theorem on homology, we know that $\left\{ H_n \right\}_{n \in \Z}$,
	for $H_n\colon\mathrm{Ch}_{\geq 0}(\mathsf{A}) \to \mathsf{A}$,
	is a homological $\partial$-functor
\end{rem}

\begin{lem}
	Moreover one can prove that $\left\{ H_n \right\}_{n \in \Z}$ is a {\em universal}
	homological $\partial$-functor.
\end{lem} 

\begin{lem}
	Let $\mathsf{A}$ and $\mathsf{B}$ be abelian categories, s.t. $\mathsf{A}$ has enough projectives.
	Consider $F\colon\mathsf{A} \to \mathsf{B}$ an exact functor, then
	\begin{equation}
		L_i F(A) = 0 \qquad
		\,\forall\, A \in \mathrm{Ob} \left(\mathsf{A}\right), \,\forall\, i > 0
	.\end{equation} 
	Moreover we also know that $L_0 F \simeq F$.
\end{lem} 

\subsection{Right derived functors}
\begin{rem}[Standard assumption]\label{rem:RDFStdAssumption}
	In the following section we will assume the following:
	$\mathsf{A}$ and $\mathsf{B}$ are abelian categories.
	Moreover we assume that $\mathsf{A}$ has enough injectives, and
	$F\colon\mathsf{A} \to \mathsf{B}$ is a left exact functor.
\end{rem}

\begin{defn}[Right derived funtors]
	Let $\mathsf{A}$, $\mathsf{B}$ and $F$ be as in remark \ref{rem:RDFStdAssumption}.
	We define the right derived functors
	$R^i F\colon\mathsf{A} \to \mathsf{B}$ s.t.
	$R^iF(A) \coloneqq H^i \left( F(I^\bullet) \right)$,
	for $0 \to A \to I^\bullet$ an injective coresolution of $A$, and $i \geq 0$.
\end{defn}

\begin{rem}[Important!]
	Recall that $A \in \mathrm{Ob} \left(\mathsf{A}\right)$ is injective
	iff $A$ is projective in $\mathsf{A}^{op}$.
	Then, given an injective coresolution $0 \to A \to I^\bullet$ for $A$,
	then $I_{\bullet} \to A \to 0$ becomes a projective resolution in $\mathsf{A}^{op}$.

	Then, given $F\colon\mathsf{A} \to \mathsf{B}$ a left exact functor, 
	we define $F^{op}\colon\mathsf{A}^{op} \to \mathsf{B}^{op}$ a covariant functor.
	Clearly $\mathsf{A}^{op}$ has enough projectives, moreover $F^{op}$ is right exact
	(in fact $F$ is left exact iff $F^{op}$ is right exact).
	Then we can define the left derived functor $L_iF^{op}(A)$.
	Finally we have the equality
	\begin{equation}
		\left( L_i F^{op} \right)^{op} (A) = R^i F(A)
	.\end{equation} 
	In particular $\left\{ R^iF \right\}_{i \geq 0}$ form a universal cohomological $\partial$-functor.
	Moreover, dualizing the previous results, we obtain that:
	\begin{itemize}
		\item $R^0F \simeq F$,
		\item Given a short exact sequence in $\mathsf{A}$
			\begin{equation}
			0 \to A \to B \to C \to 0
			\end{equation} 
			there is an associated long exact sequence
			\begin{equation}
				0 \to F(A) \to F(B) \to F(C) \xrightarrow{\partial^0} 
				R^1F(A) \to R^1F(B) \to R^iF(C) \xrightarrow{\partial^1} 
				\ldots
			.\end{equation} 
	\end{itemize}
\end{rem}

\begin{defn}[$F$-acyclic objects]
	Let $\mathsf{A}$, $\mathsf{B}$ and $F$ be as in remark \ref{rem:RDFStdAssumption}.
	An object $A \in \mathrm{Ob} \left(\mathsf{A}\right)$ is
	$F$-{\em acyclic} iff
	\begin{equation}
		R^iF(A) = 0 \qquad \,\forall\, i > 0
	.\end{equation} 
\end{defn}

\begin{rem}[]
	Any injective object $Q \in \mathrm{Ob} \left(\mathsf{A}\right)$ is 
	$F$-acyclic for any left-exact functor $F$.
\end{rem}

\begin{lem}
	Consider $\mathsf{A}$ an abelian category with enough injectives, and
	$F\colon\mathsf{A} \to \mathsf{B}$ an exact functor, then
	$R^iF = 0$ for all $i > 0$.
\end{lem} 

\begin{ex}
	Let $\mathsf{A}$ an abelian category with enough injectives.
	Fix $M \in \mathrm{Ob} \left(\mathsf{A}\right)$, then consider
	\begin{equation}
	H_M \coloneqq \mathrm{Hom}_{\mathsf{A}} \left( M, - \right)\colon\mathsf{A} \to \mathsf{Ab}
	\end{equation} 
	the covariant Hom functor.
	We know that $H_M$ is left exact.
	Then we can define the right derived functors of $H_M$.
	In particular they are defined as follows.
	For an object $A \in \mathrm{Ob} \left(\mathsf{A}\right)$,
	take an injective coresolution of $A$, $0 \to A \to I^\bullet$, then
	\begin{equation}
		R^iH_M(A) = H^i \left( \mathrm{Hom}_{\mathsf{A}} \left( M, I^\bullet \right) \right)
	.\end{equation} 
	Moreover one introduces the notation (which is especially useful in the category of modules)
	\begin{equation}
		\mathrm{Ext}_{\mathsf{A}}^i \left( M,A \right) \coloneqq R^i H_M (A)
	.\end{equation} 
\end{ex} 

\begin{prop}
	Let $\mathsf{A}$ be abelian with enough injectives
	(e.g. $\mathsf{A} = \mathsf{Mod}\text{-}R$).
	Fix $A \in \mathrm{Ob} \left(\mathsf{A}\right)$, then the following are equivalent:
	\begin{enumerate}
		\item $A$ is injective,
			i.e. $\mathrm{Hom}_{\mathsf{A}} \left( -, A \right)$ is exact;
		\item $\mathrm{Ext}^i_{\mathsf{A}}(M,A) = 0$ for all $M \in \mathrm{Ob} \left(\mathsf{A}\right)$
			and for all $i \geq 0$;
		\item $\mathrm{Ext}^1_{\mathsf{A}}(M,A) = 0$ for all $M \in \mathrm{Ob} \left(\mathsf{A}\right)$.
	\end{enumerate}
\end{prop} 
We can dualize the above proposition and obtain
\begin{prop}
	Let $\mathsf{A}$ be abelian with enough injectives
	(e.g. $\mathsf{A} = \mathsf{Mod}\text{-}R$).
	Fix $M \in \mathrm{Ob} \left(\mathsf{A}\right)$, then the following are equivalent:
	\begin{enumerate}
		\item $M$ is projective,
			i.e. $\mathrm{Hom}_{\mathsf{A}} \left( M, - \right)$ is exact;
		\item $\mathrm{Ext}^i_{\mathsf{A}}(M,A) = 0$ for all $A \in \mathrm{Ob} \left(\mathsf{A}\right)$
			and for all $i \geq 0$;
		\item $\mathrm{Ext}^1_{\mathsf{A}}(M,A) = 0$ for all $A \in \mathrm{Ob} \left(\mathsf{A}\right)$.
	\end{enumerate}
\end{prop} 

\subsection{Derived functors of contravariant functors}
\begin{rem}[Right derived functors of a contravariant functor]
	Let $\mathsf{A}$ and $\mathsf{B}$ be abelian categories and $F\colon\mathsf{A} \to \mathsf{B}$
	a contravariant left-exact functor (e.g. $F = H^M \coloneqq \mathrm{Hom}_{\mathsf{A}} \left( -, M \right)$
	for $M \in \mathrm{Ob} \left(\mathsf{A}\right)$).
	Then $F\colon\mathsf{A}^{op} \to \mathsf{B}$ is covariant and, still, left-exact.
	If $\mathsf{A}^{op}$ has enough injectives (iff $\mathsf{A}$ has enough projectives)
	we can define the right derived functors $R^iF\colon\mathsf{A}^{op} \to \mathsf{B}$, for $i \geq 0$.
	In particular this is computed by taking a projective resolution of $A \in \mathrm{Ob} \left(\mathsf{A}\right)$:
	$P_{\bullet} \to A \to 0$, which gives an injective coresolution
	$0 \to A \to P^\bullet$ of $A$ in $\mathsf{A}^{op}$.
	Then we define
	\begin{equation}
		R^iF(A) \coloneqq H^i(F(P_\bullet))
	.\end{equation} 
	Notice that given a chain complex $P_{\bullet}$, then $F(P_{\bullet})$ is a 
	cochain complex.
\end{rem}

\begin{rem}[]
	Let $\mathsf{A}$ be an abelian category with enough injectives and projectives (e.g. for
	$\mathsf{A} = \mathsf{Mod}\text{-}R$).
	Then, fixed $M \in \mathrm{Ob} \left(\mathsf{A}\right)$, $H_M \coloneqq \mathrm{Hom}_{\mathsf{A}} \left( M, - \right)$
	is a covariant, left-exact, functor.
	In particular it admits right-derived funtors
	\begin{equation}
		R^iH_M(A) = \mathrm{Ext}^i_{\mathsf{A}} \left( M, A \right) =
		H^i \left( H_M (I^\bullet) \right)
	,\end{equation} 
	for an injective coresolution $0 \to A \to I^\bullet$ of $A$.
	Moreover we can consider $H^A \coloneqq \mathrm{Hom}_{\mathsf{A}} \left( -, A \right)$, which
	is a contravariant, left-exact, functor.
	Also this admits right-derived functors
	\begin{equation}
		R^iH^A(M) = H^i \left( H^A(P_{\bullet}) \right)
	,\end{equation} 
	for $P_{\bullet} \to M \to 0$ a projective resolution of $M$.
\end{rem}

\begin{thm}[Balancing of Ext]
	\begin{equation}
		R^i H_M(A) = \mathrm{Ext}^i_{\mathsf{A}}(M,A)\simeq R^i H^A (M)
	.\end{equation} 
\end{thm}

\begin{rem}[Consequence]
	This theorem means that $\mathrm{Ext}^i_{\mathsf{A}}(M,A)$
	can be computed in two equivalent ways:
	We can consider $0 \to A \to I^\bullet$ an injective coresolution of $A$,
	or $P_{\bullet} \to M \to 0$ a projective resolution of $M$
	and
	\begin{equation}
		H^i \left( \mathrm{Hom}_{\mathsf{A}} \left( M, I^\bullet \right) \right) \simeq
		\mathrm{Ext}^i_{\mathsf{A}} (M,A) \simeq
		H^i \left( \mathrm{Hom}_{\mathsf{A}} \left( P_{\bullet}, A \right) \right)
	.\end{equation} 
\end{rem}

\begin{rem}[]
	Let $\mathsf{A}$ be an abelian category with arbitrary coproducts.
	Consider $\left( X_i^{\bullet}, d_{X_i} \right)_{i \in I}$ a family
	of cochain complexes in $\mathrm{Ch}(\mathsf{A})$.
	Then the cochain complex $( \widetilde{X}^{\bullet}, d_{\widetilde{X}} )$, with
	objects $( \widetilde{X}^\bullet )^n \coloneqq \coprod_{i \in I} X_i^n$
	and differentials $d^n_{\widetilde{X}} \coloneqq \coprod_{i \in I} d^n_{X_i}$,
	is a coproduct of $X_i^\bullet$ in $\mathrm{Ch}(\mathsf{A})$.
	Then one checks that
	\begin{equation}
	H^{n}( \widetilde{X} ) = 
	\coprod_{i \in I} H^{n}\left( X_i \right)
	\qquad \,\forall\, n \in \Z
	.\end{equation} 
	Analogously, if $\mathsf{A}$ admits arbitrary products,
	consider $( X_i^{\bullet}, d_{X_i} )_{i \in I}$ a family
	of cochain complexes in $\mathrm{Ch}(\mathsf{A})$.
	Then the cochain complex $( \widetilde{X}^{\bullet}, d_{\widetilde{X}} )$, with
	objects $( \widetilde{X}^\bullet )^n \coloneqq \prod_{i \in I} X_i^n$
	and differentials $d^n_{\widetilde{X}} \coloneqq \prod_{i \in I} d^n_{X_i}$,
	is a product of $X_i^\bullet$ in $\mathrm{Ch}(\mathsf{A})$.
	Then one checks that
	\begin{equation}
	H^{n}( \widetilde{X} ) = 
	\prod_{i \in I} H^{n}\left( X_i \right)
	\qquad \,\forall\, n \in \Z
	.\end{equation} 
\end{rem}

\begin{lem}
	Let $\left(L, R\right)$ be an adjoint pair of functors
	$L\colon\mathsf{A} \to \mathsf{B}$ and $R\colon\mathsf{B} \to \mathsf{A}$,
	between additive categories.
	Then $\left(L, R\right)$, thanks to naturality of the adjunction,
	induces an adjoint pair of morphisms
	\begin{equation}
	L\colon\mathrm{Ch}(\mathsf{A}) \to \mathrm{Ch}(\mathsf{B})
	\qquad \text{ and } \qquad
	R\colon\mathrm{Ch}(\mathsf{B}) \to \mathrm{Ch}(\mathsf{A})
	.\end{equation} 
\end{lem} 

\begin{prop}
	Let $\mathsf{A}$ and $\mathsf{B}$ be abelian categories.
	Consider an adjoint pair of functors $\left(F, G\right)$,
	for $F\colon\mathsf{A} \to \mathsf{B}$ and $G\colon\mathsf{B} \to \mathsf{A}$.
	Assume that $\mathsf{A}$ has enough projectives and arbitrary coproducts,
	whereas $\mathsf{B}$ has enough injectives and arbitrary products.
	Let $\left\{ A_\alpha \right\}_{\alpha \in \mathcal{A}} \subset \mathrm{Ob} \left(\mathsf{A}\right)$
	be a family of objects of $\mathsf{A}$
	and $\left\{ B_\beta \right\}_{\beta \in \mathcal{B}} \subset \mathrm{Ob} \left(\mathsf{B}\right)$
	be a family of objects of $\mathsf{B}$.
	Then
	\begin{equation}
		L_iF \bigg( \coprod_{\alpha \in \mathcal{A}} A_\alpha \bigg) \simeq
		\coprod_{\alpha \in \mathcal{A}} L_i F(A_\alpha)
	\end{equation} 
	and
	\begin{equation}
		R^iF \bigg( \prod_{\beta \in \mathcal{B}} B_\beta \bigg) \simeq
		\prod_{\beta \in \mathcal{B}} R^iG \left( B_\beta \right)
	.\end{equation} 
\end{prop} 

\subsection{Derived functors of tensor product functors}
Recall that, for a ring $R$, and $M_R \in \mathsf{Mod}\text{-}R$,
then, as seen in proposition \ref{prop:tnshomadj},
\begin{equation}
M_R \otimes_R - \colon R\text{-}\mathsf{Mod} \to \mathsf{Ab}
\qquad \text{ and } \qquad
\mathrm{Hom}_{\Z}\left( M, - \right)\colon R\text{-}\mathsf{Mod} \to \mathsf{Ab}
\end{equation} 
constitute an adjoint pair $\left(M_R \otimes_R -, \mathrm{Hom}_{ \Z}\left( M, - \right)\right)$.

As a consequence $T_M \coloneqq M_R \otimes_R -$ is a left adjoint, hence it is
right exact and preserves arbitrary coproducts and $\varinjlim$.

\begin{defn}[Flat module]
	Consider $M_R \in \mathsf{Mod}\text{-}R$.
	We say that $M_R$ is {\em flat} iff $T_M \coloneqq M_R \otimes_R -$ is exact
	(i.e. iff $T_M$ is also left exact).
	Simmetrically ${}_RN \in R\text{-}\mathsf{Mod}$ is flat iff $- \otimes_R N$ is exact.
\end{defn}

\begin{prop}
	Let $M_R \in \mathsf{Mod}\text{-}R$.
	The following are equivalent:
	\begin{enumerate}
		\item $M_R$ is flat;
		\item for every mono $0 \to {}_RA \xrightarrow{\mu} {}_RB$ of left $R$-modules, then
			$\mathrm{id}_{ M } \otimes \mu\colon M \otimes A \to M \otimes B$
			is mono (in $\mathsf{Ab}$);
		\item $L_i \left( M \otimes_R - \right) (N) = 0$ for all $i \geq 1$ and for all $N \in R\text{-}\mathsf{Mod}$;
		\item $L_1 \left( M \otimes_R - \right) (N) = 0$ for all $N \in R\text{-}\mathsf{Mod}$.
	\end{enumerate}
\end{prop} 
Dually:
\begin{prop}
	Let ${}_RN \in R \text{-}\mathsf{Mod}$.
	The following are equivalent:
	\begin{enumerate}
		\item ${}_RN$ is flat;
		\item for every mono $0 \to A_R \xrightarrow{\mu} B_R$ of right $R$-modules, then
			$\mu \otimes \mathrm{id}_{ N } \colon A \otimes N \to B \otimes N$
			is mono (in $\mathsf{Ab}$);
		\item $L_i \left( - \otimes_R N\right) (M) = 0$ for all $i \geq 1$ and for all $M \in \mathsf{Mod}\text{-}R$;
		\item $L_1 \left( - \otimes_R N \right) (M) = 0$ for all $M \in \mathsf{Mod}\text{-}R$.
	\end{enumerate}
\end{prop} 

\begin{rem}[]
	Combining the above propositions we obtain that
	$M_R$ is flat iff $M_R$ is $\left( - \otimes_R N \right)$-acyclic for all ${}_R N$ left $R$-modules.
	Analogously ${}_RN$ is flat iff ${}_RN$ is $\left( M \otimes_R - \right)$-acyclic for all $M_R$ right $R$-modules.
\end{rem}

\begin{defn}[Notation]
	Called $T_M \coloneqq M \otimes_R -$, then we define
	\begin{equation}
		\mathrm{Tor}^R_i (M,N) \coloneqq L_i \left( M \otimes_R - \right) (N)
	.\end{equation} 
\end{defn}

\begin{thm}[Balancing of Tor]
	\begin{equation}
		\mathrm{Tor}_i^R(M,N) = L_i \left( M \otimes_R - \right)(N) =
		L_i \left( - \otimes_R N \right)(M)
	\end{equation}
	for all $i \geq 0$, all $M \in \mathsf{Mod}\text{-}R$ and all $N \in R\text{-}\mathsf{Mod}$.
\end{thm}

\begin{rem}[Consequence]
	The above theorem means that $\mathrm{Tor}_i^R(M,N)$ can be computed in two equivalent ways:
	Consider $P_{\bullet} \to {}_RN \to 0$ a projective resolution of ${}_RN$
	or $Q_{\bullet} \to M_R \to 0$ a projective resolution of $M_R$, then
	\begin{equation}
		H_i \left( M \otimes_R P_{\bullet} \right) \simeq
		\mathrm{Tor}_i^R (M,N) \simeq
		H_i \left( Q_{\bullet} \otimes_R N \right)
	.\end{equation} 
\end{rem}

\begin{prop}
	Let $\left\{ M_i \right\}_{i \in I}$ be a family of right $R$-modules.
	Then
	\begin{enumerate}
		\item $\bigoplus_{i \in I} M_i$ is flat iff $M_i$ is flat for all $i \in I$,
		\item If $\left\{ M_i \right\}_{i \in I}$ is a direct system of flat $R$-modules, 
			then the filtered direct limit $\varinjlim_{i \in I} M_i$ is flat.
	\end{enumerate}
\end{prop} 

\begin{rem}[]
	For every ${}_RN$ $T_N \coloneqq - \otimes_R N$ is a left adjoint.
	This means that $T_N$ preserves colimts, 
	in particular, for every direct system $\left\{ M_i, F_{ij} \right\}_{i \leq j}$, then
	\begin{equation}
		\big( \varinjlim_{i \in I} M_i \big) \otimes_R N \simeq
		\varinjlim_{i \in I} \left( M_i \otimes_R N \right)
	.\end{equation} 
\end{rem}

\begin{rem}[]
	\begin{equation}
	\varinjlim_{i \in I} M_i \text{ flat } \centernot\implies M_i \text{ flat}
	.\end{equation} 
	In fact every module is the filtered direct limit of its finitely generated
	submodules.
	Though it is not true that, given $M$ flat, then its finitely generated submodules are flat.

	As an example any ring $R$ is a free, hence flat, $R$-module.
	Though this doesn't imply that its (finitely generated) ideals are flat.
	For instance, take $R \coloneqq \K[x,y]$, for a field $\K$.
	Consider $\mathfrak{m} \coloneqq (x,y)$ the maximal ideal generated by $x$ and $y$.
	Consider the mono $0 \to \mathfrak{m} \xrightarrow{\epsilon} R$, and
	\begin{align}
		\mathrm{id}_{ \mathfrak{m} } \otimes \epsilon\colon \mathfrak{m} \otimes_R \mathfrak{m} &\to 
		\mathfrak{m} \otimes_R R \simeq \mathfrak{m} \\
		a \otimes b &\mapsto a \cdot b
	.\end{align} 
	In fact $0 \neq x \otimes y - y \otimes x \mapsto xy - yx = 0$,
	then $\mathfrak{m}$ is finitely generated, but not flat.
\end{rem}

\begin{prop}
	Let $\mathsf{A}$ and $\mathsf{B}$ be abelian categories with enough projectives.
	Consider $F\colon\mathsf{A} \to \mathsf{B}$ a right exact functor.
	Then $L_iF$ can be computed using $F$-acyclic resolutions, instead of projective resolutions.
	More explicitly, given
	$Q_{\bullet} \to A \to 0$ a resolution of $A$ s.t. $Q_i$ is $F$-acyclic for each $i$, then
	\begin{equation}
		L_iF(A) \simeq H_i \left( F(Q_{\bullet}) \right)
	.\end{equation} 
\end{prop} 

\begin{rem}[]
	In particular $\mathrm{Tor}_i^R(-,-)$ can be computed using flat resolutions.
\end{rem}

\begin{rem}[Flat modules]
	Clearly any projective $P_R$ right $R$-module is flat, since it is $- \otimes_R N$-acyclic for
	all ${}_RN$ modules.
	Analogously a projective left $R$-module ${}_RP$ is flat.
	In particular any free module is flat.
\end{rem}

\begin{ex}[Flat modules]
	Recall the definition of localization:
	given a commutative ring $R$ and a multiplicatively closed subset $S \subset R$, 
	i.e. s.t. $0 \notin S, 1 \in S$ and $st \in S$ for all
	$s,t \in S$, 
	we can consider the localization
	\begin{equation}
	R_S = R \left[ S^{-1} \right] \coloneqq
	\left\{ \frac{r}{s} \ \middle|\ r \in R, \, s \in S \text{ and } \frac{r}{s} = \frac{r'}{s'}
	\iff \exists\, t \in S \text{ s.t. } t \left( rs' - r's \right) = 0 \right\}
	.\end{equation} 
	Notice, moreover, that given any module $M$, then
	\begin{equation}
	M \otimes_R R_S =: M_S =
	\left\{ \frac{x}{s} \ \middle|\ x \in M, \, s \in S \right\}
	\end{equation} 
	and $x/s = x'/s'$ iff there exists $t \in S \text{ s.t. } t \left( xs' - x's \right) = 0$.
	In particular $x/1 = 0$ iff $\exists\, t \in S$ s.t. $tx = 0$.
	Moreover any element $\zeta \in M_R \otimes_R R_S$ can be represented as
	$y \otimes 1/s$, for $s \in S$ and $y \in M$.

	Let's prove that $R_S$ is a flat $R$-module.
	Consider a mono $\mu\colon A_R \rightarrowtail B_R$, we have to prove that
	\begin{equation*}
	\begin{tikzcd}[row sep = 0ex
		,/tikz/column 1/.append style={anchor=base east}
		,/tikz/column 2/.append style={anchor=base west}]
		\mu \otimes 1_{R_S}\colon A_R \otimes_R R_S \arrow[r, "", rightarrow] &
		B_R \otimes_R R_S
	\end{tikzcd}
	\end{equation*} 
	is still mono.
	Let's consider $x/t \coloneqq x \otimes 1/t \in A_R \otimes_R R_S$, then 
	$\mu(x/t) \coloneqq \mu(x)/t$.
	Assume $\mu(x)/t = 0$, i.e. there exists $s \in S$ s.t. $s \mu(x) = 0$,
	which means $\mu(sx) = 0$, hence $sx = 0$, since $\mu$ is mono.
	But this means that $x/t = 0$.

	As a consequence, for any $M_R$ flat $R$-module, we obtain that its localization
	at $S$, i.e. $M_S \coloneqq M_R \otimes_R R_S$, is still flat.
	This is because, for all $N_R \in \mathsf{Mod}\text{-}R$, associativity of
	tensor product implies
	\begin{equation*}
	N \otimes_R M_S \coloneqq
	N \otimes_R \left( M \otimes_R R_S \right) \simeq
	\left( N \otimes_R M \right) \otimes_R R_S
	.\end{equation*}
\end{ex}

\begin{thm}[Lazard]
	A module is flat iff
	it is a filtered direct limit of projective modules,
	or a direct limit of finitely generated free modules.
	(It can be specialized to left or right modules, then every module in the
	statement has to be either left or right, accordingly).
\end{thm}

\begin{lem}
	Let $\mathsf{C}$ and $\mathsf{D}$ be abelian categories, and $L\colon \mathsf{C} \to \mathsf{D}$ and
	$R\colon \mathsf{D} \to \mathsf{C}$ be an adjoint pair $\left(L, R\right)$.
	Assume that $L$ is an exact functor.
	Then, if $I$ is an injective object of $\mathsf{D}$,
	then $R(I)$ is injective in $\mathsf{C}$.
	Dually, if $R$ is exact, and $P$ is a projective object of $\mathsf{C}$, then
	$L(P)$ is a projective object of $\mathsf{D}$.
\end{lem} 

\begin{prop}
	Let ${}_SF_R$ be an $S$-$R$-bimodule and ${}_SE$ be an injective left $S$-module, then
	\begin{itemize}
		\item If $F_R$ is flat, then $\mathrm{Hom}_{S}\left( {}_SF_R, {}_SE \right)$ is 
			an injective left $R$-module.
		\item Conversely, if ${}_SE$ is an injective cogenerator of $S$-Mod
			and $\mathrm{Hom}_{S}\left( {}_SF_R, {}_SE \right)$
			is an injective left $R$-module, then $F_R$ is flat.
	\end{itemize}
\end{prop} 

\begin{cor}
	Since $\mathbb{Q}/\Z$ is an injective cogenerator in the category $\mathsf{Ab} = \mathsf{Mod}\text{-}\Z$:
	it is the direct sum of the injective envelopes of the simple modules $\mathbb{Z}/p\mathbb{Z}$:
	\begin{equation}
		\mathbb{Q}/\Z = \bigoplus_{p \in P} E \left( \mathbb{Z}/p\mathbb{Z} \right)
	.\end{equation} 
	The module ${}_{\Z}F_R$ is flat iff
	$\mathrm{Hom}_{\Z}\left( F_R, \mathbb{Q}/\Z \right)$ is an injective left
	$R$-module.
	We introduce the following notation
	\begin{equation}
	F_R^*\coloneqq \mathrm{Hom}_{\Z}\left( F_R, \mathbb{Q}/\Z \right)
	\end{equation} 
	and call this important module, the {\em character module} of $F_R$.
\end{cor} 

\begin{thm}[Dimension shifting for right derived functors]\leavevmode\vspace{-.2\baselineskip}
	\begin{enumerate}
		\item Let $F\colon \mathsf{A} \to \mathsf{B}$ be a covariant left-exact functor
			between abelian categories, with $\mathsf{A}$ having enough injectives.
			Let $Q$ be an $F$-acyclic object (e.g. $Q$ injective) and
			\begin{equation}
			0 \to K \to Q \to A \to 0
			\end{equation} 
			be a short exact sequence.
			Then, for all $i \geq 1$,
			\begin{equation}
				R^iF(A) \simeq R^{i+1}F(K)
			.\end{equation} 
		\item Let $F\colon \mathsf{A} \to \mathsf{B}$ be a contravariant left-exact functor
			between abelian categories, with $\mathsf{A}$ having enough projectives.
			Let $Q$ be an $F$-acyclic object (e.g. $Q$ projective) and
			\begin{equation}
			0 \to K \to Q \to A \to 0
			\end{equation} 
			be a short exact sequence.
			Then, for all $i \geq 1$,
			\begin{equation}
				R^iF(K) \simeq R^{i+1}F(A)
			.\end{equation} 
	\end{enumerate}
\end{thm}
\begin{proof}\leavevmode\vspace{-.2\baselineskip}
	\begin{enumerate}
		\item Consider the long exact sequence
			\begin{equation}
				R^1F(K) \to R^1F(Q) = 0 \to R^1F(A) \to
				R^2F(K) \to R^2F(Q) = 0 \to \ldots
			.\end{equation} 
			Since $R^iF(Q) = 0$ for all $i$, we have our thesis.
		\item Consider the long exact sequence
			\begin{equation}
				R^1F(A) \to R^1F(Q) = 0 \to R^1F(K) \to
				R^2F(A) \to R^2F(Q) = 0 \to \ldots
			.\end{equation} 
			Since $R^iF(Q) = 0$ for all $i$, we have our thesis.\qedhere
	\end{enumerate}
\end{proof}

\begin{rem}[]
	Assume that $\mathsf{A}$ is an abelian category with enough projectives.
	Consider $M \in \mathrm{Ob} \left(\mathsf{A}\right)$ such that,
	for all $N \in \mathrm{Ob} \left(\mathsf{A}\right)$,
	\begin{equation}
		\mathrm{Ext}^{n+i}_{\mathsf{A}}(M,N) = 0 \qquad \,\forall\, i \geq 0
	.\end{equation} 
	By dimension shifting $\mathrm{Ext}^1_{\mathsf{A}}(K_n,N) = \mathrm{Ext}^2_{\mathsf{A}}(K_{n-1},N) =
	\ldots = \mathrm{Ext}^{n+1}_{\mathsf{A}}(M,N)$ for all $N \in \mathrm{Ob} \left(\mathsf{A}\right)$.
	And, moreover, for $K_n = \Omega_n(M)$, the $n$-th syzygy of $M$, we have
	\begin{equation}
		\mathrm{Ext}^{n+1}_{\mathsf{A}}(M,N) \simeq
		\mathrm{Ext}^1_{\mathsf{A}}(K_n,N)
	.\end{equation} 
	In particular the above condition holds iff $K_n$ is projective.

	Analogously, if $\mathsf{A}$ has enough injectives we obtain:
	By dimension shifting $\mathrm{Ext}^1_{\mathsf{A}}(M,C_n) = \mathrm{Ext}^2_{\mathsf{A}}(M,C_{n-1}) =
	\ldots = \mathrm{Ext}^{n+1}_{\mathsf{A}}(M,N)$ for all $N \in \mathrm{Ob} \left(\mathsf{A}\right)$.
	And, moreover, for $C_n = \Omega^n(N)$, the $n$-th cosyzygy of $N$, we have
	\begin{equation}
		\mathrm{Ext}^{n+1}_{\mathsf{A}}(M,N) \simeq
		\mathrm{Ext}^1_{\mathsf{A}}(M,C_n)
	.\end{equation} 
	In particular $\mathrm{Ext}^{n+i}_{\mathsf{A}}(M,N) = 0 \,\forall\, i \geq 0$
	iff $C_n$ is injective.
\end{rem}

\begin{lem}[Schanuel]
	Let $\mathsf{A}$ be an abelian category. Let $P,Q \in \mathrm{Ob} \left(\mathsf{A}\right)$
	be projective objects.
	Assume that the following are short exact sequences
	\begin{equation}
	0\to K\to P \to M \to 0
	\qquad \text{ and } \qquad
	0 \to H \to Q \to M \to 0
	.\end{equation} 
	Then $K \oplus Q \simeq H \oplus P$.
	In particular $K$ is projective iff $H$ is projective.
\end{lem} 
\begin{cor}
	Consider the two long exact sequences with $P_{i}, Q_{i}$ projective
	\begin{equation}
	0 \to K_n \to P_{n-1} \to P_{n-2} \to \ldots \to P_1 \to P_0 \to M \to 0
	\end{equation} 
	and
	\begin{equation}
	0 \to H_n \to Q_{n-1} \to Q_{n-2} \to \ldots \to Q_1 \to Q_0 \to M \to 0
	.\end{equation} 
	Then
	\begin{equation}
	K_n \oplus Q_{n-1} \oplus P_{n-2} \oplus \ldots \simeq
	K_n \oplus P_{n-1} \oplus Q_{n-2} \oplus \ldots
	.\end{equation} 
	In particular $K_n$ is projective iff $H_n$ is projective.
\end{cor} 

\begin{defn}[Projective dimension]
	Let $\mathsf{A}$ be an abelian category, with enough proectives.
	Consider $M \in \mathrm{Ob} \left(\mathsf{A}\right)$.
	We define the {\em projective dimension} of $M$, denoted by $\mathrm{p.d.}(M)$, 
	as the smallest integer $n \in \N$ s.t. there exist $P_i \in \mathrm{Ob} \left(\mathsf{A}\right)$
	projective and an exact sequence
	\begin{equation}
	0 \to P_n \to P_{n-1} \to \ldots \to P_1 \to P_0 \to M \to 0
	,\end{equation} 
	i.e. it is the minimal length of a projective resolution of $M$.
	Equivalently $n$ is the minimal index s.t. the $n$-th syzygy of $M$ is
	already a projective object.
	If no finite resolution exists, we define $\mathrm{p.d.}\, M = \infty$.
\end{defn}

\begin{rem}[]
	The projective dimension is well defined thanks to our corollary to Schanuel lemma.
\end{rem}

\begin{ex}[Infinite projective dimension]
	Let $R \coloneqq \mathbb{Z}/2\mathbb{Z}$ and $M \coloneqq \mathbb{Z}/2\mathbb{Z}$ as an $R$-module.
	Then
	\begin{equation}
	0 \to \mathbb{Z}/2\mathbb{Z} \to \mathbb{Z}/4\mathbb{Z} \xrightarrow{\pi} \mathbb{Z}/2\mathbb{Z} \to 0
	\end{equation} 
	is exact.
	This means that $\Omega_1(M) = M$, hence $\mathrm{p.d.}\, M = \infty$.

	Analogously for $R \coloneqq \mathbb{K}[x]/ \left( x^2 \right)$, for a field $\K$,
	$R$ is called the ring of dual numbers.
	Then $M_R \coloneqq (x) / (x^2)$ has infinite projective dimension.
\end{ex} 

\begin{prop}
	Let $\mathsf{A}$ be an abelian category with enough projectives.
	Let $M \in \mathrm{Ob} \left(\mathsf{A}\right)$, then the following are equivalent:
	\begin{enumerate}
		\item $\mathrm{p.d.}\, M \leq n$;
		\item $\mathrm{Ext}^{n+i}_{\mathsf{A}} (M,N) = 0$ for all $N \in \mathsf{A}$, and all $i \geq 1$;
		\item $\mathrm{Ext}^{n+1}_{\mathsf{A}} (M,N) = 0$ for all $N \in \mathsf{A}$.
	\end{enumerate}
\end{prop} 
\begin{cor}
	If $M \in \mathrm{Ob} \left(\mathsf{A}\right)$ (for $\mathsf{A}$ as above) 
	has $\mathrm{p.d.}\, M = n$, then
	$\mathrm{Ext}^{n+1}_{\mathsf{A}}(M,N) = 0$ for all $N \in \mathrm{Ob} \left(\mathsf{A}\right)$
	and $\exists\, N_0 \in \mathrm{Ob} \left(\mathsf{A}\right)$ s.t.
	$\mathrm{Ext}^n_{\mathsf{A}}(M,N_0) \neq 0$.
\end{cor} 
Let's now dualize everything for injectives
\begin{lem}[Schanuel for injectives]
	Let $\mathsf{A}$ an abelian category and $M \in \mathrm{Ob} \left(\mathsf{A}\right)$.
	Let $I, E \in \mathrm{Ob} \left(\mathsf{A}\right)$ be injective objects.
	Assume the following are short exact sequences
	\begin{equation}
	0 \to M \to I \to C \to 0
	\qquad \text{ and } \qquad
	0 \to M \to E \to D \to 0
	.\end{equation} 
	Then $C \oplus E \simeq I \oplus D$.
	In particular $D$ is injective iff $C$ is injective.
\end{lem} 

\begin{cor}
	Consider the two long exact sequences with $I^n, E^n$ injective
	\begin{equation}
	0 \to M \to I^0 \to I^1 \to \ldots \to I^{n-1} \to C \to 0
	\end{equation} 
	and
	\begin{equation}
	0 \to M \to E^0 \to E^1 \to \ldots \to E^{n-1} \to D \to 0
	.\end{equation} 
	Then 
	\begin{equation}
	C \oplus E^{n-1} \oplus I^{n-2} \oplus \ldots \simeq
	D \oplus I^{n-1} \oplus E^{n-2} \oplus \ldots
	.\end{equation} 
	In particular $C$ is injective iff $D$ is injective.
\end{cor} 

\begin{defn}[Injective dimension]
	Let $\mathsf{A}$ be an abelian category with enough injectives.
	Consider $M \in \mathrm{Ob} \left(\mathsf{A}\right)$.
	We define the {\em injective dimension} of $M$, 
	denoted by $\mathrm{i.d.}\, M$, as the smallest integer $n \in \N$ s.t.
	there exist $I^j \in \mathrm{Ob} \left(\mathsf{A}\right)$ injective and an exact sequence
	\begin{equation}
	0 \to M \to I^0 \to I^1 \to \ldots \to
	I^{n-1} \to I^n \to 0
	,\end{equation} 
	i.e. $n$ is the minimal length of an injective coresolution of $M$.
	Equivalently $n$ is the minimal index s.t. the $n$-th cosyzygy of $M$ is already
	an injective object.
	If no finite resolution exists, we define $\mathrm{i.d.}\, M = \infty$.
\end{defn}

\begin{ex}[Infinite injective dimension]
	Consider $R \coloneqq \mathbb{Z}/4\mathbb{Z}$.
	Prove that $R$ is self injective, i.e. $R$ is injective as an $R$-module
	(you can prove using Baer's criterion).
	Let $M_R \coloneqq \mathbb{Z}/2\mathbb{Z}$.
	Prove that $\mathrm{i.d.}\, M_R = \infty$.
\end{ex} 

\begin{prop}
	Let $\mathsf{A}$ be an abelian category with enough injectives.
	Let $M \in \mathrm{Ob} \left(\mathsf{A}\right)$, then the following are equivalent:
	\begin{enumerate}
		\item $\mathrm{i.d.}\, M \leq n$;
		\item $\mathrm{Ext}^{n+i}_{\mathsf{A}} (N,M) = 0$ for all $N \in \mathsf{A}$, and all $i \geq 1$;
		\item $\mathrm{Ext}^{n+1}_{\mathsf{A}} (N,M) = 0$ for all $N \in \mathsf{A}$.
	\end{enumerate}
\end{prop} 

\begin{defn}[Right global dimension of $R$]
	Let $R$ be a ring.
	We define the {\em right global dimension} of $R$, denoted by
	$\mathrm{r.gld}\, R$, as
	\begin{equation}
	\mathrm{r.gld}\, R \coloneqq \sup \left\{ \mathrm{p.d.}\, M_R \ \middle|\ M_R \in \mathsf{Mod}\text{-}R \right\}
	.\end{equation} 
\end{defn}

\begin{thm}[Global dimension]
	Consider the following numbers:
	\begin{align}
		(2) &\coloneqq \sup \left\{ \mathrm{i.d.}\, M_R \ \middle|\ M_R \in \mathsf{Mod}\text{-}R \right\}\\
		(3) &\coloneqq \sup \left\{ \mathrm{p.d.}\, R/I_R \ \middle|\ I_R \triangleleft R 
		\text{ is a right ideal}\, \right\}\\
		(4) &\coloneqq \sup \left\{n \in \N \ \middle|\ \mathrm{Ext}^n_{\mathsf{A}}(M,N) \neq 0
		\text{ for some } M_R, N_R \in \mathsf{Mod}\text{-}R \right\}
	.\end{align} 
	Let's call $(1) \coloneqq \mathrm{r.gld}\, R$.
	Then, if finite, $(1) = (2) = (3) = (4)$.
	Moreover, if any is infinite, also all the others are.
\end{thm}

\begin{lem}
	Let $R$ be a ring.
	Consider the short exact sequences
	\begin{equation}
	0 \to K \to F \to M \to 0
	\qquad \text{ and } \qquad
	0 \to H \to G \to M \to 0
	,\end{equation} 
	with $F, G$ flat.
	Then $K$ is flat iff $H$ is flat.
\end{lem} 

\begin{defn}[Flat (weak) dimension]
	Let $R$ be a ring and $M_R \in \mathsf{Mod}\text{-}R$.
	We define the {\em flat (or weak) dimension} of $M_R$,
	denoted by $\mathrm{f.d.}\, M_R$ or $\mathrm{w.d.}\, M_R$,
	as the minimum length of a flat resolution of $M$.
\end{defn}
\begin{rem}[]
	As before, by the above lemma, this is a good definition.
\end{rem}

\begin{prop}
	For $M_R \in \mathsf{Mod}\text{-}R$, the following are equivalent:
	\begin{enumerate}
		\item $\mathrm{w.d.}\, M \leq n$;
		\item $\mathrm{Tor}_{n+i}^R (N,M) = 0$ for all $N \in \mathsf{A}$, and all $i \geq 1$;
		\item $\mathrm{Tor}_{n+1}^R (N,M) = 0$ for all $N \in \mathsf{A}$.
	\end{enumerate}
\end{prop} 

\begin{defn}[Right weak-global dimension]
	Let $R$ be a ring.
	We define the {\em right weak global dimension} of $R$,
	denoted by $\mathrm{r.w.gld}\, R$,
	as
	\begin{equation}
	\mathrm{r.w.gl.dim}\, R \coloneqq \sup \left\{ \mathrm{w.d.}\, M_R \ \middle|\ M_R \in \mathsf{Mod}\text{-}R \right\}
	.\end{equation} 
\end{defn}

\begin{rem}[]
	Notice that $\mathrm{r.w.gl.dim}\, R = \sup \left\{ \mathrm{w.d.}\, {}_RN \ \middle|\ {}_RN \in R\text{-}\mathsf{Mod} \right\}$.
	Then we can analogously define the {\em left weak global dimension} of $R$
	and $\mathrm{r.w.gl.dim}\, R = \mathrm{l.w.gl.dim}\, R$.
\end{rem}

\begin{rem}[]
	Consider $M_R \in \mathsf{Mod}\text{-}R$
	and its character module $M^* \coloneqq \mathrm{Hom}_{\Z}\left( M, \mathbb{Q}/\Z \right)$.
	We proved that $M_R$ is flat iff $M^*$ is injective.
	Moreover we can define a canonical map $\mu\colon M \to M^{**}$ that
	acts sending an element of $M$ to the corresponding valuation map on $M^*$.
	More explicitly, given $x \in M_R$, 
	$\mu(x) \in \mathrm{Hom}_{ \Z}\left( M^*, \mathbb{Q}/\Z \right)$
	is the map which, on $f \in M^* = \mathrm{Hom}_{\Z}\left( M, \mathbb{Q}/\Z \right)$,
	acts by $\mu(x) (f) \coloneqq f(x)$.
	Notice that $\mu$ is mono, since $\mathbb{Q}/\Z$ is an injective cogenerator of $\mathsf{Ab}$.
	In fact, for any $0 \neq x \in M$, we can define a nonzero map
	$g\colon \left\langle x \right\rangle_{\Z} \to \mathbb{Q}/\Z$
	which can be extended to the whole $M$.
\end{rem}

\begin{prop}
	Let $M_R$ be a right $R$-module, then the following are equivalent:
	\begin{enumerate}
		\item $M_R$ is flat;
		\item $M^*$ is an injective left $R$-module;
		\item for all ${}_RI \triangleleft {}_RR$ (a left ideal)
			\begin{equation}
			M_R \otimes_R I \simeq MI =
			\left\{ \sum_{i=1}^{n} x_i a_i \ \middle|\ x_i \in M_R,\, a_i \in {}_RI,\, n \in \N \right\}
			;\end{equation} 
		\item $\mathrm{Tor}^R_1(M, R/{}_RI) = 0$ for all ${}_RI \triangleleft {}_RR$.
	\end{enumerate}
	(Clearly all of the above holds true even for left $R$-modules).
\end{prop} 

\begin{rem}[]
	Consider the embedding $\epsilon\colon I \hookrightarrow R$ of $I$ into $R$
	and $M_R \in \mathsf{Mod}\text{-}R$,
	then we can take the tensor
	$\mathrm{id}_{ M } \otimes \epsilon\colon M \otimes_R I \to M \otimes_R R \simeq M$ acting as
	$x \otimes a \mapsto xa$.
	Then
	\begin{equation}
		\ima (\mathrm{id}_{ M } \otimes \epsilon) =
		\left\{ \sum_{i=1}^{n} x_ia_i \ \middle|\ x_i \in M, a_i \in I \right\} =
		MI
	.\end{equation} 
	Thus $M \otimes I \simeq MI$ iff $id_M \otimes \epsilon$ is mono.
\end{rem}

\begin{lem}
	Consider $f\colon M_R \to N_R$ a morphism in the catogory of right $R$-modules.
	Let $f^* \coloneqq \mathrm{Hom}_{\Z}\left( f, \mathbb{Q}/\Z \right)$.
	\begin{itemize}
		\item $f$ is mono iff $f^*$ is epi,
		\item $f$ is epi iff $f^*$ is mono.
	\end{itemize}
\end{lem} 

\begin{lem}
	Consider $M_R$ and ${}_RN$. Then there are canonical isomorphisms:
	\begin{itemize}
		\item $M_R \otimes_R R \simeq R$ as right $R$-modules 
			(resp. $R \otimes_R N \simeq {}_RN$ as left $R$-modules),
		\item $\mathrm{Hom}_{ R}\left( R, M \right) \simeq M$ as right $R$-modules
			(resp. $\mathrm{Hom}_{ R}\left( R, N \right) \simeq N$ as left $R$-modules),
		\item $M \otimes_R R/{}_RI \simeq M/MI$ as abelian groups
			(resp. $R/I_R \otimes N \simeq N/IN$ as abelian groups).
	\end{itemize}
\end{lem} 

\begin{exr}
	Let $\K$ be a field.
	Consider $R \coloneqq \K[x,y]$ and $\mathfrak{m} \coloneqq (x,y)$.
	Then $R/\mathfrak{m} \simeq \K$.
	\begin{itemize}
		\item Show that $\K$ has a projective resolution
			\begin{equation}
			0 \to R \xrightarrow{\beta} R \oplus R \xrightarrow{\alpha} R \to
			R/\mathfrak{m} \simeq \K \to 0
			,\end{equation} 
			where $\beta = \begin{bmatrix}-y \\ x\end{bmatrix}$ and $\alpha(e_1) = x$, $\alpha(e_2) = y$.
		\item Show that $\mathrm{Tor}^R_2(\K,\K) \simeq \mathrm{Tor}^R_1(\mathfrak{m}, \K) \simeq \K$,
			sot that $\mathfrak{m}$ is torsion-free and not flat.
		\item $\mathrm{p.d.}\, \mathfrak{m} = 1$, $\mathrm{p.d.}\, \K = 2$
			and $\mathrm{w.d.}\, \K = 2$.
	\end{itemize}
\end{exr} 

\begin{prop}\leavevmode\vspace{-.2\baselineskip}
	\begin{enumerate}
		\item $\mathrm{Tor}^R_n \left( \bigoplus_{i \in I} M_i, N \right) \simeq
			\bigoplus_{i \in I} \mathrm{Tor}^R_n(M_i,R)$

		\item For a (from the proof I guess it is filtered)
			direct system of modules $\left\{ M_i, f_{ji} \right\}_{i \leq j}$
			\begin{equation}
				\mathrm{Tor}^R_n \big( \varinjlim_{i \in I} M_I, N \big) \simeq
				\varinjlim_{i \in I} \mathrm{Tor}^R_n(M_i, N)
			.\end{equation} 
	\end{enumerate}
	This theorem holds also for the second component of $\mathrm{Tor}$.
\end{prop}

\begin{lem}
	Let $M_R \in \mathsf{Mod}\text{-}R$.
	$M$ is flat iff 
	\begin{equation}
		\mathrm{Tor}^R_1 (M, R/I) = 0
	\end{equation} 
	for all ${}_RI \triangleleft R$ finitely generated left ideal.
\end{lem} 

\begin{prop}
	Let $\mathsf{A}$ be an abelian category with products, coproducts,
	enough injectives and projectives.
	For any family $\left\{ M_i \right\}_{i \in I}$, $\left\{ N_i \right\}_{i \in I}$ and
	any object $M,N$ of $\mathsf{A}$ and any $n \in \N$:
	\begin{enumerate}
		\item $\mathrm{Ext}^n_{\mathsf{A}} \left( M, \prod_{i \in I} N_i\right) \simeq
			\prod_{i \in I} \mathrm{Ext}^n_{\mathsf{A}}(M,N_i)$
		\item $\mathrm{Ext}^n_{\mathsf{A}} \left( \bigoplus_{i \in I}M_i, N \right) \simeq
			\prod_{i \in I} \mathrm{Ext}^n_{\mathsf{A}}(M_i,N)$
	\end{enumerate}
\end{prop}

\begin{rem}[]
	Let $\left\{ M_i, f_{ji} \right\}_{i \leq j}$ be a directed system of modules.
	In general
	\begin{equation}
		\mathrm{Ext}^R_n \big( \varinjlim_{i \in I} M_i, N\big)
		\not\simeq \varinjlim_{i \in I} \mathrm{Ext}^R_n (M_i, N)
	.\end{equation} 
\end{rem}

\begin{ex}
	Let $F_R$ be a flat, but not projective right $R$-module.
	Then there exists a module $N$ s.t. $\mathrm{Ext}^1_R(F,N) \neq 0$.
	Moreover $F = \varinjlim_{i \in I} G_i$,
	for $G_i$ finitely generated free modules (by Lazard theorem).
	Then, for all $i \in I$, $\mathrm{Ext}^1_R(G_i, N) = 0$.
	In other words we have a counterexample to the above "equality".

	(Notice that there exist module such as $F_R$, in fact $\mathbb{Q}$
	is a flat, but not projective, module;
	in particular it is flat, since it is a localization of $\Z$).
\end{ex} 

\begin{defn}[Right (left) hereditary ring]
	A ring $R$ is right (resp left) {\em hereditary} iff every submodule of a projective right
	(resp left) $R$-module is projective.
\end{defn}

\begin{prop}[Characterization of hereditary rings]
	Let $R$ be a ring, then the following are equivalent
	\begin{enumerate}
		\item $R$ is right hereditary;
		\item $\mathrm{r.gl.dim}\, R \leq 1$;
		\item $\mathrm{Ext}^2_R(M,N) = 0$ for all $M, N \in \mathsf{Mod}\text{-}R$;
		\item $\mathrm{Ext}^2_R(R/I,N) = 0$ for all $N_R \in \mathsf{Mod}\text{-}R$ and $I_R \triangleleft R$
			right ideal;
		\item $I_R$ is projective for every right ideal $I_R \triangleleft R$.
	\end{enumerate}
	Clearly there exists also a left version of this proposition.
\end{prop} 

\begin{ex}
	Being right or left hereditary is not symmetrical.
	In particular Kaplansky constructed an example of a ring $R$ which is right hereditary,
	but has $\mathrm{l.gl.dim}\, R = 2$, i.e. it is not left hereditary.
	Small gave another example of a right hereditary ring $R$, 
	with $\mathrm{l.gl.dim}\, R = 3$.
\end{ex} 

Recall the definition of weak global dimension of a ring $R$:
\begin{equation}
\mathrm{w.gl.dim}\, R = \sup \left\{ \mathrm{w.d.}\, M_R \ \middle|\ M_R \in \mathsf{Mod}\text{-}R \right\}
.\end{equation} 
(By symmetry of $\mathrm{Tor}$ functor this coincides with the left weak global dimension).

\begin{prop}
	Let $R$ be a ring. The following are equivalent:
	\begin{enumerate}
		\item $\mathrm{w.gl.dim}\, R \leq 1$;
		\item $\mathrm{Tor}^R_2(M,N) = 0$ for all $M \in \mathsf{Mod}\text{-}R$ and $N \in R\text{-}\mathsf{Mod}$;
		\item Every submodule of a flat module is flat;
		\item $\mathrm{Tor}^2_R(R/I_R, N) = 0$ for all $I_R \triangleleft R$ right ideals
			and $N \in R\text{-}\mathsf{Mod}$;
		\item Every right ideal $I_R \triangleleft R$ is a flat $R$-module.
	\end{enumerate}
\end{prop} 

\begin{rem}[]
	Recall that we have the following implications:
	choose a ring $R$, and an $R$-module $M$, then
	\begin{itemize}
		\item $M$ free $\implies$ $M$ projective;
		\item $M$ projective $\implies$ $M$ flat;
		\item filtered direct limits of projective modules are flat;
		\item direct limits of finitely generated free modules are flat;
		\item in general, it is not true that any flat module is projective.
	\end{itemize}
	We now want to show that, in the particular case where $M$ is finitely presented, then
	it is projective as soon as it is flat.
\end{rem}

\begin{defn}[Finitely presented module]
	Recall that $M \in R\text{-}\mathsf{Mod}$ is finitely presented iff
	there is a short exact sequence
	\begin{equation*}
	\begin{tikzcd}
		0 \arrow[r, "", rightarrow] &
		K \arrow[r, "", rightarrow] &
		R^n \arrow[r, "", rightarrow] &
		M \arrow[r, "", rightarrow] &
		0
	,\end{tikzcd}
	\end{equation*}
	with $n \in \N$ and $K$ a finitely generated $R$-module.
\end{defn}

\begin{lem}
	Let $M_R$ be a finitely presented module and consider the short exact sequence
	\begin{equation*}
	\begin{tikzcd}
		0 \arrow[r, "", rightarrow] &
		H \arrow[r, "", rightarrow] &
		P \arrow[r, "", rightarrow] &
		M \arrow[r, "", rightarrow] &
		0
	,\end{tikzcd}
	\end{equation*}
	with $P$ projective and finitely generated.
	Then $H$ is finitely generated.
\end{lem} 


\begin{rem}[]
	The above implies that $M_R$ is finitely presented iff there is an exact sequence
	\begin{equation*}
	\begin{tikzcd}
		R^m \arrow[r, "", rightarrow] &
		R^n \arrow[r, "", rightarrow] &
		M \arrow[r, "", rightarrow] &
		0
	,\end{tikzcd}
	\end{equation*}
	for some $m,n \in \N$.
\end{rem}

\begin{rem}[Pano's guess]
	I guess that any finitely generated projective $R$-module $P$ is also finitely presented.
	Let's, in fact, give a presentation of $M$:
	\begin{equation*}
	\begin{tikzcd}
		0 \arrow[r, "", rightarrow] &
		K \arrow[r, "", rightarrow] &
		R^n \arrow[r, "", rightarrow] &
		M \arrow[r, "", rightarrow] &
		0
	.\end{tikzcd}
	\end{equation*}
	By projectivity of $M$ it splits.
	Then $K$ is a direct summand of a free module, hence it is finitely generated.
\end{rem}


\begin{rem}[]
	Fix a pair of right $R$-modules $M_R, N_R \in \mathsf{Mod}\text{-}R$,
	then the character module $\mathrm{Hom}_{\Z}\left( N, \mathbb{Q}/\Z \right) = 
	\left( N_R \right)^* \in R\text{-}\mathsf{Mod}$ is a left $R$-module.
	Moreover there exists a morphism in $\mathsf{Ab}$
	\begin{equation*}
	\begin{tikzcd}[row sep = 0ex
		,/tikz/column 1/.append style={anchor=base east}
		,/tikz/column 2/.append style={anchor=base west}]
		\sigma_{M,N}\colon M \otimes N^* \arrow[r, "", rightarrow] &
		\left[ \mathrm{Hom}_{R}\left( M, N \right) \right]^*
		= \mathrm{Hom}_{\Z}\left( \mathrm{Hom}_{ R}\left( M, N \right), \mathbb{Q}/\Z \right) \\
		x \otimes f \arrow[r, "", mapsto] & g
	,\end{tikzcd}
	\end{equation*}
	where $g$ acts as follows on $\alpha \in \mathrm{Hom}_{R}\left( M, N \right)$
	\begin{equation}
		g(\alpha) \coloneqq f \left( \alpha(x) \right)
	.\end{equation} 
\end{rem}

\begin{lem}
	Let $M_R$ be a finitely presented $R$-module,
	then $\sigma_{M,N}$, as defined above, is an isomorphism for all $N \in \mathsf{Mod}\text{-}R$.
\end{lem} 

\begin{thm}[]
	A finitely presented flat module $M_R$ is projective.
\end{thm}

\subsection{Change of rings}
\subsubsection{Tor under change of rings}
Let $R,S$ be rings, and $f\colon R \to S$ a ring homomorphism.
Then $S$ is an $R$-$R$ bimodule via $f$ and every $S$-module
is an $R$-module via restriction of scalars.
Moreover, given any $M_R \in \mathsf{Mod}\text{-}R$, then
$M_R \otimes_R S$ is a right $S$-module via extension of scalars.

\begin{prop}
	Let $f\colon R \to S$ be a ring homomorphism.
	Assume that ${}_RS$ is a flat left $R$-module.
	Then, for all $M_R \in \mathsf{Mod}\text{-}R$, all $n \in \N$
	and ${}_SC \in S\text{-}\mathsf{Mod}$
	(hence ${}_SC \in R\text{-}\mathsf{Mod}$), we have
	\begin{equation}
		\mathrm{Tor}^R_n (M_R, {}_SC) \simeq
		\mathrm{Tor}^S_n (M \otimes_R S, {}_SC)
	.\end{equation} 
\end{prop} 

\begin{prop}
	Let $f\colon R \to S$ be a ring homomorphism.
	Assume that ${}_RS$ is a flat left $R$-module.
	Then, for all $M_R \in \mathsf{Mod}\text{-}R$, $C_S \in \mathsf{Mod}\text{-}S$ and $n \in \N$,
	we have
	\begin{equation}
		\mathrm{Ext}^n_R (M, C) \simeq
		\mathrm{Ext}^n_S (M \otimes_R S, C)
	.\end{equation} 
\end{prop} 

\begin{prop}
	Let $R, S$ be commutative rings, and $f\colon R \to S$ a ring homomorphism.
	Assume that $S$ is a flat $R$-module.
	Then for all modules $M$ and $N$, and $n \in \N$, we have
	\begin{equation}
		\mathrm{Tor}^R_n (M,N) \otimes_R S \simeq
		\mathrm{Tor}^S_n (M \otimes_R S, N \otimes_R S)
	.\end{equation} 
\end{prop} 


\begin{cor}
	Let $R$ be a commutative ring, $M$ and $N$ be $R$-modules and $n \in \N$.
	The following are equivalent:
	\begin{enumerate}
		\item $\mathrm{Tor}^R_n (M,N) = 0$;
		\item $\mathrm{Tor}_n^{R_{\mathfrak{p}}}(M_{\mathfrak{p}}, N_{\mathfrak{p}}) = 0$
			for all $\mathfrak{p} \in \mathrm{Spec}\, R$;
		\item $\mathrm{Tor}_n^{R_{\mathfrak{m}}} (M_{\mathfrak{m}}, N_{\mathfrak{m}})$
			for all $\mathfrak{m} \in \mathrm{MaxSpec}\, R$.
	\end{enumerate}
\end{cor} 

\subsubsection{Hom and Ext with finitely presented modules}
Let $R, S$ be commutative rings, $\varphi\colon R \to S$ be a ring homomorphism
and $S$ be a flat $R$-module 
(e.g. $T$ a multiplicatively closed subset of $R$
and $S \coloneqq R_T = R[T^{-1}]$).

\begin{prop}
	Let $R,S$ be commutative rings, $\varphi\colon R \to S$ be a ring homomorphism.
	Assume that $S$ is a flat $R$-module
	and consider $M_R$ a finitely presented $R$-module.
	Then, for any $N \in \mathsf{Mod}\text{-}R$,
	\begin{equation}
	\mathrm{Hom}_{ S}\left( M \otimes_R S, N \otimes_R S \right) \simeq
	\mathrm{Hom}_{ R}\left( M, N \right) \otimes_R S
	.\end{equation} 	
\end{prop} 

Let's now extend this result for the $\mathrm{Ext}$ functor:
\begin{defn}
	We denote by $\mathsf{mod}\text{-}R$ (using the lowercase m to differentiate from the
	bigger category) the category of right $R$-modules $M$ with a projective
	resolution of finitely generated projective modules
	(i.e. all the syzygys $\Omega_n(M)$ are finitely generated:
	at each point the kernel [i.e. the syzygy] is an epimorphic image of a finitely generated module).
\end{defn}

\begin{rem}[]
	If $R$ is a right Noetherian ring, then
	the objects of $\mathrm{mod}\text{-}R$ are exactly the finitely generated $R$-modules.
\end{rem}

\begin{defn}[Right coherent ring]
	A ring $R$ is {\em right coherent} iff every finitely generated right ideal
	is also finitely presented.
	Equivalently iff every finitely generated submodule of a finitely
	presented right $R$-module is finitely presented.
\end{defn}
\begin{rem}[]
	Let $R$ be right coherent, then
	$\mathsf{mod}\text{-}R$ is the category of finitely presented right $R$-modules.
\end{rem}

\begin{prop}
	Let $R,S$ be commutative rings and $\varphi\colon R \to S$ be a ring homomorphism.
	Assume that $S$ is a flat $R$-module and
	consider $M \in \mathsf{mod}\text{-}R$ and $N \in \mathsf{Mod}\text{-}R$.
	Then, for all $n \in \N$,
	\begin{equation}
		\mathrm{Ext}^n_S \left( M \otimes_R S, N \otimes_R S \right) \simeq
		\mathrm{Ext}^n_R \left( M,N \right) \otimes_R S
	.\end{equation} 
\end{prop} 

\begin{cor}
	Let $R$ be a commutative ring,
	$M_R \in \mathsf{mod}\text{-}R$, $N \in \mathsf{Mod}\text{-}R$ and $n \in \N$.
	The following are equivalent:
	\begin{enumerate}
		\item $\mathrm{Ext}^n_R = 0$;
		\item $\mathrm{Ext}^{n}_{R_{\mathfrak{p}}}
			\left( M_{\mathfrak{p}}, N_{\mathfrak{p}}\right) = 0$
			for all $\mathfrak{p} \in \mathrm{Spec}\, R$;
		\item $\mathrm{Ext}^{ n}_{ R_{\mathfrak{m}}} \left( M_{\mathfrak{m}}, N_{\mathfrak{m}} \right) = 0$
			for all $\mathfrak{m} \in \mathrm{MaxSpec}\, R$.
	\end{enumerate}
\end{cor} 

\subsection{Homological formulas relating Ext and Hom}
\begin{prop}
	Consider $R$ and $S$ rings.
	Let ${}_RN_S$ be an $S$-$R$ bimodule and $M_R \in \mathsf{Mod}\text{-}R$.
	Consider $C_S$ an injective right $S$-module.
	Then, for all $n \geq 0$,
	\begin{equation}
	\mathrm{Ext}^{ n}_{ R} \left( M_R, \mathrm{Hom}_{ S}\left( N_S, C_S \right)_R \right) \simeq
	\mathrm{Hom}_{ S}\left( \mathrm{Tor}^{ R}_{ n} \left( M, N \right)_S, C_S  \right)
	.\end{equation} 
	In particular, if we pick $S \coloneqq \Z$ and $C \coloneqq \mathbb{Q}/\Z$,
	we obtain, for all $n \in \geq 0$
	\begin{equation}
	\mathrm{Ext}^{ n}_{ R} \left( M_R, N^* \right) \simeq
	\left[ \mathrm{Tor}^{ R}_{ n} \left( M, N \right)\right]^*
	.\end{equation} 
\end{prop} 

\begin{prop}
	Consider $R,S$ rings.
	Let ${}_RN_S$ be an $S$-$R$ bimodule and $M_R \in \mathsf{mod}\text{-}R$.
	Consider ${}_SC$ an injective left $S$-module.
	Then, for all $n \geq 0$,
	\begin{equation}
		\mathrm{Tor}^{ R}_{ n}
		\left( M_R, \mathrm{Hom}_{ S}\left( {}_SN_R, {}_SC \right) \right) \simeq
		\mathrm{Hom}_{ S}\left( \mathrm{Ext}^{ n}_{ R} \left( M_R, {}_SN_R \right), {}_SC \right)
	.\end{equation} 
	In particular, if $S \coloneqq \Z$ and $C \coloneqq \mathbb{Q}/\Z$, then
	\begin{equation}
		\mathrm{Tor}^{ R}_{ n} \left( M, N^* \right) \simeq
		\left[ \mathrm{Ext}^{ n}_{ R} \left( M, N \right) \right]^*
	.\end{equation} 
\end{prop} 

\begin{ex}
	Let $M_R$ be a right $R$-module,
	${}_RG_S$ an $R$-$S$ bimodule and $C_S$ a right $S$-module.
	Assume that $\mathrm{Tor}^{ R}_{ 1} \left( M, G \right) = 0$, then
	there is a monomorphism (of abelian groups)
	\begin{equation*}
	\begin{tikzcd}
		\mathrm{Ext}^{ 1}_{ R} \left( M_R, \mathrm{Hom}_{ S}\left( {}_RG_S, C_S \right) \right)
		\arrow[r, "", rightarrowtail] &
		\mathrm{Ext}^{ 1}_{ S} \left( M \otimes_R G, C_S \right)
	.\end{tikzcd}
	\end{equation*}
\end{ex} 
