\section{Yoneda extension}
Our next aim is, given an abelian category $\mathsf{A}$
and objects $A,B \in \mathrm{Ob} \left(\mathsf{A}\right)$,
to define $\mathrm{Ext}^{ }_{ \mathsf{A}} \left( A, B \right)$ even though
$\mathsf{A}$ might not have enough injectives nor projectives.

\begin{defn}[Extension]
	Let $\mathsf{A}$ be an abelian category, $A,B \in \mathrm{Ob} \left(\mathsf{A}\right)$.
	An extension of $A$ by $B$ is a short exact sequence
	\begin{equation*}
	\begin{tikzcd}
		\zeta \coloneqq 0 \arrow[r, "", rightarrow] &
		B \arrow[r, "", rightarrow] &
		X \arrow[r, "", rightarrow] &
		A \arrow[r, "", rightarrow] &
		0
	.\end{tikzcd}
	\end{equation*}
	We say that two extensions $\zeta$ and $\zeta'$ are equivalent, denoted by $\zeta \sim \zeta'$,
	iff there is a commutative diagram as below
	\begin{equation}
	\begin{tikzcd}
		\zeta \colon &
		0 \arrow[r, "", rightarrow] &
		B \arrow[r, "", rightarrow] \arrow[d, "", equal] &
		X \arrow[r, "", rightarrow] \arrow[d, "\rotatebox{90}{\(\sim\)}", rightarrow] &
		A \arrow[r, "", rightarrow] \arrow[d, "", equal] &
		0\\
		\zeta' \colon &
		0 \arrow[r, "", rightarrow] &
		B \arrow[r, "", rightarrow] &
		X' \arrow[r, "", rightarrow] &
		A \arrow[r, "", rightarrow] &
		0.
	\end{tikzcd}
	\end{equation} 
	Notice that, by the five lemma, in any such diagram the middle
	vertical arrow is an isomorphism.
\end{defn}

\begin{rem}[Split extensions]
	Recall the characterization of splitting short exact sequences: an extension
	\begin{equation*}
	\begin{tikzcd}
		\zeta \coloneqq 0 \arrow[r, "", rightarrow] &
		B \arrow[r, "\mu", rightarrow] &
		X \arrow[r, "p", rightarrow] &
		A \arrow[r, "", rightarrow] &
		0
	\end{tikzcd}
	\end{equation*}
	splits iff it is equivalent to the following extension of $A$ by $B$:
	\begin{equation*}
	\begin{tikzcd}
		0 \arrow[r, "", rightarrow] &
		B \arrow[r, "\epsilon_B", rightarrow] &
		A \oplus B \arrow[r, "\pi_A", rightarrow] &
		A \arrow[r, "", rightarrow] &
		0
	.\end{tikzcd}
	\end{equation*}
	Equivalently iff there is $f\colon X \to B$ s.t. $f \circ \mu = 1_B$,
	iff there is $g\colon A \to X$ s.t. $p \circ g = 1_A$.
\end{rem}

\begin{rem}[Class of extensions and Ext]
	Denote by $\mathrm{E}(A,B)$ the class of all extensions of $A$ by $B$.
	If we denote by $\sim$ the above equivalence relation and we can
	define $\mathrm{Ext}^{ 1}_{ \mathsf{A}} \left( A, B \right)$ (i.e. if $\mathsf{A}$
	has enough injectives of projectives), then we want to construct an isomorphism
	$\theta$ of abelian groups
	\begin{equation}
	\mathrm{Ext}^{ 1}_{ \mathsf{A}} \left( A, B \right) \simeq_{\theta}
	\frac{\mathrm{E}(A,B)}{\sim}
	.\end{equation} 
	Let's define $\theta$.
	Fix $A,B \in \mathrm{Ob} \left(\mathsf{A}\right)$ and consider an extension of $A$ by $B$
	\begin{equation*}
	\begin{tikzcd}
		\zeta \coloneqq 0 \arrow[r, "", rightarrow] &
		B \arrow[r, "", rightarrow] &
		X \arrow[r, "", rightarrow] &
		A \arrow[r, "", rightarrow] &
		0
	.\end{tikzcd}
	\end{equation*}
	Assume that $\mathrm{Ext}^{ n}_{ \mathsf{A}} \left( -, B \right)$ exist,
	forming a cohomological $\partial$-functor (e.g. if $\mathsf{A}$ has enough
	projectives).
	Apply $\mathrm{Hom}_{\mathsf{A}} \left( -, B \right)$ to $\zeta$ and obtain
	\begin{equation*}
	\begin{tikzcd}
		0 \arrow[r, "", rightarrow] &
		\mathrm{Hom}_{\mathsf{A}} \left( A, B \right) \arrow[r, "", rightarrow] &
		\mathrm{Hom}_{\mathsf{A}} \left( X, B \right) \arrow[r, "", rightarrow] &
		\mathrm{Hom}_{\mathsf{A}} \left( B, B \right) \arrow[r, "\partial", rightarrow] &
		\mathrm{Ext}^1_{\mathsf{A}} \left( A, B \right)
	.\end{tikzcd}
	\end{equation*}
	Finally we define $\theta(\zeta) \coloneqq \partial(1_B) \in \mathrm{Ext}^{ 1}_{ \mathsf{A}} \left( A, B \right)$.
\end{rem}

\begin{lem}
	Fixed a category $\mathsf{A}$ and objects $A,B$ of $\mathsf{A}$ as before, 
	if $\zeta \sim \zeta' \in \mathrm{E}(A,B)$, then $\theta(\zeta) = \theta(\zeta')$.
	In other words $\theta\colon \mathrm{E}(A,B) \to \mathrm{Ext}^{ 1}_{ \mathsf{A}} \left( A, B \right)$
	induces a map on the quotient $\frac{\mathrm{E}(A,B)}{\sim}$.
\end{lem} 

\begin{thm}[]
	Given $\mathsf{A},A,B$ as before,
	$\theta$ gives a bijective correspondence
	\begin{equation}
	\begin{tikzcd}
		\frac{\mathrm{E}(A,B)}{\sim} \arrow[r, "\theta", leftrightarrow] &
		\mathrm{Ext}^{ 1}_{ \mathsf{A}} \left( A, B \right)
	\end{tikzcd}
	.\end{equation} 
\end{thm}

\begin{lem}
	Let $\mathsf{A}$ be an abelian category.
	Consider a commutative diagram
	\begin{equation}
	\begin{tikzcd}
		0 \arrow[r, "", rightarrow] &
		K \arrow[r, "\nu", rightarrow] \arrow[d, "\beta"', rightarrow] &
		M \arrow[r, "", rightarrow] \arrow[d, "h", rightarrow] \arrow[ld, "g"', dashrightarrow] &
		A \arrow[r, "", rightarrow] \arrow[d, "", equal] &
		0\\
		0 \arrow[r, "", rightarrow] &
		B \arrow[r, "", rightarrow] &
		Y \arrow[r, "", rightarrow] &
		A \arrow[r, "", rightarrow] &
		0
	\end{tikzcd}
	,\end{equation} 
	with exact rows and where the leftmost square is a pushout.
	Then there is $g\colon M \to B$ s.t. $g \circ \nu = \beta$
	iff the second row splits.
\end{lem} 

\begin{lem}
	Let $\mathsf{A}, A,B$ be as before,
	then $\mathrm{Ext}^{ 1}_{ \mathsf{A}} \left( A, B \right) = 0$ (as abelian groups)
	iff every extension of $A$ by $B$ splits.
\end{lem} 

\begin{rem}[]
	Consider $\zeta \in \mathrm{E}(A,B)$,
	then $\gamma \in \mathrm{Hom}_{\mathsf{A}} \left( A', A \right)$
	gives $\zeta\gamma \in \mathrm{E}(A',B)$
	\begin{equation}
	\begin{tikzcd}
		\zeta\gamma\colon &
		0 \arrow[r, "", rightarrow] &
		B \arrow[r, "", rightarrow] \arrow[d, "", equal] &
		X' \arrow[r, "", rightarrow] \arrow[d, "", rightarrow] &
		A' \arrow[r, "", rightarrow] \arrow[d, "\gamma", rightarrow] &
		0\\
		\zeta\colon &
		0 \arrow[r, "", rightarrow] &
		B \arrow[r, "", rightarrow] &
		X \arrow[r, "", rightarrow] &
		A \arrow[r, "", rightarrow] &
		0
	\end{tikzcd}
	,\end{equation} 
	where $X'$ is a pullback of the diagram
	\begin{equation}
	\begin{tikzcd}
		& A' \arrow[d, "\pi", rightarrow] \\
		X \arrow[r, "", rightarrow] & A
	\end{tikzcd}
	.\end{equation} 
	Analogously $\beta \in \mathrm{Hom}_{\mathsf{A}} \left( B, B' \right)$ gives
	$\beta\zeta \in \mathrm{E}(A,B')$:
	\begin{equation}
	\begin{tikzcd}
		\zeta\colon &
		0 \arrow[r, "", rightarrow] &
		B \arrow[r, "", rightarrow] \arrow[d, "\beta", rightarrow] &
		X \arrow[r, "", rightarrow] \arrow[d, "", rightarrow] &
		A \arrow[r, "", rightarrow] \arrow[d, "", equal] &
		0\\
		\beta\zeta\colon &
		0 \arrow[r, "", rightarrow] &
		B' \arrow[r, "", rightarrow] &
		X' \arrow[r, "", rightarrow] &
		A \arrow[r, "", rightarrow] &
		0
	\end{tikzcd}
	,\end{equation} 
	where $X'$ is a pushout of the diagram
	\begin{equation}
	\begin{tikzcd}
		B \arrow[r, "", rightarrow] \arrow[d, "p"', rightarrow] & X\\
		B' &
	\end{tikzcd}
	.\end{equation} 
\end{rem}


\begin{rem}
	In $\mathsf{Mod}\text{-}R$ (hence in any abelian category by
	Freyd-Mitchell), in any diagram like
	\begin{equation}
	\begin{tikzcd}
		0 \arrow[r, "", rightarrow] &
		B \arrow[r, "", rightarrow] \arrow[d, "", equal] &
		X' \arrow[r, "", rightarrow] \arrow[d, "", rightarrow] &
		A' \arrow[r, "", rightarrow] \arrow[d, "\gamma", rightarrow] &
		0\\
		0 \arrow[r, "", rightarrow] &
		B \arrow[r, "", rightarrow] &
		X \arrow[r, "", rightarrow] &
		A \arrow[r, "", rightarrow] &
		0
	\end{tikzcd}
	\end{equation} 
	the right square is a pullback, so that pullbacks of
	short exact sequences are the only diagrams of this form.
	Analogously in any diagram of the form
	\begin{equation}
	\begin{tikzcd}
		0 \arrow[r, "", rightarrow] &
		B \arrow[r, "", rightarrow] \arrow[d, "\beta", rightarrow] &
		X \arrow[r, "", rightarrow] \arrow[d, "", rightarrow] &
		A \arrow[r, "", rightarrow] \arrow[d, "", equal] &
		0\\
		0 \arrow[r, "", rightarrow] &
		B' \arrow[r, "", rightarrow] &
		X' \arrow[r, "", rightarrow] &
		A \arrow[r, "", rightarrow] &
		0
	\end{tikzcd}
	\end{equation} 
	the left square is a pushout.
\end{rem}

\begin{defn}[Baer sum in $\mathrm{E}(A,B)/\sim$]
	Consider $\mathsf{A},A,B$ as before.
	Let $\left[ \zeta \right], \left[ \zeta' \right]$ be the equivalence classes
	of $\zeta,\zeta' \in \mathrm{E}(A,B)$
	with respect to $\sim$:
	\begin{equation*}
	\begin{tikzcd}[column sep=1.5em]
		\zeta \coloneqq 0 \arrow[r, "", rightarrow] &
		B \arrow[r, "i", rightarrow] &
		X \arrow[r, "\pi", rightarrow] &
		A \arrow[r, "", rightarrow] &
		0
	\end{tikzcd}
	\quad\text{and}\quad
	\begin{tikzcd}[column sep=1.5em]
		\zeta' \coloneqq 0 \arrow[r, "", rightarrow] &
		B \arrow[r, "i'", rightarrow] &
		X' \arrow[r, "\pi'", rightarrow] &
		A \arrow[r, "", rightarrow] &
		0
	.\end{tikzcd}
	\end{equation*}
	Now consider the extension of $A \oplus A$ by $B \oplus B$, given by the direct sum
	\begin{equation*}
	\begin{tikzcd}
		\zeta \oplus \zeta'\coloneqq 
		0 \arrow[r, "", rightarrow] &
		B \oplus B \arrow[r, "i \oplus i'", rightarrow] &
		X \oplus X' \arrow[r, "\pi \oplus \pi'", rightarrow] &
		A \oplus A \arrow[r, "", rightarrow] &
		0
	.\end{tikzcd}
	\end{equation*}
	Let moreover $\Delta_A\colon A \to A \oplus A$ be the diagonal map,
	i.e. $\Delta_A = \begin{bmatrix} 1_A \\ 1_A \end{bmatrix}$,
	and $\nabla_B\colon B \to B \oplus B$ the codiagonal map,
	i.e. $\nabla_B = \begin{bmatrix} 1_B & 1_B \end{bmatrix}$.
	By the above remark we can construct
	\begin{equation*}
	\begin{tikzcd}
		\nabla_B \left( \zeta \oplus \zeta' \right) \Delta_A\colon &
		0 \arrow[r, "", rightarrow] &
		B \arrow[r, "", rightarrow] &
		Z \arrow[r, "", rightarrow] &
		A \arrow[r, "", rightarrow] \arrow[d, "", equal] &
		0\\
		\left( \zeta \oplus \zeta' \right) \Delta_A \colon &
		0 \arrow[r, "", rightarrow] &
		B \oplus B \arrow[r, "", rightarrow] \arrow[d, "", equal] \arrow[u, "\nabla_B", rightarrow] &
		Y \arrow[r, "", rightarrow] \arrow[d, "", rightarrow] \arrow[u, "", rightarrow] &
		A \arrow[r, "", rightarrow] \arrow[d, "\Delta_A", rightarrow] &
		0\\
		\zeta \oplus \zeta' \colon &
		0 \arrow[r, "", rightarrow] &
		B \oplus B \arrow[r, "", rightarrow] &
		X \oplus X' \arrow[r, "", rightarrow] &
		A \oplus A \arrow[r, "", rightarrow] &
		0
	\end{tikzcd}
	.\end{equation*} 
	Finally we define $\left[ \zeta \right] + \left[ \zeta' \right] :=
	\left[ \nabla_B \left( \zeta \oplus \zeta' \right)\Delta_A \right]$.
	We can also see that this definition is independent of the choice
	of representatives.
\end{defn}

\begin{prop}
	$\mathrm{E}(A,B)/\sim$ endowed with the Baer sum (i.e$.$ the sum we just defined)
	is an abelian group.
	Moreover the map 
	\begin{equation*}
	\begin{tikzcd}[row sep = 0ex
		,/tikz/column 1/.append style={anchor=base east}
		,/tikz/column 2/.append style={anchor=base west}]
		\theta\colon \frac{\mathrm{E}(A,B)}{\sim} \arrow[r, "", rightarrow] &
		\mathrm{Ext}^{ 1}_{ \mathsf{A}} \left( A, B \right)
	\end{tikzcd}
	\end{equation*} 
	is a group homomorphism.
\end{prop} 
By this proposition, one can also define $\mathrm{Ext}^1_R$
to be $\mathrm{E}(A,B)/\sim$, and this definition does not require the 
abelian category $\mathsf{A}$ to have enough injectives or projectives.

Analogously this construction can be carried out for every $n \in \N$:
\begin{defn}[$n$ extension]
	Let $\mathsf{A}$ be an abelian category, $A,B \in \mathrm{Ob} \left(\mathsf{A}\right)$.
	An $n$ extension of $A$ by $B$, denoted by $\zeta \in \mathrm{E}_n(A,B)$, is an exact sequence
	\begin{equation*}
	\begin{tikzcd}
		\zeta \coloneqq 0 \arrow[r, "", rightarrow] &
		B \arrow[r, "", rightarrow] &
		X_n \arrow[r, "", rightarrow] &
		X_{n-1} \arrow[r, "", rightarrow] &
		\ldots \arrow[r, "", rightarrow] &
		X_1 \arrow[r, "", rightarrow] &
		A \arrow[r, "", rightarrow] &
		0
	.\end{tikzcd}
	\end{equation*}
	We say that two extensions $\zeta$ and $\zeta'$ are equivalent, denoted by $\zeta \sim \zeta'$,
	iff there is a commutative diagram s.t. the nontrivial vertical arrows are all isomorphisms
	\begin{equation*}
	\begin{tikzcd}
		\zeta \colon
		0 \arrow[r, "", rightarrow] &
		B \arrow[r, "", rightarrow] \arrow[d, "", equal] &
		X_n \arrow[r, "", rightarrow] \arrow[d, "f_n", rightarrow] &
		X_{n-1} \arrow[r, "", rightarrow] \arrow[d, "f_{n-1}", rightarrow] &
		\ldots \arrow[r, "", rightarrow] &
		X_1 \arrow[r, "", rightarrow] \arrow[d, "f_1", rightarrow] &
		A \arrow[r, "", rightarrow] \arrow[d, "", equal] &
		0\\
		\zeta' \colon
		0 \arrow[r, "", rightarrow] &
		B \arrow[r, "", rightarrow] &
		X'_n \arrow[r, "", rightarrow] &
		X'_{n-1} \arrow[r, "", rightarrow] &
		\ldots \arrow[r, "", rightarrow] &
		X'_1 \arrow[r, "", rightarrow] &
		A \arrow[r, "", rightarrow] &
		0
	\end{tikzcd}
	.\end{equation*} 
\end{defn}

And, as expected, it gives the desired result:
\begin{prop}
	For every $n \in \N$, if $\mathrm{Ext}^{ n}_{ \mathsf{A}} \left( A, B \right)$ is defined
	in $\mathsf{A}$ abelian,
	we have the following isomorphism of abelian groups
	\begin{equation}
		\frac{\mathrm{E}_n(A,B)}{\sim} \simeq
		\mathrm{Ext}^{ n}_{ \mathsf{A}} \left( A, B \right)
	.\end{equation} 
\end{prop} 
