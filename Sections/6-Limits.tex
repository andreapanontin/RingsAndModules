\section{Limit}
We will concentrate on an arbitrary category $\mathsf{C}$, and on a small category $\mathsf{I}$, i.e. a category with $\mathrm{Ob} \left(\mathsf{I}\right)$ is a set.
Consider a functor 
\begin{equation}
F: \mathsf{I} \to \mathsf{C}
.\end{equation} 
Then $\,\forall\, i \in \mathrm{Ob} \left(\mathsf{I}\right)$, $F(i) \in \mathrm{Ob} \left(\mathsf{C}\right)$ and,
given a morphism $\lambda: i \to j$ in $\mathsf{I}$, then $F(\lambda): F(i) \to F(j)$.

\begin{defn}[Compatible family with respect to $F$]
	Consider a family $\left\{ \alpha_i \right\}_{i \in \mathrm{Ob} \left(\mathsf{I}\right)}$ of morphisms 
	$\alpha_i: X \to F(i)$ for a fixed $X \in \mathrm{Ob} \left(\mathsf{C}\right)$.
	It is said to be a \textbf{compatible family} with respect to $F$ iff
	given any morphism $\lambda: i \to j$ in $\mathsf{I}$, the following trangle commutes
	\begin{equation}
	\begin{tikzcd}
		X \arrow[r, "\alpha_i", rightarrow] \arrow[rd, "\alpha_j"', rightarrow] &
		F(i) \arrow[d, "F(\lambda)", rightarrow] \\
		&
		F(j)
	\end{tikzcd}
	.\end{equation} 
	In other words iff $\alpha_j = F(\lambda) \circ \alpha_i$ for every $i, j \in \mathrm{Ob} \left(\mathsf{I}\right)$ and every $\lambda: i \to j$.
\end{defn}

\begin{defn}[Projective (inverse) limit]
	A (projective/inverse) \textbf{limit} of $F$ is an object in $\mathsf{C}$, denoted with $\varprojlim F$, with morphisms
	$p_i: \varprojlim F \to F(i)$ for all $i \in \mathrm{Ob} \left(\mathsf{I}\right)$ stasfying the following conditions
	\begin{description}
		\item[LIM1] $\left\{ p_i \right\}_{i \in \mathrm{Ob} \left(\mathsf{I}\right)}$ is a compatible family of morphisms, i.e.
			\begin{equation}
			\begin{tikzcd}
				\varprojlim F \arrow[r, "p_i", rightarrow] \arrow[rd, "p_j"', rightarrow] &
				F(i) \arrow[d, "F(\lambda)", rightarrow] \\
				&
				F(j)
			\end{tikzcd}
			\end{equation} 
			the above diagram commutes for all $i, j \in \mathrm{Ob} \left(\mathsf{I}\right)$ and all $\lambda: i \to j$.
		\item[LIM2] For any $X \in \mathrm{Ob} \left(\mathsf{C}\right)$ and any compatible family of morphisms $\left\{ \alpha_i \right\}_{i \in \mathrm{Ob} \left(\mathsf{I}\right)}$, with $\alpha_i: X \to F(i)$, 
			$\exists\, !\, \alpha: X \to \varprojlim F$ s.t. $p_i \circ \alpha = \alpha_i$ $\,\forall\, i \in \mathrm{Ob} \left(\mathsf{I}\right)$, i.e.
			\begin{equation}
			\begin{tikzcd}
				X \arrow[r, "\alpha_i", rightarrow] \arrow[d, "\alpha"', rightarrow] &
				F(i)\\
				\varprojlim F \arrow[ru, "p_i"', rightarrow]  
			\end{tikzcd}
			.\end{equation} 
	\end{description} 
\end{defn}

\begin{rem}
	As always, since it is defined through a universal property, if $(\varprojlim F, p_i)$ exists, it is unique up to unique isomorphism.
\end{rem}

\begin{ex}
	Let $\mathsf{I}$ be a small discrete category, i.e. the morphisms in $\mathsf{I}$ are only the identities.
	Then, for any functor $F: \mathsf{I} \to \mathsf{C}$, $\varprojlim F$ exists iff 
	$\prod_{i \in \mathrm{Ob} \left(\mathsf{I}\right)} F(i)$ exists and they are isomorphic.
	In particular the $p_i$ s correspond with the projections of the product.
\end{ex} 

\begin{ex}
	Consider, in an arbitrary category $\mathsf{C}$, the following diagram
	\begin{equation}
	\begin{tikzcd}
		& A_1 \arrow[d, "f_1", rightarrow] \\
		A_2 \arrow[r, "f_2", rightarrow] &
		A_3
	\end{tikzcd}
	.\end{equation} 
	Consider the category $\mathsf{I}$ with $\mathrm{Ob} \left(\mathsf{I}\right) = \left\{ 1, 2, 3 \right\}$ and morphism (other than the identities) $1 \to 3$ and $2 \to 3$.
	Consider the functor $F: \mathsf{I} \to \mathsf{C}$ defined as follows:
	 \begin{equation}
		 \begin{matrix}
			 F(i) = A_i, &
			 F( 1 \to 3) = f_1, &
			 F( 2 \to 3) = f_2,
		 \end{matrix} 
	.\end{equation} 
	Then any $\varprojlim F$ corresponds with a pullback of the above diagram.
\end{ex} 

\begin{defn}[Complete category]
	A category $\mathsf{C}$ is called \textbf{complete} iff 
	every functor $F: \mathsf{I} \to \mathsf{C}$, from a small category $\mathsf{I}$, admits limit in $\mathsf{C}$.
\end{defn}

\begin{rem}[Some terminology]
	Assume that a preadditive category $\mathsf{C}$ has infinite products.
	Consider a functor $F: \mathsf{I} \to \mathsf{C}$, from a small category $\mathsf{I}$.
	For any morphism $\lambda: i \to j$ in $\mathsf{I}$, let's define
	\begin{equation}
		s(\lambda) := i \quad \text{ and } \quad t(\lambda) := j
	,\end{equation} 
	where $s$ denotes the source and $t$ the target of the morphism.
	Consider $\left(\prod_{i \in \mathrm{Ob} \left(\mathsf{I}\right)} F(i),  \pi_i \right)$ a product of $\left\{ F(i) \right\}_{i \in I}$ and the diagram
	\begin{equation}
	\begin{tikzcd}
		\prod_{i \in \mathrm{Ob} \left(\mathsf{I}\right)} F(i) \arrow[r, "\pi_{s(\lambda)}", rightarrow] 
		\arrow[rd, "\pi_{t(\lambda)}"', rightarrow] &
		F \left( s(\lambda) \right) \arrow[d, "F(\lambda)", rightarrow] \\
		& F \left( t(\lambda) \right)
	\end{tikzcd}
	.\end{equation} 
	In general it is not commutative, but we can define the morphism
	\begin{equation}
		\sigma_\lambda := F(\lambda) \circ \pi_{s(\lambda)} - \pi_{t(\lambda)}: \prod_{i \in \mathrm{Ob} \left(\mathsf{I}\right)} F(i) \to F \left( t(\lambda) \right)
	.\end{equation} 
	Let's now consider the product $\left(\prod_{\lambda \in \Lambda} F \left( t(\lambda) \right), q_\lambda\right)$, indexed by $\lambda \in \Lambda := \mathrm{Morph}\, \mathsf{I}$.
	By the universal property of products, the family $\left\{ \sigma_\lambda \right\}_{\lambda \in \Lambda}$ induces a unique morphism
	 \begin{equation}
		 \sigma: \prod_{i \in \mathrm{Ob} \left(\mathsf{I}\right)} F(i) \to \prod_{\lambda \in \Lambda} F \left( t(\lambda) \right)
	\end{equation} 
	s.t. $q_\lambda \circ\sigma = \sigma_\lambda$.
\end{rem}

\begin{prop}\label{prop:LimConstr}
	If a (preadditive) category $\mathsf{C}$ admits kernels and (infinite) products,
	then for every functor $F: \mathsf{I} \to \mathsf{C}$, from a small category $\mathsf{I}$, 
	$\mathsf{C}$ admits $\varprojlim F$.
	Moreover the limit is constructed by kernels and (infinite) products.
\end{prop} 
\begin{proof}
	The proof wants to show that the following construction actually is a limit for $F$.
	Consider $\left(K, \epsilon\right)$ a kernel for the above constructed morphism
	\begin{equation}
		\sigma: \prod_{i \in \mathrm{Ob} \left(\mathsf{I}\right)} F(i) \to \prod_{\lambda \in \Lambda} F \left( t(\lambda) \right)
	.\end{equation} 
	Then $\left(K, p_i\right)$, for $p_i := \pi_i \circ\epsilon$ is a projective limit of $F$.
\end{proof}

\begin{ex}
	Let $\mathsf{C} = \mathsf{Mod}\text{-}R$ and $\left( \mathsf{I}, \leq \right)$ a partially ordered set, viewed as a category.
	Consider a contravariant functor
	\begin{equation}
	F: \mathsf{I}^{op} \to \mathsf{Mod}\text{-}R
	.\end{equation} 
	This is equivalent to the data of $F(i) =: M_i \in \mathsf{Mod}\text{-}R$, and, for all $i \leq j$ of
	\begin{equation}
		F(i \to j) =: f_{ij}: M_j \to M_i
	.\end{equation} 
	Now, given $i \leq j \leq k$, then the following diagram commutes
	\begin{equation}
	\begin{tikzcd}
		M_k \arrow[r, "f_{jk}", rightarrow] \arrow[rd, "f_{ik}"', rightarrow] &
		M_j \arrow[d, "f_{ij}", rightarrow] \\
		& M_i
	\end{tikzcd}
	,\end{equation} 
	in other words $f_{ij} \circ f_{jk} = f_{ik}$.
	Moreover we require $f_{ii} = id_{M_i}$.

	We have, in fact, a correspondance, between contravariant functors from partially ordered sets and
	inverse systems of modules, which are families $\left\{ M_i, F_{ij} \right\}_{i \leq j}$ of modules $M_i$ and morphism $f_{ij}$ between them, satisfying the above compatibility conditions.

	Then, $\varprojlim F$ is called the \textbf{inverse limit} of $\left\{ M_i, f_{ij} \right\}_{i \leq j}$.
	The morphisms $f_{ij}$ are called the structural morphisms of the inverse system.
	Sometimes this is also denoted with $\varprojlim M_i$.
	
	Let's describe $\varprojlim M_i$ explicitly: for every $i \leq j$ we have the (not necessairily commutative) diagram
	\begin{equation}
	\begin{tikzcd}
		\prod_{i \in \mathrm{Ob} \left(\mathsf{I}\right)} M_i \arrow[r, "\pi_j", rightarrow] \arrow[rd, "\pi_i"', rightarrow] &
		M_j \arrow[d, "f_{ij}", rightarrow] \\
		& M_i
	\end{tikzcd} 
	.\end{equation}
	Let's define, for each $i \leq j$, $\sigma_{ij} := f_{ij} \circ\pi_j - \pi_i$.
	Then, by universal property of the product,
	$\exists\, !\, \sigma: \prod_{i \in \mathrm{Ob} \left(\mathsf{I}\right)} M_i \to \prod_{i \leq j} M_{ij}$,
	where $M_{ij} := M_i$ for every $i \leq j$.
	In the above construction $\pi_i \circ\sigma = \sigma_{ij}: \prod_{i \in \mathrm{Ob} \left(\mathsf{I}\right)} M_i \to M_{ij}$.
	Then we have
	\begin{align}
		\varprojlim M_i &\simeq \ker \sigma =
		\left\{ \mathbf{x} \in \prod_{i \in \mathrm{Ob} \left(\mathsf{I}\right)}
			M_i \ \middle|\ \sigma(\mathbf{x}) = 0
		\text{ i.e. } \sigma_{ij}(\mathbf{x}) =0 \,\forall\, i \leq j \right\} \\
				&=
		\left\{ \mathbf{x} = \left( x_i \right)_{i \in \mathrm{Ob} \left(\mathsf{C}\right)} \in
		\prod_{i \in \mathrm{Ob} \left(\mathsf{I}\right)} M_i \ \middle|\ 
		f_{ij}(x_j) = x_i, \ \,\forall\, i \leq j \right\}
	.\end{align} 
	It is a submodule of the product, in which, determined $x_j$, then $\,\forall\, i \leq j$, $x_i$ is determined by $x_j$, via the structural morphisms.
\end{ex} 

\begin{defn}[$I$-adic topology on a ring]
	Given a commutative ring $R$ and an ideal $I \triangleleft R$ of $R$.
	We define the \textbf{$\mathbf{I}$-adic topology on} $R$ as the linear topology determined by
	$\left\{ I^n \right\}_{n \in \N}$ as a basis for the neighbourhoods of $0$.
	The open subsets are generated by cosets of these ideals.
\end{defn}

\begin{rem}
	The \textbf{$\mathbf{I}$-adic topology} on $R$ is Hausdorff iff $\bigcap_{n \in \N} I^n = 0$.
\end{rem} 

\begin{ex}[Completion of a ring in the $i$-adic topology]
	Let $R$ be a commutative ring and $I \triangleleft R$ an ideal of $R$.
	For $n \leq m$, then $I^m \subset I^n$, hence the canonical projections
	\begin{align}
		\pi_{n,m}: R/I^m &\to R/I^n \\
		x + I^m &\mapsto x + I^n
	\end{align} 
	are well defined.
	We can check that $\left\{ R/I^n, \pi_{n,m} \right\}_{n \leq m}$ is a countable inverse system.
	\begin{align}
		\varprojlim R/I^n &=
		\left\{ \left( x_n + I^n \right)_{n \in \N} \in \prod_{n \in \N} R/I^n \ \middle|\ 
		\pi_{n,m}\left( x_m + I^m \right) = x_n + I^n \,\forall\, n \leq m \right\}\\
				  &=
		\left\{ \left( x_n + I^n \right)_{n \in \N} \in \prod_{n \in \N} R/I^n \ \middle|\ 
		x_m - x_n \in I^n \,\forall\, n \leq m \right\}
	.\end{align} 
	This is the \textbf{completion of} $R$ in the $I$-adic topology.
	It is called completion since, given $\left\{ x_n \right\}_{n \in \N}$ it is a \textit{Cauchy} sequence iff
	$\,\forall\, V$ neighbourhood of $0$, $\exists\, n_0 \in \N$ s.t. $x_n - x_m \in V$ for all $n,m \geq n_0$.
	Moreover we can define a \textit{neat Cauchy} sequence as a sequence $\left\{ x_n \right\}_{n \in \N}$ s.t.
	$\,\forall\, V_n := I^n$ m then $x_m - x_n \in V_n$ for all $m \geq n$.

	In particular an element $\left( x_n + I^n \right)_{n \in \N} \in \varprojlim R/I^n$ can be viewed as a limit
	of the Cauchy sequence $\left\{ x_n \right\}_{n \in \N}$.
	(This is the reason why it can be seen as the completion in the topology).

	Moreover we have a canonical projection
	\begin{align}
		\mu: R &\to \varprojlim R/I^n \\
		x &\mapsto \left( x + I^n \right)_{n \in \N}
	.\end{align} 
	Clearly $\ker \mu = \bigcap_{n \in \N} I^n$ (i.e. $\mu$ is injective iff
	$R$ is Hausdorff with the $I$-adic topology).
\end{ex} 

\begin{ex}[$p$-adic completion of the ring of integers]
	Let $R := \mathbb{Z}$ and $I := p  \mathbb{Z}$.
	\begin{equation}
		\hat{\mathbb{Z}}_p := \varprojlim \mathbb{Z}/p^n \mathbb{Z}
	\end{equation} 
	is the $p$-adic completion of the ring of integers.
	An element $\zeta \in \hat{\mathbb{Z}}$ can be written as
	\begin{equation}
	\zeta = a_0 + a_1 p + a_2 p^2 + \ldots
	,\end{equation} 
	with $0 \leq a_i < p$ for all  $i \geq 1$.
	In fact $x_0 + p \mathbb{Z} = a_0 + p\mathbb{Z}$, with $0 \leq a_0 < p$.
	Then $x_1 - x_0 \in p \mathbb{Z}$, hence $x_1 = a_0 + a_1 p$.
	Then, by induction, given  $x_n = a_0 + a_1 p + \ldots + a_{n} p^{n}$ and $x_{n+1} - x_n \in p^{n+1} \mathbb{Z}$, hence
	\begin{equation}
	x_{n+1} = a_0 + \ldots + a_{n+1} p^{n+1}
	.\end{equation} 
\end{ex} 

\subsection{The functor projective lim}
Fix $\mathsf{I}$ a small category and let $\mathsf{C}$ be a complete category.
Let $\mathsf{C}^{\mathsf{I}}$ be the functor category.

\begin{prop}
	\begin{align}
		\varprojlim: \mathsf{C}^{\mathsf{I}} &\to \mathsf{C} \\
		F &\mapsto \varprojlim F
	\end{align} 
	is a functor.
\end{prop} 	
\begin{proof}
	Given $\eta: F \to G$ a natural transformation between the functors
	$F, G \in \mathsf{C}^{\mathsf{I}}$, the functor associates it a morphism in the natural way
	\begin{equation}
	\varprojlim \eta: \varprojlim F \to \varprojlim G
	.\end{equation} 
\end{proof}

Let's study a little the category $\mathsf{C}^{\mathsf{I}}$, for a small category $\mathsf{I}$.
\begin{prop}
	$\mathsf{C}^{\mathsf{I}}$ inherits the properties of $\mathsf{C}$.
	More explicitly if $\mathsf{C}$ is preadditive/additive/abelian, then also
	$\mathsf{C}^{\mathsf{I}}$ is preadditive/additive/abelian.
	
	Morever construction in $\mathsf{C}$ can be done in $\mathsf{C}^{\mathsf{I}}$ locally, for every $i \in \mathrm{Ob} \left(\mathsf{I}\right)$.
	For instance:
	\begin{itemize}
		\item Given $\eta, \zeta \in \mathrm{Hom}_{\mathsf{C}^{\mathsf{I}}} \left( F, G \right)$, if $\mathsf{C}$ is preadditive, then
			$\left( \eta + \zeta \right)_i = \eta_i + \zeta_i$, for each object $i \in \mathrm{Ob} \left(\mathsf{I}\right)$.
		\item If $\mathsf{C}$ has products, then also $\mathsf{C}^{\mathsf{I}}$ has products.
			In particular, given two functors $F,G \in \mathrm{Ob} \left(\mathsf{C}^{\mathsf{I}}\right)$, we need to define the product
			$\left(F \Pi G, \pi_F, \pi_G\right)$, s.t. this is a product of $F$ and $G$ in $\mathsf{C}^{\mathsf{I}}$.
			On objects it is defined as expected
			\begin{equation}
				(F \prod G)(i) := F(i) \prod G(i)
			.\end{equation} 
			Moreover, on morphisms it is defined as follows: given $\lambda: i \to j$, then
			\begin{equation}
				\left( F \prod G \right)(\lambda) = 
				\begin{bmatrix}
					F(\lambda) & 0\\
					0 & G(\lambda)
				\end{bmatrix} 
			,\end{equation} 
			is our morphism $\left( F \Pi G \right)(\lambda): F(i) \Pi G(i) \to F(j) \Pi G(j)$.
			Finally we have to define the projections.
			They are constructed naturally as
			\begin{equation}
				(\pi_F)_i := \pi_{F(i)} \qquad \text{ and } \qquad \left( \pi_G \right)_i := \pi_{G(i)}
			.\end{equation} 
		\item If $\mathsf{C}$ has kernels, then also $\mathsf{C}^{\mathsf{I}}$ has kernels.
			Let $\eta: F \to G$ a natural transformation.
			Let's define $\ker \eta$ as an object of $\mathsf{C}^{\mathsf{I}}$.
			For every $i \in \mathrm{Ob} \left(\mathsf{I}\right)$ we define $K(i) := \ker \eta_i$ as an object in $\mathsf{C}$.
			This, for any morphism $\lambda: i \to j$, gives rise to the commutative diagram
			\begin{equation}
			\begin{tikzcd}
				K(i) \arrow[d, "\exists\, !\, \nu", dashrightarrow] \arrow[r, "\epsilon_i", rightarrow] &
				F(i) \arrow[r, "\eta_i", rightarrow] \arrow[d, "F(\lambda)", rightarrow] &
				G(i) \arrow[d, "G(\lambda)", rightarrow] \\
				K(j) \arrow[r, "\epsilon_j"', rightarrow] &
				F(j) \arrow[r, "\eta_j"', rightarrow] &
				G(j)
			\end{tikzcd}
			.\end{equation} 
			From this we define $K(\lambda) := \nu$.
			Then, the couple $\left(K, \left\{ \epsilon_i \right\}_{i \in \mathrm{Ob} \left(\mathsf{I}\right)} \right)$
			is the kernel of $\left\{ \eta_i \right\}_{i \in \mathrm{Ob} \left(\mathsf{I}\right)} $
		\item As an exercise to the writer: when you'll next read this line, please try to define the cokernel of a functor.
	\end{itemize}
\end{prop} 

\subsection{Characterization of projective limit}
Let, as before, $\mathsf{I}$ be a small category, and $\mathsf{C}^{\mathsf{I}}$ the category of functors $F: \mathsf{I} \to \mathsf{C}$.

\begin{defn}[Constant functor]
	Consider a fixed $X \in \mathrm{Ob} \left(\mathsf{C}\right)$.
	We define the constant functor
	\begin{equation}
	\Delta_X: \mathsf{I} \to \mathsf{C}
	.\end{equation} 
	On objects as $\Delta_X(i) = X$ for all $i \in \mathrm{Ob} \left(\mathsf{I}\right)$.
	On morphism $\Delta_X(\lambda) = id_X$ for all $\lambda: i \to j$.
\end{defn}

\begin{defn}[Diagonal functor]
	We define, in terms of the constant functor, the diagonal functor
	 \begin{equation}
	\Delta: \mathsf{C} \to \mathsf{C}^{\mathsf{I}}
	.\end{equation} 
	On objects as $\Delta(X) := \Delta_X$.
	On morphisms $f: X \to Y$, then
	\begin{equation}
		\Delta(f) := \bar{f}: \Delta_X \to \Delta_Y
	,\end{equation} 
	where $\bar{f}$ is a natural transformation s.t. for every $i \in \mathrm{Ob} \left(\mathsf{I}\right)$, $\bar{f}_i = f$.
\end{defn}

\begin{defn}[Some notation for the following proposition]
	Fix a functor $F \in \mathsf{C}^{\mathsf{I}}$.
	Let $H: \mathsf{C}^{op} \to \mathsf{Sets}$ be a contravariant functor from $\mathsf{C}$, defined as follows.
	On the objects $Y \in \mathrm{Ob} \left(\mathsf{C}\right)$, 
	$H(Y) := \mathrm{Nat} \left( \Delta_Y, F \right)$.
	On the morphisms, for $f: X \to Y$, 
	 \begin{align}
		 H(f): \mathrm{Nat} \left( \Delta_Y, F \right) &\to \mathrm{Nat} \left( \Delta_X, F \right) \\
		 \eta &\mapsto \eta \circ \bar{f}
	.\end{align} 
\end{defn}

\begin{prop}
	Given a functor $F \in \mathsf{C}^{\mathsf{I}}$, then
	$\varprojlim F$ exists iff the functor $H$ defined above is representable.
	In other words iff $\exists\, C \in \mathrm{Ob} \left(\mathsf{C}\right)$ s.t.
	the following two functors are naturally isomorphic
	\begin{equation}
	\mathrm{Hom}_{\mathsf{C}} \left( -, C \right) \simeq_{\varphi} H = \mathrm{Nat} \left( \Delta_{(-)}, F \right)
	.\end{equation} 
	In such case $C \simeq \varprojlim F$, and the compatible family is defined as
	\begin{align}
		\varphi_C: \mathrm{Hom}_{\mathsf{C}} \left( C, C \right) &\to \mathrm{Nat} \left( \Delta_C, F \right) \\
		1_C &\mapsto \bar{p} = \left\{ p_i \right\}_{i \in \mathrm{Ob} \left(\mathsf{I}\right)}
	.\end{align} 
\end{prop} 

\section{Colimit}
Let's dualize the notion of limit, to obtain the notion of colimit.
As usual we consider $\mathsf{I}$ a small category, and $F: \mathsf{I} \to \mathsf{C}$ a functor.

Before we introduce the notion of colimit let's dualize that of compatible family
\begin{defn}[Compatible family]
	Fix $X \in \mathrm{Ob} \left(\mathsf{C}\right)$ and
	consider a family $\left\{ \alpha_i \right\}_{i \in \mathrm{Ob} \left(\mathsf{I}\right)}$ of morphisms $\alpha_i: F(i) \to X$.
	This is said to be a \textbf{compatible family} with respect to $F$ iff, given any morphism $\lambda: i \to j$ in $\mathsf{I}$,
	the following triangle commutes
	\begin{equation}
	\begin{tikzcd}
		F(i) \arrow[r, "\alpha_i", rightarrow] \arrow[rd, "F(\lambda)"', rightarrow] &
		X \\
		&
		F(j) \arrow[u, "\alpha_j"', rightarrow] 
	\end{tikzcd}
	.\end{equation} 
	In other words iff $\alpha_i = \alpha_j \circ F(\lambda)$ for every $i, j \in \mathrm{Ob} \left(\mathsf{I}\right)$ and every $\lambda: i \to j$.
\end{defn}

\begin{defn}[Colimit/Injective (inverse) limit]
	A \textbf{colimit} of $F$, denoted with $\varinjlim F$ is a limit of $F$ in $\mathsf{C}^{op}$.
	More explicitly a colimit is an object in $\mathsf{C}$, still denoted with $\varinjlim F$, 
	with morphisms $\mu_i: F(i) \to \varinjlim F$ satisfying the following conditions
	\begin{description}
		\item[CoLIM1] $\left\{ \mu_i \right\}_{i \in \mathrm{Ob} \left(\mathsf{I}\right)}$ is a compatible family of morphisms, i.e.
			\begin{equation}
			\begin{tikzcd}
				F(i) \arrow[r, "\mu_i", rightarrow] \arrow[rd, "F(\lambda)"', rightarrow] &
				\varinjlim F \\
				&
				F(j) \arrow[u, "\mu_j"', rightarrow] 
			\end{tikzcd}
			\end{equation} 
			the above diagram commutes for all $i, j \in \mathrm{Ob} \left(\mathsf{I}\right)$ and all $\lambda: i \to j$.
		\item[CoLIM2] For any $X \in \mathrm{Ob} \left(\mathsf{C}\right)$ and any compatible family of morphisms
			$\left\{ \alpha_i \right\}_{i \in \mathrm{Ob} \left(\mathsf{I}\right)}$, 
			with $\alpha_i: F(i) \to X$, 
			$\exists\, !\, \alpha: \varinjlim F \to X$ s.t. 
			$\alpha \circ \mu_i = \alpha_i$ $\,\forall\, i \in \mathrm{Ob} \left(\mathsf{I}\right)$, i.e.
			\begin{equation}
			\begin{tikzcd}
				F(i) \arrow[r, "\alpha_i", rightarrow] \arrow[d, "\mu_i"', rightarrow] &
				X\\
				\varinjlim F \arrow[ru, "\alpha"', rightarrow]  
			\end{tikzcd}
			.\end{equation} 
	\end{description} 
\end{defn}

\begin{rem}
	As always, since it is defined through a universal property, if $\left(\varinjlim F, \mu_i \right)$ exists,
	it is unique up to a unique isomorphism.
\end{rem}

\begin{ex}
	Let $\mathsf{I}$ be a small discrete category, i.e. the morphisms in $\mathsf{I}$ are only the identities.
	Then, for any functor $F: \mathsf{I} \to \mathsf{C}$, $\varinjlim F$ exists iff
	$\coprod_{i \in \mathrm{Ob} \left(\mathsf{I}\right)} F(i)$ exists and they are isomorphic.
	In particular the $\mu_i$ s correspond with the embeddings of the coproduct.
\end{ex} 

\begin{ex}
	Consider, in an arbitrary category $\mathsf{C}$, the following diagram
	\begin{equation}
	\begin{tikzcd}
		A_3 \arrow[r, "f_1", rightarrow] \arrow[d, "f_2"', rightarrow] &
		A_1\\
		A_2
	\end{tikzcd}
	.\end{equation} 
	Consider the small category $\mathsf{I}$, with $\mathrm{Ob} \left(\mathsf{I}\right) := \left\{ 1, 2, 3 \right\}$ and morphisms, other than the identities, $3 \to 1$ and $3 \to 2$.
	Consider the functor $F: \mathsf{I} \to \mathsf{C}$ defined as follows:
	\begin{equation}
		F(i) = A_i, \qquad F( 3 \to 1 ) = f_1, \qquad
		F( 3 \to 2 ) = f_2
	.\end{equation} 
	Then any colimit of $F$ corresponds with a pushout of the above diagram.
\end{ex} 

\begin{defn}[Cocomplete category]
	A category $\mathsf{C}$ is called \textbf{cocomplete} iff every functor $F: \mathsf{I} \to \mathsf{C}$, from a small category $\mathsf{I}$, admits colimit in $\mathsf{C}$.
\end{defn}

\begin{prop}\label{prop:ColimConstr}
	If a (preadditive) category $\mathsf{C}$ admits cokernels and (infinite) coproducts, then
	for every functor $F: \mathsf{I} \to \mathsf{C}$, from a small category $\mathsf{I}$, $\mathsf{C}$ admits $\varinjlim F$.
	Moreover the colimit is constructed by cokernels and (infinite) coproducts.
\end{prop} 
\begin{proof}
	As above, the proof wants to show that the following construction asctually is a direct limit for $F$.
	Consider $\left(\coprod_{i \in \mathrm{Ob} \left(\mathsf{I}\right)} F(i), \epsilon_i\right)$ the coproduct of $F(i)$.
	Define
	\begin{equation}
		\psi_\lambda := \epsilon_{t(\lambda)} \circ F(\lambda) - \epsilon_{s(\lambda)}:
		F \left( s(\lambda)  \right) \to \coprod_{i \in \mathrm{Ob} \left(\mathsf{I}\right)} F(i)
	.\end{equation} 
	Then the family $\psi_\lambda$ induces a unique 
	\begin{equation}
		\psi: \coprod_{\lambda \in \Lambda} F \left( s(\lambda) \right) \to \coprod_{i \in \mathrm{Ob} \left(\mathsf{I}\right)} F(i)
	\end{equation} 
	s.t. $\psi \circ \epsilon_{s(\lambda)} = \psi_\lambda$.
	Moreover, we recall that $\Lambda := \mathrm{Morph}\, \mathsf{I}$.
	Then, denoted by $\left(C, p\right)$ a cokernel of $\psi$, $\left(C, \mu_i\right)$, where $\mu_i := p \circ\epsilon_i$, is an injective limit of $F$.
\end{proof}

Fix a small category $\mathsf{I}$ and let $\mathsf{C}$ be a cocomplete category.
\begin{prop}
	\begin{align}
		\varinjlim: \mathsf{C}^{\mathsf{I}} &\to \mathsf{C} \\
		F &\mapsto \varinjlim F
	\end{align} 
	is a functor.
\end{prop} 
\begin{proof}
	It is clear how the functor acts on objects.
	Let's define how it acts on $\eta: F \to G$ a natural transformation between the functors $F, G \in \mathsf{C}^{\mathsf{I}}$.
	It associates to $\eta$ a morphism in the natural way
	\begin{equation}
	\varinjlim \eta: \varinjlim F \to \varinjlim G
	.\end{equation} 
\end{proof}

\begin{prop}
	Given a functor $F \in \mathsf{C}^{\mathsf{I}}$, then $\varinjlim F$ exists iff the functor
	\begin{align}
		H: \mathsf{C} &\to \mathsf{Sets} \\
		Y &\mapsto \mathrm{Nat} \left( F, \Delta_Y \right)
	\end{align} 
	is corepresentable.
	In other words iff $\exists\, C \in \mathrm{Ob} \left(\mathsf{C}\right)$ s.t.
	the following two functors are naturally isomorphic
	\begin{equation}
		\mathrm{Hom}_{\mathsf{C}} \left( C, - \right) \simeq_{\varphi} H = \mathrm{Nat} \left( F, \Delta_{(-)} \right)
	.\end{equation} 
	In such case $C \simeq \varinjlim F$, and the compatible family is defined as
	\begin{align}
		\varphi_C: \mathrm{Hom}_{\mathsf{C}} \left( C, C \right) &\to \mathrm{Nat} \left( F, \Delta_C \right) \\
		1_C &\mapsto \bar{\mu} = \left\{ \mu_i \right\}_{i \in \mathrm{Ob} \left(\mathsf{I}\right)} 
	.\end{align} 
\end{prop} 

Let's now describe a particular case of colimits:
\begin{ex}
	Let $\mathsf{C} := \mathsf{Mod}\text{-}R$ and $\left( \mathsf{I}, \leq \right)$ a partially ordered set, viewed as a category.
	Consider a functor
	\begin{equation}
	F: \mathsf{I} \to \mathsf{Mod}\text{-}R
	.\end{equation} 
	This is equivalent to the data of $F(i) =: M_i \in \mathsf{Mod}\text{-}R$ and, for all $i \leq j$, of
	\begin{equation}
		F(i \to j) =: f_{ji}: M_i \to M_j
	.\end{equation} 
	Now, given $i \leq j \leq k$, then the following diagram commutes
	\begin{equation}
	\begin{tikzcd}
		M_i \arrow[r, "f_{ji}", rightarrow] \arrow[rd, "f_{ki}"', rightarrow] &
		M_j \arrow[d, "f_{kj}", rightarrow] \\
		&
		M_k
	\end{tikzcd}
	,\end{equation} 
	in other words $f_{kj} \circ f_{ji} = f_{ki}$. Moreover we require $f_{ii} = id_{M_i}$.

	We have, in fact, a correspondance, between functors from partially ordered sets and
	direct systems of modules, which are families $\left\{ M_i, f_{ij} \right\}_{i \leq j}$ of modules $M_i$ and morphism $f_{ij}$ between them, satisfying the above compatibility conditions.

	Then, $\varinjlim F$ is called the \textbf{direct limit} of $\left\{ M_i, f_{ij} \right\}_{i \leq j}$.
	The morphisms $f_{ij}$ are called the structural morphisms of the direct system.
	Sometimes this is also denoted with $\varinjlim M_i$.
	
	Let's describe $\varinjlim M_i$ explicitly: for every $i \leq j$ we have the (not necessairily commutative) diagram
	\begin{equation}
	\begin{tikzcd}
		M_i \arrow[r, "\epsilon_j", rightarrow] \arrow[d, "f_{ji}"', rightarrow] &
		\bigoplus_{i \in \mathrm{Ob} \left(\mathsf{I}\right)} M_i \\
		M_j \arrow[ru, "\epsilon_j"', rightarrow] & 
	\end{tikzcd} 
	.\end{equation}
	Let's define, for each $i \leq j$, $\psi_{ij} := \epsilon_j \circ f_{ji} - \epsilon_i$.
	Then, by universal property of the coproduct,
	$\exists\, !\, \psi: \bigoplus_{i \leq j} M_{ij} \to \bigoplus_{i \in \mathrm{Ob} \left(\mathsf{I}\right)} M_{i}$,
	where $M_{ij} := M_i$ for every $i \leq j$.
	In the above construction $\psi \circ \epsilon_i = \psi_{ij}: M_{ij} \to \bigoplus_{i \in \mathrm{Ob} \left(\mathsf{I}\right)} M_i$.
	Then we have
	\begin{align}
		\varinjlim M_i &\simeq \coker \psi =
		\frac{\bigoplus_{i \in \mathrm{Ob} \left(\mathsf{I}\right)} M_i}{\Ima \psi}
	,\end{align}
	where $\Ima \psi$ is generated by
	\begin{equation}
		\left\{ \epsilon_j \circ f_{ji}(x_i) - \epsilon_i(x_i) \ \middle|\ 
		x_i \in M_i,\ i \leq j \right\} \subset
		\bigoplus_{i \in \mathrm{Ob} \left(\mathsf{I}\right)} M_i
	.\end{equation} 
	In particualr the generators of $\Ima \psi$ are of the form
	\begin{equation}
		\left( \ldots, 0, \ldots, 
		0, x_i, 0, \ldots, 0, - f_{ji}(x_i),
		0, \ldots, 0, \ldots \right)
	.\end{equation} 	
	It is a submodule of the product, in which, $x_i$ in position $i$ and $f_{ji}(x_i)$ in position $j$ are identified.
\end{ex}

\begin{defn}[Directed poset]
	A poset $\left(I, \leq \right)$ is said \textbf{directed} (or \textbf{filtered}) iff
	\begin{equation}
	\,\forall\,  i, j \in I \ \exists\, k \in I \text{ s.t. } i \leq k \text{ and } i \leq k
	.\end{equation} 
\end{defn}
Morevoer, if $\left(M_i, f_{ji}\right)_{i \leq j}$, for $I$ filtered, then $\varinjlim M_i$ is called
directed (or filtered) limit.
In general it is easier to describe a colimit on a directed poset.

Before giving an example of one such limit, let's recall a definition:
\begin{defn}[Finitely generated module]
	A module $M_R$ is \textbf{finitely generated} iff there is an epimorphism
	$\phi: R^N := \bigoplus_{i=1}^N R \to M$, for some $N \in \N$.
	If we denote by $e_i$ the generators of $R^N$, then $\phi(e_i) = x_i$ are the generators of $M$.
	In other words we are saying that $\exists\, \left\{ x_1, \ldots, x_N \right\} \subset M$ a finite set of generators 
	s.t.
	\begin{equation}
	\,\forall\,  x \in R, \text{ then } x = \sum_{i=1}^{N} x_i r_i
	.\end{equation} 
\end{defn}

\begin{defn}[Finitely presented module]
	For a finitely generated module, with epimorphism $\phi: R^N \to M$,
	we denote by $K := \ker \phi$, the module of relations of $M$:
	\begin{equation}
	K = 
	\left\{ \left( r_1, \ldots, r_N \right) \in R^N \ \middle|\ \sum_{i=1}^{N} x_i r_i = 0 \right\}
	.\end{equation} 
	We say that $K$ is the module of relations of $M$ (also known as the first syzygy module).
	We say that $M$ is finitely presented if, being finitely generated, has also finitely generated first syzygy module.
\end{defn}

\begin{ex}
	Let $M \in \mathsf{Mod}\text{-}R$.
	Consider the family
	\begin{equation}
	\mathcal{F} := \left\{ N \leq M \ \middle|\ 
	N \text{ is finitely generated } \right\}
	.\end{equation} 
	Let's label the elements $N \in \mathcal{F}$ as $N= N_i$, for some index $i \in \mathsf{I}$, with $\mathsf{I}$ a set of indeces.
	Let's define a partial order on $\mathsf{I}$:
	$i \leq j$ iff $N_i \subset N_j$.
	Moreover, if $i, j \in \mathsf{I}$, then $N_i + N_j$ is finitely generated, hence $\exists\, k \in \mathsf{I}$ s.t.
	$N_i + N_j = N_k$, for some $k \in \mathsf{I}$.
	This makes $\left( \mathsf{I}, \leq \right)$ a filtered poset.
	We then label the inclusions as $\epsilon_{ji}: N_i \to N_j$ and $\epsilon_i: N_i \to M$.
	Clearly this makes $\left(N_i, \epsilon_{ji}\right)_{i \leq j}$ into a direct system.
\end{ex} 

\begin{prop}
	Every $R\text{-}\mathsf{Mod}$ $M$ is a directed limit of its finitely generated submodules.
	More explicitly, in the above notation,
	\begin{equation}
	\left(M, \epsilon_i\right) \simeq \varinjlim N_i
	.\end{equation} 
\end{prop} 

\begin{ex}[Prüfer group]
	An example of a direct limit construction in $\mathsf{C} = \mathsf{Ab}$.
	Let $M_n := \mathbb{Z}/p^n\mathbb{Z} = \left\langle c_n \right\rangle$, for $n \in \N$ and $p \in \N$ a prime number.
	Notice that $c_n$ ha order $p^n$, hence $p^n c_n = 0$.
	We define the structural morphisms of the direct system as
	\begin{align}
		f_{n+1, n}: \mathbb{Z}/p^n\mathbb{Z} &\to \mathbb{Z}/p^{n+1}\mathbb{Z} \\
		c_n &\mapsto p \cdot c_{n+1}
	\end{align} 
	extending it by linearity.
	Moreover, composing consecutive maps, we obtain
	\begin{align}
		f_{m, n}: \mathbb{Z}/p^n\mathbb{Z} &\to \mathbb{Z}/p^m\mathbb{Z} \\
		c_n &\mapsto p^{m-n} c_m
	\end{align} 
	and extending also this by linearity.
	Clearly $\left\{ \mathbb{Z}/p^n\mathbb{Z}, f_{m,n} \right\}_{n \leq m}$ is a direct system 
	(compatibility follows from the definition of $f_{m,n}$).
	We can consider the direct limit, denoted as follows, and called Prüfer group
	\begin{equation}
		\varinjlim \mathbb{Z}/p^n\mathbb{Z} = \Z (p^{\infty}) \simeq \bigcup_{n \in \N} \left\langle c_n \right\rangle
	,\end{equation} 
	where, in the last union, we consider the map $f_{n+1, n}$ as the inclusion of $\left\langle c_n \right\rangle$ in
	$\left\langle c_{n+1} \right\rangle$.
	Carrying out the construction described in the proposition we obtain that
	\begin{equation}
	\varinjlim \left\langle c_n \right\rangle \simeq
	\frac{\bigoplus_{n \in \N} \left\langle c_n \right\rangle}{\left\langle (c_n, -p \cdot c_{n+1}) \ \middle|\ n \in \N \right\rangle}
	.\end{equation} 
\end{ex} 

\subsection{Direct limit of modules}
\begin{lem}
	Let $\mathsf{C} = \mathsf{Mod}\text{-}R$ and $\left(\mathsf{I}, \leq\right)$ be a filtered poset.
	Let $\left\{ M_i, f_{ji} \right\}_{i \leq j}$ be a directed system of modules.
	In the notation of proposition \ref{prop:ColimConstr}, the direct limit $\left(\varinjlim M_i, \mu_i\right)$, has the
	compatible family of maps
	\begin{equation}
	\begin{tikzcd}
		M_i \arrow[r, "\mu_i", rightarrow] \arrow[d, "\epsilon_i", rightarrow] &
		\varinjlim M_i \\
		\bigoplus_{i \in \mathsf{I}} M_i \arrow[ru, "p"', rightarrow] &
	\end{tikzcd}
	.\end{equation} 
	Where $\mu_i := p \circ\epsilon_i$ and $p = \coker \psi$.
	If we denote with $D := \Ima \psi$, then
	every element $x \in \varinjlim M_i = (\bigoplus M_i)/D$ can be written as
	$\mu_i(x_i)$, for some $i \in \mathsf{I}$ and $x_i \in M_i$.

	(Then we can interpet $\varinjlim M_i = \sum_{i \in \mathsf{I}}^{} \mu_i (M_i)$).
\end{lem} 
\begin{proof}
	The idea is simply the fact that $\mathsf{I}$ is filtered 
	(hence for any finite set of indices we can find an index which is bigger than all of them).
	Given this one can easily use the relations to express any finite sum in terms of an element from a single $M_k$.
\end{proof}

\begin{lem}
	In the above notation and hypothesis, let $x = x_{i_1} + \ldots + x_{i_n} \in \bigoplus_{i \in \mathsf{I}} M_i$.
	$x \in D$ iff $\exists\, k \in \mathsf{I}$, $k \geq i_1, \ldots, i_n$ s.t.
	\begin{equation}
		f_{k, i_1}(x_{i_1}) + \ldots f_{k, i_n}(x_{i_n}) = 0 \in M_k
	.\end{equation} 
\end{lem} 

\begin{lem}
	In the above notation and hypothesis, let $x_i \in M_i$.
	Then 
	\begin{equation}
		\mu_i(x_i) = 0 \in \varinjlim M_i \iff \exists\, j \geq i \in \mathsf{I} \text{ s.t. } f_{ji}(x_i) = 0
	.\end{equation} 
\end{lem} 

\begin{prop}
	Let $M_R \in \mathsf{Mod}\text{-}R$, then $M_R$ is a direct limit of finitely presented modules.
\end{prop} 
