\section{Exactness}
\subsection{Subobjects and quotients}
\begin{defn}[Subobject]
	Let $A$ be an object of a category (abelian) $\mathsf{C}$.
	Consider two monomorphism $f: B \to A$ and $g: C \to A$.
	We say that $f \sim g$ iff 
	$\exists\, \alpha: B \to C$ an isomorphism s.t. the following diagram commutes
	\begin{equation}
	\begin{tikzcd}
		B \arrow[r, "f", rightarrow] \arrow[rd, "\alpha"', rightarrow] &
		A\\
		&
		C \arrow[u, "g"', rightarrow] 
	\end{tikzcd}
	\end{equation} 
	i.e. s.t. $g \circ \alpha = f$.
	Clearly this is an equivalence relation.
	An equivalence class of monomorphisms ending in $A$ is called a \textbf{subobject} of $A$.
	Chosen a representative $f: B \to A$ we denote the corresponding subobject by $B \subseteq A$.

	Moreover, given $B_1$ and $B_2$ subobjects of $A$, we say that $B_1 \subseteq B_2$, $B_1$ is a subobject of $B_2$, iff
	$\exists\, \alpha: B_1 \to B_2$ a morphism s.t. the following diagram commutes
	\begin{equation}
	\begin{tikzcd}
		B_1 \arrow[r, "f_1", rightarrow] \arrow[rd, "\alpha"', rightarrow] &
		A\\
		&
		B_2 \arrow[u, "f_2"', rightarrow] 
	\end{tikzcd}
	\end{equation} 
	i.e. s.t. $f_2 \circ \alpha = f_1$.
	Notice that, in this case, $\alpha$ has to be mono.
\end{defn}

\begin{rem}
	If $B_1 \subseteq B_2 \subseteq A$ and $B_2 \subseteq B_1$, then $B_1$ and $B_2$ represent the same subobject of $A$.
\end{rem}

Let's now give the dual definition.
\begin{defn}[Quotient]
	Consider $f: A \to B$ and $g: A \to C$ two epimorphisms.
	We say that $f \sim g$ iff $\exists\, \alpha: B \to C$ an isomorphism s.t. the following diagram commutes
	\begin{equation}
	\begin{tikzcd}
		A \arrow[r, "f", rightarrow] \arrow[rd, "g"', rightarrow] &
		B \arrow[d, "\alpha", rightarrow] \\
		& C
	\end{tikzcd}
	\end{equation} 
	i.e. s.t. $\alpha \circ f = g$.
	Given one such morphism $f: A \to B$ we call the equivalence class a \textbf{quotient} of $A$.
\end{defn}
\begin{rem}[notation]
	Assume that $f: B \to A$ is a subobject of $A$ (i.e. $f$ is a mono).
	We write $A/B$ for the quotient object represented by $\coker f$.
\end{rem}

\begin{lem}
	Let $\mathsf{C}$ be an abelian category.
	Consider two composable morphisms $A \xrightarrow{f} B \xrightarrow{g} C$.
	Then $g \circ f = 0$ iff $\Ima f \subseteq \ker g$, viewed as subobjects of $B$.
\end{lem} 

\begin{lem}
	Let $\mathsf{C}$ be an abelian category.
	Consider two composable morphisms $A \xrightarrow{f} B \xrightarrow{g} C$.
	Then $\ker g \subseteq \Ima f$, viewed as subobjects of $B$, iff $\,\forall\, h: D \to B$ s.t.
	$g \circ h = 0$, $\exists\, !\, h': D \to \Ima f$ s.t. $\mu \circ h' = h$, where $\Ima f \xrightarrow{\mu} B$ is the natural morphism.
	In other words s.t. the following diagram commutes
	\begin{equation}
	\begin{tikzcd}
		A \arrow[r, "f", rightarrow] \arrow[d, "\beta"', rightarrow] &
		B\\
		\Ima f \arrow[ur, "\mu"', rightarrow] &
		D \arrow[u, "h"', rightarrow] \arrow[l, "\exists\, !\, h'", dashrightarrow] 
	\end{tikzcd}
	.\end{equation} 
\end{lem} 

\begin{defn}[Exact sequence]
	Let $\mathsf{C}$ be an abelian category.
	Consider a sequence of composable morphisms in $\mathsf{C}$
	\begin{equation}
	\ldots \to A_n \xrightarrow{f_n} 
	A_{n+1}  \xrightarrow{f_{n + 1}} 
	A_{n+2} \xrightarrow{f_{n + 2}} \ldots
	.\end{equation} 
	The sequence is \textbf{exact} at $n$ iff
	$\Ima f_n = \ker f_{n+1}$ as subobjects of $A_{n + 1}$.
	It is said to be \textbf{exact} iff it is exact at $n$ for every $n$.
\end{defn}

\begin{defn}[Short exact sequence]
	An \textbf{exact} sequence of the form
	\begin{equation}
	0 \to A_1 \xrightarrow{f_1} A_2 \xrightarrow{f_2} A_3 \to 0 
	\end{equation} 
	is called \textbf{short exact sequence}, abbreviated as s.e.q.
	In particular this sequence is exact iff
	$f_1$ is a mono, $f_2$ is an epi, and $\Ima f_1 = \ker f_2$.
\end{defn}

\begin{lem}
	Consider the following exact sequence
	\begin{equation}
	0 \to A \xrightarrow{f} B \xrightarrow{g} C
	.\end{equation} 
	Then $f = \ker g$.
\end{lem} 

\begin{lem}
	Consider the following exact sequence
	\begin{equation}
	A \xrightarrow{f} B \xrightarrow{g} C \to 0
	.\end{equation} 
	Then $g = \coker f$.
\end{lem} 
Let's combine the above lemmas

\begin{prop}
	Consider the following sequence
	\begin{equation}
	0 \to A \xrightarrow{f} B \xrightarrow{g} C \to 0
	.\end{equation} 
	This is exact (i.e. a s.e.q.) iff $f = \ker g$ and $g = \coker f$.
\end{prop} 

\subsection{Functors}
In this section we'll always work with abelian categories $\mathsf{C}$ and $\mathsf{D}$.

\begin{defn}[Exact functor]
	Let $F: \mathsf{C} \to \mathsf{D}$ be an additive functor.
	We say that $F$ is \textbf{exact} iff, for every exact sequence
	\begin{equation}
	A \xrightarrow{f} B \xrightarrow{g} C \quad \text{ in } \mathsf{C}
	,\end{equation} 
	then the image sequence is exact in $\mathsf{D}$ 
	\begin{equation}
	F(A) \xrightarrow{F(f)} F(B) \xrightarrow{F(g)} F(C)
	.\end{equation} 
	Equivalently $F$ is exact if given $\Ima f = \ker g$ in $\mathsf{C}$, then
	$\Ima F(f) = \ker F(g)$ in $\mathsf{D}$.
\end{defn}

\begin{defn}[Left (resp. right) exact functor]
	An additive functor $F: \mathsf{C} \to \mathsf{D}$ is \textbf{left} (resp. \textbf{right}) exact iff,
	for every exact sequence in $\mathsf{C}$
	\begin{equation}
		0 \to A \xrightarrow{f} B \xrightarrow{g} C \quad ( \text{resp. }
		A \xrightarrow{f} B \xrightarrow{g} C \to 0\, )
	,\end{equation} 
	then the image sequence is exact in $\mathsf{D}$ 
	\begin{equation}
		0 \to F(A) \xrightarrow{F(f)} F(B) \xrightarrow{F(g)} F(C) \quad ( \text{resp. }
		F(A) \xrightarrow{F(f)} F(B) \xrightarrow{F(g)} F(C) \to 0\, )
	.\end{equation} 
\end{defn}

\begin{prop}
	An additive functor $F: \mathsf{C} \to \mathsf{D}$ between abelian categories,
	is exact iff it is both left and right exact.
\end{prop} 

\begin{defn}[Split exact sequence]
	A short exact sequence in $\mathsf{C}$ (as usual an abelian category)
	\begin{equation}
	0 \to A \xrightarrow{f} B \xrightarrow{g} C \to 0 
	\end{equation} 
	is said \textbf{split exact} iff $\exists\, \alpha: B \to A \oplus C$ s.t. the following diagram is commutative
	\begin{equation}
	\begin{tikzcd}
		0 \arrow[r, "", rightarrow] &
		A \arrow[r, "f", rightarrow] \arrow[d, "", equal] &
		B \arrow[r, "g", rightarrow] \arrow[d, "\alpha", rightarrow] &
		C \arrow[r, "", rightarrow] \arrow[d, "", equal] &
		0\\
		0 \arrow[r, "", rightarrow] &
		A \arrow[r, "\epsilon_A", rightarrow] &
		A \oplus C \arrow[r, "\pi_C", rightarrow] &
		C \arrow[r, "", rightarrow] &
		0
	\end{tikzcd}
	.\end{equation} 
	Recall that, in matrix notation, the embedding and projection can be written as
	 \begin{equation}
	\epsilon_A = 
	\begin{bmatrix}
		1_A\\ 0
	\end{bmatrix} \qquad \text{ and } \qquad
	\pi_C = 
	\begin{bmatrix}
		0 & 1_c
	\end{bmatrix} 
	.\end{equation} 
\end{defn}

\begin{prop}
	Let $\mathsf{C}$ be an abelian category.
	TFAE
	\begin{enumerate}
		\item The sequence $0 \to A \xrightarrow{f} B \xrightarrow{g} C \to 0$ is split exact,
		\item $\exists\, f': B \to A$ s.t. $f' \circ f = 1_A$,
		\item $\exists\, g': C \to B$ s.t. $g \circ g' = 1_B$.
	\end{enumerate}
	In such a case $f'$ is called a section of $f$, and $g'$ a retraction of $g$.
\end{prop} 

\subsubsection{Some examples}
Recall that, given a category $\mathsf{C}$, we have the natural bifunctor
\begin{equation}
F = \mathrm{Hom}_{\mathsf{C}} \left( - , - \right): \mathsf{C}^{op} \cross \mathsf{C} \to \mathsf{Sets}
.\end{equation} 
Clearly, if $\mathsf{C}$ is preadditive, $F$ is an additive functor. Moreover

\begin{prop}
	Let $\mathsf{C}$ be an abelian category, then
	\begin{equation}
	\mathrm{Hom}_{\mathsf{C}} \left( -, - \right): \mathsf{C}^{op} \cross \mathsf{C} \to \mathsf{Ab}
	\end{equation} 
	is left exact in both variables.
\end{prop} 

\begin{rem}
	Recall that a contravariant functor $F: \mathsf{C} \to \mathsf{D}$, i.e. a covariant functor $F: \mathsf{C}^{op} \to \mathsf{D}$, 
	is left exact iff given any exact sequence in $\mathsf{C}$ 
	\begin{equation}
	A \to B \to C \to 0
	,\end{equation} 
	i.e. $0 \to C \to B \to A$ exact in $\mathsf{C}^{op}$, then the image sequence is exact in $\mathsf{D}$
	\begin{equation}
		0 \to F(A) \to F(B) \to F(C)
	.\end{equation} 
\end{rem}
\begin{rem}
Consider $\mathsf{C} = \mathsf{Mod}\text{-}R$ and $\left(\mathsf{I}, \leq \right)$ a filtered poset, then
the functors $F: \mathsf{I} \to \mathsf{Mod}\text{-}R$ are in correspondance with the direct systems of modules
$\left\{ M_i, f_{ji} \right\}_{i \leq j }$.
Consider the functors $F, G, L \in \mathsf{C}^{\mathsf{I}}$ and their corresponding
direct systems $\left\{ M_i, f_{ji} \right\}_{i \leq j}$, $\left\{ N_i, g_{ji} \right\}_{i \leq j}$ and $\left\{ L_i, l_{ji} \right\}_{i \leq j}$.
Then the the sequence 
\begin{equation}
0 \to F \xrightarrow{\eta} G \xrightarrow{\zeta} L \to 0
\end{equation} 
is exact iff the following diagram is commutative and has exact rows
\begin{equation}
\begin{tikzcd}
	0 \arrow[r, "", rightarrow] &
	M_i \arrow[r, "\eta_i", rightarrow] \arrow[d, "f_{ji}", rightarrow] &
	N_i \arrow[r, "\zeta_i", rightarrow] \arrow[d, "g_{ji}", rightarrow] &
	L_i \arrow[r, "", rightarrow] \arrow[d, "l_{ji}", rightarrow] &
	0 \\
	0 \arrow[r, "", rightarrow] &
	M_i \arrow[r, "\eta_j", rightarrow] &
	N_i \arrow[r, "\zeta_j", rightarrow] &
	L_i \arrow[r, "", rightarrow] &
	0
\end{tikzcd}
,\end{equation} 
for each $i \leq j$ in $\mathsf{I}$.
\end{rem} 

\begin{prop}
	Let $\mathsf{C} = \mathsf{Mod}\text{-}R$ and $\left(\mathsf{I}, \leq \right)$ be a filtered poset.
	Then the functor $\varinjlim: \mathsf{Mod}\text{-}R^{\mathsf{I}} \to \mathsf{Mod}\text{-}R$ is exact.
\end{prop} 

\begin{rem}
	Colimits, in general, are not exact, even in $\mathsf{Mod}\text{-}\Z = \mathsf{Ab}$.
	Consider, in fact, the category $\mathsf{I}$, characterized by
	$\mathrm{Ob} \left(\mathsf{I}\right) := \left\{ 1, 2, 3 \right\}$ and nontrivial arrows
	$1 \to 2$ and $1 \to 2$.
	Consider $F, G, H \in \mathsf{Ab}^{\mathsf{I}}$, defined as follows:
	\begin{equation}
	F : \ \, 
	\begin{tikzcd}
		\Z \arrow[r, "\dot{4}", rightarrow] \arrow[d, "0"', rightarrow] & \Z \\
		\Z &
	\end{tikzcd}\qquad
	G : \ \,
	\begin{tikzcd}
		\Z \arrow[r, "\dot{4}", rightarrow] \arrow[d, "0"', rightarrow] & \Z \\
		\Z &
	\end{tikzcd}\qquad
	H : \ \,
	\begin{tikzcd}
		\mathbb{Z}/2\mathbb{Z} \arrow[r, "0", rightarrow] \arrow[d, "0"', rightarrow] & \mathbb{Z}/2\mathbb{Z} \\
		\mathbb{Z}/2\mathbb{Z} &
	\end{tikzcd}
	.\end{equation} 
	Then $\varinjlim F = \coker \dot{4} = \varinjlim G$ and $\varinjlim H \simeq \mathbb{Z}/2\mathbb{Z}$.
	Consider the natural transofrmation $\dot{2}$ and $\pi$, that give rise to the sequence
	\begin{equation}
	\begin{tikzcd}
		0 \arrow[r, "", rightarrow] &
		F \arrow[r, "\dot{2}", rightarrow] &
		G \arrow[r, "\pi", rightarrow] &
		H \arrow[r, "", rightarrow] &
		0
	\end{tikzcd}
	,\end{equation} 
	which is exact in $\mathsf{Mod}\text{-}\Z^{\mathsf{I}}$.
	In fact this corresponds to
	\begin{equation}
	\begin{tikzcd}
		0 \arrow[r, "", rightarrow] &
		\Z \arrow[r, "\dot{2}", rightarrow] \arrow[d, "\dot{4}", rightarrow] &
		\Z \arrow[r, "\pi", rightarrow] \arrow[d, "\dot{4}", rightarrow] &
		\mathbb{Z}/2\mathbb{Z} \arrow[r, "", rightarrow] \arrow[d, "0", rightarrow] &
		0 \\
		0 \arrow[r, "", rightarrow] &
		\Z \arrow[r, "\dot{2}", rightarrow]&
		\Z \arrow[r, "\pi", rightarrow] &
		\mathbb{Z}/2\mathbb{Z} \arrow[r, "", rightarrow] &
		0
	\end{tikzcd}
	\end{equation} 
	\begin{equation}
	\begin{tikzcd}
		0 \arrow[r, "", rightarrow] &
		\Z \arrow[r, "\dot{2}", rightarrow] \arrow[d, "0", rightarrow] &
		\Z \arrow[r, "\pi", rightarrow] \arrow[d, "0", rightarrow] &
		\mathbb{Z}/2\mathbb{Z} \arrow[r, "", rightarrow] \arrow[d, "0", rightarrow] &
		0 \\
		0 \arrow[r, "", rightarrow] &
		\Z \arrow[r, "\dot{2}", rightarrow]&
		\Z \arrow[r, "\pi", rightarrow] &
		\mathbb{Z}/2\mathbb{Z} \arrow[r, "", rightarrow] &
		0
	\end{tikzcd}
	.\end{equation}
	And both are commutative with exact rows.
	Taking the image by $\varinjlim$ we obtain
	\begin{equation}
	\begin{tikzcd}
		0 \arrow[r, "", rightarrow] &
		\varinjlim F \simeq \mathbb{Z}/4\mathbb{Z} \arrow[r, "\dot{2}", rightarrow] &
		\varinjlim G \simeq \mathbb{Z}/4\mathbb{Z} \arrow[r, "\pi", rightarrow] &
		\varinjlim H \simeq \mathbb{Z}/2\mathbb{Z} \arrow[r, "", rightarrow] &
		0
	\end{tikzcd}
	\end{equation} 
	which is not exact, since $\dot{2}: \mathbb{Z}/4\mathbb{Z} \to \mathbb{Z}/4\mathbb{Z}$ is not injective.
\end{rem}

\begin{prop}
	Let $C := \mathsf{Mod}\text{-}R$ and $\left(\mathsf{I}, \leq\right)$ be a filtered poset.
	Then the functor $\varprojlim: \mathsf{Mod}\text{-}R^{\mathsf{I}} \to \mathsf{Mod}\text{-}R$ is left exact.
\end{prop} 

\begin{ex}
	In general, even in the category $\mathsf{Mod}\text{-}\Z = \mathsf{Ab}$, the functor $\varprojlim$ is not exact.
	It is enough to construct an epimorphism
	\begin{equation}
		\left\{ M_i, f_{ji} \right\}_{i \leq j} \xrightarrow{\zeta} \left\{ N_i, g_{ji} \right\}_{i \leq j} \to 0 
	\end{equation} 
	s.t. the induced $\varprojlim \zeta: \varprojlim M_i \to \varprojlim N_i$  is not epi.

	Let $\mathsf{I} = \N$, with the usual order.
	Let $M_n = \Z$ for every $n$, with structural morphisms $M_m \xrightarrow{3^{m-n}} M_n$, 
	that acts as $x \mapsto x \cdot 3^{m-n}$, for all $m \leq n$, for all $x \in \Z$.
	Let $N_n = \mathbb{Z}/2\mathbb{Z}$ for every $n$, with structural morphisms $id: \mathbb{Z}/2\mathbb{Z} \to \mathbb{Z}/2\mathbb{Z}$.
	We define
	\begin{equation}
	\left\{ M_n, 3^{m-n} \right\}_{ m \leq n} \xrightarrow{\pi} \left\{ \mathbb{Z}/2\mathbb{Z}, id \right\}_{ m \leq n}
	,\end{equation} 
	defined for all $n$ as the canonical projection.
	It clearly is both surjective for all $n$, and (as can be easily checked) it is a natural transformation, hence it is an epi in the category of functors.
	Notice that, given $\left( x_n \right)_{n \in \N} \in \varprojlim M_n$, then $x_1 = 3 \cdot x_2 = \ldots = 3^n x_{n+1}$,
	hence $x_1 \in \bigcap_{n \in \N} 3^n \Z = \emptyset$.
	In other words $\varprojlim M_n = \emptyset$.
	Instead, clearly, $\varprojlim N_n = \mathbb{Z}/2\mathbb{Z}$.
	Then, obviously, $\varprojlim \pi$ cannot be surjective.
\end{ex} 
