\section{Injective and projective objects}
Let $\mathsf{C}$ be an \textit{arbitrary} category.

\begin{defn}[Projective object]
	Let $P \in \mathrm{Ob} \left(\mathsf{C}\right)$.
	$P$ is \textbf{projective} iff given any $\varphi: B \to C$ epimorphism in $\mathsf{C}$, 
	and any morphism $f: P \to C$, then there exists $g: P \to B$ s.t. $\varphi \circ g = f$, 
	i.e. s.t. the following diagram commutes
	\begin{equation}
	\begin{tikzcd}
		B \arrow[r, "\varphi", rightarrow] &
		C \arrow[r, "", rightarrow] &
		0 \\
		&
		P \arrow[lu, "\exists\, g", dashrightarrow] \arrow[u, "f"', rightarrow]  &
	\end{tikzcd}
	.\end{equation} 
	In such case $g$ is called a \textit{lift} of $f$.\newline
	Equivalently:
	$P$ is projective iff
	\begin{equation}
	\mathrm{Hom}_{\mathsf{C}} \left( P, B \right) \xrightarrow{\mathrm{Hom}_{\mathsf{C}} \left( P, \varphi \right)} 
	\mathrm{Hom}_{\mathsf{C}} \left( P, C \right)
	\end{equation} 
	is an epimorphism (a surjection in $\mathsf{Sets}$) for every $\varphi$ epi.
\end{defn}

\begin{rem}
	If, moreover, $\mathsf{C}$ is abelian, then
	$P$ is \textbf{projective} iff $\mathrm{Hom}_{\mathsf{C}} \left( P, - \right)$ is exact.
	Hence $P$ is projective iff $\mathrm{Hom}_{\mathsf{C}} \left( P, -\right)$ is also \textbf{right} exact.
\end{rem}

\begin{defn}[Injective object]
	Let $I \in \mathrm{Ob} \left(\mathsf{C}\right)$.
	$I$ is \textbf{injective} iff $I$ is projective in $\mathsf{C}^{op}$.
	More explicitly, iff given any $\mu: A \to B$ mono in $\mathsf{C}$,
	and any morphism $f: A \to I$, then there exists $g: B \to I$ s.t. $g \circ \mu = f$,
	i.e. s.t. the following diagram commutes
	\begin{equation}
	\begin{tikzcd}
		0 \arrow[r, "", rightarrow]  &
		A \arrow[r, "\mu", rightarrow] \arrow[d, "f"', rightarrow] &
		B \arrow[dl, "\exists\, g", dashrightarrow] \\
		& I &
	\end{tikzcd}
	.\end{equation} 
	In such case $g$ is called an \textit{extension} of $f$.\newline
	Equivalently:
	$I$ is injective iff
	\begin{equation}
	\mathrm{Hom}_{\mathsf{C}} \left( B, I \right) \xrightarrow{\mathrm{Hom}_{\mathsf{C}} \left( \mu, I \right)} 
	\mathrm{Hom}_{\mathsf{C}} \left( A, I \right)
	\end{equation} 
	is an epimorphism (a surjection in $\mathsf{Sets}$) for every $\mu$ mono.
\end{defn}

\begin{rem}
	If, moreover, $\mathsf{C}$ is abelian, then
	$I$ is \textbf{injective} iff $\mathrm{Hom}_{\mathsf{C}} \left( -, I \right)$ is exact.
	Hence $I$ is injective iff $\mathrm{Hom}_{\mathsf{C}} \left( -, I\right)$ is also \textbf{right} exact.
\end{rem}

\begin{prop}
	Consider $\left\{ P_i \right\}_{i \in I}$ a family of objects in $\mathsf{C}$ an arbitrary category.
	Assume that $\coprod_{i \in I} P_i$ exists.
	Then $\coprod_{i \in I} P_i$ is a projective object in $\mathsf{C}$ 
	iff $P_i$ is projective $\,\forall\, i \in I$.
\end{prop} 

Dually we have the following

\begin{prop}
	Consider $\left\{ I_\lambda \right\}_{\lambda \in \Lambda}$ a family of objects in $\mathsf{C}$ an arbitrary category.
	Assume that $\prod_{\lambda \in \Lambda} I_\lambda$ exists.
	Then $\prod_{\lambda \in \Lambda} I_\lambda$ is an injective object in $\mathsf{C}$ 
	iff $I_\lambda$ is injective $\,\forall\, \lambda \in \Lambda$.
\end{prop} 

\begin{prop}[Baer's criterion for injectivity]
	Let $\mathsf{C} = \mathsf{Mod}\text{-}R$.
	$E_R \in \mathsf{Mod}\text{-}R$ is an injective module iff
	for any ideal $I_R \triangleleft R$ and every $f: I \to E$, there exists $g: R \to E$ s.t.
	the following diagram commutes
	 \begin{equation}
	\begin{tikzcd}
		I_R \arrow[r, "\mu", rightarrow] \arrow[d, "f"', rightarrow] &
		R \arrow[ld, "\exists\, g", dashrightarrow] \\
		E &
	\end{tikzcd}
	,\end{equation} 
	where $\mu: I_R \to R$ is the inclusion.
	In other words we ask $g \circ \mu = f$.
\end{prop} 

\begin{defn}[Enough projectives/injectives]
	Consider a category $\mathsf{C}$.
	\begin{itemize}
		\item We say that $\mathsf{C}$ has \textbf{enough projectives} iff,
	given any $C \in \mathrm{Ob} \left(\mathsf{C}\right)$, 
	there exists a projective object $P \in \mathrm{Ob} \left(\mathsf{C}\right)$ and an epimorphism $\varphi: P \to C$.
	\item We say that $\mathsf{C}$ has \textbf{enough injectives} iff, 
		given any object $C \in \mathrm{Ob} \left(\mathsf{C}\right)$,
		there exists an injective object $E \in \mathrm{Ob} \left(\mathsf{C}\right)$ and a monomorphism
		$\mu: C \to E$.
	\end{itemize}
\end{defn}

\begin{defn}[Free module]
	Let $\mathsf{C} := \mathsf{Mod}\text{-}R$.
	$M_R \in \mathsf{Mod}\text{-}R$ is \textbf{free} iff it has a free set of generators $\left\{ x_i \right\}_{i \in I}$, 
	with $x_i \in M$ for all $i$, s.t. $\,\forall\, x \in M$ it can be written in a unique way as a linear combination of the generators.
	More explicitly
	\begin{equation}
	x = \sum_{i \in I}^{} x_i r_i \qquad \text{with } r_i \text{ almost all zero}
	.\end{equation} 
	Clearly $M$ is free iff $M = \bigoplus_{i \in I}R_i$
	(clearly interpreting the direct sum as a coproduct in the infinite case),
	with $R_i \simeq R$ for all $i$.
	In such case it has $\left\{ e_i \right\}_{i \in I}$ as a basis.
	Another notation for $\bigoplus_{i \in I} R_i$ is $R^{(I)}$.
\end{defn}

\begin{rem}
	It is easy to show that any free module is projective.
\end{rem}

\begin{prop}
	Let $\mathsf{C} = \mathsf{Mod}\text{-}R$.
	$P_R \in \mathsf{Mod}\text{-}R$ is projective iff 
	it is a direct summand of a free module.
\end{prop} 

\begin{rem}
	Projective modules are easy to describe.
	For injective ones we are able to do so only for a specific class of rings, for example for PIDs.

	For this purpose, recall that a module $M_R$ is \textbf{divisible} iff
	\begin{equation}
	\,\forall\, x \in M, \ \,\forall\, 0 \neq r \in R, \quad \exists\, y \in M \text{ s.t. } x = yr
	.\end{equation} 
\end{rem}

\begin{prop}
	Let $R$ be a PID, consider the category $\mathsf{C} := \mathsf{Mod}\text{-}R$.
	$E_R \in \mathsf{Mod}\text{-}R$ is injective iff it is divisible.
\end{prop} 
\begin{proof}
	It seems to me that any injective module is also divisible
	as soon as $xR \simeq R$ for any $x \in R$, i.e. I guess for integral domains
	(Baer's lemma still holds and we can check it on every principal ideal).
	The converse, however, requires that all ideals are principal.
\end{proof}

\begin{ex}[category with no nonzero projective objects]
	Let $\mathsf{C} := \mathsf{T}$ the full subcategory of $\mathsf{Ab}$ of torsion abelian groups.
	Then $\mathsf{T}$ has enough injectives, but no nonzero projective objects.
	\begin{itemize}
		\item Notice that $\mathsf{T} \subset \mathsf{Ab} = \mathsf{Mod}\text{-}\Z$,
			and $\Z$ is a PID.
			Then a torsion group is injective iff it is divisible.

			Consider an arbitrary $T \in \mathrm{Ob} \left(\mathsf{T}\right)$.
			Then $T$ has a set of generators $\left\{ x_i \right\}_{i \in I}$, each of order
			$o(x_i) = n_i \in \N$.
			Then we have an epimorphism
			\begin{equation}
			\varphi: \bigoplus_{i \in I} \mathbb{Z}/n_i\mathbb{Z} \twoheadrightarrow T
			.\end{equation} 
			Then an injective element $I \in \mathsf{T}$ containing $T$ is
			\begin{equation}
			\bigoplus_{i \in I} \mathbb{Q}/n_i \Z
			,\end{equation} 
			which is divisible, hence injective, and contains
			\begin{equation}
			\bigoplus_{i \in I} \mathbb{Z}/n_i\mathbb{Z}
			.\end{equation} 
			Finally we have an injection, given by the inclusion, which 
			states that $\mathsf{T}$ has enough injectives:
			\begin{equation}
			\frac{\bigoplus_{i \in I} \mathbb{Z}/n_i\mathbb{Z}}{\ker \varphi} \hookrightarrow 
			\frac{\bigoplus_{i \in I} \mathbb{Q}/n_i\Z}{\ker \varphi}
			.\end{equation} 
		\item There is a well-known fact saying that a subgroup of a direct sum
			of cyclic abelian groups is a direct sum of cyclic abelian groups.

			Consider $0 \neq T \in \mathrm{Ob} \left(\mathsf{T}\right)$,
			and assume it is projective.
			Then, for $\left\{ x_i \right\}_{i \in I}$ the generators of $T$, as above,
			\begin{equation}
			\begin{tikzcd}
				\bigoplus_{i \in I} \mathbb{Z}/n_i\mathbb{Z} \arrow[r, "\varphi", rightarrow] &
				T \arrow[r, "", rightarrow] &
				0 \\
				&
				T \arrow[ul, "\psi", dashrightarrow] \arrow[u, "1_T"', equals] &
			\end{tikzcd}
			.\end{equation} 
			Then $T$ is a subgroup of a direct sum of cyclic groups
			($1_T$ is injective, hence so has to be $\psi$).
			By the above remark
			\begin{equation}
			T \simeq \bigoplus_{j \in J} \mathbb{Z}/m_j\mathbb{Z}
			.\end{equation} 
			Since $T \neq 0$, then there exists $m_0$ s.t.
			$\mathbb{Z}/m_0\mathbb{Z} \neq 0$, and it is a projective object, since
			it is a direct summand of a projective object.
			Let's now consider the epimorphism
			\begin{equation}
			\mathbb{Z}/m_0^2\mathbb{Z} \twoheadrightarrow \mathbb{Z}/m_0\mathbb{Z}
			.\end{equation} 
			Reasoning as before we obtain that $\mathbb{Z}/m_0\mathbb{Z}$ is a direct
			summand of $\mathbb{Z}/m_0^2\mathbb{Z}$, which is a contradiction.
	\end{itemize}
\end{ex} 

\subsection{Functor categories}
\begin{rem}
	Let $\mathsf{I}$ be a small and preadditive category.
	Let $\mathsf{C}$ be an abelian category.
	Define
	$\underline{\mathrm{Hom}_{\mathsf{}} \left( \mathsf{I}, \mathsf{C} \right)} 
	\subset \mathsf{C}^{\mathsf{I}}$
	the subcategory of all additive functors $F: \mathsf{I} \to \mathsf{C}$.
	In this situation 
	$\underline{\mathrm{Hom}_{\mathsf{}} \left( \mathsf{I}, \mathsf{C} \right)}$ is abelian.
\end{rem}

\begin{lem}[Yoneda]
	Let $\mathsf{I}$ be as above.
	Let $\mathsf{C} := \mathsf{Ab}$.
	Fix $X \in \mathrm{Ob} \left(\mathsf{I}\right)$ and
	$F \in \underline{\mathrm{Hom}_{\mathsf{}} \left( \mathsf{I}^{op}, \mathsf{Ab} \right)}$.
	There is an isomorphism
	\begin{equation}
	\mathrm{Nat} \left( h^X, F \right) \xrightarrow{\theta_{X,F}} 
	F(X)
	,\end{equation} 
	natural in $X$ and in $F$.
	Recall that $h^X := \mathrm{Hom}_{\mathsf{I}} \left( -, X \right)$.
\end{lem} 

\begin{rem}[An application of Yoneda lemma]
	Consider $X, X' \in \mathrm{Ob} \left(\mathsf{I}\right)$. 
	Let $F := h^{X'}$, then \textit{Yoneda lemma} implies
	\begin{equation}
		\mathrm{Nat} \left( h^X, h^{X'} \right) \simeq h^{X'}(X) = \mathrm{Hom}_{\mathsf{I}} \left( X, X' \right)
	.\end{equation} 
\end{rem}

\begin{defn}[Yoneda embedding]
	Consider $\mathsf{I}$ small and preadditive, $\mathsf{C} = \mathsf{Ab}$.
	We define the \textbf{Yoneda embedding} as the functor
	$Y: \mathsf{I} \to \underline{\mathrm{Hom}_{\mathsf{}} \left( \mathsf{I}^{op}, \mathsf{Ab} \right)}$,
	defined on objects as
	\begin{align}
	Y: \mathsf{I} &\to \underline{\mathrm{Hom} \left( \mathsf{I}^{op}, \mathsf{Ab} \right)} \\
		X &\mapsto h^X
	\end{align} 
	and on morphisms, given $f: X \to X'$, by
	\begin{equation}
		Y(f):= \mathrm{Hom}_{\mathsf{I}} \left( -, f \right): h^X \to h^{X'}
	.\end{equation} 
\end{defn}

\begin{prop}
	The Yoneda embedding $Y$ is \textbf{fully faithful}.
	Moreover it sends distinct objects of $\mathsf{I}$ to distinct objects of
	$\underline{\mathrm{Hom}\left( \mathsf{I}^{op}, \mathsf{Ab} \right)}$.
\end{prop} 

\begin{cor}
	Consider a small preadditive category $\mathsf{I}$.
	Then $\mathsf{I}$ is equivalent to the full subcategory of
	$\underline{\mathrm{Hom}\left( \mathsf{I}^{op}, \mathsf{Ab} \right)}$ 
	consisting of the representable functors.
\end{cor} 

\begin{prop}
	For $\mathsf{I}$ as before (small and preadditive) and
	$X \in \mathrm{Ob} \left(\mathsf{I}\right)$, then
	$h^X$ is a projective object of 
	$\underline{\mathrm{Hom} \left( \mathsf{I}^{op}, \mathsf{Ab} \right)}$.
\end{prop} 

\begin{defn}[Generator of a category]
	Let $\mathsf{C}$ be a category.
	An object $G \in \mathrm{Ob} \left(\mathsf{C}\right)$ is a \textbf{generator} of $\mathsf{C}$ iff
	$\mathrm{Hom}_{\mathsf{C}} \left( G, - \right): \mathsf{C} \to \mathsf{Sets}$ is faithful.
	In other words iff the maps of sets
	 \begin{equation}
	\mathrm{Hom}_{\mathsf{C}} \left( C, D \right) \to
	\mathrm{Hom}_{\mathsf{Sets}} \left( \mathrm{Hom}_{\mathsf{C}} \left( G, C \right), 
	\mathrm{Hom}_{\mathsf{C}} \left( G, D \right) \right)
	\end{equation} 
	is injective for every $C, D \in \mathrm{Ob} \left(\mathsf{C}\right)$.
\end{defn}

\begin{rem}[Equivalent definition]
	$G$ is a generator, iff for every pair $f,g: C \to D$
	s.t. $\mathrm{Hom}_{\mathsf{C}} \left( G, f \right) = \mathrm{Hom}_{\mathsf{C}} \left( G, g \right)$,
	i.e. $g \circ \alpha = f \circ \alpha$ for all $\alpha: G \to C$, then $f = g$.

	In the case of a preadditive category $\mathsf{C}$, then $G$ is a generator iff
	for all morphisms $f$ in $\mathsf{C}$ s.t. $\mathrm{Hom}_{\mathsf{C}} \left( G, f \right) = 0$,
	i.e. s.t. $f \circ \alpha = 0$ (whenever admissible), then $f = 0$.
\end{rem}

\begin{defn}[Alternative notation for (co)products]
	Fix $X \in \mathrm{Ob} \left(\mathsf{C}\right)$ and $I$ a set.
	\begin{itemize}
		\item If $\prod_{i \in I} X_i$, with $X_i := X$ for all $i \in I$, exists we define the notation
			\begin{equation}
			X^I := \prod_{i \in I} X_i
			.\end{equation} 
		\item If $\coprod_{i \in I} X_i$, with $X_i := X$ for all $i \in I$, exists we define the notation
			\begin{equation}
				X^{(I)} := \coprod_{i \in I} X_i
			.\end{equation} 
	\end{itemize}
\end{defn}

\begin{prop}
	Assume that $\mathsf{C}$ has arbitrary coproducts.
	TFAE
	\begin{enumerate}
		\item $G$ is a generator of $\mathsf{C}$,
		\item $\,\forall\, X \in \mathrm{Ob} \left(\mathsf{C}\right)$, there is an epimorphism
			$G^{(I)} \twoheadrightarrow X$, for some set $I$.
	\end{enumerate}
\end{prop} 

\begin{defn}[Cogenerator of a category]
	Let $\mathsf{C}$ be a category.
	An object $C \in \mathrm{Ob} \left(\mathsf{C}\right)$ is a \textbf{cogenerator} of $\mathsf{C}$ iff
	$C$ is a generator in $\mathsf{C}^{op}$, i.e. iff
	$\mathrm{Hom}_{\mathsf{C}} \left( -, C \right): \mathsf{C}^{op} \to \mathsf{Sets}$ is faithful.
	In other words iff the maps of sets
	 \begin{equation}
	\mathrm{Hom}_{\mathsf{C}} \left( A, B \right) \to
	\mathrm{Hom}_{\mathsf{Sets}} \left( \mathrm{Hom}_{\mathsf{C}} \left( B, C \right), 
	\mathrm{Hom}_{\mathsf{C}} \left( A, C \right) \right)
	\end{equation} 
	is injective for every $A, B \in \mathrm{Ob} \left(\mathsf{C}\right)$.
\end{defn}

\begin{rem}[Equivalent definition]
	$C$ is a cogenerator, iff for every pair $f,g: A \to B$
	s.t. $\mathrm{Hom}_{\mathsf{C}} \left( f, C \right) = \mathrm{Hom}_{\mathsf{C}} \left( g, C \right)$,
	i.e. $\alpha \circ f = \alpha \circ g$ for all $\alpha: B \to C$, then $f = g$.

	In the case of a preadditive category $\mathsf{C}$, then $C$ is a generator iff
	for all morphisms $f$ in $\mathsf{C}$ s.t. $\mathrm{Hom}_{\mathsf{C}} \left( f, C \right) = 0$,
	i.e. s.t. $\alpha \circ f = 0$ (whenever admissible), then $f = 0$.
\end{rem}

\begin{prop}
	Assume that $\mathsf{C}$ has arbitrary products.
	TFAE
	\begin{enumerate}
		\item $C$ is a cogenerator of $\mathsf{C}$,
		\item $\,\forall\, X \in \mathrm{Ob} \left(\mathsf{C}\right)$, there is a monomorphism
			$\mu: X \rightarrowtail C^{I}$, for some set $I$.
	\end{enumerate}
\end{prop} 

\begin{ex}
	Let $\mathsf{C} := \mathsf{Mod}\text{-}R$.
	$R$ is a generator of $\mathsf{Mod}\text{-}R$:
	given a module $M_R$, and $\left\{ x_i \right\}_{i \in I}$ a set of generators for $M$, then
	\begin{equation}
		R^{(I)} = \bigoplus_{i \in I} R_i \xrightarrow{\phi} M \to 0
	,\end{equation} 
	in which $\phi(e_i) := x_i$.
	Moreover $R$ is projective, hence it is a projective generator.
\end{ex} 

\begin{rem}[A not-so-easy-to-prove fact about modules]
	Let $\mathsf{C} := \mathsf{Mod}\text{-}R$.
	Every module $M$ can be embedded in an injective module
	(i.e. $\mathsf{Mod}\text{-}R$ has enough injectives).
	Moreover every module $M$ admits an injective envelope, denoted $E(M)$,
	where the envelope is a minimal injective module containing $M$.
\end{rem}

\begin{ex}
	Let $\mathsf{C}:= \mathsf{Mod}\text{-}R$ as before.
	Let $\mathcal{S}$ be the set of simple modules $S \in \mathsf{Mod}\text{-}R$ (i.e. modules with no proper submodules).
	Recall that $S \in \mathcal{S}$ iff $S \simeq R/\mathfrak{m}_R$, for some maximal ideal $\mathfrak{m}_R \triangleleft R$.
	Given $S \in \mathcal{S}$, consider its injective envelope $E(S)$, and finally
	let's define
	\begin{equation}
		C := \prod_{S \in \mathcal{S}} E(S) \simeq
		\prod_{\mathfrak{m}_R \in \mathrm{Max}\, R} E \left( R / \mathfrak{m}_R \right)
		\in \mathrm{Ob} \left(\mathsf{C}\right)
	.\end{equation} 
	Then $C$ is an injective cogenerator of $\mathsf{Mod}\text{-}R$.
	In fact, consider $0 \neq X_R \in \mathsf{Mod}\text{-}R$, and $0 \neq x \in X_R$.
	Then $\left\langle x \right\rangle \simeq R/I$, 
	for $I = \left\{ r \in R \ \middle|\ xr = 0 \right\} \triangleleft R$.
	Consider any maximal ideal $\mathfrak{m}_R \triangleleft R$ s.t. $I \subset \mathfrak{m}_R$,
	then, since $E(R/\mathfrak{m}_R)$ is injective, we have the commutative diagram
	\begin{equation}
	\begin{tikzcd}
		0 \arrow[r, "", rightarrow] &
		\left\langle x \right\rangle \arrow[r, "", hookrightarrow] \arrow[d, "\pi"', rightarrow] &
		X \arrow[ldd, "\exists\, f_x \neq 0", rightarrow] \\
		& R/\mathfrak{m}_R \arrow[d, "", hookrightarrow] & \\
		& E(R/\mathfrak{m}_R) &
	\end{tikzcd}
	.\end{equation} 
	Then, for every $0 \neq x \in X$, we have the map
	\begin{equation}
		0 \neq f_x: X \to E \left( R/\mathfrak{m}_R \right) \hookrightarrow
		\prod_{\mathfrak{m}_R \in \mathrm{Max}\, R} E \left( R / \mathfrak{m}_R \right) =: C
	.\end{equation} 
	Then, by the universal property of products, viewing $X$ as a set,
	\begin{equation}
	\exists\, !\, f: X \hookrightarrow C^X
	\end{equation} 
	induced by the various $f_x$.
	Moreover this $f$ is mono, since $f_x(x) \neq 0$ for any $0 \neq x$.
\end{ex} 

\begin{rem}
	Notice that, if $\mathsf{C}$ has a projective generator, then $\mathsf{C}$
	has enough projectives.
	Analogously, if $\mathsf{C}$ has an injective cogenerator, then $\mathsf{C}$ 
	has enough injectives.
\end{rem} 

\subsection{Grothendieck categories}
\begin{defn}[Grothendieck category]
	An abelian category $\mathsf{C}$ is a \textbf{Grothendieck} category iff
	it is cocomplete, it has a generator, and filtered direct limits are exact in $\mathsf{C}$.
\end{defn}

\begin{rem}[Important fact]
	A Grothendieck category has injective envelopes, in particular injective cogenerators.
	Though it might have no nonzero projective objects.
\end{rem}

\begin{ex}\leavevmode\vspace{-.2\baselineskip}
	\begin{itemize}
		\item $\mathsf{Mod}\text{-}R$ and $R\text{-}\mathsf{Mod}$ are both Grothendieck categories.
		\item The category of coherent sheaves is Grothendieck, but has no nonzero projective objects.
		\item It can be shown that also the category of torsion abelian groups is Grothendieck.
	\end{itemize}
\end{ex} 
